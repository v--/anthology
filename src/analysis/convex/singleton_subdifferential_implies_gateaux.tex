\begin{definition}
  Let $X$ be a Banach space, let $D \subseteq X$ be open and convex and let $f: D \to \R$ a continuous convex function. We define the \uline{one-sided derivative at $x \in X$ for the direction $h \in X$} as
  \begin{align*}
    d^+ f(x)(h) \coloneqq \lim_{t \downarrow 0} \frac {f(x + th) - f(x)} t.
  \end{align*}
\end{definition}

\begin{lemma}
  \label{thm:analysis:convex:difference-quotient-grows}
  For every point $x \in X$ and every direction $h \in X$ the difference quotient is a monotone function of $t > 0$, i.e. for $0 < s < t$
  \begin{align*}
    \frac {f(x + sh) - f(x)} s
    \leq
    \frac {f(x + th) - f(x)} t
  \end{align*}
\end{lemma}
\begin{proof}
  \begin{align*}
    \frac {f(x + sh) - f(x)} s
    =
    \frac t s \frac {f(x + \frac s t t h) - f(x)} t
    =
    \frac t s \frac {f\left(\frac s t (x + th) + (1 - \frac s t) x \right) - f(x)} t
    \leq \\ \leq
    \frac t s \frac {\frac s t f(x + t h) + (1 - \frac s t) f(x) - f(x)} t
    =
    \frac t s \frac s t \frac {f(x + th) - f(x)} t
    =
    \frac {f(x + th) - f(x)} t
  \end{align*}
\end{proof}

\begin{proposition}
  \label{thm:analysis:convex:one-sided-derivatives-exist}
  For every point $x \in X$ and every direction $h \in X$ the one-sided derivative $d^+ f(x)(h)$ exists.
\end{proposition}
\begin{proof}
  We use the convexity of $f$ to obtain
  \begin{align*}
    f(x) = f \left(x + \frac {th} 2 - \frac {th} 2 \right) \leq \frac {f(x + th) + f(x - th)} 2,
    \\
    0 \leq [f(x - th) - f(x)] + [f(x + th) - f(x)],
    \\
    -[f(x - th) - f(x)] \leq [f(x + th) - f(x)],
    \\
    -\frac {f(x + t(-h)) - f(x)} t \leq \frac {f(x + th) - f(x)} t,
  \end{align*}
  thus the difference quotient in $d^+ f(x)(h)$ is bounded below by the difference quotient for $-d^+ f(x)(-h)$.

  \Cref{thm:analysis:convex:difference-quotient-grows} implies that the right difference quotient is non-increasing, thus both limits exist and
  \begin{align*}
    -d^+ f(x)(-h) \leq d^+ f(x)(h).
  \end{align*}
\end{proof}

\begin{proposition}
  \label{thm:analysis:convex:one-sided-derivative-is-max}
  For every direction $h \in X$, we have that
  \begin{align*}
    d^+ f(x)(h) = \max\{ \Prod {x^*} h \mid x^* \in \partial f(x) \}.
  \end{align*}
\end{proposition}
% TODO: prove

\begin{theorem}
  \label{thm:analysis:convex:singleton-subdifferential-implies-gateaux}
  If the subdifferential $\partial f(x)$ at $x \in X$ is a singleton with element $x^*$, then $f$ is Gateaux differentiable at $x$ and $f_G'(x) = x^*$.
\end{theorem}
\begin{proof}
  Let $h \in X$ be arbitrary.~\Cref{thm:analysis:convex:one-sided-derivatives-exist} implies that the one-sided derivatives $d^+ f(x)(-h)$ and $d^+ f(x)(h)$ exist and
  \begin{align*}
    -d^+ f(x)(-h) \leq d^+ f(x)(h).
  \end{align*}

  Assume that $f$ is not Gateaux differentiable at $x$, i.e. for some $h_0 \in X$, we have a strict inequality. Then by~\cref{thm:analysis:convex:one-sided-derivative-is-max}
  \begin{align*}
    \min\{ \Prod {x^*} {h_0} \mid x^* \in \partial f(x) \}
    =
    -\max\{ \Prod {x^*} {-h_0} \mid x^* \in \partial f(x) \}
    =
    -d^+ f(x)(-h_0)
    < \\ <
    d^+ f(x)(h_0)
    =
    \max\{ \Prod {x^*} {h_0} \mid x^* \in \partial f(x) \},
  \end{align*}
  which implies that there is more that one functional $x^* \in \partial_C f(x)$. This contradicts the assumption of the theorem.

  Thus $f$ is Gateaux differentiable at $x$.
\end{proof}
