\subsection{Topological continuity}\label{subsec:topological_continuity}

\begin{definition}\label{def:function_continuous_at_point}
  We say that the function \( f: X \to Y \) between topological spaces is \Def{continuous} at the point \( x_0 \in X \) if \( f(x_0) \) is a limit point\Tinyref{def:convergence_of_function_at_point} of \( f \) at \( x_0 \).

  If limit point is unique (e.g. in Hausdorff spaces\Tinyref{thm:t2_iff_singleton_limits}), this condition can be formulated by \enquote{interchanging} \( \lim \) and \( f \) as follows:
  \begin{equation*}
    f(x_0) = f\left( \lim_{x \to x_0} x \right) = \lim_{x \to x_0} f(x).
  \end{equation*}
\end{definition}

\begin{definition}\label{def:continuous_function}
  We say that the function \( f: X \to Y \) between topological spaces is \Def{everywhere continuous} or simply \Def{continuous} if and of the following conditions hold:
  \begin{defenum}
    \DItem{def:continuous_function/limits} \( f \) is continuous at every point of \( X \) in the sense of \cref{def:function_continuous_at_point}.
    \DItem{def:continuous_function/open} For every open set \( V \in \CT \), the preimage\Tinyref{def:function_preimage} \( f^{-1}(V) \) is open.
    \DItem{def:continuous_function/closed} For every closed set \( F \in \CF_{\CT_Y} \), the preimage \( f^{-1}(F) \) is closed.
    \DItem{def:continuous_function/base} There exists a base\Tinyref{def:topological_base} \( \CB_{\CT_Y} \subseteq \CT_Y \), such that for every \( V \in \CB_{\CT_Y} \), the preimage \( f^{-1}(V) \) is open.
    \DItem{def:continuous_function/subbase} There exists a subbase\Tinyref{def:topological_subbase} \( \CP_{\CT_Y} \subseteq \CT_Y \), such that for every \( V \in \CP_{\CT_Y} \), the preimage \( f^{-1}(V) \) is open.
    \DItem{def:continuous_function/closure} For every set \( A \subseteq X \), \( f(\Cl(A)) \subseteq \Cl(f(A)) \).
  \end{defenum}

  We denote the set of all continuous functions from \( X \) to \( Y \) by \( C(X, Y) \).
\end{definition}
\begin{proof}
  \begin{description}
    \Implies[def:continuous_function/limits][def:continuous_function/open] Follows from \cref{def:convergence_of_function_at_point/neighborhoods}.
    \Implies[def:continuous_function/open][def:continuous_function/closed] If \( F \in \CF_{\CT_Y} \) is a closed set, \( Y \setminus F \) is open, therefore \( f^{-1}(Y \setminus F) = X \setminus f^{-1}(F) \) is also open. Hence \( f^{-1}(F) \) is closed.
    \Implies[def:continuous_function/open][def:continuous_function/base] \( \CT \) is a base of itself.
    \Implies[def:continuous_function/base][def:continuous_function/subbase] Every base is also a subbase.
    \Implies[def:continuous_function/subbase][def:continuous_function/limits] Follows from the equivalences in \cref{def:convergence_of_function_at_point}.
    \Implies[def:continuous_function/closed][def:continuous_function/closure] Note that
    \begin{equation*}
      A
      \overset {\ref{thm:function_image_preimage_composition/image_first}} \subseteq
      f^{-1}(f(A))
      \overset {\ref{thm:function_preimage_properties/monotonicity}} \subseteq
      f^{-1}(\Cl(f(A))).
    \end{equation*}

    Apply \( f \circ \Cl \) to the above chain of inclusions to obtain
    \begin{equation*}
      f(\Cl(A))
      \subseteq
      f(\underbrace{\Cl}_{\ref{def:continuous_function/closed}}(f^{-1}(\Cl(f(A)))))
      \overset {\ref{thm:function_image_preimage_composition/preimage_first}} \subseteq
      \Cl(f(A)),
    \end{equation*}
    which proves the implication.

    \Implies[def:continuous_function/closure][def:continuous_function/closed] Fix a closed set \( F \subseteq Y \). Then
    \begin{equation}\label{def:continuous_function/closure_implies_closed_right}
      f(\Cl(f^{-1}(F)))
      \overset {\ref{def:continuous_function/closure}} \subseteq
      \Cl(f(f^{-1}(F)))
      \overset {\ref{thm:function_image_preimage_composition/preimage_first}} \subseteq
      \Cl(F)
      =
      F.
    \end{equation}

    Since \( \Cl \) is monotone, we have
    \begin{equation}\label{def:continuous_function/closure_implies_closed_left}
      f(\Cl(f^{-1}(F)))
      \supseteq
      f(f^{-1}(F))
      \overset {\ref{thm:function_image_preimage_composition/image_first}} \supseteq
      F.
    \end{equation}

    From \cref{def:continuous_function/closure_implies_closed_right} and \cref{def:continuous_function/closure_implies_closed_left} it follows that
    \begin{equation*}
      F = f(\Cl(f^{-1}(F))).
    \end{equation*}

    By taking the preimage, we obtain
    \begin{equation*}
      f^{-1}(F)
      =
      f^{-1}(f(\Cl(f^{-1}(F))))
      \overset {\ref{thm:function_image_preimage_composition/preimage_first}} \supseteq
      \Cl(f^{-1}(F)).
    \end{equation*}

    Therefore \( f^{-1}(F) \) is closed.
  \end{description}
\end{proof}

\begin{definition}\label{def:homeomorphism}
  We say that the continuous function \( f: X \to Y \) is \Def{open} (resp. \Def{closed}), if the image \( f(U) \) of an open (resp. closed) in \( \CT_X \) set is open (resp. closed) in \( \CT_Y \).

  If \( f \) is an open bijection, we say that \( f \) is a \Def{homeomorphism}. If \( f \) is only an open injection, we say that \( f \) is a \Def{homeomorphic embedding}.
\end{definition}

\begin{definition}\label{def:parametric_curve}
   Let \( I \) be an interval (of any type) in \( \BR \) with endpoints \( a < b \), not necessarily finite. Depending on the use case, we define a \Def{parametric curve} on \( I \) by any of the non-equivalent definitions

  \begin{defenum}
    \DItem{def:parametric_curve/function} A continuous function \( \gamma: I \to X \) is called a parametric curve.

    \DItem{def:parametric_curve/image} The image \( \Img(\gamma) \) of a parametric curve \( \gamma \) is also called a parametric curve.

    \DItem{def:parametric_curve/equivalence_class} The equivalence class of all continuous functions from \( I \) to \( X \) with
    \begin{equation*}
      \gamma \cong \beta \iff \Img(\gamma) = \Img(\beta) \text{ and the endpoints of } \gamma \text{ and } \beta \text{ coincide}
    \end{equation*}
    is also called a parametric curve.
  \end{defenum}

  The points \( \gamma(a) \) and \( \gamma(b) \) are called the \Def{endpoints} of the curve, \( \gamma(a) \) is the \Def{start} and \( \gamma(b) \) is the \Def{end}. We say that \( \gamma \) \Def{connects} \( a \) and \( b \).

  Parametric curves on \( I = [0, 1] \) are also called \Def{paths}.

  We define some fundamental types of curves:
  \begin{defenum}
    \DItem{def:parametric_curve/closed} The curve \( \gamma \) is called \Def{closed} if its endpoints coincide, i.e. \( \gamma(a) = \gamma(b) \).

    \DItem{def:parametric_curve/simple} The curve \( \gamma \) is called \Def{simple} if the function \( \gamma: I \to Y \) is injective with the possible exception of the endpoints (in which case we speak of \Def{simple closed curves}.
  \end{defenum}

  If \( \gamma: I \to X \) is a parametric curve, related curves are:
  \begin{defenum}
    \DItem{def:parametric_curve/function_graph}\cite[definition 1.20]{Иванов2017} The graph\Tinyref{def:function_graph} \( \Gph(\gamma) \) of \( \gamma \) is a the image of the curve \( \Ol{\gamma}(t, x) \coloneqq (t, \gamma(x)) \) in the topological space \( I \times X \).

    \DItem{def:parametric_curve/implicit}\cite[definition 1.24]{Иванов2017} If \( M \) is a subset of \( X \) and if there exists a curve \( \gamma: I \to X \) such that \( \Im(\gamma) = M \), we call \( M \) an \Def{implicit parametric curve}.
  \end{defenum}
\end{definition}

\begin{definition}\label{def:parametric_hypersurface}
  In analogy to \cref{def:parametric_curve} (and with the caveats of \cref{def:parametric_curve}), we define \Def{parametric hypersurfaces} as follows:

  Let \( \xi \) is a potentially infinite cardinal number, let \( \Card \CA = \xi \) and let \( \{ I_\alpha \}_{\alpha \in \CA} \) be a family of intervals in \( \BR \). We define a parametric hypersurface to be a continuous image from the product space\Tinyref{def:topological_product} \( \prod_{\alpha \in \CA} I_\alpha \) to \( Y \).

  We call \( \xi \) the \Def{dimension} of the hypersurface.
\end{definition}
