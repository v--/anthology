\subsection{Ideals}\label{subsec:ideals}

\begin{definition}\label{def:magma_ideal}
  Let \( M \) be a magma\Tinyref{def:magma/magma} and \( I \) be a subset of \( M \). We say that \( I \) is a \Def{left ideal} (resp. \Def{right ideal}) of \( G \) if the inclusion \( IM \subseteq I \) holds, where we use the convention in \cref{remark:vector_space_set_operations}, that is,
  \begin{equation*}
    IM = \{ x \cdot i \;\colon\; x \in M, i \in I \}.
  \end{equation*}

  If \( I \) is both a left ideal and a right ideal, we say that it is a \Def{two-sided ideal}.
\end{definition}

\begin{proposition}\label{thm:magma_ideal_is_submagma}
  Every two-sided magma ideal is a submagma.
\end{proposition}
\begin{proof}
  Let \( I \) be a two-sided ideal for the magma \( M \). For \( i, j \in I \), since \( I \) is a left ideal, we have \( ij \in I \) and similarly \( ji \in I \) since \( I \) is a right ideal. Thus \( II = I \) and \( I \) is a submagma of \( M \).
\end{proof}

\begin{example}\label{ex:subgroup_is_not_ideal}
  We explicitly give a counterexample to the converse of \cref{thm:magma_ideal_is_submagma}. Define \( G \coloneqq \Z \times \Z \) to be the abelian group\Tinyref{def:magma/abelian_group} given by pointwise addition. Define
  \begin{equation*}
    G' \coloneqq \{ (n, n) \colon n \in \Z \}.
  \end{equation*}

  The set \( G' \) is a subgroup of \( G \) since it is closed under addition and it contains the unit element \( (0, 0) \). It is not an ideal, however, since
  \begin{equation*}
    (n, n) + (n, 0) = (2n, n) \not\in G'.
  \end{equation*}
\end{example}

\begin{proposition}\label{thm:unital_magma_ideal_is_submagma_iff_contains_identity}
  A two-sided ideal of a unital magma is a unital submagma if and only if it contains the identity.
\end{proposition}
\begin{proof}
  Follows from \cref{thm:magma_ideal_is_submagma} and \cref{thm:proper_ideals_containing_identity}.
\end{proof}

\begin{proposition}\label{thm:proper_ideals_containing_identity}
  A left or right ideal of a unital magma contains the identity if and only if it is not proper.
\end{proposition}
\begin{proof}
  Let \( M \) be a unital magma and \( I \) be a left ideal of \( M \). We will prove that \( e \in I \iff I = M \).

  \begin{description}
    \Implies Let \( e \in I \). Then \( ex = x \) for any \( x \in M \), thus \( IM = M \). But \( I \) is an ideal, hence we have that \( IM = I \), thus \( I = IM = M \).
    \ImpliedBy If \( I = M \), then obviously \( e \in M = R \).
  \end{description}

  An analogous proof follows for the case when \( I \) is a right ideal.
\end{proof}

\begin{proposition}\label{thm:commutative_magma_ideals}
  In a commutative\Tinyref{def:algebraic_theory/commutativity} magma\Tinyref{def:magma/magma} \( M \), a subset \( I \subseteq M \) is a left ideal if and only if it is a right ideal. That is, in commutative magmas, it makes no sense to distinguish between left, right and two-sided ideals.
\end{proposition}
\begin{proof}
  For \( i \in I \) and \( x \in M \) by commutativity we have \( ix = xi \), thus \( I \) is a left ideal if and only if it is a right ideal.
\end{proof}

\begin{proposition}\label{thm:product_of_semigroup_ideals_is_in_intersection}
  Fix a semigroup \( G \). If \( I \) and \( J \) are two-sided ideals, so are \( IJ \) and \( I \cap J \) and
  \begin{equation*}
    IJ \subseteq I \cap J.
  \end{equation*}
\end{proposition}
\begin{proof}
  We first show that \( IJ \) is an ideal.

  Take \( i \in I \), \( j \in J \). If \( g \in G \), then associativity gives us
  \begin{equation*}
    g(ij) = (gi)j \in (gi)J \subseteq IJ
  \end{equation*}
  and
  \begin{equation*}
    (ij)g = i(jg) \in I(jg) \subseteq IJ.
  \end{equation*}

  Hence \( IJ \) is closed under the semigroup operation. This makes \( IJ \) a two-sided ideal.

  If \( i \in I \cap J \) and \( g \in G \), obviously \( ig \in I \) and \( ig \in J \), hence \( ig \in I \cap J \). Then \( I \cap J \) is also a two-sided ideal.

  For the inclusion
  \begin{equation*}
    IJ \subseteq I \cap J,
  \end{equation*}
  observe that \( ij \in IJ \) means that \( ij \in iJ = J \) and \( ij \in Ij = I \), thus \( ij \in I \cap J \) and the inclusion holds.
\end{proof}

\begin{definition}\label{def:semiring_ideal}
  Let \( R \) be a semiring and \( I \) be a subset of \( R \). We say that \( I \) is an \Def{ideal} (left, right or two-sided) if \( (I, +) \) is a subgroup of \( (R, +) \) and \( (I, \cdot) \) is a magma ideal\Tinyref{def:magma_ideal} of \( (R, \cdot) \).
\end{definition}

\begin{proposition}\label{thm:semiring_ideal_is_nonunital_subsemiring}
  Two-sided semiring ideals are subsemirings. In the special case where the semiring is unital, an ideal is a unital subsemiring if and only if it is not a proper ideal.
\end{proposition}
\begin{proof}
  Follows from \cref{thm:magma_ideal_is_submagma} and \cref{thm:unital_magma_ideal_is_submagma_iff_contains_identity}.
\end{proof}

\begin{definition}\label{thm:semiring_ideal_iff_kernel}
  A subset of a ring is a two-sided ideal\Tinyref{def:semiring_ideal} if and only if it is the kernel\Tinyref{def:semiring_kernel} of some ring homomorphism.
\end{definition}
\begin{proof}
  \Implies Let \( I \) be a two-sided ideal. Since it is an abelian group, \( I \) is a normal subgroup and thus we can form the quotient group\Tinyref{def:normal_subgroup} \( R / I \) with the canonical projection
  \begin{align*}
    &\pi: R \to R / I \\
    &\pi(x) \coloneqq x + I
  \end{align*}

  Multiplication in \( R \) induces multiplication in \( R / I \) by
  \begin{equation*}
    (x + I) \cdot (y + I) \coloneqq (xy + I).
  \end{equation*}

  It is well defined since if \( x + I = x' + I \) and \( y + I = y' + I \), then
  \begin{align*}
    (x + I) (y + I)
    &=
    xy + (Iy + xI + II)
    = \\ &=
    xy + I
    = \\ &=
    x'y' + I
    = \\ &=
    x'y' + (Iy' + x'I + II)
    = \\ &=
    (x' + I) (y' + I).
  \end{align*}

  Thus the ring structure on \( R \) induces a ring structure on \( R / I \).

  The canonical projection \( \pi \) is an additive group homomorphism. Since we just showed that \( \pi(xy) = \pi(x) \pi(y) \), it follows that it is also a ring homomorphism.

  It only remains to show that \( \ker(\pi) = I \). Since \( I \) is closed under addition, naturally \( I \subseteq \ker(\pi) \). Conversely, if \( x \in \ker(\pi) \), then \( \pi(x) = \pi(0) = I \), i.e. \( x \in I \). Hence \( \ker(\pi) = I \).

  \ImpliedBy Let \( f: R \to T \) is a ring homomorphism. We must show that \( \ker(f) \) is an ideal. If \( i \in \ker(f) \) and \( x \in R \), then
  \begin{equation*}
    f(ix) = f(i) f(x) = 0 f(x) = 0.
  \end{equation*}

  Thus \( ix \in \ker(f) \). Similarly, we can show that \( xi \in \ker(f) \). Thus \( R \ker(f) = \ker(f) R = \ker(f) \) and \( \ker(f) \) is a two-sided ideal.
\end{proof}

\begin{definition}\label{def:generated_ring_ideal}
  Let \( R \) be a commutative ring so that left and right ideals coincide. Let \( S \subseteq R \) be any nonempty subset of \( R \). We define the ideal generated by \( S \) equivalently as either
  \begin{defenum}
    \DItem{def:generated_ring_ideal/minimal} the smallest ideal of \( R \) that contains \( S \).
    \DItem{def:generated_ring_ideal/direct} the ideal
    \begin{equation*}
      \Gen S \coloneqq \left\{ \sum_{i=1}^n r_i s_i \mid r_1, \ldots, r_n \in R, s_1, \ldots, s_n \in S, n \in \Z_{>0} \right\}
    \end{equation*}
    of finite linear combinations.

    \DItem{def:generated_ring_ideal/polynomials} the ideal
    \begin{equation*}
      \Gen S \coloneqq \left\{ p(s_1, s_2, \ldots, s_n) \mid s_1, \ldots, s_n \in S, p \in R[X_1, \ldots, X_n], n \in \Z_{>0} \right\}
    \end{equation*}
  \end{defenum}

  If \( S \) is finite, then \( \Gen S \) is called \Def{finitely generated}. If \( S = \{ s_1, \ldots, s_n \} \), then
  \begin{equation*}
    \Gen S = s_1 R + s_2 R + \cdots s_n R.
  \end{equation*}
\end{definition}

\begin{definition}\label{def:principal_ideal}
  If an ideal \( I \) is generated\Tinyref{def:generated_ring_ideal} by a single element, it is called a \Def{principal ideal}.
\end{definition}

\begin{proposition}\label{thm:product_of_principal_ideals}
  In a commutative unital ring \( R \) the product of the principal ideals \( \Gen{x} \) and \( \Gen{y} \) is \( \Gen{xy} \).
\end{proposition}
