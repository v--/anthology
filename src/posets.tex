\begin{definition}\label{def:poset}\cite[10]{Lectures:general_topology}
  Let $(X, \leq)$ be a partially ordered set (poset) as defined in~\cref{def:order/partial}. We say that
  \begin{defenum}
    \item\label{def:poset/dual} The partially ordered set $(X, \geq)$ where $\geq$ is defined as $a \geq b \iff b \leq a$ is the called the \uline{dual poset}. See~\cref{def:thin_category} for a discussion of the duality.

    \item\label{def:poset/subset_order} For every subset $A \subseteq X$ the pair $(A, \leq_A)$ is a partially ordered set and the restriction
    \begin{align*}
      \leq_A \coloneqq \{ (a, b) \in \leq \colon a \in A \land b \in A \}.
    \end{align*}
    is called the \uline{subset order}.

    \item\label{def:poset/comparable_elements} $x, y \in X$ are \uline{comparable} if either $x \leq y$ or $y \leq x$.
    \item\label{def:poset/total_order} $\leq$ is a \uline{total order on $X$} or a \uline{linear order} if any two elements are comparable.
    \item\label{def:poset/chain} The subset $A \subseteq X$ is a \uline{chain in $X$} if $(A, \leq_A)$ is a total order.
    \item\label{def:poset/antichain} The subset $A \subseteq X$ is an \uline{antichain in $X$} if no two elements of $A$ are compatible.
    \item\label{def:poset/upper_lower_bound} $x \in X$ is an \uline{upper bound for $A \subseteq X$} (or \uline{lower bound} for $A$ in the dual poset $(X, \geq)$) if $y \leq x$ for any $y \in A$.
    \item\label{def:poset/greatest_least_element} $x \in X$ is a \uline{greatest element of $X$} (\uline{least element of $(X, \geq)$}) if $y \leq x$ any $y \in X$.
    \item\label{def:poset/maximal_minimal_element} $x \in X$ is a \uline{maximal element of $X$} (\uline{minimal element of $(X, \geq)$}) if $x \leq y$ implies $x = y$ for any $y \in X$.
    \item\label{def:poset/supremum_infimum} $x \in X$ is a \uline{supremum for $A \subseteq X$} (\uline{infimum for $A$ in $(X, \geq)$}) if $x$ is the least upper bound of $A$, i.e. the least element of the set $U \subset X$ consisting of all upper bounds of $X$.
    \item\label{def:poset/well_order} $\leq$ is a \uline{well-order on $X$} of $X \neq \varnothing$ and if every nonempty subset has a least element.
  \end{defenum}
\end{definition}

\begin{definition}\label{def:thin_category}\cite{nLab:thin_category}
  A category\Tinyref{def:category} $\Bold{P}$ is called a \uline{thin category} if, for every two objects $A, B \in \Bold{P}$, whenever $f, g \in \Bold{P}(A, B)$, we have $f = g$.

  If $\Bold{P}$ is locally small, this is equivalent to saying that any set of morphisms $\Bold{P}(A, B)$ is at most a singleton.
\end{definition}

\begin{proposition}\label{thm:small_thin_category_equivalent_to_poset}
  Any small thin category\Tinyref{def:thin_category} is equivalent to a poset category (in the sense of \cref{def:standard_categories/ord}).
\end{proposition}
\begin{definition}
  Let $\Bold{P}$ be a thin category and let $\Bold{S}$ be a skeletal subcategory of $\Bold{P}$.

  Define $X$ to be the set of objects in $\Bold{S}$ and define the relation $\leq$ by $x \leq y \iff \exists f \in \Bold{S}(x, y)$. Then
  \begin{description}
    \item[\ref{def:order/partial/reflexivity}] $x \leq x$ because of the existence of identity morphisms.
    \item[\ref{def:order/partial/antisymmetry}] If $x \leq y$ and $y \leq x$ implies that there exist morphisms $f: x \to y$ and $g: y \to x$ and, furthermore, these morphisms are unique because $\Bold{S}$ is thin. So we necessarily have that $f \circ g = \Id_x$ and $g \circ f = \Id_y$ so $x$ and $y$ are isomorphic and, since $\Bold{S}$ is skeletal, $x = y$.
    \item[\ref{def:order/partial/transitivity}] If $x \leq y$ and $y \leq z$, then, by composition of morphisms, $x \leq z$.
  \end{description}

  Since $\Bold{S}$ is thin, the category constructed in from $X$ must be equivalent to $\Bold{S}$.
\end{definition}

\begin{note}\label{note:small_thin_category_isomorphic_to_preorder}
  A more general result than \cref{thm:small_thin_category_equivalent_to_poset} states that any small thin category is isomorphic to a preordered set. The proof is the same except we do not consider the skeletal subcategories necessary for proving antisymmetry.
\end{note}
