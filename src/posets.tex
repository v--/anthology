\section{Posets}\label{sec:posets}

\begin{note}\label{note:preodered_sets_as_posets}
  The definitions in \cref{def:poset} mostly make sense for preorders. The study of partial orders\Tinyref{def:order/partial}, however, appears to be much broader than the study of preorders\Tinyref{def:order/preorder}. Especially considering \cref{thm:preorder_to_partial_order}, which gives us a \enquote{normal form} for preorders.
\end{note}

\begin{definition}\label{def:poset}
  Let $(X, \leq)$ be a partially ordered set (poset) as defined in~\cref{def:order/partial}. We say that
  \begin{defenum}
    \item\label{def:poset/strict_order} the order $<$ defined by $x < y \iff (x \leq y \land x \neq y)$ is the corresponding \uline{strict order}\Tinyref{def:order/strict_partial}.

    \item\label{def:poset/dual} the partially ordered set $(X, \geq)$ where $\geq$ is the converse relation\Tinyref{def:derived_relations/converse}, is the called the \uline{dual poset}. See~\cref{def:thin_category} for a discussion of the duality.

    \item\label{def:poset/subset_order} for every subset $A \subseteq X$ the pair $(A, \leq_A)$ is a partially ordered set and the restriction
    \begin{align*}
      \leq_A \coloneqq \{ (a, b) \in \leq \colon a \in A \land b \in A \}.
    \end{align*}
    is called the \uline{subset order}.

    \item\label{def:poset/comparable_elements} $x, y \in X$ are \uline{comparable} if either $x \leq y$ or $y \leq x$.
    \item\label{def:poset/total_order} $\leq$ is a \uline{total order on $X$} or a \uline{linear order} if any two elements are comparable.
    \item\label{def:poset/chain} The subset $A \subseteq X$ is a \uline{chain in $X$} if $(A, \leq_A)$ is a total order.
    \item\label{def:poset/antichain} The subset $A \subseteq X$ is an \uline{antichain in $X$} if no two elements of $A$ are compatible.
    \item\label{def:poset/upper_lower_bound} $x \in X$ is an \uline{upper bound for $A \subseteq X$} (resp. \uline{lower bound} for $A$ in the dual poset $(X, \geq)$) if $y \leq x$ for any $y \in A$.
    \item\label{def:poset/greatest_least_element} $x \in X$ is a \uline{greatest element of $X$} (resp. \uline{least element of $(X, \geq)$}) if $y \leq x$ any $y \in X$.
    \item\label{def:poset/maximal_minimal_element} $x \in X$ is a \uline{maximal element of $X$} (resp. \uline{minimal element of $(X, \geq)$}) if $x \leq y$ implies $x = y$ for any $y \in X$.
    \item\label{def:poset/supremum_infimum} $x \in X$ is a \uline{supremum for $A \subseteq X$} (resp. \uline{infimum for $A$ in $(X, \geq)$}) if $x$ is the least upper bound of $A$, i.e. the least element of the set $U \subset X$ consisting of all upper bounds of $X$.
    \item\label{def:poset/well_order} $\leq$ is a \uline{well-order on $X$} of $X \neq \varnothing$ and if every nonempty subset has a least element.
  \end{defenum}
\end{definition}

\begin{definition}\label{def:thin_category}\cite{nLab:thin_category}
  A category\Tinyref{def:category} $\Bold{P}$ is called a \uline{thin category} if, for every two objects $A, B \in \Bold{P}$, whenever $f, g \in \Bold{P}(A, B)$, we have $f = g$.

  If $\Bold{P}$ is locally small, this is equivalent to saying that any set of morphisms $\Bold{P}(A, B)$ is at most a singleton.
\end{definition}

\begin{definition}\label{def:poset_category}
  We say that $\Bold{P}$ is a \uline{poset category} if it is a small\Tinyref{def:category_cardinality} thin\Tinyref{def:thin_category} skeletal\Tinyref{def:skeletal_category} category.

  See \cref{thm:poset_iff_poset_category}.
\end{definition}

\begin{proposition}\label{thm:poset_iff_poset_category}
  Let $(P, \leq)$ be a poset\Tinyref{def:poset}. Then $P$ is also a poset category\Tinyref{def:poset_category} with objects $P$ and morphisms ordered pairs from $\leq$ (viewed as a relation\Tinyref{def:relation}).

  Conversely, if $\Bold{P}$ is poset category, the pair $\left(\Bold{P}, \bigcup_{x, y \in \Bold{P}} \Bold{P}(x, y) \right)$ is a preordered set.

  Furthermore, infima\Tinyref{def:poset/supremum_infimum} correspond to categorical products\Tinyref{def:categorical_product}, while suprema to coproducts\Tinyref{def:categorical_coproduct}.
\end{proposition}
\begin{proof}
  We will only prove the equivalence of products and infima since the argument for suprema and coproducts is completely dual.

  \begin{description}
    \Implies The relation $\leq$ satisfies
    \begin{description}
      \item[\ref{def:order/partial/transitivity}] hence composition of morphisms is well defined and is associative and $P$ is a thin category\Tinyref{def:thin_category}.
      \item[\ref{def:order/partial/reflexivity}] hence there exist identity morphisms.
      \item[\ref{def:order/partial/antisymmetry}] hence the category is skeletal\Tinyref{def:skeletal_category}.
    \end{description}

    Let $p$ be the product of the set $A \subseteq X$. Then $p \leq x$ for all $x \in A$, hence it is a lower bound. If $q$ is another lower bound, then by definition of product\Tinyref{def:categorical_product}, there exists a unique morphism $q \leq p$. Hence $p$ is the infimum.

    \ImpliedBy\mbox{}
    \begin{description}
      \item[\ref{def:order/partial/reflexivity}] $x \leq x$ because of the existence of identity morphisms.
      \item[\ref{def:order/partial/antisymmetry}] If $x \leq y$ and $y \leq x$ implies that there exist morphisms $f: x \to y$ and $g: y \to x$ and, furthermore, these morphisms are unique because $\Bold{P}$ is thin. So we necessarily have that $f \circ g = \Id_x$ and $g \circ f = \Id_y$ so $x$ and $y$ are isomorphic and, since $\Bold{P}$ is skeletal, $x = y$.
      \item[\ref{def:order/partial/transitivity}] If $x \leq y$ and $y \leq z$, then, by composition of morphisms, $x \leq z$.
    \end{description}

    Now since the category is thin, the infimum of $A$ (if it exists) has a unique morphism $\inf A \leq x$ for any $x \in A$. If $b \leq x$ for all $x \in A$ if another cone\Tinyref{def:categorical_cone}, then necessarily $b \leq \inf A$. Thus the infimum is the categorical product.
  \end{description}
\end{proof}

\begin{note}\label{note:small_thin_category_isomorphic_to_preorder}
  A more general result than \cref{thm:poset_iff_poset_category} states that any small thin category is a preordered set. The proof is the same except we only have isomorphisms in the antisymmetry, not equality.
\end{note}

\begin{proposition}\label{thm:dual_poset_dual_poset_category}
  Dual posets correspond to dual poset categories.
\end{proposition}
