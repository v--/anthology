\subsection{Posets}\label{subsec:posets}

\begin{remark}\label{remark:only_partial_orders}
  Most definitions in \cref{def:poset} make sense for arbitrary preorders\Tinyref{def:order/preorder} and strict partial orders\Tinyref{def:order/strict_partial}. However, we restrict our attention only to partial orders, especially since \cref{thm:preorder_to_partial_order} gives us a \enquote{normal form} for preorders and since strict partial orders\Tinyref{def:order/strict_partial} can easily be converted to partial orders and vice versa as in \cref{thm:strict_partial_order_conversion}.
\end{remark}

\begin{definition}\label{def:poset}
  Let \( (X, \leq) \) be a partially ordered set (poset) as defined in \cref{def:order/partial}. We say that

  \begin{defenum}
    \DItem{def:poset/dual} the partially ordered set \( (X, \geq) \) where \( \geq \) is the converse relation\Tinyref{def:derived_relations/converse}, is the called the \Def{dual poset}. See \cref{def:thin_category} for a discussion of the duality.

    \DItem{def:poset/subset_order} for every subset \( A \subseteq X \) the pair \( (A, \leq_A) \) is a partially ordered set and the restriction
    \begin{equation*}
      \leq_A \coloneqq \{ (a, b) \in \leq \colon a \in A \land b \in A \}.
    \end{equation*}
    is called the \Def{subset order}.

    \DItem{def:poset/comparable_elements} \( x, y \in X \) are \Def{comparable} if either \( x \leq y \) or \( y \leq x \).

    \DItem{def:poset/chain} The subset \( A \subseteq X \) is a \Def{chain in \( X \)} if \( (A, \leq_A) \) is a totally ordered set.

    \DItem{def:poset/antichain} the subset \( A \subseteq X \) is an \Def{antichain in \( X \)} if no two elements of \( A \) are compatible.

    \DItem{def:poset/upper_lower_bound}\cite[170]{Enderton1977} \( x \in X \) is an \Def{upper bound for \( A \subseteq X \)} (resp. \Def{lower bound} for \( A \) in the dual poset \( (X, \geq) \)) if \( y \leq x \) for any \( y \in A \).

    \DItem{def:poset/bounded_set} The set \( A \subseteq X \) is called \Def{bounded from above} (resp. \Def{bounded from below}) if it has an upper bound (resp. lower bound). If a set is bounded from both directions, we simply say that it is \Def{bounded}.

    \DItem{def:poset/largest_smallest_element}\cite[171]{Enderton1977} \( x \in X \) is a \Def{largest element or maximum \( \max X \) of \( X \)} (resp. \Def{smallest element or minimum \( \min X \) of \( (X, \geq) \)}) if \( y \leq x \) any \( y \in X \).

    \DItem{def:poset/maximal_minimal_element}\cite[170]{Enderton1977} \( x \in X \) is a \Def{maximal element of \( X \)} (resp. \Def{minimal element of \( (X, \geq) \)}) if \( x \leq y \) implies \( x = y \) for any \( y \in X \).

    \DItem{def:poset/supremum_infimum}\cite[170]{Enderton1977} \( x \in X \) is a \Def{supremum for \( A \subseteq X \)} (resp. \Def{infimum for \( A \) in \( (X, \geq) \)}) if \( x \) is the least upper bound of \( A \), i.e. the smallest element of the set \( U \subseteq X \) consisting of all upper bounds of \( X \).
  \end{defenum}
\end{definition}

\begin{definition}\label{def:total_order_interval}\cite{nLab:order_topology}
  In a totally ordered set\Tinyref{def:order/partial} \( (P, \leq) \), for any \( a, b \in P \) with \( a \leq b \), we define
  \begin{defenum}
    \DItem{def:total_order_interval/closed} the \Def{closed interval}
    \begin{equation*}
      [a, b] \coloneqq \{ x \in P \colon a \leq x \leq b \}
    \end{equation*}

    \DItem{def:total_order_interval/open} the \Def{open interval}
    \begin{equation*}
      (a, b) \coloneqq \{ x \in P \colon a < x < b \}
    \end{equation*}

    \DItem{def:total_order_interval/half_open} the \Def{half-open intervals}
    \begin{align*}
      (a, b] &\coloneqq \{ x \in P \colon a < x \leq b \}
      \\
      [a, b) &\coloneqq \{ x \in P \colon a \leq x < b \}
    \end{align*}

    \DItem{def:total_order_interval/open_ray} the \Def{open rays}
    \begin{align*}
      (a, \infty) &\coloneqq \{ x \in P \colon a < x \}
      \\
      (-\infty, b) &\coloneqq \{ x \in P \colon x < b \}
    \end{align*}

    \DItem{def:total_order_interval/closed_ray} the \Def{closed rays}
    \begin{align*}
      [a, \infty) &\coloneqq \{ x \in P \colon a \leq x \}
      \\
      (-\infty, b] &\coloneqq \{ x \in P \colon x \leq b \}
    \end{align*}
  \end{defenum}
\end{definition}

\begin{definition}\label{def:thin_category}\cite{nLab:thin_category}
  A category\Tinyref{def:category} \( \Cat{P} \) is called a \Def{thin category} if, for every two objects \( A, B \in \Bold{P} \), whenever \( f, g \in \Bold{P}(A, B) \), we have \( f = g \).

  If \( \Cat{P} \) is locally small, this is equivalent to saying that any set of morphisms \( \Bold{P}(A, B) \) is at most a singleton.
\end{definition}

\begin{definition}\label{def:poset_category}
  We say that \( \Cat{P} \) is a \Def{poset category} if it is a small\Tinyref{def:category_cardinality} thin\Tinyref{def:thin_category} skeletal\Tinyref{def:skeletal_category} category.

  See \cref{thm:poset_iff_poset_category}.
\end{definition}

\begin{proposition}\label{thm:poset_iff_poset_category}
  Let \( (P, \leq) \) be a poset\Tinyref{def:poset}. Then \( P \) is also a poset category\Tinyref{def:poset_category} with objects \( P \) and morphisms ordered pairs from \( \leq \) (viewed as a relation\Tinyref{def:relation}).

  Conversely, if \( \Cat{P} \) is poset category, the pair \( \left(\Bold{P}, \bigcup_{x, y \in \Bold{P}} \Bold{P}(x, y) \right) \) is a preordered set.

  Furthermore, infima\Tinyref{def:poset/supremum_infimum} correspond to categorical products\Tinyref{def:categorical_product}, while suprema to coproducts\Tinyref{def:categorical_coproduct}.
\end{proposition}
\begin{proof}
  We will only prove the equivalence of products and infima since the argument for suprema and coproducts is completely dual.

  \begin{description}
    \Implies The relation \( \leq \) satisfies
    \begin{description}
      \RItem{def:order/partial/transitivity} hence composition of morphisms is well defined and is associative and \( P \) is a thin category\Tinyref{def:thin_category}.
      \RItem{def:order/partial/reflexivity} hence there exist identity morphisms.
      \RItem{def:order/partial/antisymmetry} hence the category is skeletal\Tinyref{def:skeletal_category}.
    \end{description}

    Let \( p \) be the product of the set \( A \subseteq X \). Then \( p \leq x \) for all \( x \in A \), hence it is a lower bound. If \( q \) is another lower bound, then by definition of product\Tinyref{def:categorical_product}, there exists a unique morphism \( q \leq p \). Hence \( p \) is the infimum.

    \ImpliedBy\mbox{}
    \begin{description}
      \RItem{def:order/partial/reflexivity} \( x \leq x \) because of the existence of identity morphisms.
      \RItem{def:order/partial/antisymmetry} If \( x \leq y \) and \( y \leq x \) implies that there exist morphisms \( f: x \to y \) and \( g: y \to x \) and, furthermore, these morphisms are unique because \( \Cat{P} \) is thin. So we necessarily have that \( f \circ g = \Id_x \) and \( g \circ f = \Id_y \) so \( x \) and \( y \) are isomorphic and, since \( \Bold{P} \) is skeletal, \( x = y \).
      \RItem{def:order/partial/transitivity} If \( x \leq y \) and \( y \leq z \), then, by composition of morphisms, \( x \leq z \).
    \end{description}

    Now since the category is thin, the infimum of \( A \) (if it exists) has a unique morphism \( \inf A \leq x \) for any \( x \in A \). If \( b \leq x \) for all \( x \in A \) if another cone\Tinyref{def:categorical_cone}, then necessarily \( b \leq \inf A \). Thus the infimum is the categorical product.
  \end{description}
\end{proof}

\begin{remark}\label{remark:small_thin_category_isomorphic_to_preorder}
  A more general result than \cref{thm:poset_iff_poset_category} states that any small thin category is a preordered set. The proof is the same except we only have isomorphisms in the antisymmetry, not equality.
\end{remark}

\begin{proposition}\label{thm:dual_poset_dual_poset_category}
  Dual posets correspond to dual poset categories.
\end{proposition}

\begin{definition}\label{def:order_homomorphism}
  Let \( (P, \leq_P) \) and \( (Q, \leq_Q) \) be two posets\Tinyref{def:poset}. We say that the function \( f: P \to Q \) is an \Def{order preserving map} or \Def{order homomorphism} if
  \begin{equation*}
    x \leq_P y \implies f(x) \leq_Q f(y).
  \end{equation*}

  The terminology from \cref{def:morphism_invertibility} applies to order homomorphisms because of the category \( \Cat{Pos} \) of posets\Tinyref{def:category_of_posets}.
\end{definition}

\begin{definition}\label{def:category_of_posets}
  Posets with homomorphisms form the category \( \Cat{Pos} \) in the manner described in \cref{def:first_order_model_category}.
\end{definition}

\begin{definition}\label{def:order_topology}\cite{nLab:order_topology}
  Let \( (P, <) \) be a totally ordered set\Tinyref{def:order/partial}. The \Def{order topology induced by \( < \)} is the topology generated by the subbase\Tinyref{def:topological_subbase} of open rays\Tinyref{def:total_order_interval/open_ray}
  \begin{equation*}
    \Cal{P} \coloneqq \{ (a, \infty) \colon a \in P \} \cup \{ (-\infty, b) \colon b \in P \}.
  \end{equation*}
\end{definition}

\begin{lemma}[Zorn's lemma]\label{thm:zorns_lemma}\cite{nLab:zorns_lemma}
  If any chain\Tinyref{def:poset/chain} in a partially ordered set\Tinyref{def:poset} has an upper bound\Tinyref{def:poset/upper_lower_bound}, there exists\AOC a maximal\Tinyref{def:poset/maximal_minimal_element} set in \( X \).
\end{lemma}
