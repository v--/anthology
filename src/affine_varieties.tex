\subsection{Affine varieties}\label{subsec:affine_varieties}

\begin{definition}\label{def:affine_variety}\cite[69]{Kocev2016}
  For each ideal \( I \) of the polynomial ring\Tinyref{def:multivariate_polynomial} \( R[X_1, \ldots, X_k] \), we define its \Def{affine variety} as the locus
  \begin{equation*}
    \Cal{V}(I) \coloneqq \{ (x_1, \ldots, x_k) \in R^n \colon \forall p \in I, p(x_1, \ldots, x_k) = 0 \}
  \end{equation*}
  of the simultaneous roots of all polynomials in \( I \).
\end{definition}

\begin{example}\label{ex:affine_varieties}
  We will work in the ring \( \R[X, Y] \) of real polynomials in two indeterminates.

  \begin{itemize}
    \item The variety of the ideal \( I \coloneqq \Gen{X^2 + Y^2 - 1} \) is the unit circle\Tinyref{def:metric_space/sphere}
    \begin{equation*}
      \Cal{V}(I) = \{ (x, y) \in \R^2 \colon x^2 + y^2 = 1 \}.
    \end{equation*}

    \item A more interesting example is
    \begin{equation*}
      I \coloneqq \Gen{X + Y, X - Y - 1},
    \end{equation*}
    whose variety is
    \begin{equation*}
      \Cal{V}(I) = \{ (x, y) \in \R^2 \colon x = -y \land x = y + 1 \} = \{ (x, y) \in \R^2 \colon 2y = -1 \}.
    \end{equation*}

    This can also be shown algebraically, since the ideal \( I \) is principal and generated by
    \begin{equation*}
      \gcd(X + Y, X - Y - 1) = 2Y + 1.
    \end{equation*}

    \item The ideal
    \begin{equation*}
      I \coloneqq \{ p(X, Y) \in R[X, Y] \colon p(X) = 0 \lor \deg p \neq 0 \}
    \end{equation*}
    contains all polynomials except for the units - the nonzero constants. It is not principal because the only common divisors for all of \( I \) are the units.

    The variety \( \Cal{V}(I) \) for \( I \) is the empty set since it contains polynomials with no common roots - for example, \( X - Y \) and \( X - Y - 1 \).
  \end{itemize}
\end{example}

\begin{definition}\label{def:ideal_of_affine_variety}\cite[70]{Kocev2016}
  Dually to \cref{def:affine_variety}, for each subset \( V \subseteq R^n \), we define its \Def{ideal} as
  \begin{equation*}
    \Cal{I}(V) \coloneqq \{ p \in R[X_1, \ldots, X_n] \colon \forall (x_1, \ldots, x_n) \in V, p(x_1, \ldots, x_n) = 0 \}.
  \end{equation*}
\end{definition}

\begin{remark}\label{remark:nullstelletsatz_etymology}
  The word \enquote{nullstellensatz} is German for \enquote{zero locus theorem}.
\end{remark}

\begin{theorem}[Algebraic nullstellensatz]\label{thm:algebraic_nullstellensatz}\cite[64]{Kocev2016}
  Let \( \K \) be a field, \( A \) be a finitely-generated \( \K \)-algebra\Tinyref{def:algebra_over_ring} and \( M \) be a maximal ideal of \( A \). Then the field \( A / M \) is a finite extension of \( \K \).

  In the special case that \( \K \) is algebraically closed\Tinyref{def:algebraically_closed_field}, we have
  \begin{equation*}
    \K = A / M.
  \end{equation*}
\end{theorem}

\begin{theorem}[Geometric nullstellensatz]\label{thm:geometric_nullstellensatz}\cite[70]{Kocev2016}
  Let \( \K \) be an algebraically closed\Tinyref{def:algebraically_closed_field} field, then for each ideal \( I \subseteq \K[X_1, \ldots, X_n] \) we have the equality
  \begin{equation*}
    \Cal{I}(\Cal{V}(I)) = \sqrt I,
  \end{equation*}
  where \( \sqrt I \) is the radical\Tinyref{def:radical_ideal} of \( I \).
\end{theorem}

\begin{corollary}\label{thm:nullstellensatz_implies_fundamental_theorem_of_algebra}
  \Cref{thm:geometric_nullstellensatz} directly generalizes \cref{thm:fundamental_theorem_of_algebra} and highlights the latter's algebro-geometric significance.
\end{corollary}
\begin{proof}
  Let \( p(X) \in \Co[X] \) be a nonconstant polynomial. We will show that \( p(X) \) has at least one root. Assume\LEM that it is not so. Then its variety \( \Cal{V}(\Gen{p(X)}) \) is empty.

  Non-zero constant polynomials, for example \( 1 \), also have no roots. Therefore
  \begin{equation*}
    1 \in \Cal{I}(\varnothing) = \Cal{I}(\Cal{V}(\Gen{p(X)})).
  \end{equation*}

  By \cref{thm:geometric_nullstellensatz}, we have
  \begin{equation*}
    \Cal{I}(\Cal{V}(\Gen{p(X)})) = \sqrt{\Gen{p(X)}},
  \end{equation*}
  thus \( 1 \) belongs to the radical \( \sqrt{\Gen{p(X)}} \).

  In that case, there exists a positive integer \( k \) such that \( 1^k \in \Gen{p(X)} \). But \( 1^k = 1 \), hence \( 1 \in \Gen{p(X)} \). But \( 1 \) is a unit, hence \( p(X) \) is also a unit, i.e. a nonzero constant polynomial. This contradicts our choice of \( p(X) \).

  The obtained contradiction proves the theorem.
\end{proof}
