\subsection{Affine varieties}\label{subsec:affine_varieties}

As in \cref{sec:commutative_algebra}, by \( R \) will denote a nontrivial commutative unital ring\Tinyref{def:semiring/commutative_unital_ring}.

\begin{definition}\label{def:affine_variety}\cite[69]{Коцев2016}
  For each ideal \( I \) of the polynomial ring\Tinyref{def:multivariate_polynomial} \( R[X_1, \ldots, X_k] \), we define its \Def{affine variety} as the locus
  \begin{equation*}
    \Cal{V}(I) \coloneqq \{ (x_1, \ldots, x_k) \in R^n \colon \forall p \in I, p(x_1, \ldots, x_k) = 0 \}
  \end{equation*}
  of the simultaneous roots of all polynomials in \( I \).

  The \Def{dimension} of \( \Cal{V}(I) \) is defined as the Krull dimension\Tinyref{def:krull_dimension} of the quotient \( R / I \).
\end{definition}

\begin{example}\label{ex:affine_varieties}
  We will work in the ring \( \R[X, Y] \) of real polynomials in two indeterminates.

  \begin{itemize}
    \item The variety of the ideal \( I \coloneqq \Gen{X^2 + Y^2 - 1} \) is the unit circle\Tinyref{def:metric_space/sphere}
    \begin{equation*}
      \Cal{V}(I) = \{ (x, y) \in \R^2 \colon x^2 + y^2 = 1 \}.
    \end{equation*}

    \item A more interesting example is
    \begin{equation*}
      I \coloneqq \Gen{X + Y, X - Y - 1},
    \end{equation*}
    whose variety is
    \begin{equation*}
      \Cal{V}(I) = \{ (x, y) \in \R^2 \colon x = -y \land x = y + 1 \} = \{ (x, y) \in \R^2 \colon 2y = -1 \}.
    \end{equation*}

    This can also be shown algebraically, since the ideal \( I \) is principal and generated by
    \begin{equation*}
      \gcd(X + Y, X - Y - 1) = 2Y + 1.
    \end{equation*}

    \item The ideal
    \begin{equation*}
      I \coloneqq \{ p(X, Y) \in R[X, Y] \colon p(X) = 0 \lor \deg p \neq 0 \}
    \end{equation*}
    contains all polynomials except for the units - the nonzero constants. It is not principal because the only common divisors for all of \( I \) are the units.

    The variety \( \Cal{V}(I) \) for \( I \) is the empty set since it contains polynomials with no common roots - for example, \( X - Y \) and \( X - Y - 1 \).
  \end{itemize}
\end{example}

\begin{definition}\label{def:ideal_of_affine_variety}\cite[70]{Коцев2016}
  Dually to \cref{def:affine_variety}, for each subset \( V \subseteq R^n \), we define its \Def{ideal} as
  \begin{equation*}
    \Cal{I}(V) \coloneqq \{ p \in R[X_1, \ldots, X_n] \colon \forall (x_1, \ldots, x_n) \in V, p(x_1, \ldots, x_n) = 0 \}.
  \end{equation*}
\end{definition}

\begin{remark}\label{remark:nullstelletsatz_etymology}
  The word \enquote{nullstellensatz} is German for \enquote{zero locus theorem}\Tinyref{def:zero_locus}.
\end{remark}

\begin{theorem}[Algebraic nullstellensatz]\label{thm:algebraic_nullstellensatz}\cite[64]{Коцев2016}
  Let \( \K \) be a field, \( A \) be a finitely-generated \( \K \)-algebra\Tinyref{def:algebra_over_ring} and \( M \) be a maximal ideal of \( A \). Then the field \( A / M \) is a finite extension of \( \K \).

  In the special case that \( \K \) is algebraically closed\Tinyref{def:algebraically_closed_field}, we have
  \begin{equation*}
    \K = A / M.
  \end{equation*}
\end{theorem}

\begin{example}\label{ex:algebraic_nullstellensatz_real_over_complex}
  For \( \K = \R \) and \( A = \R[X] \), by \cref{def:complex_numbers/polynomials} \( \C = \R[X] / \Gen{X^2 + 1} \) is a field so the ideal \( \Gen{X^2 + 1} \) is maximal.

  By \cref{thm:algebraic_nullstellensatz}, \( \C \) is a finite degree extension of \( \R \).
\end{example}

\begin{theorem}[Geometric nullstellensatz]\label{thm:geometric_nullstellensatz}\cite[70]{Коцев2016}
  If \( \K \) is an algebraically closed\Tinyref{def:algebraically_closed_field} field, then for each ideal \( I \subseteq \K[X_1, \ldots, X_n] \) we have the equality
  \begin{equation*}
    \Cal{I}(\Cal{V}(I)) = \sqrt I,
  \end{equation*}
  where \( \sqrt I \) is the radical\Tinyref{def:radical_ideal} of \( I \).
\end{theorem}

\begin{example}\label{ex:geometric_nullstellensatz_does_not_hold_for_reals}
  By \cref{thm:reals_not_algebraically_closed}, the field \( \R \) of real numbers\Tinyref{def:real_numbers} is not algebraically closed since \( x^2 + 1 \) has no root. Denote \( I \coloneqq \Gen{x^2 + 1} \)

  Then \( \Cal{V}(I) = \varnothing \) so \( \Cal{I}(\Cal{V})(I) = \R[X] \).

  But \( \sqrt{I} = I \) since \( x^2 + 1 \) is irreducible and thus forms a a prime ideal by \cref{thm:pid_prime_iff_irreducible}.

  Thus \( \Cal{V}(I) \neq \sqrt{I} \) and \cref{thm:geometric_nullstellensatz} does not hold.
\end{example}

\begin{corollary}\label{thm:weak_nullstellensatz}\cite{Tao:nullstellensatz}
  If \( \K \) is an algebraically closed\Tinyref{def:algebraically_closed_field} field, then for each finite collection
  \begin{equation*}
    p_i(X_1, \ldots, X_n), i = 1, \ldots, k
  \end{equation*}
  of polynomials over \( n \) variables either
  \begin{itemize}
    \DItem{thm:weak_nullstellensatz/roots} the system of equations
    \begin{equation}\label{thm:weak_nullstellensatz/system}
      \begin{cases}
        p_1(x_1, \ldots, x_n) = 0 \\
        p_2(x_1, \ldots, x_n) = 0 \\
        \vdots \\
        p_k(x_1, \ldots, x_n) = 0
      \end{cases}
    \end{equation}
    has a solution.

    \DItem{thm:weak_nullstellensatz/bezout} there exist polynomials
    \begin{equation*}
      q_1(X_1, \ldots, X_n), i = 1, \ldots, k
    \end{equation*}
    such that
    \begin{equation*}
      p_1 q_1 + \cdots + p_k q_k = 1.
    \end{equation*}
  \end{itemize}
\end{corollary}
\begin{proof}
  Define the ideal
  \begin{equation*}
    I \coloneqq \Gen{p_1, \ldots, p_k}.
  \end{equation*}

  The following are equivalent:
  \begin{itemize}
    \item The variety \( \Cal{V}(I) \) is not empty.
    \item The system \cref{thm:weak_nullstellensatz/system} has a solution.
    \item The ideal \( \Cal{I}(\Cal{V}(I)) \) is not the whole space \( \K[X_1, \ldots, X_n] \).
    \item By \cref{thm:geometric_nullstellensatz}, the radical \( \sqrt I \) is not the whole space.
    \item The units of \( \K[X_1, \ldots, X_n] \) are not contained in \( \sqrt I \), hence also not contained in \( I \).
    \item There exists a set of polynomials satisfying \cref{thm:weak_nullstellensatz/bezout}.
  \end{itemize}
\end{proof}

\begin{corollary}\label{thm:polynomial_over_closed_field_is_either_invertible_or_has_root}
  A multivariate polynomial over an algebraically closed field either has a root or is invertible.
\end{corollary}
\begin{proof}
  \Cref{thm:weak_nullstellensatz} with \( k = 1 \).
\end{proof}
