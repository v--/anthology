\subsection{Noetherian rings}\label{subsec:noetherian_rings}

\begin{remark}\label{remark:commutative_modules}
  Since \( R \) is commutative, left and right modules over \( R \) are equivalent. We will only refer to either of them as simply \enquote{modules}.
\end{remark}

\begin{definition}\label{def:noetherian_module}\cite[proposition 8.30]{Knapp2016BAlg}
  A module \( M \) over \( R \) is called \Def{Noetherian} if it satisfies any of the following conditions:
  \begin{defenum}
    \DItem{def:noetherian_module/ascending_chain} Every strict chain of submodules of \( M \)
    \begin{equation*}
      M_1 \subsetneq M_2 \subsetneq \ldots
    \end{equation*}
    is finite.

    \DItem{def:noetherian_module/finite_basis} Every submodule of \( M \) is finitely generated\Tinyref{def:free_left_module}.
  \end{defenum}
\end{definition}
\begin{proof}
  \begin{description}
    \Implies[def:noetherian_module/ascending_chain][def:noetherian_module/finite_basis] We can construct a basis as follows: choose\AOC any element \( x_1 \) of \( M \). Next, choose\AOC an element \( x_2 \in M \setminus \Gen {x_1} \), then \( x_3 \in M \setminus \Gen {x_1, x_2} \) and so on.

    This process must stop after finitely many steps because we obtain the strict chain
    \begin{equation*}
      \Gen {x_1} \subsetneq \Gen {x_1, x_2} \cdots
    \end{equation*}
    of submodules.

    \Implies[def:noetherian_module/finite_basis][def:noetherian_module/ascending_chain] Suppose that all submodules of \( M \) are finitely generated. Let
    \begin{equation*}
      M_1 \subsetneq M_2 \subsetneq \ldots
    \end{equation*}
    be a strictly ascending chain of submodules.

    Then the submodule \( N \coloneqq \bigcup_{i=1}^\infty M_i \) is also finitely generated. There exists a member \( M_N \) of the chain containing all the generators of \( N \). Then no further strict inclusion of modules is possible. We conclude that the chain
    \begin{equation*}
      M_1 \subsetneq M_2 \subsetneq \ldots
    \end{equation*}
    is finite.
  \end{description}
\end{proof}

\begin{definition}\label{def:noetherian_ring}
  A \Def{Noetherian ring} is an Noetherian submodule over itself, i.e. it satisfies the conditions in \cref{def:noetherian_module} on its ideals.
\end{definition}

\begin{theorem}[Hilbert basis theorem]\label{thm:hilbert_basis_theorem}\cite[418]{Knapp2016BAlg}
  If \( R \) is Noetherian, so is \( R[X] \).
\end{theorem}
