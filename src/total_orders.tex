\subsection{Total orders}\label{subsec:total_orders}

\begin{Definition}\label{def:totally_ordered_set}
  We say that the partially ordered set \( P \) is \Def{totally ordered} if either the nonstrict order \( \leq \) is \hyperref[def:binary_relation/total]{total} or if the strict order \( < \) is \hyperref[def:binary_relation/trichotomic]{trichotomic}.
\end{Definition}
\begin{proof}
  The two conditions are equivalent because \( x < y \) is trichotomic if and only if \( x \leq y \) is total.
\end{proof}

\begin{Definition}\label{def:poset_chain}
  We call a subset \( A \subseteq X \) of a \hyperref[def:poset]{poset} \( (X, \leq) \) a \Def{chain} if \( (A, \leq_A) \) is a totally \hyperref[def:totally_ordered_set]{ordered} set.

  Dually, we call the subset \( A \subseteq X \) an \Def{antichain} if no two elements of \( A \) are \hyperref[def:preordered_set/comparability]{comparable}.
\end{Definition}

\begin{Definition}\label{def:total_order_interval}\cite{nLab:order_topology}
  In a totally ordered \hyperref[def:poset]{set} \( (P, \leq) \), for any \( a, b \in P \) with \( a \leq b \), we define
  \begin{DefEnum}
    \ILabel{def:total_order_interval/closed} the \Def{closed interval}
    \begin{equation*}
      [a, b] \coloneqq \{ x \in P \colon a \leq x \leq b \}
    \end{equation*}

    \ILabel{def:total_order_interval/open} the \Def{open interval}
    \begin{equation*}
      (a, b) \coloneqq \{ x \in P \colon a < x < b \}
    \end{equation*}

    \ILabel{def:total_order_interval/half_open} the \Def{half-open intervals}
    \begin{align*}
      (a, b] &\coloneqq \{ x \in P \colon a < x \leq b \}
      \\
      [a, b) &\coloneqq \{ x \in P \colon a \leq x < b \}
    \end{align*}

    \ILabel{def:total_order_interval/open_ray} the \Def{open rays}
    \begin{align*}
      (a, \infty) &\coloneqq \{ x \in P \colon a < x \}
      \\
      (-\infty, b) &\coloneqq \{ x \in P \colon x < b \}
    \end{align*}

    \ILabel{def:total_order_interval/closed_ray} the \Def{closed rays}
    \begin{align*}
      [a, \infty) &\coloneqq \{ x \in P \colon a \leq x \}
      \\
      (-\infty, b] &\coloneqq \{ x \in P \colon x \leq b \}
    \end{align*}
  \end{DefEnum}
\end{Definition}

\begin{Definition}\label{def:order_topology}\cite{nLab:order_topology}
  Let \( (P, <) \) be a totally ordered \hyperref[def:poset]{set}. The \Def{order topology induced by \( < \)} is the topology generated by the \hyperref[def:topological_subbase]{subbase} of open \hyperref[def:total_order_interval/open_ray]{rays}
  \begin{equation*}
    \Cal{P} \coloneqq \{ (a, \infty) \colon a \in P \} \cup \{ (-\infty, b) \colon b \in P \}.
  \end{equation*}
\end{Definition}

\begin{Lemma}[Zorn's lemma]\label{thm:zorns_lemma}\cite{nLab:zorns_lemma}
  If any \hyperref[def:poset_chain]{chain} in a \hyperref[def:poset]{partially ordered set} has an upper \hyperref[def:preordered_set/upper_lower_bound]{bound}, there exists a \hyperref[def:preordered_set/maximal_minimal_element]{maximal} set in \( X \).
\end{Lemma}

\begin{Definition}\label{def:well_ordered_set}
  A totally ordered \hyperref[def:totally_ordered_set]{set} \( (X, \leq) \) is said to be \Def{well-ordered} if every nonempty subset \( A \subseteq X \) has a \hyperref[def:preordered_set/largest_smallest_element]{minimum}.
\end{Definition}

\begin{Theorem}[Well-Ordering Principle]\label{thm:well_ordering_principle}\cite[196]{Enderton1977}
  Any \hyperref[def:set_zfc]{set} can be well-\hyperref[def:well_ordered_set]{ordered}.
\end{Theorem}
