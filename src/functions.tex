\subsection{Functions}\label{subsec:functions}

\begin{remark}\label{remark:function_definition}
  It is not straightforward to formalize the notion of correspondence between two values. We will reserve the term~\Def{mapping} for this informal notion and use~\Def{function} in the sense of \cref{def:function}. There are several drawbacks of using set theory for defining functions:
  \begin{enumerate}
    \item Mappings are often more general than what can be formalized, i.e. there exist correspondences between logical formulas\Tinyref{def:first_order_formula} and between proper classes\Tinyref{def:set_zfc} that cannot be defined in set theory without reaching contradictions.
    \item The ambient space often has an additional structure, e.g. algebraic or topological, that is not carried by functions. This leads to definitions such as homomorphism\Tinyref{def:first_order_homomorphism} and isometry\Tinyref{def:isometry}. This is a motivating example for the benefits category theory\Tinyref{sec:category_theory}, where the notion of morphism\Tinyref{def:category} is able to capture this additional structure (see \cref{def:category_of_sets}).
    \item Several generalizations of set-theoretic functions are often used, e.g. multivalued\Tinyref{def:function/multivalued} or partial functions\Tinyref{def:function/partial}, however most formalisms of set theory often only concern functions.
    \item Set-theoretic functions are often used in contexts where they do not refer to the intuitive notion of a mapping, e.g. for Cartesian products\Tinyref{def:cartesian_product} or for indexed families\Tinyref{def:indexed_family}.
  \end{enumerate}

  Although definitions in terms of set-valued mappings appear simpler and more general, they are also more cumbersome to work with, so we will start with the standard notion of a function.
\end{remark}

\begin{definition}\label{def:function}
  Let \( X \) and \( Y \) be (potentially empty) sets.

  Let \( f \subseteq X \times Y \) be a relation on the nonempty sets \( X \) and \( Y \). We call the triple \( (f, X, Y) \) a \Def{function from \( X \) to \( Y \)} if for every \( x \in X \), there exists exactly one \( y \in Y \) such that \( (x, y) \in f \).

  We call \( X \) the~\Def{domain of \( f \)} and denote it by \( \Dom f = X \) and we call \( Y \) the~\Def{range or codomain of \( f \)} and denote it by \( \Range f = Y \). It is customary to define a function solely in terms of the relation \( f \), however in practice the domain and range are important and the range is impossible to recover given only the relation \( f \).

  We usually write \( f(x) = y \) instead of \( (x, y) \in f \) and
  \begin{align*}
    &f: X \to Y, \\
    &f(x) = \ldots,
  \end{align*}
  where the ellipsis is called the~\Def{definition of \( f \)} and is part of a formula that is true whenever \( (x, f(x)) \in f \). \( f(x) \) is called the~\Def{value of \( f \) at \( x \)} or the~\Def{action of \( f \) on \( x \)} or the~\Def{image of \( x \) under \( f \)}. For a set \( A \subseteq X \), we define
  \begin{equation*}
    f(A) \coloneqq \cup_{a \in A} \{ f(a) \}
  \end{equation*}
  and call \( f(A) \) the~\Def{image of \( A \) under \( f \)} or the~\Def{action of \( f \) on \( A \)}. We call \( f(X) \subseteq Y \) simply the \Def{image of \( f \)} and denote it by \( \Img f \). (even if \( Y \) is a proper class, \( f(X) \) is a set by \cref{def:set_zfc/A9})

  In analogy to programming languages, we can call \( f: X \to Y \) the~\Def{type or type signature of \( f \)} (although formally we use sets rather than types) or the~\Def{declaration of \( f \)} (see, for example, \cite[section 2.4]{Kernighan1988}). It is often enough to only declare \( f \) (i.e. specify its type) without defining it in order to use it in practice.

  Functions are often called maps, mapping or operators, however:
  \begin{itemize}
    \item \Def{Mapping} is often used as a more general informal notion of a correspondence between values
    \item \Def{Operator} is usually used to refer to functions, especially linear functions, that act on sets (rather than points) of a certain ambient space
  \end{itemize}

  The following deviations from the classical notion of a function are commonly used:
  \begin{itemize}
    \DItem{def:function/partial} The function \( f: X \to Y \) is called a\Def{partial function} if there may exist points \( x \in X \) without a value \( f(x) \). In this context,
    \begin{itemize}
      \item the \Def{domain of \( f \)} refers only to the points of \( X \) that have values.
      \item standard functions are called~\Def{total functions}.
    \end{itemize}

    These notions are rarely used outside of logic.

    \DItem{def:function/multivalued} Functions of the type \( f: X \to \Power Y \), usually denoted as \( f: X \MultTo Y \), are called ~\Def{multivalued or set-valued mappings} between \( X \) and \( Y \) (these are usually referred to as mappings rather than functions). In this context,
    \begin{itemize}
      \item the \Def{domain of \( f \)} is defined as
      \begin{equation*}
        \Dom f \coloneqq \{ x \in X \colon f(x) \neq \varnothing \}
      \end{equation*}
      and if \( \Dom f = X \), we call \( f \) a \Def{total multivalued function}.

      \item the \Def{image of a set \( A \) under \( f \)} is defined as
      \begin{equation*}
        f(A) = \cup_{a \in A} f(a).
      \end{equation*}

      \item partial functions are called~\Def{single-valued functions} and if a partial function \( g: X \to Y \) agrees with a multivalued function \( f: X \MultTo Y \), i.e.
      \begin{equation*}
        \forall x \in \Dom f, g(x) \in f(x),
      \end{equation*}
      then \( g \) is called a \Def{selection of \( f \)}.

      \item if \( f \) is total then the selections of \( f \) are total functions

      \item if the value \( f(x) \) of each point \( x \in X \) is a singleton set, we call \( f \)~\Def{single-valued} and generally do not distinguish between \( f \) or its selection.
    \end{itemize}
  \end{itemize}

  We sometimes denote the set of all functions from \( X \) to \( Y \) by \( \Cat{Set}(X, Y) \) (which is consistent with \cref{def:category_of_sets}) or by \( Y^X \) (which is consistent with cardinal arithmetic\Tinyref{def:cardinal_arithmetic}).
\end{definition}

\begin{definition}\label{def:function_extension}
  Let \( X \), \( Y \) and \( X' \subset X \) be sets. We say that the function \( f: X \to Y \) is an \Def{extension} of \( f': X' \to Y \) to \( X \) if, for all \( x \in X' \), we have \( f(x) = f'(x) \). We also say that \( f' \) is a \Def{restriction} of \( f \) onto \( X' \).
\end{definition}

\begin{definition}\label{def:function_composition}
  If \( f: A \to B \) and \( g: B \to C \) are functions, we define their \Def{composition} \( gf: A \to C \) by
  \begin{equation*}
    (gf)(x) \coloneqq g(f(x))
  \end{equation*}
  for all \( x \in A \).

  If the case where \( B = B_1 \times \cdots \times B_n \) we speak of \Def{superposition}. In more concrete terms, if we are given the functions \( f_i: A \to B_i, i = 1, \ldots, n \) and \( g: B_1 \times \cdots \times B_n \to C \), we define their \Def{superposition} \( h: A \to C \) as
  \begin{equation*}
    h(x) \coloneqq g(f_1(x), \ldots, f_n(x)).
  \end{equation*}

  The terms \enquote{composition} and \enquote{superposition} are used interchangeably (e.g. see \cite[44]{Enderton1977} and \cite[\textnumero 25]{Фихтенгольц1968/1}).
\end{definition}

\begin{definition}\label{def:function_preimage}
  Let \( f: X \to Y \) be a function. We define the \Def{inverse multivalued function} as
  \begin{align*}
    &f^{-1} : Y \MultTo X \\
    &f^{-1}(y) \coloneqq \{ x \in X \colon f(x) = y \}.
  \end{align*}

  The set \( f^{-1}(y) \) is called the \Def{preimage} of \( y \).

  If \( f : X \MultTo Y \) is a multivalued mapping, we can define two types of preimages for a set \( B \subseteq Y \):
  \begin{defenum}
    \item The \Def{small preimage},
    \begin{equation*}
      f_{-1}(B) \coloneqq \{ x \in X \colon f(x) \subseteq B \}.
    \end{equation*}

    \item The \Def{large preimage},
    \begin{equation*}
      f^{-1}(B) \coloneqq \{ x \in X \colon f(x) \cap B \neq \varnothing \}.
    \end{equation*}
  \end{defenum}

  Obviously \( f_{-1}(B) \subseteq f^{-1}(B) \). The two types of preimages coincide for single-valued mappings.
\end{definition}

\begin{definition}\label{def:function_invertibility}(Compare with \cref{def:morphism_invertibility})
  We list equivalent conditions for three types of invertibility:
  \begin{defenum}
    \DItem{def:function_invertibility/injection} \( f \) is called \Def{injective}, \Def{left-invertible} or \Def{one-to-one} if either
    \begin{itemize}
      \item different points in \( X \) have different images under \( f \)
      \item the preimage of any point in \( Y \) is either empty or a singleton
      \item there exists a function \( g: Y \to X \) such that \( g \circ f = \Id_X \)
      \item the inverse is a single-valued partial function
    \end{itemize}

    We sometimes use the monomorphism\Tinyref{def:morphism_invertibility/monomorphism} notation \( f: X \hookrightarrow Y \).

    \DItem{def:function_invertibility/surjection} \( f \) is called \Def{surjective}, \Def{right-invertible} or \Def{onto} if either
    \begin{itemize}
      \item each point in \( Y \) is the image of at least one point in \( X \)
      \item the image of \( f \) equals the range of \( f \)
      \item there exists a function \( g: Y \to X \) such that \( f \circ g = \Id_Y \)
      \item the inverse is a total multivalued function
    \end{itemize}

    We sometimes use the epimorphism\Tinyref{def:morphism_invertibility/epimorphism} notation \( f: X \hookrightarrow Y \).

    \DItem{def:function_invertibility/bijection} \( f \) is called \Def{bijective} or simply \Def{invertible} if either
    \begin{itemize}
      \item it is both injective and surjective
      \item each point in \( Y \) is the image of exactly one point in \( X \)
      \item the preimage of any point in \( Y \) is a singleton
      \item there exists a function \( g: Y \to X \) such that both \( g \circ f = \Id_X \) and \( f \circ g = \Id_Y \)
      \item the inverse is a single-valued total function
    \end{itemize}

    We sometimes use the isomorphism\Tinyref{def:morphism_invertibility/isomorphism} notation \( f: X \cong Y \). See also \cref{def:equinumerous_sets}.
  \end{defenum}
\end{definition}

\begin{definition}\label{def:function_graph}
  Let \( f: X \to Y \) be a function. We define the \Def{graph of \( f \)} to be the set
  \begin{equation*}
    \Gph f \coloneqq \{ (x, y) \in X \times Y \colon f(x) = y \},
  \end{equation*}
  i.e. the underlying relation itself.

  Function graphs allows to study functions geometrically, i.e. as subsets of a space with a certain geometric structure. In low-dimensional spaces, function graphs can be plotted graphically to ease this study.

  In the case that \( Y \) is an ordered set, usually \( \R \), we also define the \Def{epigraph of f}
  \begin{equation*}
    \Epi f \coloneqq \{ (x, y) \in X \times Y \colon y \geq f(x) \},
  \end{equation*}

  and the \Def{hypograph of f}
  \begin{equation*}
    \Hypo f \coloneqq \{ (x, y) \in X \times Y \colon y \leq f(x) \},
  \end{equation*}
\end{definition}

\begin{definition}\label{def:indexed_family}
  When considering finite families of sets, it is enough to consider n-tuples. For example, given sets \( X_1, \ldots, X_n \), we can think of the family \( \{ X_k \}_k \) as the ordered tuple
  \begin{equation*}
    (X_1, \ldots, X_n)
  \end{equation*}
  where the \( i \)-th coordinate of the tuple gives us the \( i \)-th set of the family.

  This approach has two flaws:
  \begin{itemize}
    \item The family \Def{must} be ordered since the natural numbers are ordered. Families of sets often have no obvious ordering.
    \item The family \Def{must} be at most countable.
  \end{itemize}

  A more natural approach to indexed families is given by functions. We choose an arbitrary set \( I \), called the \Def{index set}. Every function \( f: I \to \Cal C \) from \( I \) into the class \( \cal C \) of all sets is then called an \Def{indexed family}. The function \( f \) maps every element \( i \) of \( I \) into a set \( X_i \coloneqq f(i) \). For convenience, this family is denoted as
  \begin{equation*}
    \{ X_i \}_{i \in I}.
  \end{equation*}

  A more general framework than indexed families that also considers relations between the family's elements is given by diagrams in category theory\Tinyref{def:categorical_diagram}.
\end{definition}

\begin{definition}\label{def:sequence}
  A \Def{sequence} \( \{ x_i \}_{i=1}^\infty \) is an indexed family with domain \( I = \BB N \). Sometimes finite \( n \)-tuples are referred to as \Def{finite sequences}, in which case the usual sequences are referred to as \Def{infinite sequences}. See \cref{def:topological_net}.

  We say that \( \{ x_{i_k} \}_{k=1}^\infty \) is a \Def{subsequence} of \( \{ x_i \}_{i=1}^\infty \) if \( \{ i_k \}_{k=1}^\infty \) is an increasing sequence of positive integers.

  Subsequences of \( \{ x_i \}_{i=1}^\infty \) are usually denoted by add another index, i.e. \( \{ x_{i_k} \}_{k=1}^\infty \).
\end{definition}

\begin{example}\label{ex:indexed_families}
  \mbox{}
  \begin{defenum}
    \item Every n-tuple \( (x_1, \ldots, x_n) \) is an indexed family with domain \( I = \{ 1, \ldots, n \} \).

    \item An important corner case is when \( I \) is the empty set. Since the only possible indexing function is then the empty function, we simply say that the resulting family is empty.

    \item In continuous stochastic processes, it is convenient to consider families of random variables \( \{ X_t \}_{t \geq 0} \) indexed by \( I = \R^+ \). The indexing parameter is often denoted by \( t \geq 0 \) is often interpreted as time.

    \item An \( n \times m \) matrix \( A = \{ a_{i,j} \} \) is a family of scalars indexed by the unordered set \( I = \{ 1, \ldots, n \} \times \{ 1, \ldots, m \} \).

    \item Nets\Tinyref{def:topological_net} is topology are indexed families where the domain is a directed set\Tinyref{def:order/directed}.
  \end{defenum}
\end{example}

\begin{definition}\label{def:family_of_functions_separates_points}
  Let \( \Cal{F} \) be a family of functions between the sets \( A \) and \( B \). We say that \( \Cal{F} \) \Def{separates points} if for every two points \( x, y \in A \) there exists a function \( f \in \Cal{F} \) such that \( f(x) \neq f(y) \).
\end{definition}

\begin{definition}\label{def:symmetric_function}
  Fix arbitrary sets \( X \) and \( Y \). A function \( f: X \times X \to Y \) is called \Def{symmetric} if, for all \( x, y \in X \), we have
  \begin{equation*}
    f(x, y) = f(y, x).
  \end{equation*}

  Note that symmetric functions are not symmetric relations\Tinyref{def:derived_relations/symmetric}.
\end{definition}
