\section{Analysis}\label{sec:analysis}
\subsection{Geometry of vector spaces}\label{subsec:geometry_of_vector_spaces}

\begin{definition}\label{def:real_linear_combinations}
  Let \( X \) be a real vector space\Tinyref{remark:real_vector_space} and let \( t_1, \ldots, t_n \in \R \) and \( x_1, \ldots, x_n \in X \). We say that their linear combination\Tinyref{def:linear_combination} \( x \coloneqq \sum_{k=1}^n t_k x_k \) is

  \begin{defenum}
    \DItem{def:real_linear_combinations/affine} an \Def{affine combination} if \( \sum_{k=1}^n t_k = 1 \).
    \DItem{def:real_linear_combinations/conic} a \Def{conic combination} if all of the coefficients are nonnegative.
    \DItem{def:real_linear_combinations/convex} a \Def{convex combination} if it is both affine and conic. A convex combination of two elements \( x, y \in X \) is usually written as \( tx + (1-t)y \) for some scalar \( t \in [0, 1] \).
  \end{defenum}
\end{definition}

\begin{definition}\label{def:linear_combination_hulls}
  Let \( A \) be a subset of the real vector space \( X \). We define the \Def{convex hull} \( \Conv A \) (resp. \Def{affine hull} or \Def{conic hull}) of \( A \) by the set of all convex (resp. affine or conic) combinations of finite subsets of \( A \).

  The convex hull \( \Conv A \) of two elements \( x, y \in X \) is often called the \Def{segment between \( x \) and \( y \)} and is denoted as
  \begin{equation*}
    [x, y] \coloneqq \{ tx + (1-t)y \colon t \in [0, 1] \}.
  \end{equation*}

  If the set \( A \) is equal to its convex (resp. affine or conic) hull, we say that it is a \Def{convex set} (resp. \Def{affine set} or \Def{cone})
\end{definition}

\begin{definition}\label{def:simplex}
  Let \( v_1, \ldots, v_n \) be linearly independent vectors and \( v_0 \) be any other vector. The convex hull \( S \) of the vectors \( v_0 + v_1, \ldots, v_0 + v_n \) is called an \Def{n-simplex}.

  The convex hull of any nonempty subset of \( \{ v_0, \ldots, v_n \} \) of cardinality \( m + 1 \) is an \( m \)-simplex and is called an \Def{\( m \)-face of \( S \)}.
\end{definition}
