\subsection{Compact spaces}\label{subsec:compact_spaces}

\begin{definition}\label{def:centered_family}\MarginCite[123]{Engelking1989}
  The nonempty family \( \CF \) of subsets of the topological space \( X \) is said to be a \Def{centered family of sets} or to have the \Def{finite intersection property} if the intersection \( F_1 \cap \cdots \cap F_n \) of any finite collection of sets is nonempty.
\end{definition}

\begin{definition}\label{def:compact_space}\MarginCite[123]{Engelking1989}
  The space \( X \) is called \Def{compact} if any of the following equivalent finiteness conditions hold:
  \begin{DefEnum}
    \ILabel{def:compact_space/finite_subcover} Every open cover of \( X \) has a finite subcover.
    \ILabel{def:compact_space/centered_family} Every centered \hyperref[def:centered_family]{family} \( \CF \) of closed subsets of \( X \) has a nonempty intersection.
    \ILabel{def:compact_space/convergent_nets} Every \hyperref[def:topological_net]{net} has a cluster point or, \hyperref[thm:net_convergence_properties/cluster_point_iff_subnet_limit_point]{equivalently}, a \hyperref[def:net_convergence]{convergent} subnet. This property is also called \enquote{generalized sequential compactness} or, when restricted to sequences instead of general nets, simply \enquote{sequential compactness}.
  \end{DefEnum}
\end{definition}
\begin{proof}
  \SubProofImplication{def:compact_space/finite_subcover}{def:compact_space/centered_family} Assume that every open cover of \( X \) has a finite subcover. Let \( \CF \) be a centered family of closed subsets of \( X \). Aiming at a contradiction, suppose\LEM that \( \bigcap \CF = \varnothing \). Then
  \begin{equation*}
    X
    =
    X \setminus \bigcap \CF
    =
    \bigcup_{F \in \CF} (X \setminus F),
  \end{equation*}
  which has a finite subcover indexed by, say, \( \CF' \subseteq \CF \). But \( \CF \) is a centered family and \( \bigcap \CF' \) is nonempty, hence
  \begin{equation*}
    X
    =
    \bigcup_{F \in \CF'} (X \setminus F)
    =
    X \setminus \bigcap \CF'
    \neq
    X.
  \end{equation*}

  The obtained contradiction shows that \( \bigcap \CF \) is nonempty.

  \SubProofImplication{def:compact_space/centered_family}{def:compact_space/finite_subcover} Assume that every centered family of closed sets has a nonempty intersection. Let \( \{ U_k \}_{k \in \CK} \) be an open cover of \( X \). By putting \( F_k \coloneqq U_k \) for all \( k \in \CK \), we obtain a family \( \{ F_k \}_{k \in \CK} \) of closed sets with an empty intersection. Therefore it is not a centered family. Then there exists at least one finite subfamily \( \{ F_k \}_{k \in \CK'} \) with an empty intersection. The complement of this subfamily is then a finite cover of \( X \), which proves our statement.

  \SubProofImplication{def:compact_space/finite_subcover}{def:compact_space/convergent_nets} Assume that every open cover of \( X \) has a finite subcover. Fix a net \( \{ x_k \}_{k \in \CK} \subseteq X \).

  Aiming at a contradiction, suppose that the net has no cluster points. For any point \( x \in X \) and any neighborhood \( U_x \) of \( x \), the net is not frequently in \( U_x \). Obviously \( \{ U_x \}_{x \in X} \) is an infinite open cover of \( X \). Then it has a finite subcover indexed by, say, \( X' \subseteq X \).

  Therefore every element of the net \( \{ x_k \}_{k \in \CK} \) if contained in one of the finitely many neighborhoods \( \{ U_x \}_{x \in X'} \) and the net itself is frequently in at least one of the neighborhoods.

  Thus one of the finitely many points in \( X' \) is a cluster point of \( \{ x_k \}_{k \in \CK} \).

  \SubProofImplication{def:compact_space/convergent_nets}{def:compact_space/centered_family}\MarginCite[thm. 3.1.23]{Engelking1989} Assume that every net has a cluster point. Let \( \{ F_k \}_{k \in \CK} \) be a central family of closed sets.

  Denote by \( \Cal C \) the family of all finite subsets of \( \CK \). For each \( C \in \Cal C \), the set \( \bigcap_{c \in C} F_c \) is a finite intersection of members of a central family and is hence nonempty. Choose\AOC an element \( x_C \in \bigcap_{c \in C} F_c \) for each \( C \in \Cal C \).

  If we order \( \Cal C \) by reverse inclusion, that is, if \( C \leq C' \iff \bigcap_{c \in C'} F_c \subseteq \bigcap_{c \in C} F_c \), then the family \( \{ x_C \}_{C \in \Cal C} \) becomes a net. Our assumption is that it has a cluster point, say \( x_0 \).

  It remains to show that \( x_0 \) belongs to the intersection of the centered family \( \{ F_k \}_{k \in \CK} \) itself. Fix an element \( F_0 \) of this family and denote by \( C_0 \) the singleton set \( \{ k_0 \} \in \Cal C \). Because \( x_0 \) is a cluster point, for every neighborhood \( U \) of \( x_0 \), there exists an index \( C \in \Cal C \) such that \( C \geq C_0 \) and \( x_C \in U \). Then \( x_C \in \bigcap_{c \in C} F_c \subseteq \bigcap_{c \in C_0} F_c = F_0 \).

  Therefore \( F_0 \cap U \neq \varnothing \) for all neighborhoods \( U \) of \( x_0 \). By \fullref{thm:closure_operator_properties/neighborhood_intersection}, \( x_0 \in F_0 \). Since \( F_0 \) was an arbitrary set from the centered family \( \{ F_k \}_{k \in \CK} \), we conclude that the intersection of the family is not empty. This proves the theorem.
\end{proof}

\begin{remark}\label{remark:precompact_set}
  If the closure of \( A \) is compact, we call \( A \) \Def{relatively compact} or \Def{precompact} (although the term \enquote{precompact} is also used for totally bounded sets, see \fullref{def:totally_bounded_set}).
\end{remark}

\begin{proposition}\label{thm:compact_space_properties}
  \hyperref[def:compact_space]{Compact spaces} have the following basic properties:
  \begin{PropEnum}
    \ILabel{thm:compact_space_properties/closed} If \( X \) is any topological space and \( A \subseteq X \) is a compact subspace, then \( A \) is closed in \( X \).
    \ILabel{thm:compact_space_properties/closed_subspace} A closed subspace of a compact space is compact.
  \end{PropEnum}
\end{proposition}
\begin{proof}
  \SubProofOf{thm:compact_space_properties/closed} If \( A \) is compact, by \fullref{def:compact_space/convergent_nets} every net in \( A \) has a cluster point in \( A \). By \fullref{thm:derived_set_properties/closed_iff_contains_all_cluster_points}, \( A \) is closed in \( X \).

  \SubProofOf{thm:compact_space_properties/closed_subspace} Let \( X \) be \hyperref[def:compact_space]{compact} and let \( X' \subseteq X \) be a closed subspace.

  Fix an open cover \( \{ U_k \}_{k \in \CK} \) of \( X' \). By definition of the subspace \hyperref[def:topological_subspace]{topology} for each \( U_k \) there exists a set \( V_k \) that is open in \( X \) and \( U_k = V_k \cap X' \). Since \( X' \) is closed, \( X \setminus X' \) is open and hence the family \( \{ V_k \}_{k \in \CK} \) together with \( X \setminus X' \) is an open cover of the space \( X \).

  By \fullref{def:compact_space/finite_subcover}, there exists a finite subcover \( \{ V_k \}_{k \in \CK'} \) that, along with \( X \setminus X' \), still covers \( X' \). Therefore \( X' \) is also compact.
\end{proof}

\begin{theorem}[Tychonoff's product theorem]\label{thm:tychonoffs_product_theorem}\MarginCite[thm. 3.2.4]{Engelking1989}
  Let \( { (X_k, \CT_k) }_{k \in \CK} \) be a family of topological spaces. Their \hyperref[def:topological_product]{product} \( (\prod_{k \in \CK} X_k, \prod_{k \in \CK} \CT_k) \) is \hyperref[def:compact_space]{compact} if and only if \( (X_k, \CT_k) \) is compact for every \( k \in \CK \).
\end{theorem}

\begin{theorem}[Weierstrass' extreme value theorem]\label{thm:weierstrass_extreme_value_theorem}\MarginCite[cor. 3.2.9]{Engelking1989}
  Let \( X \) be a compact topological space and let \( f: X \to \BR \) be a continuous function into the \hyperref[def:real_numbers]{real numbers}.

  Then \( f \) is \hyperref[def:metric_space/bounded_function]{bounded} and there exist \( m, M \in X \) such that
  \begin{align*}
    f(m) = \min_{x \in X} f(x)
     &  &
    f(M) = \max_{x \in X} f(x).
  \end{align*}
\end{theorem}

\begin{definition}\label{def:locally_compact_space}\MarginCite[148]{Engelking1989}
  A topological space is called \Def{locally compact} if every point has a relatively compact neighborhood.
\end{definition}
