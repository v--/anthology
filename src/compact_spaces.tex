\subsection{Compact spaces}\label{subsec:compact_spaces}

Let \( (X, \T) \) be a topological space.

\begin{definition}\label{def:centered_family}\cite[123]{Engelking1989}
  The nonempty family \( \F \) of subsets of \( X \) is said to be a \Def{centered family of sets} or to have the \Def{finite intersection property} if the intersection \( F_1 \cap \cdots \cap F_n \) of any finite collection of sets is nonempty.
\end{definition}

\begin{definition}\label{def:compact_space}\cite[123]{Engelking1989}
  The space \( X \) is called \Def{compact} if any of the following equivalent finiteness conditions hold:
  \begin{defenum}
    \DItem{def:compact_set/finite_subcover} Every open cover of \( X \) has a finite subcover.
    \DItem{def:compact_set/finite_intersection} Every centered family\Tinyref{def:centered_family} \( \F \) of closed subsets of \( X \) has a nonempty intersection.
  \end{defenum}
\end{definition}

\begin{remark}\label{remark:precompact_set}
  If the closure if \( A \) is compact, we call \( A \) \Def{relatively compact} or \Def{precompact} (although the term \enquote{precompact} is also used for totally bounded sets, see \ref{def:totally_bounded_set}).
\end{remark}

\begin{theorem}[Tychonoff's product theorem]\label{thm:tychonoffs_product_theorem}\cite[theorem 3.2.4]{Engelking1989}
  Let \( { (X_i, \T_i) }_{i \in I} \) be a family of topological spaces. Their product\Tinyref{def:topological_product} \( (\prod_{i \in I} X_i, \prod_{i \in I} \T_i) \) is compact\Tinyref{def:compact_set} if and only if \( (X_i, \T_i) \) is compact for every \( i \in I \).

\end{theorem}

\begin{theorem}[Weierstrass' extreme value theorem]\label{thm:weierstrass_extreme_value_theorem}
  Let \( (X, \T) \) be a compact topological space and let \( f: X \to \R \) be a continuous function into the real numbers\Tinyref{def:real_numbers}.

  Then \( f \) is bounded\Tinyref{def:metric_space/bounded_function} and there exist \( m, M \in X \) such that
  \begin{align*}
    f(m) = \min_{x \in X} f(x)
    &&
    f(M) = \max_{x \in X} f(x).
  \end{align*}
\end{theorem}

\begin{definition}\label{def:locally_compact_space}\cite[148]{Engelking1989}
  A topological space is called \Def{locally compact} if every point has a relatively compact neighborhood.
\end{definition}
