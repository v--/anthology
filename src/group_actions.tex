\subsection{Group actions}\label{subsec:group_actions}

\begin{definition}\label{def:left_group_action}\MarginCite[159]{Knapp2016BAlg}
  Let \( G \) be a group and let \( A \) be an arbitrary set or algebraic structure. A \Def{left group action} of \( G \) on \( A \) can be defined equivalently as
  \begin{DefEnum}
    \ILabel{def:left_group_action/homomorphism} a homomorphism from \( G \) to the automorphism group \( \Aut(A) \).
    \ILabel{def:left_group_action/indirect_homomorphism} an assignment of a bijection \( \tau_x: A \to A \) for each \( x \in G \) such that
    \begin{equation*}
      \tau_{xy} = \tau_x \circ \tau_y.
    \end{equation*}
    \ILabel{def:left_group_action/multiplication} an \hyperref[def:algebraic_theory]{operation} (see \fullref{remark:left_group_actions_as_algebraic_structures} for justification) \( \cdot: G \times A \to A \), written using juxtaposition, such that
    \begin{itemize}
      \item \( (g h) \cdot a = g \cdot (h \cdot a) \) whenever \( g, h \in G \) and \( a \in A \)
      \item \( e \cdot a = a \) for any \( a \in A \)
    \end{itemize}
  \end{DefEnum}
\end{definition}
\begin{proof}
  \SubProofImplication{def:left_group_action/homomorphism}{def:left_group_action/indirect_homomorphism} Let \( \tau: G \to \Aut(A) \) be a homomorphism. Thus we assign a bijection \( \tau(x) \) for each member \( x \in G \). Furthermore, since the group operation on \( \Aut(A) \) is function composition and since \( \tau \) is a homomorphism, we have
  \begin{equation*}
    \tau(xy) = \tau(x) \circ \tau(y).
  \end{equation*}

  \SubProofImplication{def:left_group_action/indirect_homomorphism}{def:left_group_action/multiplication} Assume that we have a bijection \( \tau_x: A \to A \) for each \( x \in G \) that satisfies \( \tau_{xy} = \tau_x \circ \tau_y \). Define the operation
  \begin{align*}
     & \cdot: G \times A \to A          \\
     & \cdot(g, a) \coloneqq \tau_x(a).
  \end{align*}

  It satisfies the necessary axioms:
  \begin{itemize}
    \item If \( g, h \in G \) and \( a \in A \), we have
          \begin{equation*}
            (g h) \cdot a
            =
            \tau_{g h}(a)
            =
            \tau_{g}(\tau{h}(a))
            =
            g \cdot (h \cdot a).
          \end{equation*}

    \item If \( a \in A \), we have
          \begin{equation*}
            e \cdot a
            =
            \tau_e(a)
            =
            \Id(a)
            =
            a.
          \end{equation*}
  \end{itemize}

  \SubProofImplication{def:left_group_action/multiplication}{def:left_group_action/homomorphism} Assume that we have an operation \( \cdot: G \times A \to A \) that satisfies the axioms for left action. Define the function
  \begin{align*}
     & \tau: G \to \Aut(A)                      \\
     & \tau(g) \coloneqq (a \mapsto g \cdot a).
  \end{align*}

  The function is well defined because for every \( g \in G \), the function \( \tau(g) \) is a indeed a bijection:
  \begin{equation*}
    \tau(g^{-1}) \circ \tau(g)
    =
    \tau(g^{-1} g)
    =
    \tau(e)
    =
    \Id
  \end{equation*}
  and similarly in the other direction.

  \( \tau \) is a group homomorphism because
  \begin{equation*}
    \tau(g h)
    =
    (a \mapsto gh \cdot a)
    =
    (a \mapsto g \cdot (h \cdot a))
    =
    (a \mapsto \tau(g)(\tau(h)(a))
    =
    \tau(g) \circ \tau(h).
  \end{equation*}
\end{proof}

\begin{example}\label{ex:group_actions}
  \begin{itemize}\mbox{}
    \item Any group \( G \) is a group action on itself. Every element of \( G \) simply corresponds to itself under this action. We will verify all three definitions:
          \SubProofOf{def:left_group_action/homomorphism} The identity function \( \Id: G \to G \) is the identity of the symmetric group \( S(G) \) (where \( G \) is considered as a set here). Thus the following function
          \begin{align*}
             & \varphi: G \to S(G)                 \\
             & \varphi(x) \coloneqq (y \mapsto xy)
          \end{align*}
          is a group homomorphism from \( G \) to \( S(G) \). Indeed, \( \varphi(e) = \Id \) and
          \begin{equation*}
            (z \mapsto xyz) = \varphi(xy) = \varphi(x) \circ \varphi(y) = (y \mapsto xy) \circ (z \mapsto yz) = (z \mapsto xyz)
          \end{equation*}

          \SubProofOf{def:left_group_action/indirect_homomorphism} The proof is the same as above.

          \SubProofOf{def:left_group_action/multiplication} Define the operation
          \begin{align*}
             & \circ: G \times G \to G \\
             & x \circ y \coloneqq xy
          \end{align*}

          Thus
          \begin{itemize}
            \item \( (x y) \circ z = xyz = x \circ (y \circ z) \) whenever \( x, y, z \in G \)
            \item \( e \circ x = ex = x \) for any \( x \in G \)
          \end{itemize}

    \item The group of integers \( \BZ \) acts on any group \( G \) by sending \( g \) to its power \( g^n \).
  \end{itemize}
\end{example}

\begin{remark}\label{remark:left_group_actions_as_algebraic_structures}
  We may also define left group actions as algebraic structures. Let \( \varphi: G \to \Aut(A) \) be a left group action of \( G \) on \( A \).

  We define the group action as the set \( A \) with a unary operation \( \varphi(g): A \to A \) for every element of the group \( g \in G \). We must also add the additional axioms
  \begin{equation*}
    \varphi(gh) = \varphi(g) \circ \varphi(h)
  \end{equation*}
  for every pair \( g, h \in A \).
\end{remark}

\begin{definition}\label{def:right_group_action}
  We say that \( \tau: G \to \Aut(A) \) is a \Def{right group action} of \( G \) on \( A \) if the same function is a left group \hyperref[def:left_group_action]{action} of the opposite \hyperref[def:opposite_group]{group} \( G^{-1} \) on \( A \).
\end{definition}
