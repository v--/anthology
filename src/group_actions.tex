\subsection{Group actions}\label{subsec:group_actions}

\begin{definition}\label{def:groupoid}
  A \Def{groupoid} is a \hyperref[def:category]{category} in which all morphisms are \hyperref[def:morphism_invertibility]{isomorphisms}.
\end{definition}

\begin{definition}\label{def:automorphism_group}
  Given a \hyperref[def:category_cardinality]{locally small} \hyperref[def:groupoid]{groupoid} \( \Cat{G} \), we call \( \Cat{G}(A) \) the \Def{automorphism group} over \( A \) and denote it by \( \Aut(A) \). If \( A \) is the only object in \( \Cat{G} \), we can identify the entire category \( \Cat{C} \) with the group \( \Aut(A) \).
\end{definition}

\begin{definition}\label{def:symmetric_group}
  We define the \Def{symmetric group} of order \( n \) as the group
  \begin{equation*}
    S_n \coloneqq \Aut(\{ 1, 2, \ldots, n \})
  \end{equation*}
  of all bijections from the set \( \{ 1, 2, \ldots, n \} \) to itself.

  \begin{DefEnum}
    \ILabel{def:symmetric_group/permutation} Members of \( S_n \) are called \Def{permutations}.

    \ILabel{def:symmetric_group/inversion} We say that the pair \( (p(k), p(m)) \) is an \Def{inversion} of the permutation \( p \) if \( k < m \) and \( p(k) > p(m) \).

    \ILabel{def:symmetric_group/permutation_parity} A permutation is said to be \Def{even} or \Def{odd} depending on whether it has an even or odd number of inversions.

    \ILabel{def:symmetric_group/permutation_sign} We define the \Def{sign} of a permutation as
    \begin{align*}
       &\Sign: S_n \to \{ -1, 1 \}, \\
       &\Sign(p) \coloneqq \begin{cases}
        1,  &p \text{ is even} \\
        -1, &p \text{ is odd}
      \end{cases}
    \end{align*}

    \ILabel{def:symmetric_group/alternating} The subgroup of all even permutations is denoted by \( A_n \) and is called the \Def{alternating group} of order \( n \).
  \end{DefEnum}
\end{definition}

\begin{definition}\label{def:left_group_action}
  In analogy to \hyperref[def:left_monoid_action]{monoid actions}, we can define a \Def{group actions} of the group \( \CG \) on a set \( A \) as
  \begin{DefEnum}
    \ILabel{def:left_group_action/homomorphism} A homomorphism from \( \CG \) to the automorphism group \( \Aut(A) \) (the set of all bijections on \( A \)).
    \ILabel{def:left_group_action/indirect_homomorphism} An indexed family of isomorphisms \( \{ \tau_x: A \to A \}_{x \in \CG} \) such that
    \begin{equation}\label{eq:def:left_group_action/indirect_homomorphism}
      \tau_{xy} = \tau_x \circ \tau_y \Tforall x, y \in \CG.
    \end{equation}

    \ILabel{def:left_group_action/operation} The same heterogeneous algebraic operation \( \circ: \CG \times A \to A \) as in \fullref{def:left_monoid_action/operation}.
  \end{DefEnum}
\end{definition}
\begin{proof}
  The proof of equivalence is the same as in \fullref{def:left_monoid_action}.
\end{proof}

\begin{definition}\label{def:right_group_action}
  As for \hyperref[def:right_monoid_action]{monoid actions}, we define \Def{right group actions} as \hyperref[def:left_group_action]{left group action} of the \hyperref[def:magma/opposite]{opposite group} \( \CG^{-1} \) on \( A \).
\end{definition}

\begin{lemma}\label{thm:group_multiplication_is_bijection}
  For each element \( x \) of a group \( \CG \), the function
  \begin{align*}
    &\varphi_x: \CG \to \CG \\
    &\varphi_x(y) \coloneqq xy
  \end{align*}
  is an bijection (but not necessarily an isomorphism).
\end{lemma}
\begin{proof}
  \SubProofOf{def:function_invertibility/injection} If \( y_1, y_2 \in \CG \) and \( \varphi_x(y_1) = \varphi_x(y_2) \), we have
  \begin{equation*}
    xy_1 = \varphi_x(y_1) = \varphi_x(y_2) = xy_2.
  \end{equation*}

  By \fullref{thm:group_properties/cancellative}, \( y_1 = y_2 \). Therefore \( \varphi_x \) is injective.

  \SubProofOf{def:function_invertibility/surjection} If \( z \in \CG \), then \( z = x(x^{-1} z) \). Therefore \( z = \varphi_x(x^{-1} z) \) and \( \varphi_x \) is surjective.
\end{proof}

\begin{theorem}[Cayley's theorem]\label{thm:cayleys_theorem}
  Any \hyperref[def:group]{group} \hyperref[def:left_group_action]{acts} on itself by endofunctions.

  Compare this result to \fullref{thm:monoid_is_action}.
\end{theorem}
\begin{proof}
  Follows directly from \fullref{thm:monoid_is_action} and \fullref{thm:group_multiplication_is_bijection}.
\end{proof}
