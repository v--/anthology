\subsection{Group actions}\label{subsec:group_actions}

\begin{definition}\label{def:group_action}\cite[159]{Knapp2016BAlg}
  Let \( G \) be a group and let \( A \) be an arbitrary set. A \Def{group action} of \( G \) on \( A \) can be defined equivalently as
  \begin{defenum}
    \DItem{def:group_action/homomorphism} a homomorphism from \( G \) to the symmetric group \( S(A) \).
    \DItem{def:group_action/indirect_homomorphism} an assignment of a bijection \( \tau_x: A \to A \) for each \( x \in G \) such that
    \begin{equation*}
      \tau_{xy} = \tau_x \circ \tau_y.
    \end{equation*}
    \DItem{def:group_action/left_multiplication} an operation\Tinyref{def:algebraic_theory} (see \cref{remark:group_actions_as_algebraic_structures} for justification) \( \cdot: G \times A \to A \), written using juxtaposition, such that
    \begin{itemize}
      \item \( (g h) \cdot a = g \cdot (h \cdot a) \) whenever \( g, h \in G \) and \( a \in A \)
      \item \( e \cdot a = a \) for any \( a \in A \)
    \end{itemize}
  \end{defenum}

  When necessary, we also refer to \ref{def:group_action/left_multiplication} as \Def{left group actions} to distinguish them from \Def{right group actions} (which are defined in the obvious way).
\end{definition}
\begin{proof}
  \begin{description}
    \Implies[def:group_action/homomorphism][def:group_action/indirect_homomorphism] Let \( \tau: G \to S(A) \) be a homomorphism. Thus we assign a bijection \( \tau(x) \) for each member \( x \in G \). Furthermore, since the group operation on \( S(A) \) is function composition and since \( \tau \) is a homomorphism, we have
    \begin{equation*}
      \tau(xy) = \tau(x) \circ \tau(y).
    \end{equation*}

    \Implies[def:group_action/indirect_homomorphism][def:group_action/left_multiplication] Assume that we have a bijection \( \tau_x: A \to A \) for each \( x \in G \) that satisfies \( \tau_{xy} = \tau_x \circ \tau_y \). Define the operation
    \begin{align*}
      &\cdot: G \times A \to A \\
      &\cdot(g, a) \coloneqq \tau_x(a).
    \end{align*}

    It satisfies the necessary axioms:
    \begin{itemize}
      \item If \( g, h \in G \) and \( a \in A \), we have
      \begin{equation*}
        (g h) \cdot a
        =
        \tau_{g h}(a)
        =
        \tau_{g}(\tau{h}(a))
        =
        g \cdot (h \cdot a).
      \end{equation*}

      \item If \( a \in A \), we have
      \begin{equation*}
        e \cdot a
        =
        \tau_e(a)
        =
        \Id(a)
        =
        a.
      \end{equation*}
    \end{itemize}

    \Implies[def:group_action/left_multiplication][def:group_action/homomorphism] Assume that we have an operation \( \cdot: G \times A \to A \) that satisfies the axioms for left action. Define the function
    \begin{align*}
      &\tau: G \to S(A) \\
      &\tau(g) \coloneqq (a \mapsto g \cdot a).
    \end{align*}

    The function is well defined because for every \( g \in G \), the function \( \tau(g) \) is a indeed a bijection:
    \begin{equation*}
      \tau(g^{-1}) \circ \tau(g)
      =
      \tau(g^{-1} g)
      =
      \tau(e)
      =
      \Id
    \end{equation*}
    and similarly in the other direction.

    \( \tau \) is a group homomorphism because
    \begin{equation*}
      \tau(g h)
      =
      (a \mapsto gh \cdot a)
      =
      (a \mapsto g \cdot (h \cdot a))
      =
      (a \mapsto \tau(g)(\tau(h)(a))
      =
      \tau(g) \circ \tau(h).
    \end{equation*}
  \end{description}
\end{proof}

\begin{example}\label{ex:group_actions}
  \begin{itemize}\mbox{}
    \item Any group \( G \) is a group action on itself. Every element of \( G \) simply corresponds to itself under this action. We will verify all three definitions:
    \begin{description}
      \RItem{def:group_action/homomorphism} The identity function \( \Id: G \to G \) is the identity of the symmetric group \( S(G) \) (where \( G \) is considered as a set here). Thus the following function
      \begin{align*}
        &\varphi: G \to S(G) \\
        &\varphi(x) \coloneqq (y \mapsto xy)
      \end{align*}
      is a group homomorphism from \( G \) to \( S(G) \). Indeed, \( \varphi(e) = \Id \) and
      \begin{equation*}
        (z \mapsto xyz) = \varphi(xy) = \varphi(x) \circ \varphi(y) = (y \mapsto xy) \circ (z \mapsto yz) = (z \mapsto xyz)
      \end{equation*}

      \RItem{def:group_action/indirect_homomorphism} The proof is the same as above.

      \RItem{def:group_action/left_multiplication} Define the operation
      \begin{align*}
        &\circ: G \times G \to G \\
        &x \circ y \coloneqq xy
      \end{align*}

      Thus
      \begin{itemize}
        \item \( (x y) \circ z = xyz = x \circ (y \circ z) \) whenever \( x, y, z \in G \)
        \item \( e \circ x = ex = x \) for any \( x \in G \)
      \end{itemize}
    \end{description}

    \item The group of integers \( \Z \) acts on any group \( G \) by sending \( g \) to its power \( g^n \).
  \end{itemize}
\end{example}

\begin{remark}\label{remark:group_actions_as_algebraic_structures}
  We may also define group actions as algebraic structures. Let \( \varphi: G \to S(A) \) be a group action of \( G \) on \( A \).

  We define the group action as the set \( A \) with a unary operation \( \varphi(g): A \to A \) for every element of the group \( g \in G \). We must also add the additional axioms
  \begin{equation*}
    \varphi(gh) = \varphi(g) \circ \varphi(h)
  \end{equation*}
  for every pair \( g, h \in A \).
\end{remark}
