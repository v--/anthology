\subsection{Boolean functions}\label{subsec:boolean_functions}

\begin{definition}\label{def:boolean_function}
  Fix a two-element set \( B \). A function \( f: X \to B \) is called \Def{Boolean-valued} and a function \( f: B^n \to B \) is called \Def{Boolean}.

  The concrete set \( B \) does not matter much. If \( B \) is a Boolean algebra\Tinyref{def:boolean_algebra}, we can regard it as any of the equivalent Boolean algebras of two elements (see \cref{thm:binary_boolean_algebras_are_isomorphic}).
\end{definition}

\begin{definition}\label{def:boolean_closure}
  Fix a set \( \Cal{B} \) of Boolean functions of arbitrary arities.

  The \Def{closure} \( \Cl{\Cal{B}} \) of \( \Cal{B} \) is defined inductively as follows:
  \begin{itemize}
    \item If \( f \in \Cal{B} \), then \( f \in \Cl{\Cal{B}} \)
    \item If \( f_i(x_1, \ldots, x_n) \in \Cl{\Cal{B}}, i = 1, \ldots, m \) and \( g(x_1, \ldots, x_m) \in \Cl{\Cal{B}} \), then their superposition
    \begin{equation*}
      h(x_1, \ldots, x_n) \coloneqq g(f_1(x_1, \ldots, x_n), \ldots, f_m(x_1, \ldots, x_n))
    \end{equation*}
    is also in \( \Cl{\Cal{B}} \).
  \end{itemize}

  We say that \( \Cal{B} \) is \Def{closed} if \( \Cl{\Cal{B}} = \Cal{B} \).

  We say that \( \Cal{B} \) is \Def{complete} if \( \Cl{\Cal{B}} \) is the set of all Boolean functions of arbitrary arity.
\end{definition}

\begin{definition}\label{def:boolean_functions_in_f2}
  Fix a Boolean function\Tinyref{def:boolean_function} \( f(x_1, \ldots, x_n) \) in the Galois field\Tinyref{thm:galois_field_existence} \( \BF_2 \),

  We will denote multiplication in \( \BF_2 \) by juxtaposition and addition in \( \BF_2 \) via \( \oplus \).

  \begin{defenum}
    \DItem{def:boolean_function_in_f2/dual} Its \Def{dual function} is
    \begin{equation*}
       \Ol{f}(x_1, \ldots, x_n) \coloneqq \Ol{f(\Ol{x_1}, \ldots, \Ol{x_n})}.
    \end{equation*}

    \DItem{def:boolean_functions_in_f2/self_dual} We call \( f \) \Def{self-dual} if it is its own dual\Tinyref{def:boolean_function_in_f2/dual}.

    \DItem{def:boolean_functions_in_f2/truth_preserving} We call \( f \) if \( f(1, \ldots, 1) = 1 \).

    \DItem{def:boolean_functions_in_f2/falsity_preserving} We call \( f \) if \( f(0, \ldots, 0) = 0 \).

    \DItem{def:boolean_functions_in_f2/monotone} We call \( f \) \Def{monotone} if, for any two tuples of arguments \( x_1, \ldots, x_n \in \BF_2^n \) and \( y_1, \ldots, y_n \in \BF_2^n \), the inequalities \( x_i \leq y_i \) for all \( i = 1, \ldots, n \) imply that
    \begin{equation*}
      f(x_1, \ldots, x_n) \leq f(y_1, \ldots, y_n).
    \end{equation*}
  \end{defenum}
\end{definition}

\begin{definition}\label{def:zhegalkin_polynomial}
  A \Def{Zhegalkin polynomial} is a polynomial\Tinyref{def:polynomial} in the Galois field\Tinyref{thm:galois_field_existence} \( \BF_2 \). These polynomials obviously induce Boolean functions\Tinyref{def:boolean_function}.
\end{definition}

\begin{example}\label{ex:zhegalkin_polynomials}
  We list some examples of Zhegalkin polynomials.

  \begin{itemize}
    \item \( 0 \) and \( 1 \) are the constant polynomials
    \item \( X \) and \( X^2 \) induce the same function (the identity)
    \item We can define all \ref{def:propositional_logic_language/connectives}as polynomials in two indeterminates, for example \( X \lor Y \coloneqq X \oplus Y \oplus XY \) and \( X \land Y \coloneqq XY \) from the isomorphism defined in the proof of \cref{thm:binary_boolean_algebras_are_isomorphic}. Note that both are not linear.
  \end{itemize}
\end{example}

\begin{theorem}[Post's completeness theorem]\label{thm:posts_completeness_theorem}\cite{Pelletier1990}
  Denote by \( \Cal{B} \) the any set of Boolean functions\Tinyref{def:boolean_function} on \( \BF_2 \).

  Denote by \( \Cal{T} \) the set of the following Boolean functions on \( \BF_2 \):
  \begin{thmenum}
    \DItem{thm:posts_completeness_theorem/truth_preserving} All truth-preserving\Tinyref{def:boolean_functions_in_f2/truth_preserving} functions,
    \DItem{thm:posts_completeness_theorem/falsity_preserving} All falsity-preserving\Tinyref{def:boolean_functions_in_f2/falsity_preserving} functions,
    \DItem{thm:posts_completeness_theorem/self_dual} All self-dual\Tinyref{def:boolean_functions_in_f2/self_dual} functions,
    \DItem{thm:posts_completeness_theorem/monotone} All monotone\Tinyref{def:boolean_functions_in_f2/self_dual} functions,
    \DItem{thm:posts_completeness_theorem/linear} All functions that correspond to a linear Zhegalkin polynomial\Tinyref{def:zhegalkin_polynomial}.
  \end{thmenum}

  The \( \Cal{B} \) is a complete\Tinyref{def:boolean_closure} set of Boolean functions if and only if \( \Cal{B} \not\subseteq \Cal{T} \).
\end{theorem}

\begin{example}\label{ex:posts_completeness_theorem}
  We give examples of complete and incomplete sets of Boolean functions in \( \BF_2 \).

  \begin{exenum}
    \DItem{ex:posts_completeness_theorem/and_or} The archetypic example is the triple of functions
    \begin{itemize}
     \item \( x \land y = \sup \{ x, y \} = xy \)
     \item \( x \lor y = \inf \{ x, y \} = x \oplus y \oplus xy \)
     \item \( \neg x = x \oplus 1 \)
    \end{itemize}
    that form the Boolean algebra structure on \( \BF_2 \).

    Due to analogy with connectives in propositional logic\Tinyref{def:propositional_logic_language}, we call \( \land \) \enquote{and}, \( \lor \) \enquote{or} and \( \neg \) \enquote{not}. The addition in \( \BF_2 \) corresponds to \enquote{exclusive or}, sometimes shortened as \enquote{XOR}.

    We verify that the conditions of \fullref{thm:posts_completeness_theorem} are satisfied:
    \begin{description}
      \RItem{thm:posts_completeness_theorem/truth_preserving} \( \neg \) is not truth-preserving.
      \RItem{thm:posts_completeness_theorem/falsity_preserving} \( \neg \) is not falsity-preserving.
      \RItem{thm:posts_completeness_theorem/self_dual} Both \( \land \) and \( \lor \) are not self-dual. In fact, by de Morgan's laws, \( \land \) is the dual of \( \lor \) and vice versa.
      \RItem{thm:posts_completeness_theorem/monotone} \( \neg \) is not monotone.
      \RItem{thm:posts_completeness_theorem/linear} Neither \( \land \) nor \( \lor \) have linear Zhegalkin polynomials.
    \end{description}

    Thus \( \{ \land, \lor, \neg \} \) is a complete set of Boolean functions.

    Note that having both \( \lor \) and \( \land \) is redundant and we usually include both for symmetry. The sets \( \{ \land, \neg \} \) and \( \{ \lor, \neg \} \) are both complete.

    \DItem{ex:posts_completeness_theorem/nand} We can go even further and have a single binary Boolean function generate all others. We will use the function \( x \uparrow y \coloneqq \neg(x \land y) \), which in analogy to propositional logic is called \enquote{Sheffer's stroke} or \enquote{NAND (not and)}.

    By fixing \( y = 1 \), we obtain \( x \uparrow 1 = \neg x \). Due to \cref{ex:posts_completeness_theorem/and_or}, we only need to verify that
    \begin{equation*}
      x \uparrow y = xy \oplus 1
    \end{equation*}
    corresponds to no linear Zhegalkin polynomial, which is evident.

    Hence the singleton set \( \{ \uparrow \} \) is a complete set of Boolean functions.

    \DItem{ex:posts_completeness_theorem/implies} The family \( \{ \Rightarrow, 0 \} \), where \( x \Rightarrow y \coloneqq 1 \oplus x \oplus xy \), is sometimes used.

    We verify that the conditions of \fullref{thm:posts_completeness_theorem} are satisfied:
    \begin{description}
      \RItem{thm:posts_completeness_theorem/truth_preserving}  \( 0 \) is not truth-preserving because \( 0 \) is a zero-arity function and \( 0 \neq 1 \).
      \RItem{thm:posts_completeness_theorem/falsity_preserving} \( \Rightarrow \) is not falsity-preserving because \( 0 \Rightarrow 0 = 1 \).
      \RItem{thm:posts_completeness_theorem/self_dual} \( \Rightarrow \) is not self-dual because \( \Ol{\Ol{x} \Rightarrow \Ol{y}} = \Ol{y \Rightarrow x} \).
      \RItem{thm:posts_completeness_theorem/monotone} \( \Rightarrow \) is not monotone because \( 1 \Rightarrow 0 = 0 \).
      \RItem{thm:posts_completeness_theorem/linear} \( \Rightarrow \) has no linear Zhegalkin polynomial.
    \end{description}
  \end{exenum}
\end{example}
