\subsection{Bernstein inequalities}\label{subsec:bernstein_inequalities}

\begin{definition}\label{def:modulus_of_continuity}\MarginCite[27]{Николов2020}
  Let \( f: X \to Y \) be a function between metric spaces. We define the \Def{modulus of continuity} as
  \begin{align*}
     & \omega: \Cat{Set}(X, Y) \to [0, \infty]                                                         \\
     & \omega(f, \delta) \coloneqq \sup \Big\{ \rho_Y(f(x), f(y)) \colon \rho_X(x, y) < \delta \Big\}.
  \end{align*}
\end{definition}

\begin{proposition}\label{thm:modulus_of_continuity_properties}
  The \hyperref[def:modulus_of_continuity]{modulus of continuity} has the following basic properties:
  \begin{PropEnum}
    \ILabel{thm:modulus_of_continuity_properties/continuity_condition} \( f \) is globally \hyperref[def:uniform_continuity]{uniformly continuous} if and only if for every \( \varepsilon > 0 \) there exists \( \delta > 0 \) such that \( \omega(f, \delta) \).

    \ILabel{thm:modulus_of_continuity_properties/monotone} \( \omega(f, \delta) \) is monotone in \( \delta \).

    \ILabel{thm:modulus_of_continuity_properties/cauchy_inequality}\MarginCite[28]{Николов2020}For all \( \lambda, \delta > 0 \), we have the following analog of \fullref{thm:cauchy_bunyakovsky_schwarz_inequality}
    \begin{equation}\label{thm:modulus_of_continuity_properties/cauchy_inequality/inequality}
      \omega(f, \lambda \delta) \leq \omega(f, \lambda^2) + \omega(f, \delta^2).
    \end{equation}

    \ILabel{thm:modulus_of_continuity_properties/single_inequality}\MarginCite[28]{Николов2020}For all \( \lambda, \delta > 0 \),
    \begin{equation}\label{thm:modulus_of_continuity_properties/single_inequality/inequality}
      \omega(f, \lambda \delta) \leq (\lambda + 1) \omega(f, \delta).
    \end{equation}
  \end{PropEnum}
\end{proposition}
\begin{proof}
  \SubProofOf{thm:modulus_of_continuity_properties/continuity_condition} Follows directly from \fullref{def:uniform_continuity}.

  \SubProofOf{thm:modulus_of_continuity_properties/monotone} A supremum on a larger set is larger.

  \SubProofOf{thm:modulus_of_continuity_properties/cauchy_inequality} If \( \lambda \leq \delta \), clearly \( \lambda \delta \leq \delta^2 \). Otherwise, \( \lambda \delta < \lambda^2 \).

  Combining the two inequalities with \fullref{thm:modulus_of_continuity_properties/monotone}, we obtain \fullref{thm:modulus_of_continuity_properties/cauchy_inequality/inequality}.

  \SubProofOf{thm:modulus_of_continuity_properties/single_inequality} Note that
  \begin{equation*}
    \rho_X(x, y) < \delta \Timplies \rho_Y(f(x), f(y)) < \omega(f, \delta).
  \end{equation*}

  We can multiply this by \( \lambda \) to obtain
  \begin{equation*}
    \lambda \rho_X(x, y) < \lambda \delta \Timplies \lambda \rho_Y(f(x), f(y)) < \lambda \omega(f, \delta).
  \end{equation*}

  If \( \lambda \geq 1 \), then \( \rho_X(x, y) \leq \lambda \rho_X(x, y) \) and \( \rho_Y(f(x), f(y)) \leq \lambda \rho_Y(f(x), f(y)) \) and hence
  \begin{equation*}
    \omega(f, \lambda \delta) \leq \lambda \omega(f, \delta).
  \end{equation*}

  Otherwise, \( \lambda < 1 \) and clearly \( \lambda \delta < \delta \), which by \fullref{thm:modulus_of_continuity_properties/monotone} implies
  \begin{equation*}
    \omega(f, \lambda \delta) \leq \omega(f, \delta).
  \end{equation*}

  Combining the two cases, we obtain
  \begin{equation*}
    \omega(f, \lambda \delta) \leq \lambda \omega(f, \delta) + \omega(f, \delta),
  \end{equation*}
  which we wanted to prove.
\end{proof}

\begin{definition}\label{def:real_function_space_operators}
  Consider the \hyperref[def:function/single_valued]{operator} \( T: C([a, b]) \to C([a, b]) \).

  \begin{DefEnum}
    \ILabel{def:real_function_space_operators/positive} If \( f([a, b]) \subseteq [0, \infty) \) implies \( T(f)([a, b]) \subseteq [0, \infty) \), we say that \( T \) is \Def{positive}.

    \ILabel{def:real_function_space_operators/monotone} If \( f(x) \leq g(x) \) for all \( x \in [a, b] \) implies \( T(f)(x) \leq T(g)(x) \) for all \( x \in [a, b] \), we say that \( T \) is \Def{monotone}.
  \end{DefEnum}
\end{definition}

\begin{definition}\label{def:periodic_function_space}\MarginCite[44]{Николов2020}
  We denote by \( \Ti{C}([a, b]) \) the subspace of \( C([a, b]) \) consisting of all continuous functions in \( [a, b] \) which are periodic with minimal period \( b - a \).
\end{definition}

\begin{definition}\label{def:approximation_error}\MarginCite[44]{Николов2020}
  We introduce two operators.

  \begin{DefEnum}
    \ILabel{def:approximation_error/algebraic} The \Def{algebraic approximation error}
    \begin{align*}
      E_n: C([a, b]) \to [0, \infty] \\
      E_n(f) \coloneqq \inf_{p \in \pi_n} \Norm{f - p}.
    \end{align*}

    \ILabel{def:approximation_error/trigonometric} The \Def{trigonometric approximation error}
    \begin{align*}
      \Ti{E}_n: C([a, b]) \to [0, \infty] \\
      \Ti{E}_n(f) \coloneqq \inf_{p \in \tau_n} \Norm{f - p}.
    \end{align*}
  \end{DefEnum}
\end{definition}

\begin{theorem}[Jackson's trigonometric theorem]\label{thm:jacksons_trigonometric_theorem}\MarginCite[47]{Николов2020}
  For \( f \in \Ti{C}[-\pi, \pi] \) we have
  \begin{equation*}
    \Ti{E}_n(f) \leq \frac {6^{k+1}} {n^k} \omega\left(f^{(k)}, \frac 1 n \right).
  \end{equation*}
\end{theorem}

\begin{theorem}[Szego’s inequality]\label{thm:szegos_trigonometric_inequality}\MarginCite[55]{Николов2020}
  For any nonnegative integer \( n \) and any \( s \in S_{\tau_n} \), we have
  \begin{equation}\label{eq:thm:szegos_trigonometric_inequality}
    [s'(\theta)]^2 + n^2 s^2 (\theta) \leq n^2 \quad\forall \theta \in [-\pi, \pi].
  \end{equation}
\end{theorem}
\begin{proof}
  Fix \( n = 1, 2, \ldots \) and \( \alpha \in [-1, 1] \).

  For brevity, denote \( c(\theta) \coloneqq \cos(n \theta) \). Let \( \theta_s \) and \( \theta_c \) be numbers in \( [-\pi, \pi] \) such that
  \begin{equation*}
    s(\theta_s) + c(\theta_c) = \alpha.
  \end{equation*}

  We will show that
  \begin{equation}\label{eq:thm:szegos_trigonometric_inequality/abs}
    \Abs{s'(\theta_s)} \leq \Abs{c'(\theta_c)}
  \end{equation}

  This will, in turn, imply that
  \begin{equation*}
    [s'(\theta_s)]^2
    \leq
    [c'(\theta_c)]^2
    =
    n^2 [\sin(n \theta_c)]^2
    \overset {\ref{thm:trigonometric_identities/pythagorean_identity}}
    =
    n^2 [1 - \cos(n \theta_c)^2]
    =
    n^2 [1 - s(\theta_s)]
  \end{equation*}
  which is equivalent to \eqref{eq:thm:szegos_trigonometric_inequality}.

  Now we will prove \eqref{eq:thm:szegos_trigonometric_inequality/abs}. If \( \theta_c \) is a critical point of \( c \), i.e. if \( r'(\theta_c) = 0 \), then \( c'(\theta_c) = s'(\theta_s) \) and \eqref{eq:thm:szegos_trigonometric_inequality/abs} holds. Suppose that \( \theta_c \) is not a critical point. Denote by
  \begin{equation*}
    \theta_n \coloneqq \tfrac k n \pi, k = -n, -n+1, \ldots, n-2, n-1.
  \end{equation*}
  the extrema of \( c(\theta) \) in \( [-\pi, \pi) \).

  Define the auxiliary function
  \begin{equation*}
    r(\theta) \coloneqq c(\theta) - s(\theta - \theta_c + \theta_s).
  \end{equation*}

  We now have \( r(\theta_c) = c(\theta_c) - s(\theta_s) = 0 \).

  Furthermore, since \( \Norm{s} = 1 \), then \( \Abs{s(\theta)} \leq 1 \) for all \( \theta \in [-\pi, \pi) \). Therefore \( r(\theta_k) \leq 0 \) for all odd \( k \) and \( r(\theta_k) \geq 0 \) for all even \( k \).

  If \( \theta_c \) coincides with any of the extrema \( \theta_k \), then \( r(\theta_c) \) holds trivially. Suppose that \( \theta_c \) is between \( \theta_{k-1} \) and \( \theta_k \). Without loss of generality, assume that \( k \) is even.

  By the intermediate value theorem, there exists a zero of \( r \) between \( r(\theta_c) \) and \( r(\theta_k) \). If \( r(\theta_c) < 0 \), then \( \theta_c \) is a local minimum and hence there exists a point \( \theta_{c'} \) between \( r(\theta_{k-1}) \) and \( \theta_c \) such that \( r(\theta_{c'}) = r(\theta_c) = 0 \). But this would imply that \( r \) has more than \( 2n \) different roots in the interval \( [-\pi, \pi) \), which is a contradiction.
\end{proof}

\begin{corollary}[Bernstein's trigonometric inequality]\label{thm:bernsteins_trigonometric_inequality}\MarginCite[53]{Николов2020}
  For any nonnegative integer \( n \) and any \( s \in \tau_n \) we have
  \begin{equation}\label{eq:thm:bernsteins_trigonometric_inequality}
    \Abs{s'(\theta)} \leq n \Norm{s} \quad\forall \theta \in [-\pi, \pi].
  \end{equation}
\end{corollary}
\begin{proof}
  The case \( n = 0 \) is trivial. If \( n > 0 \), for any \( s \in \tau_n \), we can apply \eqref{eq:thm:szegos_trigonometric_inequality} to \( \frac s {\Norm s} \) to obtain
  \begin{equation}
    [s'(\theta)]^2 + n^2 s^2 (\theta) \leq \Norm{s}^2 n^2 \quad\forall \theta \in [-\pi, \pi].
  \end{equation}

  \eqref{eq:thm:bernsteins_trigonometric_inequality} follows directly.
\end{proof}

\begin{theorem}[Bernstein's trigonometric theorem]\label{thm:bernsteins_trigonometric_theorem}\MarginCite[55]{Николов2020}
  Let \( f \in \Ti(C)[-\pi, \pi] \) and
  \begin{equation*}
    \Ti{E}_n(f) \leq \frac A {n^{k+\alpha}} \quad n = 0, 1, 2, \ldots,
  \end{equation*}
  where \( A \in \BR \) and \( \alpha \in (0, 1) \).

  Then \( f \in C^{(k)}[-\pi, \pi] \) and \( f^{(k)} \) is \( \alpha \)-H\"older.
\end{theorem}
\begin{proof}
  Since \( \Ti{E}_n(f) \) is bounded by \( \frac A {n^{k+\alpha}} \) on a compact interval, there exists a sequence \( \{ s_k \}_{k=0}^\infty \) such that
  \begin{equation*}
    \Norm{f - s_n} \leq A {n^{k+\alpha}}.
  \end{equation*}

  Define the sequence of polynomials
  \begin{equation*}
    v_j \coloneqq \begin{cases}
      s_1,                  &j = 0, \\
      s_{2^j} - s_{2^{j-1}} &j > 0.
    \end{cases}
  \end{equation*}

  It is now clear that
  \begin{equation*}
    \Norm{f - \sum_{j=0}^n v_j} \xrightarrow[]{j \to \infty} 0
  \end{equation*}
  because
  \begin{equation*}
    \Abs{f(\theta) - \sum_{j=0}^n v_j (\theta)}
    =
    \Abs{f(\theta) - s_{2^n}(\theta)}
    \leq
    A {n^{k+\alpha}}.
  \end{equation*}

  For each term of the series, we have
  \begin{equation*}
    \Abs{v_j(\theta)}
    \leq
    \Abs{f(\theta) - s_{2^j}(\theta)} + \Abs{f(\theta) - s_{2^{j-1}}(\theta)}
    \leq
    \frac A {2^{j(k + \alpha)}} + \frac A {2^{(j - 1) (k + \alpha)}}.
  \end{equation*}

  By setting \( B \coloneqq A (2^{k + \alpha}) \), we obtain
  \begin{equation*}
    \Abs{v_j(\theta)} \leq \frac B {2^{j(k + \alpha)}}.
  \end{equation*}

  By \fullref{thm:bernsteins_trigonometric_inequality},
  \begin{equation*}
    \Norm{v_j^{(r)}} \leq 2^{jr} \Norm{v_j} \leq \frac B {2^{j(k - r + \alpha)}}.
  \end{equation*}

  Then
  \begin{equation*}
    \sum_{j=0}^\infty v_j^{(r)} (\theta)
  \end{equation*}
  converges uniformly, therefore
  \begin{equation*}
    f^{(r)}(\theta) = \sum_{j=0}^\infty v_j^{(r)} (\theta).
  \end{equation*}
\end{proof}

\begin{theorem}[Bernstein's algebraic inequality]\label{thm:bernsteins_algebraic_inequality}\MarginCite[59]{Николов2020}
  For any nonnegative integer \( n \) and any \( p \in \pi_n \) and \( x \in (a, b) \) we have
  \begin{equation*}
    \Abs{p'(x)} \leq n \frac 1 {(b - a)(b - x)} \Norm{p}.
  \end{equation*}
\end{theorem}

\begin{theorem}[Bernstein's algebraic theorem]\label{thm:bernsteins_algebraic_theorem}\MarginCite[60]{Николов2020}
  Let \( f \in C[a, b] \) and
  \begin{equation*}
    E_n(f) \leq \frac A {n^{k+\alpha}} \quad n = 0, 1, 2, \ldots,
  \end{equation*}
  where \( A \in \BR \) and \( \alpha \in (0, 1) \).

  Then \( f \in C^{(k)}(a, b) \) and \( f^{(k)} \) is \( \alpha \)-H\"older in every \( [a_1, b_1] \) such that \( a_1 > a \) and \( b_1 < b \).
\end{theorem}
