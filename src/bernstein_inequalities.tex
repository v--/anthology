\subsection{Bernstein inequalities}\label{subsec:bernstein_inequalities}

\begin{definition}\label{def:modulus_of_continuity}\cite[27]{Николов2020}
  Let \( f: X \to Y \) be a function between metric spaces. We define the \Def{modulus of continuity} as
  \begin{align*}
    &\omega: \Cat{Set}(X, Y) \to [0, \infty] \\
    &\omega(f, \delta) \coloneqq \sup \Big\{ \rho_Y(f(x), f(y)) \colon \rho_X(x, y) < \delta \Big\}.
  \end{align*}
\end{definition}

\begin{proposition}\label{thm:modulus_of_continuity_properties}
  The \hyperref[def:modulus_of_continuity]{modulus of continuity} has the following basic properties:
  \begin{propenum}
    \DItem{thm:modulus_of_continuity_properties/continuity_condition} \( f \) is globally \hyperref[def:uniform_continuity]{uniformly continuous} if and only if for every \( \varepsilon > 0 \) there exists \( \delta > 0 \) such that \( \omega(f, \delta) \).

    \DItem{thm:modulus_of_continuity_properties/monotone} \( \omega(f, \delta) \) is monotone in \( \delta \).

    \DItem{thm:modulus_of_continuity_properties/cauchy_inequality}\cite[28]{Николов2020} For all \( \lambda, \delta > 0 \), we have the following analog of \fullref{thm:cauchy_bunyakovsky_schwarz_inequality}
    \begin{equation}\label{thm:modulus_of_continuity_properties/cauchy_inequality/inequality}
      \omega(f, \lambda \delta) \leq \omega(f, \lambda^2) + \omega(f, \delta^2).
    \end{equation}

    \DItem{thm:modulus_of_continuity_properties/single_inequality}\cite[28]{Николов2020} For all \( \lambda, \delta > 0 \),
    \begin{equation}\label{thm:modulus_of_continuity_properties/single_inequality/inequality}
      \omega(f, \lambda \delta) \leq (\lambda + 1) \omega(f, \delta).
    \end{equation}
  \end{propenum}
\end{proposition}
\begin{proof}\mbox{}
  \begin{description}
    \RItem{thm:modulus_of_continuity_properties/continuity_condition} Follows directly from \fullref{def:uniform_continuity}.

    \RItem{thm:modulus_of_continuity_properties/monotone} A supremum on a larger set is larger.

    \RItem{thm:modulus_of_continuity_properties/cauchy_inequality} If \( \lambda \leq \delta \), clearly \( \lambda \delta \leq \delta^2 \). Otherwise, \( \lambda \delta < \lambda^2 \).

    Combining the two inequalities with \fullref{thm:modulus_of_continuity_properties/monotone}, we obtain \fullref{thm:modulus_of_continuity_properties/cauchy_inequality/inequality}.

    \RItem{thm:modulus_of_continuity_properties/single_inequality} Note that
    \begin{equation*}
      \rho_X(x, y) < \delta \Timplies \rho_Y(f(x), f(y)) < \omega(f, \delta).
    \end{equation*}

    We can multiply this by \( \lambda \) to obtain
    \begin{equation*}
      \lambda \rho_X(x, y) < \lambda \delta \Timplies \lambda \rho_Y(f(x), f(y)) < \lambda \omega(f, \delta).
    \end{equation*}

    If \( \lambda \geq 1 \), then \( \rho_X(x, y) \leq \lambda \rho_X(x, y) \) and \( \rho_Y(f(x), f(y)) \leq \lambda \rho_Y(f(x), f(y)) \) and hence
    \begin{equation*}
      \omega(f, \lambda \delta) \leq \lambda \omega(f, \delta).
    \end{equation*}

    Otherwise, \( \lambda < 1 \) and clearly \( \lambda \delta < \delta \), which by \fullref{thm:modulus_of_continuity_properties/monotone} implies
    \begin{equation*}
      \omega(f, \lambda \delta) \leq \omega(f, \delta).
    \end{equation*}

    Combining the two cases, we obtain
    \begin{equation*}
      \omega(f, \lambda \delta) \leq \lambda \omega(f, \delta) + \omega(f, \delta),
    \end{equation*}
    which we wanted to prove.
  \end{description}
\end{proof}

\begin{definition}\label{def:real_function_space_operators}
  Consider the \hyperref[def:function/single_valued]{operator} \( T: C([a, b]) \to C([a, b]) \).

  \begin{defenum}
    \DItem{def:real_function_space_operators/positive} If \( f([a, b]) \subseteq [0, \infty) \) implies \( T(f)([a, b]) \subseteq [0, \infty) \), we say that \( T \) is \Def{positive}.

    \DItem{def:real_function_space_operators/monotone} If \( f(x) \leq g(x) \) for all \( x \in [a, b] \) implies \( T(f)(x) \leq T(g)(x) \) for all \( x \in [a, b] \), we say that \( T \) is \Def{monotone}.
  \end{defenum}
\end{definition}

\begin{definition}\label{def:periodic_function_space}\cite[44]{Николов2020}
  We denote by \( \Ti{C}([a, b]) \) the subspace of \( C([a, b]) \) consisting of all continuous functions in \( [a, b] \) which are periodic with minimal period \( b - a \).
\end{definition}

\begin{definition}\label{def:approximation_error}\cite[44]{Николов2020}
  We introduce two operators
  \begin{defenum}
    \DItem{def:approximation_error/algebraic}
    \begin{align*}
      E_n: C([a, b]) \to [0, \infty] \\
      E_n(f) \coloneqq \inf_{p \in \pi_n} \Norm{f - p}.
    \end{align*}

    \DItem{def:approximation_error/trigonometric}
    \begin{align*}
      \Ti{E}_n: C([a, b]) \to [0, \infty] \\
      \Ti{E}_n(f) \coloneqq \inf_{p \in \tau_n} \Norm{f - p}.
    \end{align*}
  \end{defenum}
\end{definition}

\begin{theorem}\label{thm:jacksons_trigonometric_theorem}\cite[47]{Николов2020}
  For \( f \in \Ti{C}[-\pi, \pi] \) we have
  \begin{equation*}
    \Ti{E}_n(f) \leq \frac {6^{k+1}} {n^k} \omega\left(f^{(k)}, \frac 1 n \right).
  \end{equation*}
\end{theorem}

\begin{theorem}\label{thm:bernsteins_trigonometric_inequality}[Bernstein's trigonometric inequality]\cite[53]{Николов2020}
  For any nonnegative integer \( n \) and any \( s \in \tau_n \) we have
  \begin{equation*}
    \Abs{s'(\theta)} \leq n \Norm{s} \quad\forall \theta \in [-\pi, \pi].
  \end{equation*}
\end{theorem}

\begin{theorem}\label{thm:bernsteins_trigonometric_theorem}[Bernstein's trigonometric theorem]\cite[55]{Николов2020}
  Let \( f \in \Ti(C)[-\pi, \pi] \) and
  \begin{equation*}
    \Ti{E}_n(f) \leq \frac A {n^{k+\alpha}} \quad n = 0, 1, 2, \ldots,
  \end{equation*}
  where \( A \in \BR \) and \( \alpha \in (0, 1) \).

  Then \( f \in C^{(k)}[-\pi, \pi] \) and \( f^{(k)} \) is \( \alpha \)-H\"older.
\end{theorem}

\begin{theorem}\label{thm:bernsteins_trigonometric_inequality}[Bernstein's algebraic inequality]\cite[59]{Николов2020}
  For any nonnegative integer \( n \) and any \( p \in \pi_n \) and \( x \in (a, b) \) we have
  \begin{equation*}
    \Abs{p'(x)} \leq n \frac 1 {(b - a)(b - x)} \Norm{p}.
  \end{equation*}
\end{theorem}

\begin{theorem}\label{thm:bernsteins_trigonometric_theorem}[Bernstein's algebraic theorem]\cite[60]{Николов2020}
  Let \( f \in C[a, b] \) and
  \begin{equation*}
    E_n(f) \leq \frac A {n^{k+\alpha}} \quad n = 0, 1, 2, \ldots,
  \end{equation*}
  where \( A \in \BR \) and \( \alpha \in (0, 1) \).

  Then \( f \in C^{(k)}(a, b) \) and \( f^{(k)} \) is \( \alpha \)-H\"older in every \( [a_1, b_1] \) such that \( a_1 > a \) and \( b_1 < b \).
\end{theorem}
