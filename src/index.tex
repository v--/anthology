\documentclass[numbers=endperiod, bibliography=totocnumbered]{scrartcl}

% Base packages
\usepackage[T2A]{fontenc}
\usepackage[utf8]{inputenc}
\usepackage[english]{babel}
\usepackage[pdfencoding=unicode]{hyperref}
\usepackage[style=alphabetic, citestyle=alphabetic]{biblatex}
\usepackage{csquotes}

% Base math packages
\usepackage{amsmath}
\usepackage{amssymb}
\usepackage{amsthm}
\usepackage{mathtools}
\usepackage{cleveref}

% Misc packages
\usepackage{ulem}
\usepackage{import}
\usepackage{enumitem}

% Custom packages
\usepackage{common/macros}
\usepackage{common/theorem_styles}

% Bibliography
\addbibresource{references.bib}

% Document
\title{Anthology}
\author{Ianis Vasilev, \Email{ianis@ivasilev.net}}
\date{}

% https://tex.stackexchange.com/questions/13715/how-to-suppress-overfull-hbox-warnings-up-to-some-maximum
\hfuzz=5pt

% https://tex.stackexchange.com/questions/156049/how-can-i-get-cleveref-enumitem-to-print-not-item-e-but-proposition-1
\newlist{defenum}{enumerate}{1}
\setlist[defenum]{label=\alph*), ref=\theproposition~(\alph*)}
\crefalias{defenumi}{definition}

\begin{document}

\maketitle

\begin{abstract}
  This is my attempt to formulate various mathematical concepts and theorems in a unified manner. The notes are quite useful for later reference. Most of the notions were new to me at the time of writing, but sometimes I come back to a fundamental notion and expand upon it with additional commentary. I do not use a strict note-taking process and so the document structure follows a natural evolution.
\end{abstract}

\tableofcontents

\section{Analysis}\label{sec:analysis}
\subsection{Fundamental notions}\label{sec:analysis/fundamental_notions}
\subsubsection{Differentiability in Banach spaces}\label{sec:banach_space_differentiability}
\import{}{banach_space_differentiability.tex}

\subsection{Convex analysis}\label{sec:convex_analysis}
\subsubsection{Convex functions}\label{sec:convex_functions}
\import{}{convex_functions.tex}

\subsection{Asplund spaces}\label{sec:asplund_spaces}
\subsubsection{Dentable sets}\label{sec:dentable_sets}
\import{}{dentable_sets.tex}
\subsubsection{Asplund space characterizations}\label{sec:asplund_space_char}
\import{}{asplund_space_char.tex}

\subsection{Nonsmooth analysis}\label{sec:nonsmooth_analysis}
\subsubsection{Clarke generalized gradients}\label{sec:clarke_gradients}
\import{}{clarke_gradients.tex}

\printbibliography

\end{document}
