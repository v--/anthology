\documentclass[numbers=endperiod, bibliography=totocnumbered]{scrartcl}

% Base packages
\usepackage[T2A]{fontenc}
\usepackage[utf8]{inputenc}
\usepackage[english]{babel}
\usepackage[pdfencoding=unicode]{hyperref}
\usepackage[style=alphabetic, citestyle=alphabetic]{biblatex}
\usepackage{csquotes}

% Base math packages
\usepackage{amsmath}
\usepackage{amssymb}
\usepackage{amsthm}
\usepackage{mathtools}
\usepackage{cleveref}

% Misc packages
\usepackage{ulem} % Line-breaking underlines
\usepackage{import} % Importing nested files

% Custom packages
\usepackage{common/macros}
\usepackage{common/theorem_styles}

% Bibliography
\addbibresource{references.bib}

% Document
\title{Anthology}
\author{Ianis Vasilev, \Email{ianis@ivasilev.net}}
\date{}

% https://tex.stackexchange.com/questions/171999/overfull-hbox-in-biblatex
\emergencystretch=1em

\begin{document}

\maketitle

\begin{abstract}
  This is my attempt to formulate various mathematical concepts and theorems in a unified manner. The notes are quite useful for later reference. Most of the notions were new to me at the time of writing, but sometimes I come back to a fundamental notion and expand upon it with additional commentary. I do not use a strict note-taking process and so the document structure follows a natural evolution.
\end{abstract}

\tableofcontents

\section{Analysis}
\subsection{Basic notions}
\import{analysis/basic/}{differentiability.tex}
\import{analysis/basic/}{combinations.tex}

\subsection{Convex analysis}
\import{analysis/convex/}{functions.tex}
\import{analysis/convex/}{singleton_subdifferential_implies_gateaux.tex}

\subsection{Asplund spaces}
\import{analysis/asplund/}{dentable_set_definitions.tex}
\import{analysis/asplund/}{weak_dentable_sets_are_dentable.tex}
\import{analysis/asplund/}{asplund_space_definition.tex}

\subsection{Nonsmooth analysis}
\import{analysis/nonsmooth/}{generalized_gradient_definition.tex}
\import{analysis/nonsmooth/}{generalized_gradient_existence.tex}

% \section{Topology}
% \subsection{}

\printbibliography

\end{document}
