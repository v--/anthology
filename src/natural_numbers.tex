\section{Numbers}\label{sec:numbers}
\subsection{Natural numbers}\label{subsec:natural_numbers}

\begin{definition}\label{def:peano_arithmetic}\MarginCite[1]{Peano1889}
  The Peano arithmetic is a \hyperref[def:first_order_logic_language]{first-order logic language} for describing \hyperref[def:natural_numbers]{natural numbers} and their operations. It consists of:
  \begin{DefEnum}
    \ILabel{def:peano_arithmetic/eq} A formal equality \( \doteq \) symbol.
    \ILabel{def:peano_arithmetic/zero} A \enquote{zero} constant \( 0 \) (see \fullref{remark:peano_arithmetic_zero}).
    \ILabel{def:peano_arithmetic/succ} A unary \enquote{successor} functional symbol \( s \).
  \end{DefEnum}

  We impose the following axioms:
  \begin{DefEnum}
    \IAxiom{def:peano_arithmetic/PA1}{PA1} \( \forall \xi \forall \eta (\xi \doteq \eta \iff s(\xi) \doteq s(\eta)) \)
    \IAxiom{def:peano_arithmetic/PA2}{PA2} \( \forall \xi \neg (s(\xi) \doteq 0) \)
    \IAxiom{def:peano_arithmetic/PA3}{PA3} The following, called the \Def{axiom schema of induction}, holds: for all formulas \( \varphi(\xi) \), we have:
    \begin{equation*}
      \varphi(0) \wedge \forall \xi (\varphi(\xi) \implies \varphi(s(\xi))).
    \end{equation*}

    We allow \( \varphi \) to have other free variables, however we avoid writing them to avoid cluttering our notation.
  \end{DefEnum}

  A \hyperref[def:first_order_model]{model} of this theory that only contains \( 0 \) and the \hyperref[remark:induction]{inductively defined} members \( s(0), s(s(0)), \ldots \) is said to be a \Def{standard model of arithmetic}. Non-standard models are know to exist but are mostly of theoretical interest to logicians.
\end{definition}

\begin{definition}\label{def:natural_numbers}
  We define the set of \Def[bg=естествени числа,ru=натуральные числа]{natural numbers} \( \BN \) as any \hyperref[sec:set_theory]{set-theoretic} standard \hyperref[def:first_order_model]{model} of the \hyperref[def:peano_arithmetic]{Peano arithmetic}. Unless mentioned otherwise, it is safe to assume that the model is the \hyperref[ex:natural_numbers_models/omega]{smallest inductive set} \( \omega \).

  See \fullref{ex:natural_numbers_models} for concrete examples of models.
\end{definition}

\begin{remark}\label{remark:peano_arithmetic_zero}
  It is common to consider the first natural numbers to be zero (e.g. \cite[67]{Enderton1977}). Peano himself, however, considered one to be the first - see \cite[1]{Peano1889}.

  Whether \( 0 \) is a natural number or not, i.e. whether \( \varnothing \) encodes \( 0 \) or \( 1 \), is a matter of convention and notation like \( \BZ_{\geq 0} \) and \( \BZ_{>0} \) or \( \BN_0 \) and \( \BN_+ \) is sometimes used.

  Peano's original axioms referred to sets rather than first-order logic languages, however this more general framework is more suitable nowadays.

  It is easier to construct models of natural numbers that semantically include zero (see \fullref{thm:natural_numbers_structure}), so we will construct the natural numbers along with zero and later on mostly use the set \( \BZ_{>0} \).
\end{remark}

\begin{example}\label{ex:natural_numbers_models}
  A model of the \hyperref[def:natural_numbers]{natural numbers} is based on a nonempty set \( N \) and additionally satisfies:
  \begin{RefList}
    \IRef{def:peano_arithmetic/eq} The equality is the usual \hyperref[def:set_zfc]{set-theoretic equality}.

    \IRef{def:peano_arithmetic/zero} There exists a constant \( e \in N \) acting as zero.

    \IRef{def:peano_arithmetic/succ} There exists a unary \enquote{successor} function \( S: N \to N \) such that \( S(x) \neq e \) for any \( x \in N \).
  \end{RefList}

  Any of the following (and many more) can be taken as a definition for the natural numbers:
  \begin{DefEnum}
    \ILabel{ex:natural_numbers_models/omega} A good fit for the set of natural numbers is \hyperref[def:smallest_inductive_set]{the smallest inductive set \( \omega \)}. In this case,
    \begin{RefList}
      \IRef{def:peano_arithmetic/zero} \( e = \varnothing \).
      \IRef{def:peano_arithmetic/succ} \( S \) is the set-theoretic \hyperref[def:successor_operator]{successor function}.
    \end{RefList}

    \ILabel{ex:natural_numbers_models/language} Let \( N = \{ a \}^{*} \) be the \hyperref[def:language]{Kleene star} of the singleton alphabet \( \{ a \} \). In this case,
    \begin{RefList}
      \IRef{def:peano_arithmetic/zero} \( e = \varepsilon \) is the empty word.
      \IRef{def:peano_arithmetic/succ} \( S(w) \coloneqq w \cdot a \) concatenates the letter at the end of a word.
    \end{RefList}

    \ILabel{ex:natural_numbers_models/vector_spaces} The \hyperref[def:first_order_model_category]{model category} \( \Cat{FinVect}_{\BK} \) of all \hyperref[def:vector_space_dimension]{finite-dimensional} \hyperref[def:vector_space]{vector spaces} over the field \( \BK \) can also be used as a model for the natural numbers. In this case,
    \begin{RefList}
      \IRef{def:peano_arithmetic/zero} \( e = 0_{\BK} \) is the trivial (zero-dimensional) vector space.
      \IRef{def:peano_arithmetic/succ} \( S(V) \coloneqq V \times \BK \) is the \hyperref[def:left_module_direct_product]{module direct sum} of \( V \) and \( \BK \).
    \end{RefList}
  \end{DefEnum}
\end{example}

\begin{definition}\label{def:natural_numbers_structure}
  We define the usual \hyperref[def:algebraic_theory]{arithmetic operations} and \hyperref[remark:order_infix_notation]{ordering} on the \hyperref[def:natural_numbers]{natural numbers} \( \BN \):
  \begin{DefEnum}
    \ILabel{def:natural_numbers_structure/addition} We define addition of \( x \) and \( y \) using \hyperref[remark:induction]{structural induction}\IND on \( y \):
    \begin{equation*}
      x + y \coloneqq \begin{cases}
        x,         & y = 0,     \\
        S(x + y'), & y = S(y').
      \end{cases}
    \end{equation*}

    \ILabel{def:natural_numbers_structure/multiplication} We define multiplication inductively\IND on \( y \):
    \begin{equation*}
      x \cdot y \coloneqq \begin{cases}
        0,              & y = 0,     \\
        x \cdot y' + x, & y = s(y').
      \end{cases}
    \end{equation*}

    \ILabel{def:natural_numbers_structure/order} We define the ordering of natural numbers via the \hyperref[def:first_order_formula]{formula}:
    \begin{equation*}
      x \leq y \iff \exists z \in \BN: y = x + z.
    \end{equation*}
  \end{DefEnum}
\end{definition}

\begin{proposition}\label{thm:natural_numbers_structure}\mbox{}
  \begin{PropEnum}
    \ILabel{thm:natural_numbers_structure/addition} The set \( \BN \) is a commutative \hyperref[def:magma]{monoid} under addition with \( 0 \) as identity.

    \ILabel{thm:natural_numbers_structure/multiplication} The set \( \BN \setminus \{ 0 \} \) is a commutative monoid under multiplication with \( 1 \coloneqq S(0) \) as identity.

    \ILabel{thm:natural_numbers_structure/order} The set \( \BN \) is \hyperref[def:well_ordered_set]{well-ordered} by \( \leq \).
  \end{PropEnum}
\end{proposition}
\begin{proof}
  \SubProofOf{thm:natural_numbers_structure/addition} Fix \( x + y \). We use nested induction\IND. We start with induction on \( y \):
  \begin{itemize}
    \item If \( y = 0 \), induction\IND by \( x \) yields:
    \begin{itemize}
      \item If \( x = y = 0 \),
      \begin{equation*}
        x + y = 0 + 0 = 0 = 0 + 0 = y + x
      \end{equation*}

      \item If \( y \) and \( x' \) commute with each other and if \( x = S(x') \), we have
      \begin{BreakableAlign*}
        x + y
         & =
        S(x') + 0
        =    \\ &=
        S(x')
        =    \\ &=
        S(x' + 0)
        =    \\ &=
        S(x' + y)
        =    \\ &=
        S(y + x')
        =    \\ &=
        y + S(x')
        =
        y + x.
      \end{BreakableAlign*}
    \end{itemize}
  \end{itemize}

  \item If \( y = S(y') \) and \( y' \) commutes with \( x \), we use induction\IND on \( x \):
  \begin{itemize}
    \item If \( x = 0 \),
    \begin{equation*}
      x + y = 0 + S(y') = S(0 + y') = S(y' + 0) = S(y') = y = y + x.
    \end{equation*}

    \item If \( y \) and \( x' \) commute with each other and if \( x = S(x') \), we have
    \begin{BreakableAlign*}
      x + y
       & =
      x + S(y')
      =    \\ &=
      S(x + y')
      =    \\ &=
      S(y' + x)
      =    \\ &=
      S(y' + S(x'))
      =    \\ &=
      S(S(y' + x'))
      =    \\ &=
      S(S(x' + y'))
      =    \\ &=
      S(x' + S(y'))
      =    \\ &=
      S(S(y') + x')
      =    \\ &=
      S(y') + S(x')
      =
      y + x.
    \end{BreakableAlign*}
  \end{itemize}

  \SubProofOf{thm:natural_numbers_structure/multiplication} Analogous.

  \SubProofOf{thm:natural_numbers_structure/order} See \fullref{thm:natural_numbers_are_well_ordered}.
\end{proof}
