\section{Numbers}\label{sec:numbers}
\subsection{Natural numbers}\label{subsec:natural_numbers}

\begin{definition}\label{def:peano_arithmetic}\cite[1]{Peano1889}
  The Peano arithmetic (see \cref{remark:peano_arithmetic_zero}) is a first-order logic language\Tinyref{def:first_order_logic_language} with 
  \begin{itemize}
    \item formal equality \( \doteq \).
    \item a constant \( 0 \) (see \cref{remark:peano_arithmetic_zero}).
    \item a unary function \( s \).
  \end{itemize}
  and axioms (see \cref{def:propositional_theory})
  \begin{defenum}
    \DItem{def:peano_arithmetic/PA1}[PA1] \( \forall \xi \forall \eta (\xi \doteq \eta \iff s(\xi) \doteq s(\eta)) \)
    \DItem{def:peano_arithmetic/PA2}[PA2] \( \forall \xi \neg s(\xi) \)
    \DItem{def:peano_arithmetic/PA3}[PA3] (the \Def{axiom schema of induction}) for all formulas \( \varphi(\xi) \) (it is possible for \( \varphi \) to have other free variables), we have
    \begin{equation*}
      \varphi(0) \land \forall \xi (\varphi(\xi) \implies \varphi(s(\xi)))
    \end{equation*}
  \end{defenum}

  Any of the models of this theory is called \Def{the set of natural numbers \( \N \)}. See \cref{def:natural_numbers} for a list of some models.
\end{definition}

\begin{remark}\label{remark:peano_arithmetic_zero}
  It is common to consider the first natural numbers to be zero (e.g. \cite[67]{Enderton1977}). Peano himself, however, considered one to be the first - see \cite[1]{Peano1889}.

  Whether \( 0 \) is a natural number or not, i.e. whether \( \varnothing \) encodes \( 0 \) or \( 1 \), is a matter of convention and notation like \( \Z_{\geq 0} \) and \( \Z_{>0} \) or \( \N^0 \) and \( \N^+ \) is sometimes used.

  Peano's original axioms referred to sets rather than first-order logic languages, however this more general framework is more suitable nowadays.

  It is easier to construct models of natural numbers that semantically include zero (see \cref{def:natural_numbers}), so we will construct the natural numbers along with zero and later on mostly use the set \( \Z_{>0} \).
\end{remark}

\begin{definition}\label{def:peano_arithmetic_models}
  We can define the natural numbers as any model of Peano arithmetic (see \cref{def:peano_arithmetic}). A set-theoretic model of the Peano arithmetic consists of a triple \( N, S, e \), where
  \begin{itemize}
    \item \( N \) is the universum, i.e. the set of natural numbers.
    \item \( S: N \to N \) is the successor function that increments a number.
    \item \( e \) is the first number, taken to be one (see \cref{remark:peano_arithmetic_zero}).
  \end{itemize}

  Any of the following (and many more) can be taken as a definition for the natural numbers:
  \begin{defenum}
    \DItem{def:peano_arithmetic_models/omega} the number of nested sets in the model
    \begin{itemize}
      \item \( N = \omega \) is the smallest inductive set\Tinyref{def:smallest_inductive_set}.
      \item \( S \) is the set-theoretic successor function\Tinyref{def:successor_operator}.
      \item \( e = \varnothing \).
    \end{itemize}

    \DItem{def:peano_arithmetic_models/language} the word length in the model
    \begin{itemize}
      \item \( N = \{ a \}^{*} \) is the Kleene star\Tinyref{def:language} of the singleton alphabet \( \{ a \} \).
      \item \( S(w) \coloneqq w \cdot a \) concatenates the letter at the end of a word.
      \item \( e = \varepsilon \) is the empty word.
    \end{itemize}

    \DItem{def:peano_arithmetic_models/vector_spaces} the vector space dimension\Tinyref{def:vector_space_dimension} in the model
    \begin{itemize}
      \item \( N \) is the set of all finite-dimensional vector spaces\Tinyref{def:vector_space} over a field\Tinyref{def:semiring/field} \( F \).
      \item \( S(V) \coloneqq V \times F \) is the module direct sum\Tinyref{def:left_module_direct_product} of \( V \) and \( F \).
      \item \( e = 0_F \) is the field considered as a vector space.
    \end{itemize}
  \end{defenum}
\end{definition}

\begin{definition}\label{def:natural_number_operations}
  Fix any set-theoretic model \( \N \) of the natural numbers\Tinyref{def:natural_numbers}. We use induction on \( y \in \N \) to define addition and multiplication:
  \begin{align*}
    x + y &\coloneqq \begin{cases}
      x, &y = 0, \\
      S(x + y'), &y = S(y'),
    \end{cases}
    \\
    x \cdot y &\coloneqq \begin{cases}
      0, &y = 0, \\
      x \cdot y' + x, &y = S(y'),
    \end{cases}
  \end{align*}

  We also define a partial order relation as
  \begin{equation*}
    x \leq y \iff \exists z \in \N: y = x + z.
  \end{equation*}
\end{definition}

\begin{proposition}\label{def:natural_numbers}
  \begin{itemize}\mbox{}
    \item The set \( \N \) is a commutative monoid\Tinyref{def:magma} under addition with \( 0 \) as identity.
    \item The set \( \N \setminus \{ 0 \} \) is a commutative monoid under multiplication with \( 1 \) as identity.
    \item The set \( \N \) is well-ordered by \( \leq \).
  \end{itemize}
\end{proposition}
\begin{proof}
  \begin{itemize}\mbox{}
    \item Fix \( x + y \). We use simultaneous induction by \( y \):
    \begin{itemize}
      \item if \( y = 0 \), induction by \( x \):
      \begin{itemize}
        \item if \( x = y = 0 \),
        \begin{equation*}
          x + y = 0 + 0 = 0 = 0 + 0 = y + x
        \end{equation*}

        \item if \( x = S(x') \) and \( y \) and \( x' \) commute with each other, we have
        \begin{align*}
          x + y
          &=
          S(x') + 0
          = \\ &=
          S(x')
          = \\ &=
          S(x' + 0)
          = \\ &=
          S(x' + y)
          = \\ &=
          S(y + x')
          = \\ &=
          y + S(x')
          =
          y + x.
        \end{align*}
      \end{itemize}

      \item if \( y = S(y') \) and \( y' \) commutes with \( x \), we use induction by \( x \):
      \begin{itemize}
        \item if \( x = 0 \),
        \begin{equation*}
          x + y = 0 + S(y') = S(0 + y') = S(y' + 0) = S(y') = y = y + x.
        \end{equation*}

        \item if \( x = S(x') \) and \( y' \) and \( x' \) commute with each other, we have
        \begin{align*}
          x + y
          &=
          x + S(y')
          = \\ &=
          S(x + y')
          = \\ &=
          S(y' + x)
          = \\ &=
          S(y' + S(x'))
          = \\ &=
          S(S(y' + x'))
          = \\ &=
          S(S(x' + y'))
          = \\ &=
          S(x' + S(y'))
          = \\ &=
          S(S(y') + x')
          = \\ &=
          S(y') + S(x')
          =
          y + x.
        \end{align*}
      \end{itemize}
    \end{itemize}

    \item Analogous.

    \item See \cref{thm:natural_numbers_are_well_ordered}.
  \end{itemize}
\end{proof}
