\subsection{The category of sets}\label{subsec:category_of_sets}

\begin{Definition}\label{def:category_of_sets}
  We define the \hyperref[def:category]{category} \( \Cat{Set} \) as follows:
  \begin{RefList}
    \IRef{def:category/objects} The \hyperref[def:set_zfc]{class} of objects is the class of all sets.

    \IRef{def:category/morphisms} The morphisms between two sets \( A, B \) are the \hyperref[def:function]{functions} \( f: A \to B \).

    \IRef{def:category/composition} Composition of morphisms is the usual \hyperref[def:function/composition]{function composition}.
  \end{RefList}
\end{Definition}
\begin{proof}
  To see that \( \Cat{Set} \) is indeed a category, we verify the axioms
  \begin{RefList}
    \IRef{def:category/identity} For any set \( A \in \Cat{Set} \), we have the identity function
    \begin{align*}
      &\Id_A: A \to A \\
      &\Id_A(x) \coloneqq A.
    \end{align*}

    If \( f: A \to B \) is any function, for all \( x \in A \) we have
    \begin{align*}
      [\Id_B \circ f](x) = \Id_B(f(x)) = f(x),
      &&
      [f \circ \Id_A](x) = f(\Id_A(x)) = f(x),
    \end{align*}
    thus \( \Id_A \) and \( \Id_B \) are indeed identity morphisms.

    \IRef{def:category/associativity} Let \( f: A \to B \), \( g: B \to C \) and \( h: C \to D \) be arbitrary functions. For any \( x \in A \), we have
    \begin{align*}
      [[h \circ g] \circ f](x)
      =
      [h \circ g](f(x))
      =
      h(g(f(x)))
      =
      h([g \circ f](x))
      =
      [h \circ [g \circ f]](x).
    \end{align*}
  \end{RefList}
\end{proof}

\begin{Corollary}\label{thm:functions_over_set_form_monoid}
  Let \( X \) be a set. Then the family \( \Cat{Set}(X) \) of all \hyperref[remark:category_obj_hom]{functions} is a \hyperref[def:magma/monoid]{monoid} with function composition as the operation.
\end{Corollary}

\begin{Proposition}\label{thm:set_is_locally_small}
  The category \( \Cat{Set} \) is \hyperref[def:category_cardinality]{locally small}.
\end{Proposition}
\begin{proof}
  Since a function \( f: X \to Y \) is \hyperref[def:function]{formally} a subset of the \hyperref[def:cartesian_product]{Cartesian product} \( X^2 \), which is a set, it is itself a set.

  The class of all functions \( \Cat{Set}(X, Y) \) from \( X \) to \( Y \) is then a subset of \( \Pow(X \times Y) \), which is also a set. Thus \( \Cat{Set} \) is a locally small category.
\end{proof}

\begin{Proposition}\label{thm:set_categorical_limits}
  We are interested in \hyperref[def:categorical_limit]{categorical limits} and \hyperref[def:categorical_colimit]{colimits} in \( \Cat{Set} \). Fix an indexed \hyperref[def:indexed_family]{family} \( \{ X_k \}_{k \in \CK} \) of sets.
  \begin{DefEnum}
    \ILabel{thm:set_categorical_limits/product} Their \hyperref[def:categorical_product]{categorical product} is their \hyperref[def:cartesian_product]{Cartesian product} \( \prod_{k \in \CK} X_k \), the projection morphisms being
    \begin{align*}
      &\pi_\beta: \prod_{k \in \CK} X_k \to X_\beta \\
      &\pi_\beta(\{ x_k \}_{k \in \CK}) \coloneqq x_\beta.
    \end{align*}

    \ILabel{thm:set_categorical_limits/coproduct} Their \hyperref[def:categorical_coproduct]{categorical coproduct} is their disjoint \hyperref[def:disjoint_union]{union} \( \coprod_{k \in \CK} X_k \), the injection morphisms being
    \begin{align*}
      &\iota_\beta: X_\beta \to \coprod_{k \in \CK} X_k \\
      &\iota_\beta(x_k) \coloneqq (\beta, x_k).
    \end{align*}

    \ILabel{thm:set_categorical_limits/equalizer} An equalizer of two functions \( f, g: X \to Y \) in \( \Cat{Set} \) is the set
    \begin{equation*}
      \{ x \in X \colon f(x) = g(x) \}.
    \end{equation*}

    Compare this with pullbacks in \( \Cat{Set} \) (see \fullref{thm:set_categorical_limits/pullback}).

    \ILabel{thm:set_categorical_limits/pullback} The pullback of two functions \( f: X \to Z \) and \( g: Y \to Z \) in \( \Cat{Set} \) is the set
    \begin{equation*}
      \{ (x, y) \in X \times Y \colon f(x) = g(y) \}.
    \end{equation*}

    Compare this with equalizers in \( \Cat{Set} \) (see \fullref{thm:set_categorical_limits/equalizer}).

    \ILabel{thm:set_categorical_limits/coequalizer} A coequalizer of two functions \( f, g: X \to Y \) in \( \Cat{Set} \) is the quotient set formed by the reflexive, symmetric and transitive closure of the relation \( x \sim y \iff f(x) = g(x) \).

    In particular, if \( \sim \subseteq X^2 \) is an \hyperref[def:equivalence_relation]{equivalence relation}) on \( X \), then the coequalizer of the two projection maps of the product \( X^2 \) is the pair \( (X / ~, \pi) \), where \( \pi \) is the quotient map
    \begin{align*}
      &\pi: X \to X / \sim \\
      &\pi(x) = [x].
    \end{align*}

    \ILabel{thm:set_categorical_limits/pushout} Let \( X \) and \( Y \) be two sets, let \( Z \) be a subset of \( X \) and let \( i: Z \to X \) be the inclusion map. For any function \( f: Z \to Y \), we define a pushout of \( i \) and \( f \) in \( \Cat{Set} \) to be the set obtained as the quotient of the coproduct \( X \coprod Y \) and the relation \( x \sim y \iff i^{-1}(x) = f^{-1}(y) \).
  \end{DefEnum}
\end{Proposition}

\begin{Proposition}\label{thm:set_is_monoidal}
  The category \( \Cat{Set} \) is monoidal with
  \begin{itemize}
    \item the \hyperref[def:cartesian_product]{Cartesian product} acting as a monoidal product
    \item the singleton set \( \{ \varnothing \} \) acting as an identity object
    \item natural transformations
    \begin{align*}
      \sigma &\coloneqq \Id \\
      \lambda(\{ \varnothing \} \times A) &\coloneqq A \\
      \rho(A \times \{ \varnothing \}) &\coloneqq A
    \end{align*}
  \end{itemize}
\end{Proposition}
\begin{proof}
  All conditions in \fullref{def:monoidal_category} are trivially satisfied.
\end{proof}
