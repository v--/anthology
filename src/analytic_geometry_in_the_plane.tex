\subsection{Analytic geometry in the plane}\label{subsec:analytic_geometry_in_the_plane}

\begin{remark}\label{remark:analytic_geometry}
  Analytic geometry is a XVII-century branch of mathematics that studies geometric figures using coordinate \hyperref[remark:coordinate_systems]{systems}. The term \enquote{analytic geometry} may refer to a modern subbranch of algebraic geometry, however we refrain from using \enquote{analytic geometry} in that sense. Historically, most of these definitions were given either for the Euclidean \hyperref[def:euclidean_plane]{plane} or for the three-dimensional Euclidean space.

  Most of the definitions from \fullref{subsec:vector_space_geometry} are generalizations of concepts from analytic geometry. We will state definitions in the language of linear algebra and refrain from using synthetic (axiomatic) geometry. When working in the plane (resp. three-dimensional space), we will assume that we have fixed an \hyperref[def:orthonormal_system]{orthonormal} coordinate \hyperref[def:euclidean_plane_coordinate_system]{system} \( Oxy \) (resp. \( Oxyz \)), which allows us to visualize geometric figures.
\end{remark}

\begin{definition}\label{def:plane_line_equations}
  \hyperref[def:geometric_line]{Lines} in \( \BR^2 \) are so ubiquitous that they can be represented by a lot of standard \hyperref[remark:equations]{equations}.

  \begin{defenum}
    \DItem{def:plane_line_equations/vector_parametric} When regarding a line as a parametric curve as in \fullref{def:geometric_line/parametric}, the \hyperref[def:first_order_formula]{formula}
    \begin{equation}\label{def:plane_line_equations/parametric_equation}
      l(t) = tx + a
    \end{equation}
    is called a \Def{vector parametric equation}.

    \DItem{def:plane_line_equations/scalar_parametric} Given \fullref{def:plane_line_equations/parametric_equation}, the \Def{scalar parametric equations} of the line are
    \begin{equation}\label{def:plane_line_equations/scalar_parametric_equations}
      \begin{cases}
        &l_1(t) = t x_1 + a_1 \\
        &l_2(t) = t x_2 + a_2.
      \end{cases}
    \end{equation}

    \DItem{def:plane_line_equations/general} When regarding a line as an algebraic curve as in \fullref{def:geometric_line/algebraic}, the equation
    \begin{equation}\label{def:plane_line_equations/general_equation}
      p(x, y) \coloneqq Ax + By + C = 0
    \end{equation}
    is called the \Def{general equation} or simply \Def{equation} of a line in a plane. Either \( A \) or \( B \) must be nonzero so that \( \deg(p) = 1 \).

    Note that multiple general equations can have the same locus (e.g. the entire polynomial ideal \( \Gen{p} \)).

    \DItem{def:plane_line_equations/normal} If \( A^2 + B^2 = 1 \), we call \fullref{def:plane_line_equations/general_equation} a \Def{normal equation}. This leaves us with only two representatives of \( \Gen{p} \).

    \DItem{def:plane_line_equations/cartesian} Given \( k, m \in \BR \) and  \( k \neq 0 \), we define the \Def{Cartesian equation} of a line:
    \begin{equation}\label{def:plane_line_equations/cartesian_equation}
      y = kx + m.
    \end{equation}

    We call \( k \) the \Def{slope} of the line.

    This is a special case of \fullref{def:plane_line_equations/general} with \( A = -k \), \( B = -1 \) and \( C = m \). Unlike the general equation, the Cartesian equation of a line is unique.

    Conversely, if \( B \neq 0 \) in \fullref{def:plane_line_equations/general_equation}, we can define \( k = -\tfrac A B \) and \( m = -\tfrac C B \) to form a Cartesian equation.

    \begin{figure}
      \centering
      \begin{mplibcode}
        input metapost/plotting;

        u := 1.5cm;

        beginfig(1);
          path l, x_axis, y_axis;

          x_axis = (-1, 0) scaled u -- (1, 0) scaled u;
          y_axis = (0, -1) scaled u -- (0, 1) scaled u;
          l = (-1 / 2, -1) * u -- (1, 3 / 4) * u;

          drawarrow x_axis;
          label.bot("$x$", point 0.9 of x_axis);
          drawarrow y_axis;
          label.lft("$y$", point 0.9 of y_axis);
          draw l;
          label.lrt("$y = kx + m$", endpoint of l);
        endfig;
      \end{mplibcode}
      \caption{A line in \( \BR^2 \) defined using its Cartesian equation}\label{def:plane_line_equations/cartesian_equation_drawing}
    \end{figure}

    \DItem{def:plane_line_equations/intercept} Given nonzero \( a, b \in \BR \), we define the \Def{intercept equation} of a line:
    \begin{equation}\label{def:plane_line_equations/intercept_equation}
      \frac x a + \frac y b = 1,
    \end{equation}

    This is a special case of \fullref{def:plane_line_equations/general} with \( A = \frac 1 a \), \( B = \frac 1 b \) and \( C = -1 \). The intercept equation of a line is also unique.

    If \( A, B, C \neq 0 \) in \fullref{def:plane_line_equations/general}, we can define an \Def{intercept equation} as \( a = -\tfrac C A \) and \( b = -\tfrac C B \)).
  \end{defenum}
\end{definition}

\begin{definition}\label{def:angle}
  A \Def{directed angle} is a tuple of two closed \hyperref[def:geometric_ray]{rays} with a common vertex. It is a closed cone.

  Suppose that the rays have scalar parametric equations
  \begin{equation*}
    t \mapsto
    \begin{cases}
      tx_i + a_i \\
      ty_i + b_i,
    \end{cases}
    i = 1, 2.
  \end{equation*}

  The condition of the rays having a common vertex is equivalent to \( a_1 = a_2 \) and \( b_1 = b_2 \). If not specified otherwise, we assume that \( a_1 = a_2 = b_1 = b_2 = 0 \).

  \begin{figure}
    \centering
    \begin{mplibcode}
      input metapost/plotting;
      u := 1.5cm;

      beginfig(1)
        drawarrow (-1 / 2, 0) scaled u -- (2, 0) scaled u;
        drawarrow (0, -1 / 2) scaled u -- (0, 2) scaled u;

        z0 = (1 / 2, 1 / 6) scaled u;
        z1 = (2, 11 / 12) scaled u;
        z2 = (1, 13 / 6) scaled u;

        draw z0 -- (x0, max(y1, y2)) dashed withdots;

        drawarrow z0 -- z1;
        draw (x0, y1) -- z1 dashed evenly;
        label.top("$x_1$", midpoint of ((x0, y1) -- z1));

        drawarrow z0 -- z2;
        draw (x0, y2) -- z2 dashed evenly;
        label.bot("$x_2$", midpoint of ((x0, y2) -- z2));
      endfig;
    \end{mplibcode}
    \caption{An angle with the measurement segments marked.}\label{def:angle/figure}
  \end{figure}

  The measure of a directed angle, often called the angle itself, can be defined as the number
  \begin{equation*}
    \alpha \coloneqq \Rem\left(\arccos\left(\frac {x_2} {\sqrt{x_2^2 + y_2^2}} \right) - \arccos\left(\frac {x_1} {\sqrt{x_1^2 + y_1^2}} \right), 2\pi \right).
  \end{equation*}

  We can classify angles based on their measure as
  \begin{defenum}
    \DItem{def:angle/zero} \Def{zero} if \( \alpha = 0 \),
    \DItem{def:angle/acute} \Def{acute} if \( \alpha < \tfrac \pi 2 \),
    \DItem{def:angle/right} \Def{right} if \( \alpha = \tfrac \pi 2 \),
    \DItem{def:angle/obtuse} \Def{obtuse} if \( \alpha \in (\tfrac \pi 2, \pi) \),
    \DItem{def:angle/straight} \Def{straight} if \( \alpha = \pi \), in which case the angle is actually a line,
    \DItem{def:angle/reflex} \Def{reflex} if \( \alpha > \pi \).
  \end{defenum}

  We often do not care about the order of the two rays and speak of an \Def{undirected angle}. In this case, the measure of the undirected angle is the smaller of the measures of the two oriented angles. Thus we cannot speak of straight and reflex undirected angles.
\end{definition}

\begin{definition}\label{def:quadratic_plane_curve}
  The \Def{quadratic plane curves} are algebraic \hyperref[def:hypersurface/algebraic]{curves} given by a bivariate polynomial of degree \( 2 \). The \Def{general equation} of a quadratic plane curve is
  \begin{equation}\label{def:quadratic_plane_curve/general_equation}
    c(x, y) \coloneqq A x^2 + B xy + C y^2 + Dx + Ey + F = 0.
  \end{equation}

  Multiple equation can correspond to the same curve. Not all general equations, however, define algebraic curves. We will not concern ourselves with the details. See \fullref{ex:affine_varieties} for a proof that the unit circle is an algebraic curve. It turns out that the algebraic curves given \fullref{def:quadratic_plane_curve/general_equation} are precisely the ones listed here, collectively known as \Def{conic sections}. We give only canonical forms of the equations; any linear transformation of the corresponding loci is described by another general equation.

  \begin{defenum}
    \DItem{def:quadratic_plane_curve/ellipse} An \Def{ellipse} is a quadratic curve whose canonical equation has the form
    \begin{equation}\label{def:quadratic_plane_curve/ellipse/canonical_equation}
      c(x, y) \coloneqq \frac {x^2} {a^2} + \frac {x^y} {b^2} - 1 = 0,
    \end{equation}
    where \( a, b > 0 \).

    If \( a = b \), we say that the ellipse is a \Def{circle} and we call \( a \) the circle's \Def{radius}. Circles generalize to \hyperref[def:metric_space/sphere]{spheres} in metric spaces.

    We are often interested in defining ellipses via \Def{scalar parametric equations} using \hyperref[def:trigonometric_functions]{trigonometric functions} as follows:
    \begin{equation}\label{def:quadratic_plane_curve/ellipse/parametric_equations}
      \begin{cases}
        x = a \cos(t) \\
        y = b \sin(t),
      \end{cases}
    \end{equation}
    where \( t \in [0, 2\pi) \).

    \begin{figure}
      \centering
      \begin{mplibcode}
        input metapost/plotting;

        vardef scaled_cos(expr x) =
          2 * cos(x)
        enddef;

        beginfig(1)
          fill dot shifted (2u, 0);

          drawarrow (-pi, 0) scaled u -- (pi, 0) scaled u;
          drawarrow (0, -pi / 2) scaled u -- (0, pi / 2) scaled u;

          drawarrow path_of_curve(scaled_cos, sin, -1 / 4 * pi, 3 / 4 * pi, 0.01, u);
          drawarrow path_of_curve(scaled_cos, sin, 3 / 4 * pi, 7 / 4 * pi, 0.01, u);
        endfig;
      \end{mplibcode}
      \caption{An ellipse defined via its parametric equations. The starting point is highlighted and the direction of the parametric curves is shown.}\label{def:quadratic_plane_curve/ellipse/parametric_equations_figure}
    \end{figure}

    \DItem{def:quadratic_plane_curve/hyperbola} A \Def{hyperbola} is a quadratic curve whose canonical equation has the form
    \begin{equation}\label{def:quadratic_plane_curve/hyperbola/canonical_equation}
      c(x, y) \coloneqq \frac {x^2} {a^2} + \frac {y^2} {b^2} + 1 = 0,
    \end{equation}
    where \( a, b > 0 \).

    Similarly to ellipses, we are can define hyperbolas via \Def{scalar parametric equations} using \hyperref[def:hyperbolic_trigonometric_functions]{hyperbolic trigonometric functions} as follows:
    \begin{equation}\label{def:quadratic_plane_curve/hyperbola/parametric_equations}
      \begin{cases}
        x = a \cosh(t) \\
        y = b \sinh(t),
      \end{cases}
    \end{equation}
    where \( t \in \BR \). This only defines the \Def{right part} of the hyperbola. The left part is defined by replacing \( a \) with \( -a \).

    \begin{figure}
      \centering
      \begin{mplibcode}
        input metapost/plotting;

        vardef minus_cosh(expr x) =
          -cosh(x)
        enddef;

        beginfig(1)
          drawarrow (-pi / 2, 0) scaled u -- (pi / 2, 0) scaled u;
          drawarrow (0, -pi / 2) scaled u -- (0, pi / 2) scaled u;

          drawarrow path_of_curve(cosh, sinh, -pi / 3, 0, 0.01, u);
          drawarrow path_of_curve(cosh, sinh, 0, pi / 3, 0.01, u);

          drawarrow path_of_curve(minus_cosh, sinh, -pi / 3, 0, 0.01, u);
          drawarrow path_of_curve(minus_cosh, sinh, 0, pi / 3, 0.01, u);
        endfig;
      \end{mplibcode}
      \caption{A hyperbola defined via its parametric equations.}\label{def:quadratic_plane_curve/hyperbola/parametric_equations_figure}
    \end{figure}

    \DItem{def:quadratic_plane_curve/parabola} A \Def{parabola} is a quadratic curve whose canonical equation has the form
    \begin{equation}\label{def:quadratic_plane_curve/parabola/canonical_equation}
      c(x, y) \coloneqq y^2 - 2px = 0,
    \end{equation}
    where \( p \neq 0 \).

    Unlike ellipses and hyperbolas, we do not define parametric equations. Instead, we define \( y \) as a function of \( x \) separately for the lower half-plane and upper half-plane:
    \begin{equation}\label{def:quadratic_plane_curve/parabola/cartesian_equation}
      y(x) = \pm \sqrt{2px}.
    \end{equation}

    \begin{figure}
      \centering
      \begin{mplibcode}
        input metapost/plotting;

        beginfig(1)
          fill dot;

          drawarrow (-pi / 2, 0) scaled u -- (pi / 2, 0) scaled u;
          drawarrow (0, -pi / 2) scaled u -- (0, pi / 2) scaled u;

          vardef y_upper(expr x) =
            sqrt(x)
          enddef;

          vardef y_lower(expr x) =
            -sqrt(x)
          enddef;

          drawarrow path_of_plot(y_upper, 0, pi / 3, 0.01, u);
          drawarrow path_of_plot(y_lower, 0, pi / 3, 0.01, u);
        endfig;
      \end{mplibcode}
      \caption{A parabola defined via its parametric equations.}\label{def:quadratic_plane_curve/parabola/parametric_equations_figure}
    \end{figure}
  \end{defenum}

  Ellipses, hyperbolas and parabolas are collectively called \Def{conic sections}.
\end{definition}
