\subsection{Vector spaces}\label{subsec:vector_spaces}

\begin{definition}\label{def:vector_space}
  A \Def{vector space} \( (V, +, \cdot) \) is a module\Tinyref{def:module} over a field \( F \).

  We call elements of \( F \) \Def{scalars} and elements of \( V \) \Def{vectors}.
\end{definition}

\begin{definition}\label{def:vector_field}
  Let \( V \) be a vector space over \( F \). Functions of the type
  \begin{equation*}
    f: F \to V
  \end{equation*}
  are called \Def{vector fields}. To avoid confusion, \( F \) is sometimes referred to as a \Def{scalar field}. This convention comes from physics and is dominant in areas that are far from algebraic field theory, hence in practice it does not cause a lot of confusion.
\end{definition}

\begin{remark}\label{remark:real_vector_space}
  Outside of algebra, we are usually only interested in vector spaces over the fields \( \R \) or \( \BB{C} \). We call them \Def{real vector spaces} and \Def{complex vector spaces}, respectively.
\end{remark}

\begin{proposition}\label{thm:field_extension_is_vector_space}
  Let \( F \) be a field extension\Tinyref{def:field_extension} of \( G \). Then \( F \) is a vector space over \( G \).
\end{proposition}
\begin{proof}
  Since \( F \) already has the structure of an abelian group, we must only define scalar multiplication
  \begin{align*}
    &\circ: G \times F \to F, \\
    &g \circ f \coloneqq gf,
  \end{align*}
  where the product in the definition is simply multiplication in \( F \). The well-definedness of \( \circ \) follows from the well-definedness of multiplication in \( F \).
\end{proof}

\begin{theorem}\label{thm:all_vector_spaces_are_free_modules}
  All vector spaces have a basis\Tinyref{def:module_basis}. Equivalently, all vector spaces are free modules\Tinyref{def:free_module}. See \cref{thm:aoc/vector_space_bases}.
\end{theorem}
\begin{proof}
  Let \( V \) be a vector space. Assume that it does not have a basis. Let \( \Cal{B} \) be the family of all linearly independent subsets\Tinyref{def:linear_combination} of \( V \).

  The family \( \Cal{B} \) is obviously nonempty since any singleton\Tinyref{remark:singleton_sets} from \( V \) belongs to \( \Cal{B} \). The union of any chain \( \Cal{B}' \subseteq \Cal{B} \) can then contain only linearly independent elements since otherwise\LEM we would have that some set in \( \Cal{B}' \) is not linearly independent. Thus we can apply Zorn's lemma\Tinyref{thm:aoc/zorn} to obtain a maximal element \( B \).

  Assume\LEM that \( B \) is not a basis, that is,
  \begin{equation*}
    \Span B \subsetneq V.
  \end{equation*}

  Take \( V \in V \setminus \Span B \). Then the set \( B \cup \{ v \} \) is linearly independent, which contradicts the assumption that \( B \) is not a basis. Thus \( B \) is a basis of \( V \) and \( V \) is a free module.
\end{proof}

\begin{definition}\label{def:vector_space_dimension}
  The free module rank\Tinyref{def:free_module} of a vector space \( V \) is called the \Def{dimension} \( \dim V \) of \( V \).
\end{definition}

\begin{definition}\label{def:linear_operator}
  An \( F \)-module homomorphism\Tinyref{def:module_homomorphism} between the \( F \)-vector spaces \( U \) and \( V \) is called a \Def{linear map} or \Def{linear operator}.
\end{definition}

\begin{proposition}\label{thm:linear_operator_iff_function_on_basis}
  Let \( U \) and \( V \) be \( F \)-vector spaces and let \( B \) be a basis of \( U \). Then there exists a bijection between the linear maps from \( U \) to \( V \) and the functions from \( B \) to \( V \).
\end{proposition}
