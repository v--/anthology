Let $(X, \rho)$ be a metric space. Let $\B$ be the family of bounded sets in $X$.

\begin{definition}\label{def:totally_bounded_set}
  The space $A \subseteq X$ is called \uline{totally bounded} if any of the following equivalent conditions hold:

  \begin{defenum}
    \item\label{def:totally_bounded_set/sets} For every $\varepsilon > 0$ there exists a finite cover of $A$ with sets with diameter at most $\varepsilon$.
    \item\label{def:totally_bounded_set/balls} For every $\varepsilon > 0$ there exists a finite cover of $A$ with balls of radius $\varepsilon$.
    \item\label{def:totally_bounded_set/zero_noncompactness/sets} The ball noncompactness measure (see~\cref{def:noncompactness_measures/sets}) $\alpha(A)$ is zero.
    \item\label{def:totally_bounded_set/zero_noncompactness/balls} The ball noncompactness measure (see~\cref{def:noncompactness_measures/balls}) $\beta(A)$ is zero.
    \item\label{def:totally_bounded_set/fundamental_subsequences} Every sequence in $A$ admits a fundamental subsequence.
  \end{defenum}

  Totally bounded sets are sometimes called \uline{precompact} (see \cref{def:compact_sets}) because of~\cref{thm:metric_sequentially_compact_iff_compact}. This equivalence requires the metric space to be complete, however.
\end{definition}
\begin{proof}
  The equivalences \ref{def:totally_bounded_set/sets} $\iff$ \ref{def:totally_bounded_set/zero_noncompactness/sets} and \ref{def:totally_bounded_set/balls} $\iff$ \ref{def:totally_bounded_set/zero_noncompactness/balls} are straightforward.

  (\ref{def:totally_bounded_set/balls} $\implies$ \ref{def:totally_bounded_set/sets}) Given $\varepsilon > 0$, any cover of $A$ with balls of radius $\frac \varepsilon 2$ is a cover with sets of diameter $\varepsilon$.

  (\ref{def:totally_bounded_set/sets} $\implies$ \ref{def:totally_bounded_set/balls}) Fix $\varepsilon > 0$ and $\mu \in (0, \varepsilon)$ and let $A_1, \ldots, A_n \subseteq \PS X$ be a finite cover of $A$ with sets of diameter at most $\mu$.

  Choose\AOC~a point $x_k$ from every $A_k$, $k = 1, \ldots, n$. We then have that for every $k = 1, \ldots, n$,
  \begin{align*}
    A_k \subseteq \Cl B(x_k, \mu) \subsetneq B(x_k, \varepsilon)
    \\
    \implies A \subseteq \bigcup_{k=1}^n A_k \subseteq \bigcup_{k=1}^n B(x_k, \mu) \subsetneq \bigcup_{k=1}^n B(x_k, \varepsilon),
  \end{align*}
  hence $x_1, \ldots, x_n$ are centers of $\varepsilon$-balls that cover $A$.

  (\ref{def:totally_bounded_set/balls} $\implies$ \ref{def:totally_bounded_set/fundamental_subsequences}) Let $\{ x_n \} \subseteq A$ be any sequence.

  If we assume\LEM\ that $\{ x_n \}$ has no fundamental subsequence, then there exists $\varepsilon_0 > 0$ such that $\rho(x_k, x_m) > \varepsilon_0$ for any $n, m \in \ZPos$.

  Consider a finite cover of $A$ with $\varepsilon_0$-balls. By the pigeonhole principle, at least one of the balls contains more than one element of the sequence, which contradicts the assumption that all elements of the sequence have a distance of at least $\varepsilon_0$.

  Hence an arbitrary sequence in $A$ has a fundamental subsequence.

  (\ref{def:totally_bounded_set/fundamental_subsequences} $\implies$ \ref{def:totally_bounded_set/balls}) Assume\LEM\ that there exists $\varepsilon_0 > 0$, such that $A$ admits no finite cover by $\varepsilon_0$-balls.

  Define $x_1 \in X, x_2 \in X \setminus B(x_1, \varepsilon_0), \ldots$, so that every two elements of the sequence $\{ x_n \}$ have a distance of at least $\varepsilon_0$. But then the sequence is does not admit a fundamental subsequence, which contradicts our assumption.

  This contradiction proves that $A$ admits a finite cover by $\varepsilon$-balls for every $\varepsilon > 0$.
\end{proof}

\begin{corollary}\label{thm:metric_compact_iff_closed_totally_bounded}
  Assume that $X$ is complete. The set $A \subseteq X$ is sequentially compact if and only if it is closed and totally bounded.
\end{corollary}
\begin{proof}
  The property that every sequence has a fundamental subsequence is equivalent to sequential compactness for a closed set in a complete metric space.
\end{proof}

\begin{proposition}\label{thm:closure_of_totally_bounded_is_totally_bounded}
  If a set $A \subseteq X$ is totally bounded, then so is its closure $\Cl A$.
\end{proposition}
\begin{proof}
  Let $\varepsilon > 0$ and $\mu \in (0, \varepsilon)$ and let $x_1, \ldots, x_n \in X$ be the centers of a cover of $A$ with $\mu$-balls.

  If $y$ is a point in $\Cl A \setminus A$, there exists a point $z \in A$ with $\rho(y, z) < \varepsilon - \mu$. Let $x_k \in A$ be one of the centers whose $\mu$-balls contain $z$. We then have that $y \in B(x_k, \varepsilon)$ since
  \begin{align*}
    \rho(x_k, z) \leq \rho(x_k, y) + \rho(y, z) < \mu + \varepsilon - \mu = \varepsilon.
  \end{align*}

  Hence the balls $\Cl B(x_k, \varepsilon)$ cover $\Cl A$, i.e.
  \begin{align*}
    \Cl A \subseteq \bigcup_{k=1}^n B(x_k, \varepsilon).
  \end{align*}
\end{proof}

\begin{lemma}[Lebesgue's covering lemma]\label{thm:lebesgue_covering_lemma}
  Assume that $X$ is complete. Let $A \subseteq X$ be sequentially compact. Given an open cover $\F \subseteq \PS A$, there exists a number $\delta > 0$ such that every $\delta$-ball with a center in $A$ is contained in some set of the cover $\F$.
\end{lemma}
\begin{proof}
  Assume\LEM\ that no such number $\delta > 0$ exists. Then for any natural number $n \in \ZPos$, there exists an element $x_n \in A$ such that the ball $B(x_n, \frac 1 n)$ is not contained in any set of the cover $\F$. Since $A$ is sequentially compact, the sequence $\{ x_n \}_n$ contains a convergent subsequence $\{ x_{n_k} \}_k$.

  Define
  \begin{align*}
    x \coloneqq \lim_{k \to \infty} x_{n_k}.
  \end{align*}

  Let\AOC\ $E$ be a set in $\F$ that contains $x$. Since $E$ is open, there exists some radius $r > 0$ such that $B(x, r) \subseteq E$.

  Choose any $k_0 > \frac 2 r$ such that $\rho(x_{n_{k_0}}, x) < \frac r 2$. By the triangle inequality,
  \begin{align*}
    B \left(x_{n_k}, \frac 1 k \right) \subsetneq B \left(x_k, \frac r 2 \right) \subseteq B(x, r) \subseteq E,
  \end{align*}
  which contradicts the choice of the sequence $\{ x_n \}_n$.

  Hence there exists a $\delta > 0$ such that for every $x \in A$, the ball $B(x, \delta)$ is contained in some element $E$ of the cover $\F$.
\end{proof}

\begin{theorem}\label{thm:metric_compact_iff_sequentially_compact}
  Assume that $X$ is complete. The set $A \subseteq X$ is compact if and only if it is sequentially compact
\end{theorem}
\begin{proof}
  ($\implies$) Let $\F \subseteq \PS X$ be an open cover of $A$.

  By the Lebesgue covering lemma (\cref{thm:lebesgue_covering_lemma}), there exists $\delta > 0$ such that for every $x \in A$, the ball $B(x, \delta)$ is contained in some set of the cover $\F$. Let $x_1, \ldots, x_n$ be a cover of $A$ with $\delta$-balls.

  For each $k = 1, \ldots, n$ we have that the ball $B(x_k, \delta)$ is contained in some set $E_k \in \F$. Hence $E_1, \ldots, E_n$ is a finite subcover of $A$, because
  \begin{align*}
    A \subseteq \bigcup_{k=1}^\infty B(x_k, \delta) \subseteq \bigcup_{k=1}^\infty E_k.
  \end{align*}

  Thus $A$ is compact.

  ($\impliedby$) Let $A$ be compact. Fix $\varepsilon > 0$ and take the cover
  \begin{align*}
    \F \coloneqq \{ B(a, \varepsilon) \colon a \in A \}.
  \end{align*}

  By compactness of $A$, there exists a finite subcover. Thus a finite cover of $A$ with $\varepsilon$-balls exists for every $\varepsilon > 0$. \Cref{def:totally_bounded_set} then implies that total boundedness is equivalent to sequential compactness because $X$ is complete and $A$ is closed.
\end{proof}

\begin{definition}\label{def:noncompactness_measures}(\cite[definition 7.1]{Deimling1985})
  We define the following functions
  \begin{defenum}
    \item\label{def:noncompactness_measures/sets} The \uline{Kuratowski measure of noncompactness},
    \begin{align*}
      &\alpha: \B \to \RPos \\
      &\alpha(A) \coloneqq \inf \{d > 0 \colon \exists U_1, \ldots, U_n \subseteq X: \Diam {U_k} < d \land A \subseteq \cup_{k=1}^n U_k \}
    \end{align*}

    \item\label{def:noncompactness_measures/balls} The \uline{ball measure of noncompactness},
    \begin{align*}
      &\beta: \B \to \RPos \\
      &\beta(A) \coloneqq \inf \{r > 0 \colon \exists x_1, \ldots, x_2 \in X: A \subseteq \cup_{k=1}^n B(x_k, r) \}
    \end{align*}
  \end{defenum}
\end{definition}
