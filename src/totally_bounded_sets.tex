\subsection{Totally bounded sets}\label{subsec:totally_bounded_sets}

Let \( (X, \rho) \) be a metric space\Tinyref{def:metric_space}. Let \( \Cal{B} \) be the family of bounded sets in \( X \).

\begin{definition}\label{def:epsilon_net}
  We say that \( A \subseteq X \) is an \Def{\( \varepsilon \)-net} for \( B \subseteq X \) if
  \begin{equation}
    B \subseteq \bigcup_{a \in A} B(a, \varepsilon).
  \end{equation}
\end{definition}

\begin{definition}\label{def:totally_bounded_set}
  The space \( A \subseteq X \) is called \Def{totally bounded} if any of the following equivalent conditions hold:

  \begin{defenum}
    \DItem{def:totally_bounded_set/sets} For every \( \varepsilon > 0 \) there exists a finite cover of \( A \) with sets with diameter at most \( \varepsilon \).
    \DItem{def:totally_bounded_set/epsilon_net} For every \( \varepsilon > 0 \) there exists a finite \Def{\( \varepsilon \)-net}\Tinyref{def:epsilon_net} of \( A \).
    \DItem{def:totally_bounded_set/zero_noncompactness/sets} Kuratowski's noncompactness measure\Tinyref{def:noncompactness_measures/sets} \( \alpha(A) \) is zero.
    \DItem{def:totally_bounded_set/zero_noncompactness/balls} The ball noncompactness measure\Tinyref{def:noncompactness_measures/balls} \( \beta(A) \) is zero.
    \DItem{def:totally_bounded_set/fundamental_subsequences} Every sequence in \( A \) admits a fundamental subsequence\Tinyref{def:fundamental_sequence}.
  \end{defenum}

  Totally bounded sets are sometimes called \Def{precompact}\Tinyref{def:compact_set} because of \cref{thm:metric_compact_iff_sequentially_compact}. This equivalence requires the metric space to be complete, however.
\end{definition}
\begin{proof}
  \begin{description}
    \Iff[def:totally_bounded_set/sets][def:totally_bounded_set/zero_noncompactness/sets] Straightforward.

    \Iff[def:totally_bounded_set/epsilon_net][def:totally_bounded_set/zero_noncompactness/balls] Straightforward.

    \Implies[def:totally_bounded_set/epsilon_net][def:totally_bounded_set/sets] Given \( \varepsilon > 0 \), any cover of \( A \) with balls of radius \( \frac \varepsilon 2 \) is a cover with sets of diameter \( \varepsilon \).

    \Implies[def:totally_bounded_set/sets][def:totally_bounded_set/epsilon_net] Fix \( \varepsilon > 0 \) and \( \rho \in (0, \varepsilon) \) and let \( A_1, \ldots, A_n \subseteq \Power X \) be a finite cover of \( A \) with sets of diameter at most \( \rho \).

    Choose\AOC~a point \( x_k \) from every \( A_k \), \( k = 1, \ldots, n \). We then have that for every \( k = 1, \ldots, n \),
    \begin{align*}
      A_k \subseteq \Cl B(x_k, \rho) \subsetneq B(x_k, \varepsilon)
      \\
      \implies A \subseteq \bigcup_{k=1}^n A_k \subseteq \bigcup_{k=1}^n B(x_k, \rho) \subsetneq \bigcup_{k=1}^n B(x_k, \varepsilon),
    \end{align*}
    hence \( x_1, \ldots, x_n \) are centers of \( \varepsilon \)-balls that cover \( A \).

    \Implies[def:totally_bounded_set/epsilon_net][def:totally_bounded_set/fundamental_subsequences] Let \( \{ x_n \} \subseteq A \) be any sequence.

    If we assume\LEM that \( \{ x_n \} \) has no fundamental subsequence, then there exists \( \varepsilon_0 > 0 \) such that \( \rho(x_k, x_m) > \varepsilon_0 \) for any \( n, m \in \Z_{>0} \).

    Consider a finite cover of \( A \) with \( \varepsilon_0 \)-balls. By the pigeonhole principle, at least one of the balls contains more than one element of the sequence, which contradicts the assumption that all elements of the sequence have a distance of at least \( \varepsilon_0 \).

    Hence an arbitrary sequence in \( A \) has a fundamental subsequence.

    \Implies[def:totally_bounded_set/fundamental_subsequences][def:totally_bounded_set/epsilon_net] Assume\LEM that there exists \( \varepsilon_0 > 0 \), such that \( A \) admits no finite cover by \( \varepsilon_0 \)-balls.

    Define \( x_1 \in X, x_2 \in X \setminus B(x_1, \varepsilon_0), \ldots \), so that every two elements of the sequence \( \{ x_n \} \) have a distance of at least \( \varepsilon_0 \). But then the sequence is does not admit a fundamental subsequence, which contradicts our assumption.

    This contradiction proves that \( A \) admits a finite cover by \( \varepsilon \)-balls for every \( \varepsilon > 0 \).
  \end{description}
\end{proof}

\begin{corollary}\label{thm:metric_space_compact_iff_closed_totally_bounded}
  Assume that \( X \) is complete. The set \( A \subseteq X \) is sequentially compact if and only if it is closed and totally bounded.
\end{corollary}
\begin{proof}
  The property that every sequence has a fundamental subsequence is equivalent to sequential compactness for a closed set in a complete metric space.
\end{proof}

\begin{proposition}\label{thm:totally_bounded_sets_are_bounded}
  Totally bounded sets are bounded.
\end{proposition}
\begin{proof}
  Fix a totally bounded set \( A \subseteq X \). Let \( \varepsilon > 0 \) and let \( x_1, x_2, \ldots, x_n \) be a finite \( \varepsilon \)-net\Tinyref{def:totally_bounded_set/epsilon_net} of \( A \). The distance between two points of the \( \varepsilon \)-net is at most \( 2\varepsilon \). Then
  \begin{equation*}
    A \subseteq \bigcup_{i=1}^n B(x_i, \varepsilon) \subseteq B(x_i, 2 n \varepsilon).
  \end{equation*}

  Hence \( A \) is bounded\Tinyref{def:metric_space/bounded_set}.
\end{proof}

\begin{proposition}\label{thm:closure_of_totally_bounded_is_totally_bounded}
  If a set \( A \subseteq X \) is totally bounded, then so is its closure \( \Cl A \).
\end{proposition}
\begin{proof}
  Let \( \varepsilon > 0 \) and \( \rho \in (0, \varepsilon) \) and let \( x_1, \ldots, x_n \in X \) be the centers of a cover of \( A \) with \( \rho \)-balls.

  If \( y \) is a point in \( \Cl A \setminus A \), there exists a point \( z \in A \) with \( \rho(y, z) < \varepsilon - \rho \). Let \( x_k \in A \) be one of the centers whose \( \rho \)-balls contain \( z \). We then have that \( y \in B(x_k, \varepsilon) \) since
  \begin{equation*}
    \rho(x_k, z) \leq \rho(x_k, y) + \rho(y, z) < \rho + \varepsilon - \rho = \varepsilon.
  \end{equation*}

  Hence the balls \( \Cl B(x_k, \varepsilon) \) cover \( \Cl A \), i.e.
  \begin{equation*}
    \Cl A \subseteq \bigcup_{k=1}^n B(x_k, \varepsilon).
  \end{equation*}
\end{proof}

\begin{lemma}[Lebesgue's covering lemma]\label{thm:lebesgue_covering_lemma}
  Assume that \( X \) is complete. Let \( A \subseteq X \) be sequentially compact. Given an open cover \( \Cal{F} \subseteq \Power A \), there exists a number \( \delta > 0 \) such that every \( \delta \)-ball with a center in \( A \) is contained in some set of the cover \( \Cal{F} \).
\end{lemma}
\begin{proof}
  Assume\LEM that no such number \( \delta > 0 \) exists. Then for any natural number \( n \in \Z_{>0} \), there exists an element \( x_n \in A \) such that the ball \( B(x_n, \frac 1 n) \) is not contained in any set of the cover \( \Cal{F} \). Since \( A \) is sequentially compact, the sequence \( \{ x_n \}_n \) contains a convergent subsequence \( \{ x_{n_k} \}_k \).

  Define
  \begin{equation*}
    x \coloneqq \lim_{k \to \infty} x_{n_k}.
  \end{equation*}

  Let\AOC \( E \) be a set in \( \Cal{F} \) that contains \( x \). Since \( E \) is open, there exists some radius \( r > 0 \) such that \( B(x, r) \subseteq E \).

  Choose any \( k_0 > \frac 2 r \) such that \( \rho(x_{n_{k_0}}, x) < \frac r 2 \). By the triangle inequality,
  \begin{equation*}
    B \left(x_{n_k}, \frac 1 k \right) \subsetneq B \left(x_k, \frac r 2 \right) \subseteq B(x, r) \subseteq E,
  \end{equation*}
  which contradicts the choice of the sequence \( \{ x_n \}_n \).

  Hence there exists a \( \delta > 0 \) such that for every \( x \in A \), the ball \( B(x, \delta) \) is contained in some element \( E \) of the cover \( \Cal{F} \).
\end{proof}

\begin{theorem}\label{thm:metric_compact_iff_sequentially_compact}
  Assume that \( X \) is complete. The set \( A \subseteq X \) is compact if and only if it is sequentially compact.
\end{theorem}
\begin{proof}
  \begin{description}
    \Implies Let \( \Cal{F} \subseteq \Power X \) be an open cover of \( A \).

    By the Lebesgue covering lemma (\cref{thm:lebesgue_covering_lemma}), there exists \( \delta > 0 \) such that for every \( x \in A \), the ball \( B(x, \delta) \) is contained in some set of the cover \( \Cal{F} \). Let \( x_1, \ldots, x_n \) be a cover of \( A \) with \( \delta \)-balls.

    For each \( k = 1, \ldots, n \) we have that the ball \( B(x_k, \delta) \) is contained in some set \( E_k \in \Cal{F} \). Hence \( E_1, \ldots, E_n \) is a finite subcover of \( A \), because
    \begin{equation*}
      A \subseteq \bigcup_{k=1}^\infty B(x_k, \delta) \subseteq \bigcup_{k=1}^\infty E_k.
    \end{equation*}

    Thus \( A \) is compact.

    \ImpliedBy Let \( A \) be compact. Fix \( \varepsilon > 0 \) and take the cover
    \begin{equation*}
      \Cal{F} \coloneqq \{ B(a, \varepsilon) \colon a \in A \}.
    \end{equation*}

    By compactness of \( A \), there exists a finite subcover. Thus a finite cover of \( A \) with \( \varepsilon \)-balls exists for every \( \varepsilon > 0 \). \Cref{def:totally_bounded_set} then implies that total boundedness is equivalent to sequential compactness because \( X \) is complete and \( A \) is closed.
  \end{description}
\end{proof}

\begin{corollary}\label{thm:complete_metric_space_compact_conditions}
  The following are equivalent for a set \( A \) in complete metric space:
  \begin{defenum}
    \DItem{thm:complete_metric_space_compact_conditions/compact} \( A \) is compact
    \DItem{thm:complete_metric_space_compact_conditions/sequentially_compact} \( A \) is sequentially compact
    \DItem{thm:complete_metric_space_compact_conditions/closed_totally_bounded} \( A \) is closed and totally bounded
  \end{defenum}
\end{corollary}
\begin{proof}
  \Iff[thm:complete_metric_space_compact_conditions/compact][thm:complete_metric_space_compact_conditions/sequentially_compact] The equivalence is given by \cref{thm:metric_compact_iff_sequentially_compact}.

  \Implies[thm:complete_metric_space_compact_conditions/sequentially_compact][thm:complete_metric_space_compact_conditions/closed_totally_bounded] The equivalence is given by \cref{thm:metric_space_compact_iff_closed_totally_bounded}.
\end{proof}
