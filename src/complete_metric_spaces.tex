\subsection{Complete metric spaces}\label{subsec:metric_convergence}

\begin{Definition}\label{def:complete_metric_space}
  A metric space is said to be \Def{complete} if
  \begin{DefEnum}
    \ILabel{def:complete_metric_space/sequences} Every fundamental sequence converges.
    \ILabel{def:complete_metric_space/uniform} It is complete as a uniform space in the sense of \fullref{def:complete_uniform_space}
  \end{DefEnum}
\end{Definition}
\begin{proof}
  The equivalence is due to \cref{thm:metric_topology_properties/sequential} and \cref{thm:metric_topology_properties/hausdorff}.
\end{proof}

\begin{Proposition}\label{thm:fundamental_sequence_is_bounded}
  In a metric space, any \hyperref[def:fundamental_net]{fundamental sequence} \( \{ x_k \}_{i=1}^n \) is \hyperref[def:metric_space/bounded_sequence]{bounded}.
\end{Proposition}
\begin{proof}
  Since the set
  \begin{equation*}
    I \coloneqq \{ x_k \colon k \leq k_0 \}
  \end{equation*}
  is finite, it has a finite \hyperref[def:metric_space/diameter]{diameter}.

  Fix \( \varepsilon > 0 \). Since the sequence is fundamental, there exists an index \( k_0 \) such that
  \begin{equation*}
    \rho(x_k, x_m) < \varepsilon \quad\forall k, m \geq k_0.
  \end{equation*}

  We are only interested in the case \( \rho(x_{k_0}, x_m) < \varepsilon \).

  Let \( k < k_0 \) and \( m \geq k_0 \). Then
  \begin{equation*}
    \rho(x_k, x_m) \leq \rho(x_k, x_{k_0}) + \rho(x_{k_0}, x_m) < \Diam(I) + \varepsilon,
  \end{equation*}
  which is a finite number.

  Thus the distance between any two elements of the sequence is finite and the sequence is bounded.
\end{proof}

\begin{Proposition}\label{thm:fundamental_subsequence_convergence}
  In any \hyperref[def:complete_metric_space]{metric space}, a \hyperref[def:fundamental_net]{fundamental sequence} converges to a value if and only if it has a subsequence that converges to the same value.
\end{Proposition}
\begin{proof}
  Let \( (X, \rho) \) be a metric space and let \( \{ x_k \}_{k=1}^\infty \) be a fundamental sequence.

  \begin{RefList}
    \ISufficiency Obvious
    \INecessity Assume that the subsequence \( \{ x_{k_n} \}_{n=1}^\infty \) converges to \( x \). Fix \( \varepsilon > 0 \). There exist \( k_0 \) and \( n_0 \) such that
    \begin{align*}
      &\rho(x_k, x_m) < \tfrac \varepsilon 2 \quad\forall k, m \geq k_0
      &\rho(x, x_{k_n}) < \tfrac \varepsilon 2 \quad\forall n \geq n_0.
    \end{align*}

    Fix \( k \geq k_0 \) and let \( n \geq n_0 \) be such that \( k_n \geq k_0 \). Then
    \begin{equation*}
      \rho(x, x_k) \leq \rho(x, x_{k_n}) + \rho(x_{k_n}, x_k) < \varepsilon.
    \end{equation*}

    Since \( \varepsilon \) was arbitrary, we conclude that \( \lim_{k \to \infty} x_k = \lim_{n \to \infty} x_{k_n} = x \).
  \end{RefList}
\end{proof}

\begin{Lemma}\label{thm:metric_space_completion_uniqueness}
  Let \( X \) be a metric space. If both \( f: X \to Y \) and \( g: X \to Z \) are \hyperref[def:complete_metric_space]{completions} of \( X \), then \( Y \) and \( Z \) are isometric.
\end{Lemma}
\begin{proof}
  Let \( y \in Y \) and let \( \{ x_k \}_{k \to \infty} \subseteq X \) be a sequence such that
  \begin{equation*}
    f(x_k) \xrightarrow[k \to \infty]{} y.
  \end{equation*}

  Such a sequence exists since \( f(X) \) is dense in \( Y \).

  Define \( z \coloneqq \lim_{k \to \infty} g(x_k) \). Since both \( f \) and \( g \) are isometries, \( z \) does not depend on the choice of sequence \( \{ x_k \}_{k \to \infty} \) such that \( f(x_k) \to y \). Furthermore, if \( z \in Z \) is given rather than \( y \in Y \), an analogous process allows us to determine \( y \) uniquely based on \( z \).

  Thus we have a bijective isometry between \( Y \) and \( Z \).
\end{proof}

\begin{Theorem}[Metric space completion]\label{thm:metric_space_completion}
  Every metric space has a unique (up to an isometry) \hyperref[def:complete_metric_space]{completion}.

  This is a special case of \fullref{thm:uniform_space_completion} that we prove fully.
\end{Theorem}
\begin{proof}
  Let \( (X, \rho) \) be a metric space. Uniqueness of the completion follows from \fullref{thm:metric_space_completion_uniqueness}. We will only show existence.

  \begin{ThmEnum}
    \ILabel{thm:metric_space_completion/part_a} First, we build the pseudometric space \( (F, \rho) \). We deal with fundamental sequences and isometries in pseudometric spaces, where the definitions, however, does not change.

     Define \( F \) to be the set of all fundamental \hyperref[def:fundamental_net]{sequences} in X. Define the pseudometric
    \begin{align*}
      &\rho: F \times F \to \BR_{\geq 0} \\
      &\rho\left( \{ x_k \}_{k=1}^\infty, \{ y_k \}_{k=1}^\infty \right) \coloneqq \lim_{k \to \infty} \rho(x_k, y_k).
    \end{align*}

    We first show that is well defined as a \hyperref[def:function]{function}. Let \( \{ x_k \}_{k=1}^\infty \) and \( \{ y_k \}_{k=1}^\infty \) be two sequences. Fix \( \varepsilon > 0 \). Then there exists an \( k_0 \) such that
    \begin{equation*}
      \rho(x_k, x_k) < \tfrac \varepsilon 2 \text{ and } \rho(y_k, y_m) <  \quad\forall k, m \geq k_0.tfrac \varepsilon 2.
    \end{equation*}

    Fix \( k, m \geq k_0 \). Then
    \begin{equation*}
      \rho(x_k, y_k) \leq \rho(x_k, x_m) + \rho(x_m, y_m) + \rho(y_m, y_k) < \rho(x_m, y_m) + \varepsilon,
    \end{equation*}
    hence
    \begin{equation*}
      \Abs{\rho(x_k, y_k) - \rho(x_m, y_m)} < \varepsilon.
    \end{equation*}

    Thus the sequence \( \{ \rho(x_k, y_k) \}_{k=1}^\infty \) is fundamental and, by \fullref{def:real_numbers_complete_metric_space}, it is convergent.

    Now we check that \( \rho \) is indeed a pseudometric:
    \begin{RefList}
      \IRef{def:metric_space/pseudometric_identity} For every sequence \( x \in F \),
      \begin{equation*}
        \rho(x, x) = \lim_{k \to \infty} \rho(x_k, x_k) = 0.
      \end{equation*}
      \IRef{def:metric_space/M2} For all sequences \( x, y \in F \),
      \begin{equation*}
        \rho(x, y) = \lim_{k \to \infty} \rho(x_k, y_k) = \lim_{k \to \infty} \rho(y_k, x_k) = \rho(y, x).
      \end{equation*}

      \IRef{def:metric_space/M3} For all sequences \( x, y, z \in F \),
      \begin{equation*}
        \rho(x, z) = \lim_{k \to \infty} \rho(x_k, z_i) \leq \lim_{k \to \infty} \rho(x_k, y_k) + \lim_{k \to \infty} \rho(y_k, z_i) = \rho(x, y) + \rho(y, z).
      \end{equation*}
    \end{RefList}

    \ILabel{thm:metric_space_completion/part_b} We prove that every fundamental sequence in \( (F, \rho) \) is convergent.

    Let \( \{ c^{(k)} \}_{k=1}^\infty \) be a fundamental sequence (of sequences) in \( (F, \rho) \). Thus, for every \( k = 1, 2, \ldots \), there exists an index \( n_k \) such that
    \begin{equation*}
      \rho(c_m^{(k)}, c_{n_k}^{(k)}) < \tfrac 1 k \quad\forall m \geq n_k.
    \end{equation*}

    Define the sequence
    \begin{equation*}
      d_k \coloneqq c_{n_k}^{(k)}, k = 1, 2, \ldots.
    \end{equation*}

    To see that it is fundamental, fix \( \varepsilon > 0 \). Now since the sequence \( \{ c^{(k)} \} \) in \( F \) is fundamental, there exists \( k_0 \) such that
    \begin{equation*}
      \rho(c^{(k)}, c^{(m)}) = \lim_{k \to \infty} \rho(c_k^{(k)}, c_k^{(m)}) < \frac \varepsilon 2 \quad\forall k, m \geq k_0.
    \end{equation*}

    Let \( m_0 \geq k_0 \) be an index such that
    \begin{equation*}
      \frac 2 {m_0} < \frac \varepsilon 2.
    \end{equation*}

    Fix \( k \geq m \geq m_0 \). Let \( l \geq \max \{ n_k, n_m \} \) be such that
    \begin{equation*}
      \rho(c_l^{(k)}, c_l^{(m)}) < \frac \varepsilon 2.
    \end{equation*}

    Then
    \begin{align*}
      \rho(d_k, d_m)
      &=
      \rho(c_{n_k}^{(k)}, c_{n_m}^{(m)})
      \leq \\ &\leq
      \rho(c_{n_k}^{(k)}, c_l^{(k)}) + \rho(c_l^{(k)}, c_l^{(m)}) + \rho(c_l^{(m)}, c_{n_m}^{(m)})
      \leq \\ &\leq
      \frac 1 k + \frac \varepsilon 2 + \frac 1 m
      \leq
      \frac 2 m + \frac \varepsilon 2
      <
      \varepsilon.
    \end{align*}

    Thus we have
    \begin{equation*}
      \rho(d_k, d_m) < \varepsilon \quad\forall k \geq m \geq m_0,
    \end{equation*}
    which proves that the sequence \( \{ d_k \}_{k=1}^\infty \) is fundamental in \( (X, \rho) \).

    Now it remains to show that \( c^{(k)} \xrightarrow[k \to \infty]{} d \) in \( (F, \rho) \).

    Fix \( \varepsilon > 0 \) and let \( k_0 \) be such that
    \begin{equation*}
      \frac 1 {k_0} \leq \frac \varepsilon 2.
    \end{equation*}
    and
    \begin{equation*}
      \rho(d_k, d_m) < \frac \varepsilon 2 \quad\forall k, m \geq k_0.
    \end{equation*}

    Now fix \( i \geq k_0 \). We have, for all \( k \geq i \),
    \begin{align*}
      \rho(c_{n_k}^{(k)}, d_k)
      &=
      \rho(c_{n_k}^{(k)}, c_{n_k}^{(k)})
      \leq \\ &\leq
      \rho(c_{n_k}^{(k)}, c_{n_k}^{(k)}) + \rho(c_{n_k}^{(k)}, c_{n_k}^{(k)})
      = \\ &=
      \rho(c_{n_k}^{(k)}, c_{n_k}^{(k)}) + \rho(d_k, d_k)
      < \\ &<
      \frac 1 k + \frac \varepsilon 2
      <
      \varepsilon.
    \end{align*}

    Hence
    \begin{align*}
      \rho(c^{(k)}, d)
      =
      \lim_{k \to \infty} \rho(c_k^{(k)}, d_k)
      =
      \lim_{k \to \infty} \rho(c_k^{(k)}, c_{n_k}^{(k)})
      <
      \varepsilon.
    \end{align*}

    Thus, given \( \varepsilon > 0 \), we found an index \( k_0 \) such that
    \begin{equation*}
      \rho(c^{(k)}, d) < \varepsilon \quad\forall g \geq k_0.
    \end{equation*}

    Thus \( d = \lim_{k \to \infty} c^{(k)} \) and \( (F, \rho) \) is a complete pseudometric space.

    \ILabel{thm:metric_space_completion/part_c} We construct an isometry of \( (X, \rho) \) into \( (F, \rho) \).

    Define the function
    \begin{align*}
      &\iota: X \to F \\
      &\iota(x) \coloneqq (x, x, x, \ldots),
    \end{align*}
    which sends each element of \( X \) into the corresponding constant sequence in \( F \).

    It is an \hyperref[def:isometry]{isometry} since
    \begin{equation*}
      \rho(\iota(x),\iota(y)) = \lim_{k \to \infty} \rho(x, y) = \rho(x, y).
    \end{equation*}

    \ILabel{thm:metric_space_completion/part_d} We show that the image \( \iota(X) \) is dense in \( (F, \rho) \).

    Fix the fundamental sequence \( y \coloneqq \{ y_k \}_{k=1}^\infty \). Define the sequence \( x \) of sequences
    \begin{equation*}
      x^{(k)} \coloneqq \iota(y_k), i = 1, 2, \ldots.
    \end{equation*}

    It is fundamental in \( (F, \rho) \) since \( e \) is an isometry and since \( y \) is fundamental in \( (X, \rho) \).

    Fix \( \varepsilon > 0 \). Let \( k_0 \) be such that
    \begin{equation*}
      \rho(y_k, y_m) < \varepsilon \quad\forall k, m \geq k_0.
    \end{equation*}

    For \( i, k \geq k_0 \), we have
    \begin{align*}
      \rho(x_k^{(k)}, y_k)
      \leq
      \rho(x_k^{(k)}, y_k) + \rho(y_k, y_k)
      =
      0 + \rho(y_k, y_k)
      <
      \varepsilon,
    \end{align*}
    hence
    \begin{equation*}
      \rho(x^{(k)}, y) = \lim_{k \to \infty} \rho(x_k^{(k)}, y_k) < \varepsilon.
    \end{equation*}

    We conclude that \( x^{(k)} \xrightarrow[k \to \infty]{} y \) in \( (F, \rho) \), which implies that \( e(X) \) is dense in \( (F, \rho) \).

    \ILabel{thm:metric_space_completion/part_e} We build a complete metric space \( (C, \nu) \) from \( (F, \rho) \).

    We use \fullref{thm:pseudometric_to_metric} to construct a complete metric space \( (C, \nu) \) from the complete pseudometric space \( (F, \rho) \).

    We adapt \( \iota \) to the equivalence classes on \( C \):
    \begin{align*}
      &\hat\iota: X \to C \\
      &\hat\iota(x) \coloneqq [\iota(x)].
    \end{align*}

    Thus \( \hat\iota \) embeds \( X \) into the complete metric space \( C \).
  \end{ThmEnum}
\end{proof}

\begin{Theorem}[Cantor's nested compact theorem]\label{thm:cantors_nested_compact_theorem}
  A descending sequence of nonempty compact sets \( F_1 \supseteq F_2 \supseteq \ldots \) in a complete metric space such that \( \Diam(F_i) \to 0 \) intersects at exactly one point (compare with \fullref{thm:noncompact_kuratowskis_lemma}).
\end{Theorem}
\begin{proof}
  Choose\AOC an element \( x_k \in F_k \) for any \( i = 1, 2, \ldots \). Then the sequence \( \{ x_k \}_{k=1}^\infty \) is fundamental. To see this, let \( \varepsilon > 0 \) and let \( k_0 \) be an index such that \( \Diam(F_{k_0}) < \varepsilon \). Then if \( j \geq i \geq k_0 \), \( x_m \) is contained in \( F_k \) and \( \rho(x_k, x_m) < \varepsilon \). Thus the sequence is indeed fundamental and, since the space is complete, it has a limit point \( x \).

  The point \( x \) is contained in every set \( F_k, i = 1, 2, \ldots \) since all of the sets \( F_k \) are closed (by \fullref{thm:complete_metric_space_compact_conditions}) and contain their limit \hyperref[thm:limit_point_iff_in_closure]{points}. Thus
  \begin{equation*}
    x \in \bigcap_{k=1}^\infty F_k.
  \end{equation*}

  Furthermore,
  \begin{equation*}
    \Diam\left( \bigcap_{k=1}^\infty F_k \right) = 0,
  \end{equation*}
  hence \( x \) is the only point in the intersection.
\end{proof}
