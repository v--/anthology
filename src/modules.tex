\begin{definition}\label{def:module}\cite{Knapp2016BAlg}[374]
  Let $R$ be a commutative ring (if $R$ is not commutative, we can separately define \uline{left $R$-modules} and \uline{right $R$-modules} in the obvious way). An $R$-module $M$ is an (additive) abelian group $(M, +)$ along with an operation $\circ: R \times M \to M$, called \uline{scalar multiplication} and denoted by juxtaposition, such that for all $u, v \in M$ and all $a, b \in R$,
  \begin{description}
    \DItem{Associativity}{def:module/associativity} $a (b u) = (a b) u$.
    \DItem{Scalar distributivity}{def:module/scalar_distributivity} $(a + b) u = a u + b u$.
    \DItem{Vector distributivity}{def:module/vector_distributivity} $a (u + v) = a u + a v$.
    \DItem{Identity}{def:module/identity} If the ring $R$ has an identity, then we require the additional axiom $1_R u = u$.
  \end{description}

  In analogy with linear algebra, we call elements of $R$ scalars and elements of $M$ vectors.

  We say that
  \begin{itemize}
    \item the subset $N \subseteq M$ is an \uline{$R$-submodule of $M$} if $N$ is closed under the module operations.
    \item the module $\{ 0_R \}$ is the \uline{zero $R$-module} or \uline{trivial $R$-module} or the \uline{trivial submodule} since it is a submodule of every $R$-module.
    \item all submodules except for $M$ itself are \uline{proper submodules}.
  \end{itemize}
\end{definition}

\begin{example}\label{ex:module/ideal}
  Every commutative ring $R$ is a module over itself. Every ideal $I \unlhd R$ is an $R$-module since it is closed under multiplication with \enquote{scalars} from $R$.
\end{example}

\begin{definition}\label{def:module_homomorphism}\cite{Knapp2016BAlg}[375]
  Let $M$ and $N$ be two $R$-modules. We say that the function $f: M \to N$ is a module homomorphism if it is linear, that is,
  \begin{description}
    \DItem{Homogeneity}{def:module_homomorphism/homogeneity} $a f(u) = f(a u)$ for any $a \in R$ and $u \in M$.
    \DItem{Additivity}{def:module_homomorphism/additivity} $f(a + b) = f(a) + f(b)$ for any (that is, $f$ is a group homomorphism between $(M, +)$ and $(N, +)$).
  \end{description}

  The \uline{kernel} of $f$ is defined as the preimage\Tinyref{def:function_invertibility} $f^{-1}(0_M)$.

  The terminology from~\cref{def:morphism_invertibility} applies to module homomorphisms because of the category $\Bold{RMod}$ of $R$-modules.
\end{definition}

\begin{definition}\label{def:abelian_group_z_module}\cite{Knapp2016BAlg}[375]
  Let $G$ be an abelian group. Define the \uline{$\BB{Z}$-module $M$, associated with $G$}, with scalar multiplication
  \begin{align*}
    nu \coloneqq \begin{cases}
      0, &n = 0 \\
      u + \ldots + u, &n > 0 \\
      -((-n)u), &n < 0.
    \end{cases}
  \end{align*}

  Thus, abelian groups can be regarded as modules.
\end{definition}

\begin{theorem}\label{thm:abelian_group_iff_z_module}\cite{Knapp2016BAlg}[375]
  Every abelian group is isomorphic to exactly one $\BB{Z}$-module.
\end{theorem}
\begin{proof}
  We already saw in~\cref{def:abelian_group_z_module} how every abelian group can be regarded as a $\BB{Z}$-module. Every $\BB{Z}$-module can then be identified with its additive group.

  Scalar multiplication ensures that there is exactly one way to define a $\BB{Z}$-module structure on an abelian group since $na = (n-1)a + a$ and $0a = 0$.
\end{proof}
