\begin{definition}\label{def:functor}\cite[definitions 1.2.1, 1.2.10]{Leinster2014}
  Let $\Bold A$ and $\Bold B$ be categories. A \uline{(covariant) functor} $F: \Bold A \to \Bold B$ consists of:
  \begin{itemize}
    \item a function $\Bold A \to \Bold B$, written as $A \mapsto F(A)$.
    \item for each $A, A' \in \Bold A$, a function
    \begin{align*}
      \Bold{A}(A, A') \to \Bold{B}(F(A), F(A')),
    \end{align*}
    written as $f \mapsto F(f)$.
  \end{itemize}
  such that
  \begin{defenum}
    \item\label{def:functor/composition_axiom} $A \overset f \mapsto B \overset g \mapsto C$ implies $F(g \circ f) = F(g) \circ F(f)$.
    \item\label{def:functor/identity_axiom} $A \in \Bold A$ implies $F(\Id_A) = \Id_{F(A)}$.
  \end{defenum}

  If we replace the axiom~\cref{def:functor/composition_axiom} with
  \begin{defenum}
    \item[b')]\label{def:functor/contravariant_composition_axiom} $A \overset f \mapsto B \overset g \mapsto C$ implies $F(g \circ f) = F(f) \circ F(g)$,
  \end{defenum}
  we call $F$ a \uline{contravariant functor}. Equivalently, $F: \Bold A \to \Bold B$ is contravariant if and only if $F: \Bold{A}^{\Op} \to \Bold B$ is covariant.

  The \uline{identity functor} $\Id_A: \Bold A \to \Bold A$ simply maps a category to itself.
\end{definition}

\begin{note}\label{note:image_of_functor_maybe_not_subcategory}
  The \uline{image} $F(\Bold A)$ of a category $\Bold A$ under a functor $F: \Bold A \to \Bold B$ may not be a subcategory of $\Bold B$. A simple example is can be constructed as follows:

  Let $\Bold A$ be a category with four objects $A, B, C, D$ and two morphisms $f: A \to B$ and $g: C \to D$. If $F(B) = F(C)$, then $F(f): F(A) \to F(B)$ and $F(g): F(B) \to F(D)$, however there is no morphism from $F(A)$ to $F(D)$. Thus the image $F(\Bold A)$ is not itself category.
\end{note}

\begin{definition}\label{def:categorical_diagram}
  A generalization of set-indexed families (see \cref{def:indexed_family}) is given by diagrams. We fix a category $\Bold I$, called an \uline{index category}, which is often assumed to be small. A \uline{diagram of shape $\Bold I$} is then any functor $D: \Bold I \to \Bold A$, where $\Bold A$ is any other category.

  It is often convenient to think of diagrams in terms of their images $D(\Bold I)$, which are selections of objects and morphisms in $\Bold A$. Note the image $D(\Bold I)$ may not be a subcategory of $\Bold A$ (see \cref{note:image_of_functor_maybe_not_subcategory}).

  If the category $\Bold I$ is small, we say that the diagram is a \uline{small diagram}.
\end{definition}

\begin{example}\label{ex:categorical_diagrams}
  \mbox{}
  \begin{defenum}
    \item In the case when $\Bold I$ is a small discrete category, a diagram $D: \Bold I \to \Bold A$ is simply a mapping of each element $i$ of $\Bold I$ into an element of $\Bold A$, i.e. we can interpret any diagram of shape $\Bold I$ as a set-indexed family $\{ A_i \}_{i \in I}$, where all $A_i$ are objects in $\Bold A$.

    \item If $\Bold I$ is not discrete, a diagram $D: \Bold I \to \Bold A$ also involves morphisms. For example, if $\Bold I$ is a three-object category with two morphisms as in the following picture
    \begin{center}
      \begin{tikzcd}
        \bullet \arrow[r] & \bullet \arrow[r] & \bullet,
      \end{tikzcd}
    \end{center}
    we can interpret a diagram $D$ of shape $\Bold I$ as a selection of objects and morphisms in $\Bold A$ that satisfy the same relations as in $\Bold I$:
    \begin{center}
      \begin{tikzcd}
        A \arrow[r, "f"] & B \arrow[r, "g"] & C.
      \end{tikzcd}
    \end{center}
  \end{defenum}
\end{example}

\begin{definition}\label{def:commutative_diagram}
  A diagram $D$ is said to be \uline{commutative} if, whenever we have two chains of morphisms $X \overset {f_1} \to A_1 \overset {f_2} \to \ldots \overset {f_{n-1}} \to A_{n-1} \overset {f_n} \to Y$ and $X \overset {g_1} \to B_1 \overset {g_2} \to \ldots \overset {g_{m-1}} \to B_{m-1} \overset {f_m} \to Y$ in the diagram, where $n > 0$ and $m > 0$, then necessarily
  \begin{align*}
    f_n \circ \ldots \circ f_1 = g_m \circ \ldots \circ g_1.
  \end{align*}

  We also say that the diagram $D$ \uline{commutes}.
\end{definition}

\begin{example}\label{ex:commutative_diagrams}
  Consider the diagram
  \begin{center}
    \begin{tikzcd}
                         & A \arrow[ld, "f"'] \arrow[rd, "g"] & \\
      B \arrow[rr, "h"'] &                                    & C
    \end{tikzcd}
  \end{center}

  It is commutative if and only if $h \circ f = g$.

  For a more convoluted example, see \cref{def:categorical_pullback}.
\end{example}

\begin{definition}\label{def:opposite_functor}\cite[definition 5.2.1]{Leinster2014}
  Given a functor $F: \Bold A \to \Bold B$, we define \uline{opposite or dual functor} $F^{\Op}: \Bold{A}^{\Op} \to \Bold{B}^{\Op}$ as
  \begin{itemize}
    \item $F^{\Op}(A) = F(A)$
    \item $F^{\Op}(f^{\Op}: A' \to A) = F(f: A \to A')$
  \end{itemize}
\end{definition}

\begin{proposition}\label{thm:functors_preserve_isomorphisms}\cite[exercise 1.2.21]{Leinster2014}
  Functors preserve isomorphisms, i.e. if $F: \Bold A \to \Bold B$ is a (covariant) functor and $A \cong A'$, then $F(A) \cong F(A')$.
\end{proposition}
\begin{proof}
  Let $f: A \to A'$ be an isomorphism with inverse $f^{-1}$. From~\cref{def:functor}, we have
  \begin{align*}
    F(f^{-1}) \circ F(f)
    \overset{\ref{def:functor/composition_axiom}} =
    F(f^{-1} \circ f)
    =
    F(\Id_A)
    \overset{\ref{def:functor/identity_axiom}} =
    \Id_{F(A)}.
  \end{align*}

  Analogously, $F(f) \circ F(f^{-1}) = \Id_{F(A')}$. Thus $F(f): F(A) \to F(A')$ is an isomorphism with inverse $F(f^{-1})$.
\end{proof}

\begin{note}\label{note:forgetful_free_functor}\cite[examples 1.2.3, 1.2.4]{Leinster2014}
  An informal notion is that of the \uline{forgetful functor}. A functor $F: \Bold A \to \Bold B$ is called forgetful if the images $F(A)$ of objects $A \in \Bold A$ have \enquote{less structure} than $A$. For example, a functor which maps topological spaces to their underlying sets is forgetful since it \enquote{forgets} about the topological structure.

  A dual informal notion is that of a \uline{free functor}. In contrast to forgetful functors which \enquote{remove structure}, free functors \enquote{add structure}. For example, a functor which maps a set to its corresponding discrete topological space is a free functor.
\end{note}

\begin{definition}\label{def:presheaf}\cite[definition 1.2.15]{Leinster2014}
  A \uline{presheaf} on the category $\Bold A$ is a contravariant functor
  \begin{align*}
    F: \Bold A \to \Bold{Set}.
  \end{align*}
\end{definition}

\begin{example}\label{ex:topological_space_presheaf}\cite[24]{Leinster2014}
  Let $(X, \tau)$ be a topological space. Form the category $\Bold C$ from the poset $(\tau, \subseteq)$ as in \cref{def:standard_categories/ord}. Presheaves on $\Bold C$ are also called presheaves on the topological space $(X, \tau)$.

  Let $(Y, \rho)$ be another topological space. Then the map
  \begin{align*}
    &F: \tau \MultTo C(\tau, Y) \\
    &F(U) = C(U, Y) = \{ f: U \mapsto Y, f \text{ is continuous} \}
  \end{align*}
  is a presheaf.
\end{example}

\begin{definition}\label{def:faithful_full_functors}\cite[definition 1.2.16]{Leinster2014}
  A functor $F: \Bold A \to \Bold B$ is called \uline{faithful} (resp. \uline{full}) if the map
  \begin{align*}
    \Bold{A}(A, A) &\to \Bold{B}(F(A), F(A')) \\
    f &\mapsto F(f)
  \end{align*}
  is injective (resp. surjective) (see \cref{def:function_invertability}).
\end{definition}

\begin{example}\label{def:subcategory_functors}\cite[25]{Leinster2014}
  Let $\Bold B$ be a subcategory of $\Bold A$. We define the inclusion functor $I: \Bold B \to \Bold A$ by sending each object and each morphism of $\Bold B$ to itself within $\Bold A$.

  Then $I$ is faithful and, if the subcategory $\Bold B$ is full (see \cref{def:subcategory}), then $I$ is also full.
\end{example}

\begin{definition}\label{def:natural_transformation}\cite[definition 1.3.1]{Leinster2014}
  Let $\Bold A$ and $\Bold B$ be categories and let $F$ and $G$ be functors from $\Bold A$ to $\Bold B$.

  A \uline{natural transformation} $\alpha: F \to G$ is a family $\{ \alpha_A: F(A) \to G(A) \}_{A \in \Bold A}$ of morphisms in $\Bold B$ such that for every morphism $f: A \to A'$ in $\Bold A$, the diagram
  \begin{center}
    \begin{tikzcd}
      F(A) \arrow[r, "F(f)"] \arrow[d, "\alpha_A"] & F(A') \arrow[d, "\alpha_{A'}"] \\
      G(A) \arrow[r, "G(f)"]                       & G(A')
    \end{tikzcd}
  \end{center}
  commutes.

  The morphisms $\alpha_A$ are called the components of $\alpha$. We denote natural transformations using
  \begin{center}
    \begin{tikzcd}[column sep=huge]
      \Bold A
        \arrow[r, bend left, "F"]{}[name=F]{}
        \arrow[r, bend right, "G"']{}[name=G]{} &
      \Bold B
        \arrow[shorten <= 0.5em, Rightarrow,to path={(F) -- node[label=right:$\alpha$] {} (G)}]{}
    \end{tikzcd}
  \end{center}

  The natural transformation from $F$ to $F$ composed of identity morphisms is called the \uline{identity natural transformation}.
\end{definition}

\begin{definition}\label{def:natural_transformation_composition}
  Let $F: \Bold A \to \Bold B$, $G: \Bold A \to \Bold B$ and $H: \Bold A \to \Bold B$ be functors and let $\alpha: F \to G$ and $\beta: G \to H$ be natural transformations.

  We define the \uline{composition} (sometimes called \uline{vertical composition}) of the natural transformations $\beta$ and $\alpha$ component-wise for $A \in \Bold A$ as
  \begin{align*}
    (\beta \circ \alpha)_A \coloneqq \beta_{A} \circ \alpha_A.
  \end{align*}
\end{definition}

\begin{definition}\label{def:functor_category}
  Given categories $\Bold A$ and $\Bold B$, we define their \uline{functor category} $[\Bold A, \Bold B]$ by
  \begin{itemize}
    \item the objects in $[\Bold A, \Bold B]$ are functors $F: \Bold A \to \Bold B$.
    \item the morphisms in $[\Bold A, \Bold B](F, G)$ are the natural transformations from $F$ to $G$.
  \end{itemize}

  The functor category $[\Bold A, \Bold B]$ is often denoted by ${\Bold B}^{\Bold A}$ since, if $\Bold A$ is a finite discrete category of cardinality $n$, it is equivalent (see \cref{def:category_equivalence}) to the product category (see \cref{def:product_category}) $\Bold B \times \Bold B$
  \begin{align*}
    {\Bold B}^{\Bold A} = {\Bold B}^n = \Bold B \times \ldots \times \Bold B.
  \end{align*}

  If the natural transformation $\alpha$ is an isomorphism in $[\Bold A, \Bold B]$, we say that the categories $\Bold A$ and $\Bold B$ are \uline{naturally isomorphic} and write $\Bold A \cong \Bold B$.
\end{definition}

\begin{definition}\label{def:category_equivalence}\cite[definition 1.3.15]{Leinster2014}
  An \uline{equivalence} between the categories $\Bold A$ and $\Bold B$ consists of a pair of functors $F, G: \Bold A \to \Bold B$ and a pair of natural isomorphisms
  \begin{align*}
    \xi: \Id_{\Bold A} \to G \circ F,
    &&
    \eta: F \circ G \to \Id_{\Bold B}.
  \end{align*}

  If an equivalence between $\Bold A$ and $\Bold B$ exists, we say that \uline{the categories $\Bold A$ and $\Bold B$ are equivalent} and write $\Bold A \simeq \Bold B$.

  An equivalence of the form $\Bold{A}^{\Op} \simeq \Bold{B}$ is called a \uline{duality} between $\Bold A$ and $\Bold B$ and we say that \uline{$\Bold A$ is dual to $\Bold B$} \cite[example 1.3.22]{Leinster2014}.
\end{definition}

\begin{definition}\label{def:natural_transformation_horizontal_composition}\cite[remarks 1.3.24]{Leinster2014}
  Let $\Bold A$, $\Bold B$ and $\Bold C$ be categories, $F, G: \Bold A \to \Bold B$ and $F', G': \Bold B \to \Bold C$ be functors and $\alpha: F \to G$ and $\alpha': F' \to G'$ be natural transformations.
  \begin{center}
    \begin{tikzcd}[column sep=huge]
      \Bold A
        \arrow[r, bend left, "F"]{}[name=F]{}
        \arrow[r, bend right, "G"']{}[name=G]{} &
      \Bold B
        \arrow[shorten <= 0.5em, Rightarrow,to path={(F) -- node[label=right:$\alpha$] {} (G)}]{}
        \arrow[r, bend left, "F'"]{}[name=F']{}
        \arrow[r, bend right, "G'"']{}[name=G']{} &
      \Bold C
        \arrow[shorten <= 0.5em, Rightarrow,to path={(F') -- node[label=right:$\alpha'$] {} (G')}]{}
    \end{tikzcd}
  \end{center}

  We define the natural transformation
  \begin{align*}
    \alpha' * \alpha: F' \circ F \to G' \circ G,
  \end{align*}
  called \uline{horizontal composition of $\alpha$ and $\alpha'$}, defined by
  \begin{align*}
    (\alpha' * \alpha)_A \coloneqq \alpha'_{G(A)} \circ F'(\alpha_A) = G'(\alpha_A) \circ \alpha'_{F(A)}.
  \end{align*}
\end{definition}

\begin{note}
  We restrict our attention to locally small categories because we need to define an isomorphism of morphism sets.
\end{note}
\begin{definition}\label{def:adjoint_functor}\cite[definition 2.1.1]{Leinster2014}
  Let $\Bold A$ and $\Bold B$ be locally small categories and $F: \Bold A \to \Bold B$ and $G: \Bold B \to \Bold A$ be functors. Further assume that for every $A \in \Bold A$ and $B \in \Bold B$ we have an isomorphism
  \begin{align*}
    \Bold{A}(A, G(B)) \overset {\varphi_{A, B}} {\cong} \Bold{B}(F(A), B),
  \end{align*}
  where $\Bold{A}(A, G(B))$ and $\Bold{B}(F(A), B)$ are regarded as objects in $\Bold{Set}$.

  Given a morphism $f: A \to G(B)$, we define the \uline{transpose $\Ol f$ of $f$} as
  \begin{align*}
    &\Ol f: F(A) \to B \\
    &\Ol f \coloneqq \varphi_{A, B} (f).
  \end{align*}

  Dually, given a morphism $g: F(A) \to B$, we define
  \begin{align*}
    &\Ol g: A \to G(B) \\
    &\Ol g \coloneqq \varphi_{A, B}^{-1} (g).
  \end{align*}

  We say that the isomorphism $\varphi_{A, B}$ is \uline{natural} if,  given $A' \in \Bold A$ and morphisms $f: A \to G(B)$ and $p: A' \to A$, we have
  \begin{align*}
    \Ol{f \circ p} = \Ol f \circ F(p),
  \end{align*}
  and, given $B' \in \Bold B$ and morphisms $g: F(A) \to B$ and $q: B \to B'$, we have
  \begin{align*}
    \Ol{q \circ g} = G(q) \circ \Ol g.
  \end{align*}

  In this case, we say that $F$ is \uline{left-adjoint} to $G$ and $G$ is \uline{right-adjoint} to $F$, and write $F \dashv G$.
\end{definition}

\begin{example}\label{ex:top_adjoint_functor}\cite[example 2.1.5]{Leinster2014}
  Consider the functors
  \begin{itemize}
    \item $U: \Bold{Top} \to \Bold{Set}$, which maps topological spaces to their underlying sets.
    \item $D: \Bold{Set} \to \Bold{Top}$, which maps sets to topological spaces equipped with the discrete topology.
    \item $I: \Bold{Set} \to \Bold{Top}$, which maps sets to topological spaces equipped with the indiscrete topology.
  \end{itemize}

  Let $T \in \Bold{Top}$ and $S \in \Bold{Set}$.

  Let $f: T \to I(S)$ be any continuous function and $g: U(T) \to S$ be any function.

  Denote by $\Ol f: U(T) \to S$ the function between sets, corresponding to $f$ and by $\Ol g: T \to I(S)$ the corresponding function between the topological spaces $T$ and $I(S)$. Since any function into an indiscrete topological space is $T$ is continuous, we have that $\Ol g$ is a morphism $T \to I(S)$.

  Thus $\Ol{\Ol f} = f$ and $\Ol{\Ol g} = g$ and we have a natural isomorphism between $\Bold{Set}(U(T), S)$ and $\Bold{Top}(T, I(S))$. This proves that $U \dashv I$.

  Similarly, since any function from a discrete space is continuous, we have that $D \dashv U$.

  Hence $D \dashv U \dashv I$.
\end{example}

\begin{definition}\label{def:comma_category}\cite[definition 2.3.1]{Leinster2014}
  Let $\Bold A$, $\Bold B$ and $\Bold C$ be categories and $\Bold A \overset F \to \Bold C \overset G \leftarrow \Bold B$. We define the \uline{comma category} $(F \downarrow G)$ by
  \begin{itemize}
    \item The objects in $(F \downarrow G)$ are triples $(A, h, B)$ where $A \in \Bold A$, $B \in \Bold B$ and $F(A) \overset h \to G(B)$.
    \item The morphisms from $(A, h, B)$ to $(A', h', B')$ are pairs $(f, g) \in \Bold{A}(A, A') \times \Bold{B}(B, B')$ such that the following diagram commutes:
    \begin{equation}\label{def:comma_category/universal_property}
      \begin{tikzcd}[baseline=(current bounding box.center)]
        F(A) \arrow[r, "F(f)"] \arrow[d, "h"'] & F(A') \arrow[d, "h'"] \\
        G(B) \arrow[r, "G(g)"']                & G(B')
      \end{tikzcd}
    \end{equation}
  \end{itemize}

  As a special case, if $\Bold A$ is the one-object category, then $F$ necessarily \enquote{selects} an object $C \in \Bold C$. Thus, we can define the comma category $(C \downarrow G)$, in which objects may be regarded as pairs $(h, B)$ rather than triples and the diagram for morphisms looks like
  \begin{center}
    \begin{tikzcd}
                               & C \arrow[ld, "h"'] \arrow[rd, "h'"] & \\
      G(B) \arrow[rr, "G(g)"'] &                                     & G(B')
    \end{tikzcd}
  \end{center}

  Analogously, we can also define the category $(F \downarrow C)$ by regarding $G$ and not $F$ as a functor from the one-object category.
\end{definition}

\begin{definition}\label{def:representable_functor}\cite[definitions 4.1.3, 4.1.16]{Leinster2014}
  Let $\Bold A$ be a locally small category and $A \in \Bold A$. Define
  \begin{align*}
    &H^A: \Bold{A} \to \Bold{Set}, \\
    &H^A(B) \coloneqq \Bold{A}(A, B), \\
    &H^A(f: B \to C) \coloneqq (p: A \to B) \mapsto (f \circ p: A \to C).
  \end{align*}

  We say that the functor $F: \Bold{A} \to \Bold{Set}$ is \uline{representable} with \uline{representation} $H^A$ if $F \cong H^A$.

  Analogously, the presheaf $G: \Bold{A}^{\Op} \to \Bold{Set}$ is representable if for some $A \in \Bold{A}^{\Op}$ we have $G \cong H_A$, where
  \begin{align*}
    &H_A: \Bold{A}^{\Op} \to \Bold{Set}, \\
    &H_A(B) \coloneqq \Bold{A}(B, A), \\
    &H_A(f: C \to B) \coloneqq (p: B \to A) \mapsto (p \circ f: C \to A).
  \end{align*}
\end{definition}

\begin{example}\label{def:top_representable_functor}\cite[example 4.1.4]{Leinster2014}
  Let $U: \Bold{Top} \to \Bold{Set}$ be the forgetful functor which maps a topological space to its underlying set.

  Let $1$ be the one-element topological space. There is a correspondence between points $x$ in $T$ and continuous functions $p_x: 1 \to T$. Thus the functor $H^1$ maps
  \begin{itemize}
    \item any topological space $T$ into the set of morphisms
    \begin{align*}
      H^1(T) = \Bold{Top}(1, T) = \{ p_x: 1 \to T \} \cong U(T).
    \end{align*}
    \item any continuous function $f: T \to S$ to
    \begin{align*}
      H^1(f) = p_x \mapsto f \circ p_x \cong x \mapsto f(x) = f.
    \end{align*}
  \end{itemize}

  Thus $U$ is representable with representation $H^1$.
\end{example}

\begin{definition}\label{def:yoneda_embedding}\cite[definitions 4.1.15, 4.1.21]{Leinster2014}
  Let $\Bold A$ be a locally small category. For each pair $A, B \in \Bold A$ and morphism $f: A \to B$ we define the natural transformation $H^f: H^A \to H^B$ with $C$-components (note the reversal)
  \begin{align*}
    &H^f(C): H^A(C) \to H^B(C), \\
    &H^f(C) \coloneqq H_C(f) = p \mapsto p \circ f.
  \end{align*}

  Thus allows us to define the functor $H^\bullet: \Bold{A}^{\Op} \to [\Bold{A}, \Bold{Set}]$ by
  \begin{align*}
    H^\bullet(A) \coloneqq H^A && H^\bullet(f) \coloneqq H^f.
  \end{align*}

  Analogously, we define $H_f$ by $H_f(C) = H^C(f), C \in \Bold A$ and the \uline{Yoneda embedding} $H_\bullet: \Bold{A} \to [\Bold{A}^{\Op}, \Bold{Set}]$ by
  \begin{align*}
    H_\bullet(A) \coloneqq H_A && H_\bullet(f) \coloneqq H_f.
  \end{align*}
\end{definition}

\begin{proposition}\label{def:yoneda_embedding_is_injective}\cite[exercise 4.1.27]{Leinster2014}
  Let $\Bold A$ be a locally small category and let $A, A' \in \Bold A$ be such that $H_A \cong H_{A'}$. Then $A \cong A'$.
\end{proposition}
\begin{proof}
  First, let $A$ and $A'$ be arbitrary. Given a natural isomorphism $\eta: H_A \to H_{A'}$, its components are $\alpha_B: H_A(B) \to H_{A'}(B)$.

  We are interested in the morphisms
  \begin{align*}
    &f \coloneqq \alpha_A(\Id_A): A \to A', \\
    &g \coloneqq \alpha_{A'}^{-1}(\Id_A'): A' \to A.
  \end{align*}

  We need to show that $g$ is inverse to $f$. We will use the commutativity of the following diagram:
  \begin{center}
    \begin{tikzcd}
      H_A(A) \arrow[d, "\alpha_A"'] & H_A(A') \arrow[d, "\alpha_{A'}"] \arrow[l, "H_A(f)"'] \\
      H_{A'}(A)                     & H_{A'}(A') \arrow[l, "H_{A'}(f)"],
    \end{tikzcd}
  \end{center}
  where
  \begin{align*}
    &H_A(f: A \to A') = (p: A' \to A) \mapsto (p \circ f: A \to A), \\
    &H_{A'}(f: A \to A') = (p: A' \to A') \mapsto (p \circ f: A \to A').
  \end{align*}

  In particular,
  \begin{center}
    \begin{tikzcd}
      g \circ f \arrow[d, "\alpha_A"']              & g \arrow[d, "\alpha_{A'}"] \arrow[l, "H_A(f)"'] \\
      \alpha_A(g \circ f) = H_{A'}(f)(\Id_{A'}) = f & \alpha_{A'}(\alpha_{A'}^{-1}(\Id_{A'})) = \Id_{A'} \arrow[l, "H_{A'}(f)"],
    \end{tikzcd}
  \end{center}
  i.e.
  \begin{align*}
    \alpha_A(g \circ f) = f = \alpha_A(\Id_A).
  \end{align*}

  Since $\alpha_A$ is a bijection, we conclude that $g \circ f = \Id_A$.

  Analogously, we obtain that $f \circ g = \Id_{A'}$. Thus $f: A \to A'$ is an isomorphism, the inverse being $g: A' \to A$.
\end{proof}

\begin{theorem}(Yoneda's lemma)\label{def:yoneda_lemma}\cite[theorem 4.2.1]{Leinster2014}
  Let $\Bold A$ be a locally small category. Then there is a natural isomorphism between the functors
  \begin{align*}
    &\Bold{A}^{\Op} \times [\Bold{A}^{\Op}, \Bold{Set}] \to \Bold{Set} \\
    &(A, X) \mapsto X(A)
  \end{align*}
  and
  \begin{align*}
    &\Bold{A}^{\Op} \times [\Bold{A}^{\Op}, \Bold{Set}] \to \Bold{Set} \\
    &(A, X) \mapsto [\Bold{A}^{\Op}, \Bold{Set}](H_A, X).
  \end{align*}
\end{theorem}
