\section{Set theory}\label{sec:set_theory}
\subsection{Sets}\label{subsec:sets}

\begin{definition}\label{def:set_naive}\MarginCite[chapter 1]{Enderton1977}
  Naive set theory is not based on a strict axiom set but rather on the intuitive notion of a set as an unordered collection without repetition. Set equality \( A = B \), set membership \( x \in A \) and \hyperref[def:subset]{set inclusion} \( A \subseteq B \) are assumed to be understood. Sets can be explicitly constructed by specifying their elements, e.g.
  \begin{equation*}
    \{ 3, 7, 31, 127, 8191 \}
  \end{equation*}
  or by specifying a unary logical \hyperref[def:first_order_formula]{formula} \( \varphi(x) \) in an implicitly assumed logical language:
  \begin{equation*}
    \{ a \colon \varphi(a) \}
  \end{equation*}

  If \( \varphi(x) = x \in A \land \psi(x) \), we often write
  \begin{equation*}
    \{ x \in A \colon \psi(x) \}.
  \end{equation*}

  In a suitable context, the definitions can be made precise. For example, in the ring of integers \( \BZ \) with equality, addition, multiplication and predicates partial ordering \( \leq \) and divisibility \( \vert \), each set can be thought of an equivalence class of formulas in the corresponding \hyperref[def:first_order_logic_language]{first-order logic language}. Given formulas \( \varphi_A \) and \( \varphi_B \) with a free variable \( x \) and sets
  \begin{BreakableAlign*}
    A \coloneqq \{ x \colon \varphi_A(x) \} &  & B \coloneqq \{ x \colon \varphi_B(x) \}
  \end{BreakableAlign*}

  \begin{itemize}
    \item the membership relation \( x \in A \) holds precisely when \( \BZ \models \varphi_A(x) \).

    \item the inclusion relation \( A \subseteq B \) holds when for any \hyperref[def:first_order_variable_assignment]{evaluation} \( v \) in \( \BZ \) and any integer \( x \), we have \( \varphi_A(x) \implies \varphi_B(x) \).

    \item set equality \( A = B \) holds precisely when \( A \subseteq B \) and \( B \subseteq A \)
  \end{itemize}

  Naive set theory easily leads to \hyperref[ex:russels_paradox_sets]{paradoxes}) and so some axiomatization (e.g. \fullref{def:set_zfc}) is required.
\end{definition}

\begin{example}\label{ex:russels_paradox_sets}
  Define
  \begin{equation*}
    R \coloneqq \{ x \colon x \neq x \}.
  \end{equation*}

  We have both \( R \in R \) and \( R \not\in R \).
\end{example}

\begin{definition}\label{def:set_zfc}\MarginCite[271]{Enderton1977}
  In contrast to \hyperref[def:set_naive]{na\"ive set theory}), \Def{Z}ermelo – \Def{F}raenkel set theory with the axiom of choice (ZF\Def{C}) can be made precise. Consider the \hyperref[def:first_order_logic_language]{first-order logic language} with equality \( = \), no functional symbols and a single binary predicate \( \in \). Note that we can take the language not to have formal equality and then use \fullref{def:set_zfc/A1} as an axiom schema to define equality in terms of \( \in \).

  Given a unary formula \( \varphi(x) \), we can construct a (syntactic) object
  \begin{equation*}
    A = \{ a \colon \varphi(a) \}
  \end{equation*}
  that we call a \Def{class}. Not all classes can be defined to have meaningful semantics (e.g. the class of all classes easily leads to paradoxes like \fullref{ex:russels_paradox_sets}). We define sets in ZFC as classes with semantics given by a model for the following axioms (exclude \fullref{def:set_zfc/A8} to obtain ZF). Classes that do not satisfy these axioms are called \Def{proper classes} and are often said to be \Def{too big} to be sets, e.g. the class of all sets or the class of all vector spaces). In this document, our main limitation when working with classes rather than sets is not being able to talk about a class not being a member of another class, however this is also not necessary for us.

  \begin{DefEnum}
    \IAxiom{def:set_zfc/A1}{A1}(extensionality) Two sets are equal if they have the same elements (given by set membership)

    \IAxiom{def:set_zfc/A2}{A2}(empty set) The following class is a set
    \begin{equation*}
      \varnothing \coloneqq \{ x \colon x \neq x \}.
    \end{equation*}

    \IAxiom{def:set_zfc/A3}{A3}(pairing) If \( A \) and \( B \) are sets, then
    \begin{equation*}
      \{ A, B \}
    \end{equation*}
    is also a set. In particular, \( \{ A \} = \{ A, A \} \) is a set.

    \IAxiom{def:set_zfc/A4}{A4}(union) If \( A \) is a set, then \( \bigcup A \) (see \fullref{def:set_union}) is also a set.

    \IAxiom{def:set_zfc/A5}{A5}(power set) If \( A \) is a set, \( \Pow(A) \) (see \fullref{def:power_set}) is also a set.

    \IAxiom{def:set_zfc/A6}{A6}(specification) If \( A \) is a set and \( \varphi \) is a formula, then
    \begin{equation*}
      \{ x \in A \colon \varphi(x) \}
    \end{equation*}
    is a set.

    \IAxiom{def:set_zfc/A7}{A7}(infinity) There exists an \hyperref[def:inductive_set]{inductive set}.

    \IAxiom{def:set_zfc/A8}{A8}(choice; see \fullref{thm:aoc}) Let \( \CM \neq 0 \) and for all \( m \in \CM \), let \( X_m \) be a nonempty set and \( X_k \cap X_m = \varnothing \) whenever \( m \neq k \). Then there exists a set \( M \) such that for every \( m \in \CM \), the intersection \( M \cap X_m \) (see \fullref{def:set_intersection}) has exactly one member.

    \IAxiom{def:set_zfc/A9}{A9}(replacement) Given a set \( X \) and a formula \( \varphi(x, y) \), if for every set \( x \in X \) there exists a unique set \( y \) such that \( \varphi(x, y) \) holds, then
    \begin{equation*}
      Y \coloneqq \{ y \colon \exists x \in X, \varphi(x, y) \}
    \end{equation*}
    is a set.

    \IAxiom{def:set_zfc/A10}{A10}(regularity) For every nonempty set \( A \), there exists a member \( a \in A \) such that
    \begin{equation*}
      a \cap A \neq \varnothing.
    \end{equation*}
  \end{DefEnum}
\end{definition}

\begin{remark}\label{remark:family_of_sets}
  In ZFC \fullref{def:set_zfc}, everything is a set. However, it is often the case that we are not interested in how a set's elements are represented and only in how they behave, e.g. when working with \hyperref[def:natural_numbers]{natural numbers} we are interested in the elements of \( \BN \) and not in the way every element of \( \BN \) is encoded as a set.

  In order to reduce repetitiveness, sets whose elements we consider to be other sets, are often called \Def{families} of sets. In particular if all (different) sets are \hyperref[def:set_intersection]{disjoint}, we say that the family is a \Def{disjoint family}. We usually assume that the sets are nonempty.
\end{remark}

\begin{remark}\label{remark:singleton_sets}
  Sets with a single elements are usually called \Def{singletons}. It is sometimes convenient, especially with connection to geometry or multivalued \hyperref[def:function]{functions} (e.g. when dealing with \hyperref[def:net_convergence/limit]{limits} or \hyperref[def:subdifferentials]{subdifferentials}), to not distinguish between singleton sets and their corresponding element.
\end{remark}

\begin{definition}\label{def:subset}
  We say that \( A \) is a \Def{subset} of \( B \) and write \( A \subseteq B \) if \( x \in A \implies x \in B \). If \( A \) is a subset of \( B \), we say that B is a \Def{superset} of \( A \).

  If \( A \subseteq B \) and \( A \neq B \), we say that \( A \) is a \Def{proper subset} of \( B \) and write \( A \subsetneq B \).
\end{definition}

\begin{remark}\label{remark:subset_notation}
  Some authors, such as \cite{Kelley1955}, use the notation \( A \subseteq B \) to mean \enquote{all elements of \( A \) belong to \( B \)}, even in the case when \( A = B \). To avoid confusion, we use the notations \( A \subseteq B \) and \( A \subsetneq B \) (see \fullref{def:subset}).
\end{remark}

\begin{remark}\label{remark:subset_and_membership_relations}
  Both \( \in \) and \( \subseteq \) are binary \hyperref[def:relation]{relations}, called the \Def{membership} and \Def{inclusion} relations, correspondingly.
\end{remark}

\begin{definition}\label{def:set_intersection}\MarginCite[24]{Enderton1977}
  If \( A \) is a set, define their \Def{intersection} as
  \begin{equation*}
    \bigcap A \coloneqq \{ x \colon \forall a \in A, x \in a \}.
  \end{equation*}

  We leave \( \bigcap \varnothing \) undefined.

  By \fullref{def:set_zfc/A6}, \( \bigcap A \) is a set.

  For two sets \( A \) and \( B \), we define the \Def{binary intersection} as
  \begin{equation*}
    A \cup B \coloneqq \bigcap \{ A, B \} = \{ x \colon x \in A \land x \in B \}.
  \end{equation*}

  The class \( \{ A, B \} \) is a set by \fullref{def:set_zfc/A3} and \( A \cup B \) is a set by \fullref{def:set_zfc/A6}.

  If \( A \cap B = \varnothing \), we say that \( A \) and \( B \) are \Def{disjoint}. If they are not disjoint, we say that they \Def{intersect}.
\end{definition}

\begin{definition}\label{def:set_union}\MarginCite[24]{Enderton1977}
  If \( A \) is a set, define its \Def{union} as
  \begin{equation*}
    \bigcup A \coloneqq \{ x \colon \exists a \in A, x \in A \}.
  \end{equation*}

  In particular, \( \bigcup \varnothing = \varnothing \).

  By \fullref{def:set_zfc/A4}, \( \bigcup A \) is a set.

  For two sets \( A \) and \( B \), we define the \Def{binary union} as
  \begin{equation*}
    A \cup B \coloneqq \bigcup \{ A, B \} = \{ x \colon x \in A \lor x \in B \}.
  \end{equation*}

  The class \( \{ A, B \} \) is a set by \fullref{def:set_zfc/A3} and \( A \cup B \) is a set by \fullref{def:set_zfc/A4}.
\end{definition}

\begin{definition}\label{def:set_difference}\MarginCite[27]{Enderton1977}
  If \( A \) and \( B \) are sets, define their \Def{difference} as
  \begin{equation*}
    A \setminus B \coloneqq \{ a \in A \colon a \not\in B \}.
  \end{equation*}

  By \fullref{def:set_zfc/A6}, \( A \setminus B \) is a set.
\end{definition}

\begin{proposition}\label{thm:set_difference_properties}
  Set \hyperref[def:set_difference]{difference} has the following basic properties:
  \begin{PropEnum}
    \ILabel{thm:set_difference_properties/intersection} If \( A \) and \( B \) are subsets of \( C \), then \( A \setminus B = A \cap (C \setminus B) \).
    \ILabel{thm:set_difference_properties/double_difference} If \( A \subseteq B \), then \( B \setminus (B \setminus A) = A \)
  \end{PropEnum}
\end{proposition}
\begin{proof}
  \SubProofOf{thm:set_difference_properties/intersection} Since \( a \in A \) implies \( a \in C \), we have
  \begin{BreakableAlign*}
    A \setminus B
     & =
    \{ a \in A \colon a \not\in B \}
    =    \\ &=
    \{ a \in A \colon a \in C \text{ and } a \not\in B \}
    =    \\ &=
    A \cap (C \setminus B).
  \end{BreakableAlign*}

  \SubProofOf{thm:set_difference_properties/double_difference} By the law of the excluded middle,
  \begin{BreakableAlign*}
    B \setminus (B \setminus A)
     & =
    \{ b \in B \colon b \not\in \{ b \in B \colon b \not\in A \} \}
    =    \\ &=
    \{ b \in B \colon b \in A \}
    =    \\ &=
    A.
  \end{BreakableAlign*}
\end{proof}

\begin{definition}\label{def:power_set}\MarginCite[19]{Enderton1977}
  If \( A \) is a set, define its \Def{power set} as
  \begin{equation*}
    \Pow(A) \coloneqq \{ B \colon B \subseteq A \}.
  \end{equation*}

  By \fullref{def:set_zfc/A5}, \( \Pow(A) \) is a set.
\end{definition}

\begin{proposition}\label{thm:subsets_form_boolean_algebra}
  Let \( X \) be an arbitrary set. Then \( \Pow(X) \) is a \hyperref[def:boolean_algebra]{Boolean algebra} with the \hyperref[def:poset]{partial order} \( \subseteq \) (see \fullref{def:subset}) and complements given by \( \Ol{A} \coloneqq X \setminus A \). More concretely,
  \begin{RefList}
    \IRef{def:binary_lattice_operations/join} Joins are given by \hyperref[def:set_union]{unions \( \bigcup \)}.
    \IRef{def:binary_lattice_operations/meet} Meets are given by \hyperref[def:set_intersection]{intersections \( \bigcap \)}.
    \IRef{def:lattice/top} The top element is \( X \).
    \IRef{def:lattice/bottom} The bottom is \( \varnothing \).
  \end{RefList}
\end{proposition}
\begin{proof}
  The structure is based on \fullref{thm:propositional_logic_boolean_algebra} because every set corresponds to a certain equivalence class of first-order logic formulas.
\end{proof}

\begin{remark}\label{remark:binary_vs_arbitrary_tuples}
  We give two pairs of definitions for tuples and Cartesian products. The first pair, \fullref{def:kuratowski_pair,def:binary_cartesian_product}, is quite restricted and is mostly necessary for defining \hyperref[def:function]{functions} and ensuring that everything along the way is indeed a set. The second pair of definitions, given in \fullref{def:cartesian_product}, can then be used freely.
\end{remark}

\begin{definition}\label{def:kuratowski_pair}\MarginCite[36]{Enderton1977}
  If \( A \) and \( B \) are sets, define the \Def{(binary) tuple} or \Def{Kuratowski pair} as
  \begin{equation*}
    (A, B) \coloneqq \{ \{ A \}, \{ A, B \} \}.
  \end{equation*}

  By \fullref{def:set_zfc/A3}, \( (A, B) \) is a set.
\end{definition}

\begin{definition}\label{def:binary_cartesian_product}\MarginCite[37]{Enderton1977}
  If \( A \) and \( B \) are sets, define their \Def{binary Cartesian product} as
  \begin{equation*}
    A \times B \coloneqq \{ (a, b) \colon a \in A \land b \in B \}.
  \end{equation*}
\end{definition}

\begin{proposition}\label{thm:binary_cartesian_product_is_set}
  If \( A \) and \( B \) are sets, their product \( A \times B \) is also a set.
\end{proposition}
\begin{proof}
  Fix \( a \in A \) and \( b \in B \).
  \begin{itemize}
    \item \( \{ a \} \) is a set by \fullref{def:set_zfc/A6} since \( \{ a \} \subseteq A \)
    \item \( A \cup B \) is a set by \fullref{def:set_union}
    \item \( \{ a, b \} \) is a set by \fullref{def:set_zfc/A6} since \( \{ a \} \subseteq A \cup B \)
    \item \( (a, b) = \{ \{ a \}, \{ a, b \} \} \) is a set by \fullref{def:set_zfc/A6} since \( (a, b) \subseteq \Pow(A \cup B) \).
  \end{itemize}

  Thus \( A \times B \) is a set since \( A \times B \subseteq \Pow(\Pow(A \cup B)) \).
\end{proof}

\begin{definition}\label{def:cartesian_product}\MarginCite[54]{Enderton1977}
  Let \( \{ X_k \}_{k \in \CK} \) be a nonempty \hyperref[def:indexed_family]{family} of (potentially nonempty) sets.

  We define their \Def{Cartesian product} as
  \begin{equation*}
    \prod_{k \in \CK} X_k \coloneqq \left\{ f: \CK \to \bigcup_{k \in \CK} X_k \colon \forall m \in \CK: f(m) \in X_m \right\}.
  \end{equation*}

  The definition makes sense when any of the sets is empty because the product itself is then empty.

  Any element of the Cartesian product is called a \Def{tuple}.
\end{definition}

\begin{definition}\label{def:disjoint_union}
  Let \( \{ X_k \}_{k \in \CK} \) be a nonempty \hyperref[def:indexed_family]{family} of nonempty sets.

  We define their \Def{disjoint union} as
  \begin{equation*}
    \coprod_{k \in \CK} X_k \coloneqq \{ (k, x) \colon k \in \CK, x \in X_k \}.
  \end{equation*}
\end{definition}

\begin{definition}\label{def:successor_operator}\MarginCite[68]{Enderton1977}
  For any set \( X \), we define the \Def{successor} operation
  \begin{equation*}
    S(X) \coloneqq X \cup \{ X \}.
  \end{equation*}
\end{definition}

\begin{definition}\label{def:inductive_set}\MarginCite[68]{Enderton1977}
  A set \( A \) is called \Def{inductive} if
  \begin{DefEnum}
    \item \( \varnothing \in A \)
    \item \( a \in A \implies S(a) \in A \)
  \end{DefEnum}
\end{definition}

\begin{definition}\label{def:smallest_inductive_set}
  The smallest \hyperref[def:inductive_set]{inductive set} is
  \begin{equation*}
    \omega \coloneqq \bigcap \{ A \colon A \text{ is an inductive set} \}.
  \end{equation*}

  The elements of \( \omega \) are
  \begin{equation*}
    \varnothing, S(\varnothing), S(S(\varnothing)),
  \end{equation*}
  where \( S \) is the set-theoretic successor operator (see \fullref{def:successor_operator}).
\end{definition}

\begin{definition}\label{def:transitive_set}\MarginCite[71]{Enderton1977}
  A set \( A \) is called \Def{transitive} if \( a \in A \) implies \( a \subseteq A \).
\end{definition}

\medskip

\begin{definition}\label{def:ordinal}\MarginCite[thm. 7L]{Enderton1977}
  A set \( A \) is called an \Def{ordinal} or an \Def{ordinal numbers} if it is \hyperref[def:well_ordered_set]{well-ordered} under set membership.
\end{definition}

\begin{theorem}\label{thm:ordinals_are_well_ordered}\MarginCite[thm. 7M]{Enderton1977}
  The class of all \hyperref[def:ordinal]{ordinals} is \hyperref[def:well_ordered_set]{well-ordered} by the inclusion relation \( \subseteq \), that is, every set of ordinals has a least element.
\end{theorem}

\begin{corollary}\label{thm:natural_numbers_are_well_ordered}
  The \hyperref[def:natural_numbers]{natural numbers} are well-ordered.
\end{corollary}

\begin{definition}\label{def:pointed_set}
  A nonempty set \( A \) with a dedicated element \( x \in A \) with called a \Def{pointed set}.
\end{definition}
