\section{Sets}\label{sec:sets}

There are certain technical subtleties when defining sets and many mathematicians use a mostly informal approach towards set theory that is based on the axioms of Zermelo and Fraenkel, including (or sometimes excluding) the axiom of choice (see \ref{def:set_zfc} and~\cref{list:aoc}).

\begin{definition}\label{def:set_naive}\cite[Chapter 1]{Enderton1977}
  Naive set theory is not based on a strict axiom set but rather on the intuitive notion of a set as an unordered collection without repetition. Set equality $A = B$, set membership $x \in A$ and set inclusion $A \subseteq B$ are assumed to be understood. Sets can be explicitly constructed by specifying their elements, e.g.
  \begin{align*}
    \{ 3, 7, 31, 127, 8191 \}
  \end{align*}
  or by specifying a logical formula $\varphi(x)$ in an implicitly assumed logical language:
  \begin{align*}
    \{ x \colon \varphi(x) \}
  \end{align*}

  If $\varphi(x) = x \in A \land \psi(x)$, we often write
  \begin{align*}
    \{ x \in A \colon \psi(x) \}.
  \end{align*}

  In a suitable context, the definitions can be made precise. For example, in the ring of integers $\BB{Z}$ with equality, addition, multiplication and predicates partial ordering $\leq$ and divisibility $\vert$, each set can be thought of a formula in the corresponding first-order language\Tinyref{def:first_order_language}. Given formulas $\varphi_A$ and $\varphi_B$ with a free variable $x$ and sets
  \begin{align*}
    A \coloneqq \{ x \colon \varphi_A(x) \} && B \coloneqq \{ x \colon \varphi_B(x) \}
  \end{align*}

  \begin{itemize}
    \item the membership relation $x \in A$ holds precisely when $\BB{Z} \models \varphi_A(x)$.

    \item the inclusion relation $A \subseteq B$ holds when for any evaluation\Tinyref{def:first_order_evaluation} $v$ in $\BB{Z}$ and any integer $x$, we have $\varphi_A(x) \implies \varphi_B(x)$.

    \item set equality $A = B$ holds precisely when $A \subseteq B$ and $B \subseteq A$
  \end{itemize}

  Naive set theory easily leads to paradoxes\Tinyref{ex:russels_paradox_sets}) and so some axiomatization (e.g. \cref{def:set_zfc}) is required.
\end{definition}

\begin{example}\label{ex:russels_paradox_sets}
  Define
  \begin{align*}
    R \coloneqq \{ x \colon x \neq x \}.
  \end{align*}

  We have both $R \in R$ and $R \not\in R$.
\end{example}

\begin{definition}\label{def:set_zfc}\cite[271]{Enderton1977}
  In contrast to naive set theory\Tinyref{def:set_naive}), \textbf{Z}ermelo – \textbf{F}raenkel set theory with the axiom of choice (ZF\textbf{C}) can be made precise. Consider the first-order language\Tinyref{def:first_order_language} with equality $=$, no functional symbols and a single predicate $\in$.

  Given a formula $\varphi(x_1, \ldots, x_n)$, we can construct a (syntactic) object
  \begin{align*}
    A = \{ a_1, \ldots, a_n \colon \varphi(a_1, \ldots, a_n) \}
  \end{align*}
  that we call a \ul{class}. Not all classes can be defined to have meaningful semantics (e.g. the class of all classes easily leads to paradoxes like \cref{ex:russels_paradox_sets}). We define sets in ZFC as classes with semantics given by a model for the following axioms (exclude~\ref{def:set_zfc/A8} to obtain ZF). Classes that do not satisfy these axioms are called \ul{proper classes} and are often said to be \ul{too big} to be sets, e.g. the class of all sets or the class of all vector spaces).

  \begin{description}
    \DItem{A1}{def:set_zfc/A1}(extensionality) Two sets are equal if they have the same elements (given by set membership)

    \DItem{A2}{def:set_zfc/A2}(empty set) The following class is a set
    \begin{align*}
      \varnothing \coloneqq \{ x \colon x \neq x \}.
    \end{align*}

    \DItem{A3}{def:set_zfc/A3}(pairing) If $A$ and $B$ are sets, then
    \begin{align*}
      \{ A, B \}
    \end{align*}
    is also a set. In particular, $\{ A \} = \{ A, A \}$ is a set.

    \DItem{A4}{def:set_zfc/A4}(union) If $A$ is a set, then $\bigcup A$\Tinyref{def:set_union} is also a set.

    \DItem{A5}{def:set_zfc/A5}(power set) If $A$ is a set, $\Power(A)$\Tinyref{def:power_set} is also a set.

    \DItem{A6}{def:set_zfc/A6}(specification) If $A$ is a set and $\varphi$ is a formula, then
    \begin{align*}
      \{ x \in A \colon \varphi(x) \}
    \end{align*}
    is a set.

    \DItem{A7}{def:set_zfc/A7}(infinity) There exists an inductive set\Tinyref{def:inductive_set}.

    \DItem{A8}{def:set_zfc/A8}(choice; see \cref{list:aoc}) Let $J \neq 0$ and for all $j \in J$, $X_j$ is a nonempty set and $X_i \cap X_j = \varnothing$ when $i \neq j$. Then there exists a set $M$ such that for every $j \in J$, the intersection $M \cap X_j$\Tinyref{def:set_intersection} has exactly one member.

    \DItem{A9}{def:set_zfc/A9}(replacement) Given a set $X$ and a formula $\varphi(x, y)$, if for every set $x \in X$ there exists a unique set $y$ such that $\varphi(x, y)$ holds, then
    \begin{align*}
      Y \coloneqq \{ y \colon \exists x \in X, \varphi(x, y) \}
    \end{align*}
    is a set.

    \DItem{A10}{def:set_zfc/A10}(regularity) For every nonempty set $A$, there exists a member $a \in A$ such that
    \begin{align*}
      a \cap A \neq \varnothing.
    \end{align*}
  \end{description}
\end{definition}

\begin{note}\label{note:family_of_sets}
  In ZFC~\cref{def:set_zfc}, everything is a set. However, it is often the case that we are not interested in how a set's elements are represented and only in how they behave, e.g. when working with natural numbers\Tinyref{def:natural_numbers_zfc} we are interested in the elements of $\BB{N}$ and not in the way every element of $\BB{N}$ is encoded as a set.

  In order to reduce repetitiveness, sets whose elements we consider to be other sets, are often called \ul{families of sets}. In particular if all (different) sets are disjoint\Tinyref{def:set_intersection}, we say that the family is a \ul{disjoint family}. We usually assume that the sets are nonempty.
\end{note}

\begin{note}\label{note:singleton_sets}
  Sets with a single elements are usually called \ul{singletons}. It is sometimes convenient, especially with connection to geometry\Tinyref{sec:geometry_of_vector_spaces} or multivalued functions\Tinyref{def:function/multivalued} (e.g. when dealing with limits\Tinyref{def:net_limit_point} or subdifferentials\Tinyref{def:subdifferentials}), to not distinguish between singleton sets and their corresponding element.
\end{note}

\begin{definition}\label{def:subset}
  We say that $A$ is a \ul{subset of $B$} and write $A \subseteq B$ if $x \in A \implies x \in B$.

  If $A \subseteq B$ and $A \neq B$, we say that \ul{$A$ is a proper subset of $B$} and write $A \subsetneq B$.
\end{definition}

\begin{definition}\label{def:set_intersection}\cite[24]{Enderton1977}
  If $A$ is a set, define their \ul{intersection} as
  \begin{align*}
    \bigcap A \coloneqq \{ x \colon \forall a \in A, x \in a \}.
  \end{align*}

  We leave $\bigcap \varnothing$ undefined.

  By~\ref{def:set_zfc/A6}, $\bigcap A$ is a set.

  For two sets $A$ and $B$, we define the \ul{binary intersection} as
  \begin{align*}
    A \cup B \coloneqq \bigcap \{ A, B \} = \{ x \colon x \in A \land x \in B \}.
  \end{align*}

  The class $\{ A, B \}$ is a set by~\ref{def:set_zfc/A3} and $A \cup B$ is a set by~\ref{def:set_zfc/A6}.

  If $A \cap B = \varnothing$, we say that $A$ and $B$ are \ul{disjoint}.
\end{definition}

\begin{definition}\label{def:set_union}\cite[24]{Enderton1977}
  If $A$ is a set, define its \ul{union} as
  \begin{align*}
    \bigcup A \coloneqq \{ x \colon \exists a \in A, x \in A \}.
  \end{align*}

  In particular, $\bigcup \varnothing = \varnothing$.

  By~\ref{def:set_zfc/A4}, $\bigcup A$ is a set.

  For two sets $A$ and $B$, we define the \ul{binary union} as
  \begin{align*}
    A \cup B \coloneqq \bigcup \{ A, B \} = \{ x \colon x \in A \lor x \in B \}.
  \end{align*}

  The class $\{ A, B \}$ is a set by~\ref{def:set_zfc/A3} and $A \cup B$ is a set by~\ref{def:set_zfc/A4}.
\end{definition}

\begin{definition}\label{def:set_difference}\cite[27]{Enderton1977}
  If $A$ and $B$ are sets, define their \ul{difference} as
  \begin{align*}
    A \setminus B \coloneqq \{ a \in A \colon a \not\in B \}.
  \end{align*}

  By~\ref{def:set_zfc/A6}, $A \setminus B$ is a set.
\end{definition}

\begin{definition}\label{def:power_set}\cite[19]{Enderton1977}
  If $A$ is a set, define its \ul{power set} as
  \begin{align*}
    \Power(A) \coloneqq \{ B \colon B \subseteq A \}.
  \end{align*}

  By~\ref{def:set_zfc/A5}, $\Power(A)$ is a set.
\end{definition}

\begin{note}
  We give two pairs of definitions for tuples and Cartesian products. The first pair, \cref{def:kuratowski_pair,def:binary_cartesian_product}, is quite restricted and is mostly necessary for defining functions\Tinyref{def:function} and ensuring that everything along the way is indeed a set. The second pair of definitions, given in \cref{def:cartesian_product}, can then be used freely.
\end{note}

\begin{definition}\label{def:kuratowski_pair}\cite[36]{Enderton1977}
  If $A$ and $B$ are sets, define the \ul{(binary) tuple} or \ul{Kuratowski pair} as
  \begin{align*}
    (A, B) \coloneqq \{ \{ A \}, \{ A, B \} \}.
  \end{align*}

  By~\ref{def:set_zfc/A3}, $(A, B)$ is a set.
\end{definition}

\begin{definition}\label{def:binary_cartesian_product}\cite[37]{Enderton1977}
  If $A$ and $B$ are sets, define their \ul{binary Cartesian product} as
  \begin{align*}
    A \times B \coloneqq \{ (a, b) \colon a \in A \land b \in B \}.
  \end{align*}
\end{definition}

\begin{proposition}\label{def:binary_cartesian_product_is_set}
  If $A$ and $B$ are sets, their product $A \times B$ is also a set.
\end{proposition}
\begin{proof}
  Fix $a \in A$ and $b \in B$.
  \begin{itemize}
    \item $\{ a \}$ is a set by~\ref{def:set_zfc/A6} since $\{ a \} \subseteq A$
    \item $A \cup B$ is a set by~\cref{def:set_union}
    \item $\{ a, b \}$ is a set by~\ref{def:set_zfc/A6} since $\{ a \} \subseteq A \cup B$
    \item $(a, b) = \{ \{ a \}, \{ a, b \} \}$ is a set by~\ref{def:set_zfc/A6} since $(a, b) \subseteq \Power(A \cup B)$.
  \end{itemize}

  Thus $A \times B$ is a set since $A \times B \subseteq \Power(\Power(A \cup B))$.
\end{proof}

\begin{definition}\label{def:cartesian_product}\cite[54]{Enderton1977}
  Let $\{ X_i \}_{i \in I}$ be a nonempty family of nonempty sets\Tinyref{def:indexed_family}.

  We define their \ul{Cartesian product} as
  \begin{align*}
    \prod_{i \in I} X_i \coloneqq \left\{ f: I \to \bigcup_{j \in I} X_j \colon \forall j \in I, f(j) \in X_j \right\}.
  \end{align*}

  Any element of the Cartesian product is called a \ul{tuple}.
\end{definition}

\begin{definition}\label{def:disjoint_union}
  Let $\{ X_i \}_{i \in I}$ be a nonempty family of nonempty sets\Tinyref{def:indexed_family}.

  We define their \ul{disjoint union} as
  \begin{align*}
    \coprod_{i \in I} X_i \coloneqq \{ (i, x) \colon i \in I, x \in X_i \}.
  \end{align*}
\end{definition}

\begin{definition}\label{def:successor_operator}\cite[68]{Enderton1977}
  For any set $X$, we define the \ul{successor} operation
  \begin{align*}
    S(X) \coloneqq X \cup \{ X \}.
  \end{align*}
\end{definition}

\begin{definition}\label{def:inductive_set}\cite[68]{Enderton1977}
  A set $A$ is called \ul{inductive} if
  \begin{defenum}
    \item $\varnothing \in A$
    \item $a \in A \implies S(a) \in A$
  \end{defenum}
\end{definition}

\begin{definition}\label{def:natural_numbers_zfc}
  The smallest inductive set\Tinyref{def:inductive_set} is
  \begin{align*}
    \omega \coloneqq \bigcap \{ A \colon A \text{ is an inductive set} \}.
  \end{align*}

  Since the elements of $\omega$ are $\varnothing, S(\varnothing), S(S(\varnothing)), \ldots$, we can identify $\omega$ with the set $\BB{N}$ of natural numbers.

  Whether $0$ is a natural number or not, i.e. whether $\varnothing$ encodes $0$ or $1$, is a matter of convention and notation like $\Cal{Z}^{\geq 0}$ and $\Cal{Z}^{>0}$ or $\Cal{N}^0$ and $\Cal{N}^+$ is sometimes used.
\end{definition}

\begin{definition}\label{def:finite_intersection_operator}
  Let $X$ be a nonempty set. We define the \ul{finite intersection} operator
  \begin{align*}
    &\FI: \Power(X) \mapsto \Power(X) \\
    &\FI(P) \coloneqq \left\{ \bigcap P' \colon P' \text{ is a nonempty finite\Tinyref{def:finite_set} subset of } P \right\}
  \end{align*}
\end{definition}

\begin{proposition}\label{thm:finite_intersection_properties}
  The finite intersection operator $\FI$\Tinyref{def:finite_intersection_operator} satisfies the following
  \begin{defenum}
    \item $P \subseteq \FI(P)$
    \item $\FI(\FI(P)) = \FI(P)$
    \item $\bigcap \FI(P) = \bigcap P$
  \end{defenum}
\end{proposition}

\begin{definition}\label{def:transitive_set}\cite[71]{Enderton1977}
  A set $A$ is called \ul{transitive} if $a \in A$ implies $a \subseteq A$.
\end{definition}

\begin{definition}\label{def:ordinal}\cite[theorem 7L]{Enderton1977}
  A set $A$ is called an \ul{ordinal} or an \ul{ordinal numbers} if it is well-ordered\Tinyref{def:order/partial/well_order} under set membership.
\end{definition}

\begin{theorem}\label{thm:ordinals_are_well_ordered}\cite[theorem 7M]{Enderton1977}
  The class of all ordinals\Tinyref{def:ordinal} is well-ordered\Tinyref{def:order/partial/well_order} by the inclusion relation $\subseteq$, that is, every set of ordinals has a least element.
\end{theorem}

\begin{definition}\label{def:category_of_sets}
  The class\Tinyref{def:set_zfc} of all sets forms the category\Tinyref{def:category} $\Bold{Set}$, where for every two sets $X, Y \in \Bold{Set}$, the morphisms $\Bold{Set}(X, Y)$ are the functions\Tinyref{def:function} from $X$ to $Y$ and composition is the usual function composition\Tinyref{def:function_composition}.

  Furthermore, $\Bold{Set}$ is locally small\Tinyref{def:category_cardinality} and concrete\Tinyref{def:concrete_category}.
\end{definition}
\begin{proof}
  \begin{description}
    \item [\ref{def:category/identity}] For any set $X \in \Bold{Set}$, we have the identity function
    \begin{align*}
      &\Id_X: X \to X \\
      &\Id_X(x) \coloneqq x.
    \end{align*}

    If $f: X \to Y$ is any function, for all $x \in X$ we have
    \begin{align*}
      [\Id_Y \circ f](x) = \Id_Y(f(x)) = f(x),
      &&
      [f \circ \Id_X](x) = f(\Id_X(x)) = f(x),
    \end{align*}
    thus $\Id_x$ and $\Id_Y$ are indeed identity morphisms.

    \item [\ref{def:category/associativity}] Let $f: A \to B$, $g: B \to C$ and $h: C \to D$ be arbitrary functions. For any $x \in A$, we have
    \begin{align*}
      [[h \circ g] \circ f](x)
      =
      [h \circ g](f(x))
      =
      h(g(f(x)))
      =
      h([g \circ f](x))
      =
      [h \circ [g \circ f]](x).
    \end{align*}
  \end{description}

  Since a function $f: X \to Y$ is formally\Tinyref{def:function} a subset of the product, $X \times X$\Tinyref{def:cartesian_product}, which is a set, it is itself a set.

  The class of all functions $\Bold{Set}(X, Y)$ from $X$ to $Y$ is then a subset of $\Power(X \times Y)$, which is also a set. Thus $\Bold{Set}$ is a locally small category.

  It is concrete since it is equipped with the identity functor $\Id_{\Bold{Set}}$.
\end{proof}

\begin{theorem}\label{thm:set_categorical_limits}
  We are interested in categorical limits\Tinyref{def:categorical_limit} and colimits\Tinyref{def:categorical_colimit} in $\Bold{Set}$. If $\{ X_i \}_{i \in I}$ is an indexed family\Tinyref{def:indexed_family} of sets, then
  \begin{defenum}
    \item\label{thm:set_categorical_limits/product} their categorical product\Tinyref{def:categorical_product} is their Cartesian product\Tinyref{def:cartesian_product} $\prod_{i \in I} X_i$, the projection morphisms being
    \begin{align*}
      &\pi_j: \prod_{i \in I} X_i \to X_j \\
      &\pi_j(\{ x_i \}_{i \in I}) \coloneqq x_j.
    \end{align*}

    \item\label{thm:set_categorical_limits/coproduct} their categorical coproduct\Tinyref{def:categorical_coproduct} is their disjoint union\Tinyref{def:disjoint_union} $\coprod_{i \in I} X_i$, the injection morphisms being
    \begin{align*}
      &\iota_j: X_j \to \coprod_{i \in I} X_i \\
      &\iota_j(x_j) \coloneqq (j, x_j).
    \end{align*}

    \item\label{thm:set_categorical_limits/equalizer} An equalizer of two functions $f, g: X \to Y$ in $\Bold{Set}$ is the set
    \begin{align*}
      \{ x \in X \colon f(x) = g(x) \}.
    \end{align*}

    Compare this with pullbacks in $\Bold{Set}$\Tinyref{thm:set_categorical_limits/pullback}.

    \item\label{thm:set_categorical_limits/pullback} The pullback of two functions $f: X \to Z$ and $g: Y \to Z$ in $\Bold{Set}$ is the set
    \begin{align*}
      \{ (x, y) \in X \times Y \colon f(x) = g(y) \}.
    \end{align*}

    Compare this with equalizers in $\Bold{Set}$\Tinyref{thm:set_categorical_limits/equalizer}.

    \item\label{thm:set_categorical_limits/coequalizer} A coequalizer of two functions $f, g: X \to Y$ in $\Bold{Set}$ is the quotient space formed by the reflexive, symmetric and transitive closure of the relation $x \sim y \iff f(x) = g(x)$.

    In particular, if $\sim \subseteq X^2$ is an equivalence relation\Tinyref{def:order/equivalence}) on $X$, then the coequalizer of the two projection maps of the product $X^2$ is the pair $(X / ~, \pi)$, where $\pi$ is the quotient map
    \begin{align*}
      &\pi: X \to X / ~ \\
      &\pi(x) = [x].
    \end{align*}

    \item\label{thm:set_categorical_limits/pushout} Let $X$ and $Y$ be two sets, let $Z$ be a subset of $X$ and let $i: Z \to X$ be the inclusion map. For any function $f: Z \to Y$, we define a pushout of $i$ and $f$ in $\Bold{Set}$ to be the set obtained as the quotient of the coproduct $X \coprod Y$ and the relation $x \sim y \iff i^{-1}(x) = f^{-1}(y)$.
  \end{defenum}
\end{theorem}
