\subsection{Topological nets}\label{subsec:topological_nets}

In this section, \( X \) will denote an arbitrary nonempty topological space.

\begin{definition}\label{def:topological_net}
  A \Def{net} or \Def{generalized sequence}\Tinyref{def:sequence} or \Def{Moore-Smith sequence} in a nonempty set \( S \) is a family of elements of \( S \) indexed by a nonempty directed set\Tinyref{def:indexed_family}, i.e. a function from a nonempty directed set\Tinyref{def:order/directed} \( (\CA, \leq) \) to \( S \). We use the conventional notation for indexed sets (with the caveats described in \fullref{remark:indexed_family_notation}):
  \begin{equation*}
    \{ x_\al \}_{\al \in \CA},
  \end{equation*}
  because the preorder on the domain \( \CA \) is usually clear from the context.

  If we know that the net is a sequence, we will usually use the notation for sequences given in \fullref{def:sequence}.

  Note that this definition does not actually require a topology on \( S \). Some other important definitions also do not require topologies:
  \begin{defenum}
    \DItem{def:topological_net/frequently_in} We say that \( \{ x_\al \}_{\al \in \CA} \) is \Def{frequently in} the set \( A \subseteq S \) if for every index \( \al_0 \in \CA \) there exists an index \( \al \geq \al_0 \) such that \( x_\al \in A \).

    \DItem{def:topological_net/eventually_in} We say that \( \{ x_\al \}_{\al \in \CA} \) is \Def{eventually in} the set \( A \subseteq S \) if there exists an index \( \al_0 \) such that \( x_\al \in A \) whenever \( \al \geq \al_0 \). This is obviously a stronger condition.

    \DItem{def:topological_net/subnet}\cite[50]{Engelking1989} We say that the net \( \{ y_\beta \}_{\beta \in \CB} \subseteq S \) is a \Def{subnet} of \( \{ x_\al \}_{\al \in \CA} \subseteq S \) if there exists an embedding function \( \varphi: \CB \to \CA \) such that
    \begin{itemize}
      \item To every \( \al \in \CA \) there corresponds \( \beta \in \CB \) such that \( \varphi(\beta) \geq \al \).
      \item For every \( \beta \in \CB \) we have \( x_{\varphi(\beta)} = y_\beta \).
    \end{itemize}
  \end{defenum}
\end{definition}

\begin{proposition}\label{thm:topological_net_properties}
  Nets\Tinyref{def:topological_net} have the following basic properties:

  \begin{propenum}
    \DItem{thm:topological_net_properties/eventually_in_implies_frequently_in} \enquote{Eventually in}\Tinyref{def:topological_net/eventually_in} implies \enquote{frequently in}\Tinyref{def:topological_net/frequently_in}

    \DItem{thm:topological_net_properties/net_eventually_in_iff_not_frequently_in_complement} The net \( \{ x_\al \}_{\al \in \CA} \subseteq S \) is eventually in \( A \subseteq S \) if and only if it is not frequently in \( S \setminus A \).
  \end{propenum}
\end{proposition}
\begin{proof}\mbox{}
  \begin{description}
    \RItem{thm:topological_net_properties/eventually_in_implies_frequently_in} Suppose that the net \( \{ x_\al \}_{\al \in \CA} \subseteq S \) is eventually in \( A \subseteq S \). Then there exists an index \( \al_0 \) such that \( x_\al \in A \) for all \( \al \geq \al_0 \).

    Given any index \( \al_1 \), we choose \( \al_2 \) such that \( \al_2 \geq \al_0 \) and \( \al_2 \geq \al_1 \) (this is possible by the definition of a directed set). Then \( x_{\al_2} \in A \) and \( \al_2 \) satisfies the existence quantifier in \fullref{def:topological_net/frequently_in}.

    \RItem{thm:topological_net_properties/net_eventually_in_iff_not_frequently_in_complement} Suppose\LEM that \( \{ x_\al \}_{\al \in \CA} \) is both eventually in \( A \) and frequently in \( S \setminus A \).

    Since the net is eventually in \( A \), we can fix an index \( \al_0 \) such that \( x_\al \in A \) whenever \( \al \geq \al_0 \).

    Since the net is frequently in \( A \), we can fix an index \( \al_1 \geq \al_0 \) such that \( \al_1 \in S \setminus A \), which is a contradiction.

    This proves that the two conditions are incompatible.
  \end{description}
\end{proof}

\begin{definition}\label{def:net_convergence}
  Let \( X \) be a topological space and \( \{ x_\al \}_{\al \in \CA} \subseteq X \) be a net.

  \begin{defenum}
    \DItem{def:net_convergence/cluster} If the net is frequently in every neighborhood of \( x_0 \in X \), we say that \( x_0 \) is a \Def{cluster point} or an \Def{accumulation point} of \( \{ x_\al \}_{\al \in \CA} \subseteq X \).

    \DItem{def:net_convergence/limit} If the net is eventually in every neighborhood of \( x_0 \in X \), we say that \( x_0 \) is a \Def{limit point} of \( \{ x_\al \}_{\al \in \CA} \subseteq X \).
  \end{defenum}

  In general, there can exist multiple limit points (see \fullref{ex:multiple_limit_points_of_net}) and even more cluster points (see \fullref{ex:cluster_points/sine}). In Hausdorff spaces, however, limits are unique by \fullref{thm:t2_iff_singleton_limits}.

  If \( \{ x_\al \}_{\al \in \CA} \subseteq X \) has a unique limit, we say that the net \Def{converges} to \( x_0 \) use the notation
  \begin{equation*}
    x_0 = \lim_{\al \in \CA} x_\al.
  \end{equation*}

  If the net is a sequence\Tinyref{def:sequence}, we also use the following notations:
  \begin{itemize}
    \item \( x_0 = \lim_{k \to \infty} x_k \)
    \item \( x_0 = \lim x_k \)
    \item \( x_k \xrightarrow[k \to \infty]{} x_0 \)
    \item \( x_k \to x_0 \)
  \end{itemize}
\end{definition}

\begin{example}\label{ex:multiple_limit_points_of_net}
  Even limits of sequences need not be unique in arbitrary topological spaces. Let \( X = \{ y, z \} \) be a binary set with the indiscrete topology\Tinyref{def:standard_topologies/indiscrete} \( \{ \varnothing, X \} \). L

  Define the following sequence\Tinyref{def:sequence}
  \begin{align*}
    x_k \coloneqq \begin{cases}
      y, &k \text{ is even}, \\
      z, &k \text{ is odd}.
    \end{cases}
  \end{align*}

  The only neighborhood of \( y \), the whole space \( X \), contains all members of the sequence, therefore \( y \) is a limit point of the sequence. The same is true for \( z \), however.
\end{example}

\begin{example}\label{ex:cluster_points/sine}
  Consider the net \( \{ \sin(\al) \}_{\al \in \BR} \). It has no limit point, yet every real number in the interval \( [-1, 1] \) is a cluster point.
\end{example}

\begin{example}\label{ex:reverse_inclusion_net}
  A commonly used technique is to use a variation of a \Def{reverse set inclusion net}.

  Fix an element \( x_0 \in X \) of any topological space and choose\LEM an element \( x_U \) our of every neighborhood \( U \) of \( x_0 \). Consider the directed set \( (\CT(x), \leq) \) consisting of all neighborhoods of \( x_0 \) ordered by \Def{reverse inclusion}, i.e. \( U \leq V \iff U \supseteq V \).

  Then, by construction, \( x_0 \) is a limit point of the net \( \{ x_U \}_{U \in \CT(x_0)} \).
\end{example}

\begin{proposition}\label{thm:net_convergence_properties}
  Convergence of nets\Tinyref{def:net_convergence} has the following basic properties:

  \begin{propenum}
    \DItem{thm:net_convergence_properties/sequence_converges_iff_almost_entirely_in_neighborhood} The point \( x_0 \in X \) is a limit point of the sequence \( \{ x_k \}_{k=1}^\infty \subseteq X \) if and only if, given a neighborhood \( U \) of \( x_0 \), only finitely many elements of the sequence are outside of \( U \).

    \DItem{thm:net_convergence_properties/limit_point_is_cluster_point} Every limit point is a cluster point.

    \DItem{thm:net_convergence_properties/cluster_point_iff_subnet_limit_point} A point \( x_0 \in X \) is a cluster point of the net \( \{ x_\al \}_{\al \in \CA} \subseteq X \) if and only if \( x_0 \) is a limit point of some subnet.

    \DItem{thm:net_convergence_properties/limit_implies_no_proper_cluster_points} If a net has a limit point, all of its cluster points are limit points.

    \DItem{thm:net_convergence_properties/unique_limit_point_iff_unique_cluster_point} A net has a unique limit point if and only if has a unique cluster point.

    \DItem{thm:net_convergence_properties/unique_limit_point_iff_subnets_have_same_limit} A net has a unique limit point if and only if all subnets have the same limit point.
  \end{propenum}
\end{proposition}
\begin{proof}\mbox{}
  \begin{description}
    \RItem{thm:net_convergence_properties/sequence_converges_iff_almost_entirely_in_neighborhood} This is simply a restatement of \fullref{def:net_convergence/limit} for the special case of sequences.

    \RItem{thm:net_convergence_properties/limit_point_is_cluster_point} Follows from \fullref{thm:topological_net_properties/eventually_in_implies_frequently_in}.

    \RItem{thm:net_convergence_properties/cluster_point_iff_subnet_limit_point}\mbox{}
    \begin{description}
      \Implies The definition of a cluster point (\fullref{def:net_convergence/cluster}) allows us to build a reverse inclusion net in the style of \fullref{ex:reverse_inclusion_net}.

      \ImpliedBy Let \( \{ y_\beta \}_{\beta \in \CB} \subseteq X \) be a subnet of \( \{ x_\al \}_{\al \in \CA} \subseteq X \) with a subnet injection given by \( \varphi: \CB \to \CA \) and let \( y_0 \) be a limit point of \( \{ y_\beta \}_{\beta \in \CB} \subseteq X \).

      By the definition of \( \varphi \)\Tinyref{def:topological_net/subnet}, given an index \( \alpha_0 \in \CA \), there exist \( \beta_0 \in \CB \) such that \( \varphi(\beta_0) \geq \alpha_0 \)

      By assumption, \( \{ y_\beta \}_{\beta \in \CB} \subseteq X \) is eventually in every neighborhood of \( y_0 \). Fix a neighborhood \( U \) of \( y_0 \). There exists an index \( \beta_1 \) such that \( y_\beta \in U \) for \( \beta \geq \beta_1 \).

      Pick a \( \beta_2 \) such that \( \beta_2 \geq \beta_0 \) and \( \beta_2 \geq \beta_1 \). Then \( \varphi(\beta_2) \geq \alpha_0 \) and \( x_{\varphi(\beta_2)} = y_{\beta_2} \in U \). Therefore \( \varphi(\beta_2) \) allows \fullref{def:topological_net/eventually_in} to hold.
    \end{description}

    \RItem{thm:net_convergence_properties/limit_implies_no_proper_cluster_points} Let \( \{ x_\al \}_{\al \in \CA} \subseteq X \) be a net, let \( x_0 \) be a limit point and let \( y_0 \) be a cluster point of the net.

    Fix a neighborhood \( U \) of \( x_0 \). By \fullref{thm:topological_net_properties/net_eventually_in_iff_not_frequently_in_complement}, since \( \{ x_\al \}_{\al \in \CA} \) is eventually in \( U \), it is not frequently in \( X \setminus U \).

    Therefore, any a neighborhood of \( y_0 \) must intersect every neighborhood of \( x_0 \). Hence every neighborhood of \( x_0 \) is also a neighborhood of \( y_0 \). Consequently, the net \( \{ x_\al \}_{\al \in \CA} \subseteq X \) is eventually in every neighborhood of \( y_0 \). Then \( y_0 \) is also a limit point.

    \RItem{thm:net_convergence_properties/unique_limit_point_iff_unique_cluster_point}\mbox{}
    \begin{description}
      \Implies Follows from \fullref{thm:net_convergence_properties/limit_implies_no_proper_cluster_points}.

      \ImpliedBy Let \( \{ x_\al \}_{\al \in \CA} \subseteq X \) be a net and let \( x_0 \) be its only cluster point. By \fullref{thm:net_convergence_properties/limit_point_is_cluster_point}, it is clear that there is either a single limit point or non at all. We will show the former.

      Fix a neighborhood \( U \) of \( x_0 \). We know that the net is frequently in \( U \). By uniqueness of \( x_0 \) as a cluster point, it follows that \( x_0 \) is not frequently in all neighborhoods of any other point. Therefore there are only finitely many members of the net that do not belong to \( U \) and hence we can choose a large enough index \( \al_0 \) such that \( x_\al \in U \) for all \( \al \geq \al_0 \).
    \end{description}

    \RItem{thm:net_convergence_properties/unique_limit_point_iff_subnets_have_same_limit}\mbox{}
    \begin{description}
      \Implies Let \( x_0 \) be the unique limit of \( \{ x_\al \}_{\al \in \CA} \subseteq X \) and let \( \{ y_\beta \}_{\beta \in \CB} \subseteq X \) be a subnet with the subnet injection given by \( \varphi: \CB \to \CA \). Fix a neighborhood \( U \) of \( x_0 \). Since \( \{ x_\al \}_{\al \in \CA} \subseteq X \) is eventually in \( U \), we can fix an index \( \al_0 \) such that \( x_\al \in U \) whenever \( \al \geq \al_0 \). Then there exists an index \( \beta_0 \) such that \( \varphi(\beta_0) \geq \al_0 \). Note that when \( \beta \geq \beta_0 \), \( y_\beta = x_{\varphi(\beta)} \in U \) because \( \varphi(\beta) \geq \varphi(\beta_0) \geq \al_0 \). Therefore \( \{ y_\beta \}_{\beta \in \CB} \subseteq X \) is eventually in \( U \).

      Since our choice of the neighborhood \( U \) was arbitrary, it follows that \( x_0 \) is a limit of \( \{ y_\beta \}_{\beta \in \CB} \subseteq X \).

      \ImpliedBy Follows from \fullref{thm:net_convergence_properties/unique_limit_point_iff_unique_cluster_point}.
    \end{description}


    Follows from \fullref{thm:net_convergence_properties/unique_limit_point_iff_unique_cluster_point}.
  \end{description}
\end{proof}

\begin{proposition}\label{thm:limit_point_iff_in_closure}\cite[proposition 1.6.3]{Engelking1989}
  Fix a set \( A \subseteq X \). A point \( x_0 \in X \) belongs to \( \Cl{A} \) if and only if there exists a net \( \{ x_\al \}_{\al \in \CA} \subseteq A \) for which \( x_0 \) is a limit point.
\end{proposition}
\begin{proof}
  The complement of the empty set is the empty set, hence the statement of the proposition holds vacuously. Assume that \( A \) is nonempty.

  \begin{description}
    \Implies Suppose that \( x_0 \in \Cl{A} \). If \( x_0 \in A \), then the one-element net \( (x_0) \) converges to \( x_0 \).

    If \( x_0 \in \Bd{A} \), by \fullref{def:topological_boundary/neighborhoods}, every neighborhood of \( x_0 \) contains points from \( A \). Therefore we can build reverse inclusion net in the style of \fullref{ex:reverse_inclusion_net} that converges to \( x_0 \).

    \ImpliedBy Let \( x_0 \) be a limit point of \( \{ x_\al \}_{\al \in \CA} \subseteq A \). We will show that \( x_0 \) belongs every closed set that contains \( A \).

    Let \( F \supseteq A \) be a closed set. Denote \( U \coloneqq X \setminus F \). Suppose\LEM that \( x_0 \in U \). Then \( U \) is a neighborhood \( x_0 \) and, by \fullref{def:net_convergence/cluster}, the net \( \{ x_\al \}_{\al \in \CA} \subseteq A \) is eventually in \( U \). But \( U \) does not contains \( A \).

    The obtained contradiction shows that \( x_0 \) belongs to every closed set containing \( A \) and hence to their intersection, the closure \( \Cl A \).
  \end{description}
\end{proof}

\begin{corollary}\label{thm:cluster_point_iff_in_closure}
  A point \( x \in X \) belongs to \( \Cl{A} \) if and only if there exists a net \( \{ x_\al \}_{\al \in \CA} \subseteq A \) for which \( x \) is a cluster point (rather than a limit point as in \fullref{thm:limit_point_iff_in_closure}).

  Compare this result to \fullref{thm:derived_set_properties/closed_iff_contains_all_cluster_points}.
\end{corollary}
\begin{proof}
  Since we are only concerned with existence in \fullref{thm:limit_point_iff_in_closure}, we can replace \enquote{limit point} with \enquote{cluster point} in the statement. This is justified by \fullref{thm:net_convergence_properties/cluster_point_iff_subnet_limit_point}.
\end{proof}

\begin{proposition}\label{thm:cluster_point_of_set_iff_limit_point_of_net}
  The point \( x_0 \in X \) is a cluster point\Tinyref{def:topological_derived_set/cluster_point} of the set \( A \) if and only if it is a limit point\Tinyref{def:net_convergence/cluster} of some net in \( A \setminus \{ x_0 \} \).
\end{proposition}
\begin{proof}\mbox{}
  \begin{description}
    \Implies Let \( x_0 \in \Der(A) \). By \fullref{thm:derived_set_properties/cluster_via_neighborhoods}, every neighborhood \( U \) of \( x_0 \) intersects \( A \setminus \{ x_0 \} \). Choose\LEM \( x_U \in U \cap (A \setminus \{ x_0 \}) \) for every neighborhood \( U \) of \( x_0 \) and form the reverse inclusion net\Tinyref{ex:reverse_inclusion_net} \( \{ x_U \}_{U \in \CT(x)} \). Then \( x_0 \) is a limit point of this net. Furthermore, the net is contained in \( A \setminus \{ x_0 \} \).

    \ImpliedBy Conversely, if \( \{ x_\al \}_{\al \in \CA} \subseteq A \setminus \{ x_0 \} \) is a net and if \( x_0 \) is a limit point of this net, then for every neighborhood \( U \) of \( x_0 \) there exists an index \( \al_U \) such that for \( \al \geq \al_U \) we have \(  x_\al \in U \). In particular, \( U \cap A \) contains elements other than \( x_0 \). Since this is true for any neighborhood \( U \) of \( x_0 \), by \fullref{thm:derived_set_properties/cluster_via_neighborhoods} we conclude that \( x_0 \) is a cluster point of the set\( A \).
  \end{description}
\end{proof}

\begin{corollary}\label{thm:cluster_point_of_set_iff_cluster_point_of_net}
  The point \( x \in X \) is a cluster point\Tinyref{def:topological_derived_set/cluster_point} of the set \( A \) if and only if it is a cluster point\Tinyref{def:net_convergence/cluster} (rather than a limit point as in \fullref{thm:cluster_point_of_set_iff_limit_point_of_net}) of some net in \( A \setminus \{ x \} \).
\end{corollary}
\begin{proof}
  See the proof of \fullref{thm:cluster_point_iff_in_closure}.
\end{proof}

\begin{proposition}\label{thm:net_convergence_via_subbases}
  Fix a topological space \( X \), a point \( x_0 \) and a local subbase\Tinyref{def:topological_local_subbase} \( \CP(x_0) \). The point \( x_0 \) is a limit of the net \( \{ x_\al \}_{\al \in \CA} \subseteq X \) if and only if it is eventually in every element \( U_P \) of the local subbase \( \CP(x_0) \).
\end{proposition}
\begin{proof}\mbox{}
  \begin{description}
    \Implies Obvious consequence of the definition of local subbase.
    \ImpliedBy Fix a neighborhood \( U \) of \( x_0 \). By \fullref{def:topological_local_subbase}, there exists a finite family \( \{ U_k \}_{k=1}^n \subseteq \CP(x_0) \) such that \( \bigcap_{k=1}^n U_k \subseteq U \). Since the net \( \{ x_\al \}_{\al \in \CA} \subseteq X \) is eventually in each of \( U_k, k = 1, \ldots, n \), by transitivity of inclusion it follows that the net is eventually in \( U \).
  \end{description}
\end{proof}
