\subsection{Topological nets}\label{subsec:topological_nets}

In this section, \( X \) will denote an arbitrary nonempty topological space.

\begin{definition}\label{def:topological_net}
  A \Def{net} or \Def{generalized sequence}\Tinyref{def:sequence} or \Def{Moore-Smith sequence} on the nonempty set \( S \) is any function from a directed set\Tinyref{def:order/directed} \( (\CA, \leq) \) to \( S \). For convenience, since nets are simply indexed families\Tinyref{def:indexed_family} over directed sets, we denote them by
  \begin{equation*}
    ( x_\al )_{\al \in \CA},
  \end{equation*}
  because the preorder on the domain \( \CA \) is usually clear from the context.

  We will write \( ( x_\al )_{\al \in \CA} \subseteq S \), despite the net actually being an \( \CA \)-shaped generalized element\Tinyref{def:generalized_element} of \( S \) rather than a subset of \( S \).

  If we know that the net is a sequence, we will usually use the notation for sequences given in \cref{def:sequence}.

  Note that this definition does not actually require a topology on \( S \). Some other important definitions also do not require topologies:
  \begin{defenum}
    \DItem{def:topological_net/frequently_in} We say that \( ( x_\al )_{\al \in \CA} \) is \Def{frequently in} the set \( A \subseteq S \) if for every index \( \al_0 \in \CA \) there exists an index \( \al \geq \al_0 \) such that \( x_\al \in A \).

    \DItem{def:topological_net/eventually_in} We say that \( ( x_\al )_{\al \in \CA} \) is \Def{eventually in} the set \( A \subseteq S \) if there exists an index \( \al_0 \) such that \( x_\al \in A \) whenever \( \al \geq \al_0 \). This is obviously a stronger condition.

    \DItem{def:topological_net/subnet}\cite[50]{Engelking1989} We say that the net \( ( x'_{\al'} )_{\al' \in \CA'} \subseteq S \) is a \Def{subnet} of \( ( x_\al )_{\al \in \CA} \subseteq S \) if there exists an embedding function \( \varphi: \CA' \to \CA \) such that
    \begin{itemize}
      \item To every \( \al \in \CA \) there corresponds \( \al' \in \CA' \) such that \( \varphi(\al') \geq \al \).
      \item For every \( \al' \in \CA' \) we have \( x_{\varphi(\al')} = x_{\al'} \).
    \end{itemize}
  \end{defenum}
\end{definition}

\begin{proposition}\label{thm:topological_net_eventually_in_iff_not_frequently_in_complement}
  The net \( ( x_\al )_{\al \in \CA} \subseteq S \) is eventually in \( A \subseteq S \) if and only if it is not frequently in \( S \setminus A \).
\end{proposition}
\begin{proof}
  Suppose\LEM that \( ( x_\al )_{\al \in \CA} \) is both eventually in \( A \) and frequently in \( S \setminus A \).

  Since the net is eventually in \( A \), we can fix an index \( \al_0 \) such that \( x_\al \in A \) whenever \( \al \geq \al_0 \).

  Since the net is frequently in \( A \), we can fix an index \( \al_1 \geq \al_0 \) such that \( \al_1 \in S \setminus A \), which is a contradiction.

  This proves that the two conditions are incompatible.
\end{proof}

\begin{definition}\label{def:topological_net_convergence}
  Let \( X \) be a topological space and \( (x_\al)_{\al \in \CA} \subseteq X \) be a net.

  \begin{defenum}
    \DItem{def:topological_net_convergence/cluster} If the net is frequently in every neighborhood of \( x \), we say that \( x \) is a \Def{cluster point} of \( (x_\al)_{\al \in \CA} \subseteq X \).

    \DItem{def:topological_net_convergence/limit} If the net is eventually in every neighborhood of \( x \), we say that \( x \) is a \Def{limit point} of \( (x_\al)_{\al \in \CA} \subseteq X \).
  \end{defenum}

  In general, there can exist multiple limit points (see \cref{ex:multiple_limit_points_of_net}) and even more cluster points (see \cref{ex:cluster_points/sine}). In Hausdorff spaces, however, limits are unique by \cref{thm:t2_iff_singleton_limits}.

  If \( (x_\al)_{\al \in \CA} \subseteq X \) has a unique limit, we say that the net \Def{converges} to \( x \) use the notation
  \begin{equation*}
    x = \lim_{\al \in \CA} x_\al.
  \end{equation*}

  If the net is a sequence\Tinyref{def:sequence}, we also use the following notations:
  \begin{itemize}
    \item \( x = \lim_{k \to \infty} x_k \)
    \item \( x = \lim x_k \)
    \item \( x_k \xrightarrow[k \to \infty]{} x \)
    \item \( x_k \to x \)
  \end{itemize}
\end{definition}

\begin{example}\label{ex:multiple_limit_points_of_net}
  Even limits of sequences need not be unique in arbitrary topological spaces. Let \( X = \{ y, z \} \) be a binary set with the indiscrete topology\Tinyref{def:standard_topologies/indiscrete} \( \{ \varnothing, X \} \). L

  Define the following sequence\Tinyref{def:sequence}
  \begin{align*}
    x_k \coloneqq \begin{cases}
      y, &k \text{ is even}, \\
      z, &k \text{ is odd}.
    \end{cases}
  \end{align*}

  The only neighborhood of \( y \), the whole space \( X \), contains all members of the sequence, therefore \( y \) is a limit point of the sequence. The same is true for \( z \), however.
\end{example}

\begin{example}\label{ex:cluster_points/sine}
  Consider the net \( ( \sin \al )_{\al \in \BR} \). It has no limit point and every real number in the interval \( [-1, 1] \) is a cluster point.
\end{example}

\begin{proposition}\label{thm:sequence_converges_iff_almost_entirely_in_neighborhood}
  A sequence \( \{ x_k \}_{k=1}^\infty \subseteq X \) converges to \( x \in X \) if and only if, given a neighborhood \( U \) of \( x \), only finitely many elements of the sequence are outside of \( U \).
\end{proposition}
\begin{proof}
  This is simply a restatement of \cref{def:topological_net_convergence/limit} for the special case of sequences.
\end{proof}

\begin{proposition}\label{thm:limit_point_iff_in_closure}\cite[proposition 1.6.3]{Engelking1989}
  Fix a set \( A \subseteq X \). A point \( x \in X \) belongs to \( \Cl{A} \) if and only if there exists a net \( \{ x_\al \}_{\al \in \CA} \subseteq A \) such that \( x = \lim_{\al \in \CA} x_\al \).
\end{proposition}
\begin{proof}
  The complement of the empty set is the empty set, hence the statement of the proposition holds vacuously. Assume that \( A \) is nonempty.

  \begin{description}
    \Implies Suppose that \( x \in \Cl{A} \). If \( x \in A \), then the one-element net \( (x) \) converges to \( x \).

    If \( x \in \partial{A} \), by \cref{thm:topological_boundary_properties/neighborhoods}, every neighborhood of \( x \) contains points from \( A \). Therefore we can build a net \( (x_U)_{U \in \CT(x)} \) by choosing\AOC an element \( x \in U \cap A \) for every neighborhood \( U \) of \( x \). It is then immediate that \( (x_U)_{U \in \CT(x)} \) converges to \( x \).

    \ImpliedBy Let \( x \) be a limit point of \( (x_\al)_{\al \in \CA} \subseteq A \). We will show that \( x \) belongs every closed set that contains \( A \).

    Let \( F \supseteq A \) be a closed set. Denote \( U \coloneqq X \setminus F \). Suppose\LEM that \( x \in U \). Then \( U \) is a neighborhood \( x \) and, by \cref{def:topological_net_convergence/limit}, the net \( (x_\al)_{\al \in \CA} \subseteq A \) is eventually in \( U \). But \( U \) does not contains \( A \).

    The obtained contradiction shows that \( x \) belongs to every closed set containing \( A \) and hence to their intersection, the closure \( \Cl A \).
  \end{description}
\end{proof}

\begin{proposition}\label{thm:cluster_point_iff_in_subsequence_limit}
  A point \( x \in X \) is a cluster point of the net \( (x_\al)_{\al \in \CA} \subseteq X \) if and only if there exists a subnet that converges to \( x \).
\end{proposition}
\begin{proof}
  \begin{description}
    \Implies The definition of a cluster point (\cref{def:topological_net_convergence/cluster}) allows us to choose\AOC a subnet, indexed by neighborhoods of the point, that converges to the cluster point (similar to the sufficiency proof in \cref{thm:limit_point_iff_in_closure}).

    \ImpliedBy This is an immediate consequence of \cref{def:topological_net_convergence/cluster} and \cref{def:topological_net/subnet}.
  \end{description}
\end{proof}
