When adding additional structure to posets to obtain lattices, it is customary to set aside the order relation and work with \uline{joins} and \uline{meets} instead of suprema and infima. Additionally, lattices may be defined as algebraic structures with joins and meets as operations and with no reference to order relations.

\begin{definition}\label{def:join_meet}\cite[28]{Lectures:general_topology}
  Let $(X, \leq)$ be a poset\Tinyref{def:poset}. We define \uline{joins $\lor$} and \uline{meets $\land$} as the partial\Tinyref{def:function/partial} functions
  \begin{align*}
    &\lor: \Power(X) \to X
    &&\land: \Power(X) \to X
    \\
    &\lor(A) \coloneqq \sup X
    &&\land(A) \coloneqq \inf X.
  \end{align*}

  For finite sets, we usually use the infix notation $x_1 \lor \ldots \lor x_n$ instead of $\lor \{ x_1, \ldots, x_n \}$. Both joins and meets are obviously associative and commutative.
\end{definition}

\begin{definition}\label{def:lattice}\cite[28]{Lectures:general_topology}
  A poset $X$ is called a \uline{lattice} if it has
  \begin{description}
    \DItem{Bottom}{def:lattice/top} a \uline{top element} $\top$, such that $\top = \lor X$.
    \DItem{Top}{def:lattice/bottom} a \uline{bottom element} $\bot$, such that $\bot = \land X$.
    \DItem{Finite joins}{def:lattice/join} all finite joins\Tinyref{def:join_meet} exist.
    \DItem{Finite meets}{def:lattice/meet} all finite meets\Tinyref{def:join_meet} exist.
  \end{description}

  If the last two properties hold for all joins and meets, not necessarily finite, we say that the lattice is a \uline{full lattice}.
\end{definition}

\begin{definition}\label{def:distributive_lattice}\cref{nLab:distributive_lattice}
  A lattice $(X, \top, \bot, \lor, \land)$ is called a \uline{distributive lattice} if any of the following two equivalent distributive axioms hold for all $x, y, z \in X$:
  \begin{itemize}
    \item $x \land (y \lor z) = (x \land y) \lor (x \land z)$
    \item $x \lor (y \land z) = (x \lor y) \land (x \lor z)$
  \end{itemize}
\end{definition}

\begin{definition}\label{def:boolean_algebra}\cref{nLab:boolean_algebra}
  Let $X$ be a distributive lattice. A \uline{complement} of $x$ is an element $y$ of $X$ such that
  \begin{align*}
    x \lor y = \top && x \land y = \bot.
  \end{align*}

  Since in a distributive lattice complements are unique\Tinyref{thm:boolean_algebra_properties/unique_complement}, the complement of $x$ is denoted by $\neg x$. If all elements of $X$ have complements, then $(X, \top, \bot, \lor, \land, \neg)$ is called a \uline{Boolean algebra}.
\end{definition}

\begin{proposition}\label{thm:boolean_algebra_properties}
  A Boolean algebra $X$ has the following basic properties:
  \begin{defenum}
    \item\label{thm:boolean_algebra_properties/unique_complement} For each $x \in X$, there exists a unique complement $\neg x$.
    \item\label{thm:boolean_algebra_properties/double_complement} For each $x \in X$, we have $x = \neg \neg x$.
  \end{defenum}
\end{proposition}
