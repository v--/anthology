\subsection{Polynomials}\label{subsec:polynomials}

\begin{remark}\label{remark:polynomial_commutative_ring}
  In this subsection, \( R \) will refer to a commutative\Tinyref{def:algebraic_theory/commutativity} unital\Tinyref{def:algebraic_theory/identity} ring\Tinyref{def:semiring/ring}. Polynomials are usually only defined over commutative rings.
\end{remark}

\begin{definition}\label{def:polynomial}\cite[149]{Knapp2016BAlg}
  A \Def{polynomial} \( p \) over \( R \) is a sequence of members of \( R \) called \Def{coefficients},
  \begin{equation*}
    p \coloneqq ( a_0, a_1, a_2, \ldots ) \subseteq R,
  \end{equation*}
  such that only finitely many coefficients are nonzero.

  \begin{defenum}
    \DItem{def:polynomial/zero_polynomial} An exception to most rules is the \Def{zero polynomial}, all of whose coefficients are zeroes.

    \DItem{def:polynomial/leading_coefficient} The last nonzero coefficient of a nonzero polynomial is called the \Def{leading coefficient} and is denoted by \( \LC(p) \).

    \DItem{def:polynomial/degree} The zero-based index of the leading coefficient is called the \Def{degree} of the polynomial as is denoted by \( \deg(p) \). That is, if only \( a_0 \) is nonzero, then \( \deg(p) = a_0 \). The degree of the zero polynomial is left undefined.

    \DItem{def:polynomial/expression} It is conventional to write a polynomial of degree \( k \) as the expression\Tinyref{def:language}
    \begin{equation*}
      p(x) \coloneqq a_n x^n + a_{n-1} x^{n-1} + \ldots + a_2 x^2 + a_1 x + a_0 = \sum_{i=0}^k a_k x^k
    \end{equation*}
    and the zero polynomial as
    \begin{equation*}
      p(x) \coloneqq 0.
    \end{equation*}

    This has the downside of introducing confusion between the polynomial and its corresponding function\Tinyref{def:polynomial/function}.

    \DItem{def:polynomial/function} To each polynomial corresponds a \Def{polynomial function}, defined by the formal expression in \cref{def:polynomial/expression}. In general, multiple polynomials can have the same function.

    \DItem{def:polynomial/constant} The zero polynomial and the polynomials of degree \( 1 \) are called the \Def{constant polynomials}. We will implicitly use the canonical embedding, which sends an element \( r \) of \( R \) into the polynomial \( p(x) \coloneqq r \).
  \end{defenum}
\end{definition}

\begin{definition}\label{def:algebra_of_polynomials}
  Denote by \( R[x] \) the set of polynomials over \( R \). Note that it is bijective with \( c_{00} \) and we can inherit pointwise addition and scalar multiplication from there. That is,

  \begin{defenum}
    \DItem{def:algebra_of_matrices/addition} We define \Def{polynomial addition} point as
    \begin{align*}
      &+: R[x] \times R[x] \to R[x] \\
      &(a_0, a_1, \ldots) + (b_0, b_1, \ldots) \coloneqq (a_0 + b_0, b_0 + b_1, \ldots)
    \end{align*}

    \DItem{def:algebra_of_matrices/scalar_multiplication} We define \Def{scalar multiplication} as
    \begin{align*}
      &\cdot: R[x] \times R[x] \to R[x] \\
      &t \cdot (a_0, a_1, \ldots) \coloneqq (t a_0, t b_1, \ldots)
    \end{align*}

    \DItem{def:algebra_of_matrices/matrix_multiplication} In order to make \( R[x] \) into an algebra\Tinyref{def:algebra_over_ring}, we define \Def{polynomial multiplication} \( \odot: R[x] \times R[x] \to R[x] \) as follows: if \( (a_0, a_1, \ldots) \) and \( (b_0, b_1, \ldots) \) are polynomials, their product is defined to be the polynomial with coefficients
    \begin{equation}
      c_k \coloneqq \sum_{i+j=k} a_i b_j, k = 0, 1, \ldots.
    \end{equation}

    Polynomial multiplication is bilinear, associative and commutative.
  \end{defenum}

  We usually refer to \( R[x] \) as the \Def{polynomial ring}.
\end{definition}

\begin{theorem}\label{thm:polynomial_monomorphism_into_polynomial_functions}
  The homomorphism \( \Phi: R[x] \to \Cat{Set}(R, R) \), which assigns a polynomial function\Tinyref{def:polynomial/function} for each polynomial, is injective when \( R \) is a field of characteristic \( 0 \).
\end{theorem}

\begin{remark}\label{remark:polynomial_vs_polynomial_function}
  Since we mostly care about polynomials over \( \R \) and \( \Co \), \cref{thm:polynomial_monomorphism_into_polynomial_functions} allows us to identify polynomial functions with their sequences of coefficients.
\end{remark}
