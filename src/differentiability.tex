Let $X$ and $Y$ be Hausdorff topological vector spaces\Tinyref{def:topological_vector_space}, let $D \subseteq X$ be open and let $D: X \to Y$ be any function.

\begin{definition}\label{def:derivatives}
  We fix a point $x \in D$ and a direction $h \in S_X$. We introduce a few definitions of derivatives. In all cases we say that \uline{the derivative (of the corresponding type) exists for $f$ at $x$ in the direction $h$}. The quotient under the limit sign is called a \uline{difference quotient} (of the corresponding type).

  \begin{defenum}
    \item\label{def:derivatives/classical} The classical \uline{two-sided derivative} is defined as
    \begin{align*}
      f'(x)(h) \coloneqq \lim_{t \to 0} \frac {f(x + th) - f(x)} t
    \end{align*}

    This definition is often too strict and so there exist a few generalizations.

    \item\label{def:derivatives/onesided}\cite[lemma 1.2]{Phelps1993} The \uline{one-sided (or right-hand) directional derivative} is defined as
    \begin{align*}
      f_+'(x)(h) \coloneqq \lim_{t \downarrow 0} \frac {f(x + th) - f(x)} t
    \end{align*}

    It is also denoted as $\partial^+ f(x)(h)$ in~\cite[lemma 1.2]{Phelps1993}. We do not need to define a left-hand directional derivative because it would equal $-f_+'(x)(-h)$.

    \item\label{def:derivatives/dini}\cite[definition 11.18]{Clarke2013} The \uline{upper (resp. lower) Dini derivative} is defined as
    \begin{align*}
      \overline f'(x)(h) &\coloneqq \limsup_{t \downarrow 0} \frac {f(x + th) - f(x)} t
      \\
      \underline f(x)(h) &\coloneqq \liminf_{t \downarrow 0} \frac {f(x + th) - f(x)} t
    \end{align*}

    Dini derivatives are useful when the difference quotients are bounded but do not have a limit.

    \item\label{def:derivatives/clarke}\cite[Section 10.1]{Clarke2013} The \uline{generalized Clarke derivative} is defined as
    \begin{align*}
      f^\circ(x)(h)
      &\coloneqq
      \limsup_{\substack{y \to x \\ t \downarrow 0}} \frac {f(y + th) - f(y)} t
      =
      \lim_{\delta \to 0} \sup_{\substack{y \in B(x, \delta) \\ t \in (0, \delta)}} \frac {f(y + th) - f(y)} t.
    \end{align*}

    Refer to~\cref{sec:clarke_gradients} for their usefulness.
  \end{defenum}
\end{definition}

\begin{definition}\label{def:differentiability}
  We fix a point $x \in D$. We will now introduce several types of differentiability. Each is implied by the next one.

  \begin{defenum}
    \item We do not introduce a special name for functions that have a one-sided derivative at $x$ in the direction $h$, although some authors call these functions \uline{Gateaux differentiable at $x$ in the direction $h$}. % TODO: Give examples of such authors

    \item\label{def:differentiability/two-sided}\cite[0.2.1]{Ioffe1974} If the two-sided directional derivative $f'(x)(h)$ exists for all directions $h \in S_X$, we call it the \uline{first variation at $x$} and denote the corresponding linear operator $\delta f(x): X \to Y$.

    \item\label{def:differentiability/gateaux}\cite[definition 1.12]{Phelps1993} Let $X$ be a Banach space. We say that $f$ is Gateaux-differentiable at $x$ if there exists a continuous linear operator $f'_G(x): X \to Y$, called the Gateaux derivative of $f$ at $x$, such that
    \begin{align*}
      f'_G(x)(h) = \lim_{t \to 0} \frac {f(x + th) - f(x)} t.
    \end{align*}

    The Gateaux derivative $f'_G(x)$ exists precisely when the first variation $\delta f(x)$ operator exists and is continuous. They are obviously equal.

    The Gateaux derivative is also denoted by $df(x)$ in~\cite[definition 1.12]{Phelps1993}.

    \item\label{def:differentiability/frechet}\cite[definition 1.12]{Phelps1993} Let $X$ be a Banach space. We say that $f$ is Frechet-differentiable at $x$ if there exists a continuous linear operator $f'(x): X \to Y$, called the Frechet derivative of $f$ at $x$, such that for each $\varepsilon > 0$ there exists a radius $\delta > 0$ such that and for every direction $h \in S_X$ we have
    \begin{align*}
      t \in (0, \delta) \implies \Norm{ \frac {f(x + th) - f(x)} t - f'(x)(h)} < \varepsilon.
    \end{align*}

    Note that for each $\varepsilon > 0$, Gateaux differentiability gives us a radius $\delta_h > 0$ such that
    \begin{align*}
      t \in (0, \delta_h) \implies \Norm{ \frac {f(x + th) - f(x)} t - f_G'(x)(h)} < \varepsilon.
    \end{align*}

    If the limit is uniform over $h \in S_X$, i.e. if $\sup_{h \in S_X} \delta_h < \infty$, then $f$ is Frechet differentiable at $x$ and $f'(x) = f'_G(x)$.

    \item\label{def:differentiability/strong}\cite[33]{Dontchev2014} We say that \uline{$f$ is strictly differentiable at $x$} if there exists a continuous linear operator $f'(x): X \to Y$ such that
    \begin{align*}
      \lim_{\substack{y \to x \\ z \to x}} \frac{f(y) - f(z) - f'(x)(y - z)} {\Norm{y - z}} = 0.
    \end{align*}
  \end{defenum}
\end{definition}
