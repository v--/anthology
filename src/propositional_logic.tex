\subsection{Propositional logic}\label{subsec:propositional_logic}

\begin{remark}\label{remark:metalanguage}
  Mathematical logic uses mathematics to study languages that themselves describe mathematics. Thus a distinction should be made between \Def{the language} under study, and \Def{the metalanguage} which we use to write statements relating to the language under study. This distinction is often emphasized by using different logical symbols in the two languages.

  Outside of logic, \fullref{def:propositional_language/connectives} are frequently used, however the use of connectives in statements in the metalanguage within logic can be reduced to a minimum and not cause any confusion. At the same time, \hyperref[def:propositional_language/constants]{truth values} are used both in the language and in the metalanguage, so we use the following convention:
  \begin{itemize}
    \item The \( \top \) and \( \bot \) symbols represent truth and false values in the language.
    \item The \( T \) and \( F \) symbols represent truth and false values in the metalanguage.
  \end{itemize}

  We denote equality in the language by \( \doteq \).
\end{remark}

\begin{definition}\label{def:propositional_language}\MarginCite[102]{OpenLogic20201202}
  Propositional logic is a simple framework for describing relationships between statements. It is sometimes called boolean logic because of \fullref{thm:propositional_logic_boolean_algebra}.

  The \hyperref[def:language]{language} of propositional logic consists of \Def{propositional formulas}. These formulas are formed over a set of symbols, called the \Def{propositional alphabet}, whose constituents are described below.

  \begin{DefEnum}
    \ILabel{def:propositional_language/prop} A nonempty set \( \Bold{Prop} \) of \Def{propositional variables}.

    \ILabel{def:propositional_language/constants} Two \Def{propositional constants} (also known as \Def{truth values}):
    \begin{DefEnum}
      \ILabel{def:propositional_language/constants/top}\Def{top} \( \top \)
      \ILabel{def:propositional_language/constants/bottom}\Def{bottom} \( \bot \)
    \end{DefEnum}

    \ILabel{def:propositional_language/negation}\Def{negation} \( \neg \)
    \ILabel{def:propositional_language/connectives} The set \( \Sigma \) of \Def{propositional connectives}, namely
    \begin{DefEnum}
      \ILabel{def:propositional_language/connectives/conjunction} \Def{conjunction} \( \wedge \) (also known as \Def{and} or \hyperref[def:lattice_operations]{\Def{meet}})
      \ILabel{def:propositional_language/connectives/disjunction} \Def{disjunction}\( \vee \) (also known as \Def{or} or \hyperref[def:lattice_operations]{\Def{join}})
      \ILabel{def:propositional_language/connectives/implication} \Def{implication} \( \implies \) (see \fullref{def:material_implication})
      \ILabel{def:propositional_language/connectives/equivalence} \Def{equivalence} \( \iff \) (see \fullref{def:equivalence})
      \ILabel{def:propositional_language/connectives/pierces_arrow} \Def{Pierce's arrow} \( \downarrow \) (also known as \Def{nand})
      \ILabel{def:propositional_language/sheffer_stroke} \Def{Sheffer's stroke} \( \uparrow \) (also known as \Def{nor})
    \end{DefEnum}

    \ILabel{def:propositional_language/parentheses} Parentheses \( ( \) and \( ) \) for defining the order of operations unambiguously (see \fullref{remark:propositional_formula_parentheses}).
  \end{DefEnum}

  \Fullref{remark:minimal_propositional_language} Allows us to start with a smaller language and define the rest using \hyperref[def:boolean_closure]{Boolean closures}.
\end{definition}

\begin{definition}\label{def:propositional_formulas}
  The propositional formulas \( \CF_B \) are defined inductively as
  \begin{itemize}
    \item The variables in \( \Bold{Prop} \) are formulas.
    \item The constants \( \top \) and \( \bot \) are formulas.
    \item If \( \varphi \) is a formula, then \( \neg \varphi \) is a formula.
    \item If \( \varphi \) and \( \psi \) are formulas, so are \( (\varphi \circ \psi) \), where \( \circ \in \Sigma \).
  \end{itemize}

  Furthermore, the grammar of propositional formulas is unambiguous (seed \fullref{thm:propositional_formulas_are_unambiguous}).

  If \( \varphi \) and \( \psi \) are formulas and \( \psi \) is a subword of \( \varphi \), we say that \( \psi \) is a \Def{subformula} of \( \varphi \).

  For each formula \( \varphi \), we define the set of its variables as
  \begin{align*}
    \Bold{Var}(\varphi) \coloneqq \begin{cases}
      \varnothing,                              & \varphi \in \{ \top, \bot \}                     \\
      \{ P \},                                  & \varphi = P \in \Bold{Prop}                      \\
      \Bold{Var}(\psi),                         & \varphi = \neg \psi                              \\
      \Bold{Var}(\psi) \cup \Bold{Var}(\theta), & \varphi = (\psi \circ \theta), \circ \in \Sigma.
    \end{cases}
  \end{align*}
\end{definition}

\begin{definition}\label{def:material_implication}
  Fix the formula \( \varphi \coloneqq P \implies Q \). We call formulas of this form \Def{material implications}.

  \begin{DefEnum}
    \ILabel{def:material_implication/antecedent} We call \( P \) the \Def{antecedent} of \( \varphi \).

    \ILabel{def:material_implication/consequent} We call \( Q \) the \Def{consequent} of \( \varphi \).

    \ILabel{def:material_implication/inverse} We call the formula \( \neg P \implies \neg Q \) the \Def{inverse} of \( \varphi \).

    \ILabel{def:material_implication/converse} We call the formula \( Q \implies P \) the \Def{converse} of \( \varphi \).

    \ILabel{def:material_implication/contrapositive} We call the formula \( \neg Q \implies \neg P \) the \Def{contrapositive} of \( \varphi \). It is Boolean \hyperref[def:propositional_interpretation]{equivalent} to the original formula by \fullref{thm:boolean_equivalences}.
  \end{DefEnum}
\end{definition}

\begin{definition}\label{def:equivalence}
  Fix the formula \( \varphi \coloneqq P \iff Q \). We call formulas of this form \Def{logical equivalence}. Note that
  \begin{equation*}
    P \iff Q \models_B (P \implies Q) \wedge (Q \implies P).
  \end{equation*}

  Despite the symmetry of \( \wedge \), there is an ordering in the set \( \{ P, Q \} \) of propositions and we use this ordering. For example, instead of \( Q \implies P \), we write \( P \impliedby Q \). We say that \( P \) is a \Def{necessary condition} for \( Q \) and that \( Q \) is a \Def{sufficient condition} for \( P \).
\end{definition}

\begin{remark}\label{remark:statements_as_implications}
  Theorems in mathematics usually take the form of a material \hyperref[def:material_implication]{implication} \( P \implies Q \) or even \hyperref[def:equivalence]{equivalence} \( P \iff Q \). Therefore the terminology of \fullref{def:material_implication} and \fullref{def:equivalence} usually applies to statements about mathematics.
\end{remark}

\begin{definition}\label{def:propositional_substition}
  If \( \varphi \) and \( \rho \) are formulas and \( \psi \) is a subformula of \( \varphi \), we define the \Def{substition} of \( \psi \) with \( \rho \) in \( \varphi \) as
  \begin{align*}
    \varphi[\psi \to \rho] & \coloneqq \begin{cases}
      \rho,                                                    & \varphi = \psi                                                                        \\
      \varphi,                                                 & \varphi \neq \psi \text{ and } \varphi \in \{ \top, \bot \} \cup \Bold{Prop}          \\
      \neg \theta[\psi \to \rho],                              & \varphi \neq \psi \text{ and } \varphi = \neg \theta                                  \\
      (\theta_1[\psi \to \rho] \circ \theta_2[\psi \to \rho]), & \varphi \neq \psi \text{ and } \varphi = (\theta_1 \circ \theta_2), \circ \in \Sigma.
    \end{cases}
  \end{align*}
\end{definition}

\begin{remark}\label{remark:propositional_formula_parentheses}
  We often omit parentheses from propositional formulas when no conceptual ambiguity is possible, for example we often write \( \varphi \wedge \psi \wedge \theta \) instead of \( ((\varphi \wedge \psi) \wedge \theta) \). These are only notations shortcuts in the \hyperref[remark:metalanguage]{metalanguage} and the formulas themselves (as abstract mathematical objects) are still assumed to contain parentheses to avoid syntactic ambiguity (see \fullref{thm:propositional_formulas_are_unambiguous}).
\end{remark}

\begin{proposition}\label{thm:propositional_formulas_are_unambiguous}
  The \hyperref[def:grammar]{grammar} of \hyperref[def:propositional_language]{propositional formulas}
  \begin{displaymath}
    \begin{aligned}
       & \Phi \to p,                 &  & p \in \Bold{Prop} \\
       & \Phi \to \top \;|\; \bot    &  &                   \\
       & \Phi \to \neg \Phi          &  &                   \\
       & \Phi \to (\Phi \circ \Phi), &  & \circ \in \Sigma.
    \end{aligned}
  \end{displaymath}
  is \hyperref[def:ambiguous_grammar]{unambiguous}.
\end{proposition}
\begin{proof}
  The proof is analogous to \fullref{ex:context_free_grammar/real_arithmetic}.
\end{proof}

\begin{definition}\label{def:conjunctive_normal_form}
  We define \Def{literals} to either be propositional variables \( L = P \) or negations \( L = \neg P \) of propositional variables.

  We define \Def{disjuncts} (resp. \Def{conjuncts}) to be finite disjunctions (resp. conjunctions) of literals, i.e.
  \begin{equation*}
    L_1 \vee L_2 \vee \ldots \vee L_n.
  \end{equation*}

  We say that a propositional formula \( \varphi \) is in \Def{conjunctive normal form} (resp. \Def{disjunctive normal form}) if \( \varphi \) is a finite conjunction of disjunctions (resp. finite disjunction of conjunctions).
\end{definition}

\begin{proposition}\label{thm:conjunctive_normal_form_reduction}
  Every propositional formula \( \varphi \) is Boolean \hyperref[def:propositional_interpretation]{equivalent} to a formula in conjunctive normal \hyperref[def:conjunctive_normal_form]{form}.
\end{proposition}

\begin{definition}\label{def:truth_functions}
  We define the following auxiliary truth functions
  \begin{equation*}
    \begin{tabular}{c | c || c c | c c c c c c}
      \( x \) & \( H_\neg \) & \( x \) & \( y \) & \( H_\vee \) & \( H_\wedge \) & \( H_\Rightarrow \) & \( H_{\iff} \) & \( H_\downarrow \) & \( H_\uparrow \) \\
      \hline
      \( T \) & \( F \)      & \( T \) & \( T \) & \( T \)      & \( T \)        & \( T \)             & \( T \)        & \( F \)            & \( F \)          \\
      \( F \) & \( T \)      & \( T \) & \( F \) & \( T \)      & \( F \)        & \( F \)             & \( F \)        & \( F \)            & \( T \)          \\
              &              & \( F \) & \( T \) & \( T \)      & \( F \)        & \( T \)             & \( T \)        & \( F \)            & \( T \)          \\
              &              & \( F \) & \( F \) & \( F \)      & \( F \)        & \( T \)             & \( F \)        & \( T \)            & \( T \)
    \end{tabular}
  \end{equation*}
\end{definition}

\begin{definition}\label{def:propositional_interpretation}
  A \Def{propositional interpretation} is a function \( I: \Bold{Prop} \to \{ T, F \} \).

  We define interpretation for formulas inductively as
  \begin{align*}
    \varphi[I] \coloneqq \begin{cases}
      T,                           & \varphi = \top                                 \\
      F,                           & \varphi = \bot                                 \\
      I(P),                        & \varphi = P \in \Bold{Prop}                    \\
      H_\neg(\psi[I]),             & \varphi = \neg \psi                            \\
      H_\circ(\psi[I], \theta[I]), & \varphi = \psi \circ \theta, \circ \in \Sigma.
    \end{cases}
  \end{align*}

  We introduce some directly related notions:
  \begin{DefEnum}
    \ILabel{def:propositional_interpretation/model} If \( \varphi[I] = T \), we say that \( I \) is a \Def{Boolean model} of \( \varphi \) and write \( I \models_B \varphi \).
    \ILabel{def:propositional_interpretation/entailment} If every model for \( \Gamma \) is a model for \( \varphi \), we say that the set \( \Gamma \) of formulas \Def{entails} the formula \( \varphi \) (denoted as \( \Gamma \models_B \varphi \)).
    \ILabel{def:propositional_interpretation/tautology} If all interpretations are models for \( \varphi \), we call \( \varphi \) a \Def{tautology}.
    \ILabel{def:propositional_interpretation/contradiction} Dually, if no interpretation is a model for \( \varphi \), \( \varphi \) is a \Def{contradiction}.
    \ILabel{def:propositional_interpretation/equivalence} If \( \varphi[I] = \psi[I] \) for any interpretation \( I \), we say that \( \varphi \) and \( \psi \) are \Def{Boolean equivalent} and write \( \varphi \equiv_B \psi \).
  \end{DefEnum}
\end{definition}

\begin{definition}\label{def:propositional_theory}\MarginCite[def. 15.1]{OpenLogic20201202}
  The \Def{closure} of a set of formulas \( \Gamma \) is defined as
  \begin{equation*}
    \Cl(\Gamma) \coloneqq \{ \varphi(x) \colon \Gamma \models_B \varphi \}.
  \end{equation*}

  A set of formulas is said to be \Def{closed} if it coincides with its closure.

  If \( \Delta \subseteq \Gamma \) and \( \Gamma = \Cl(\Delta) \), we say that \( \Gamma \) is a \Def{theory} that is \Def{axiomatized} by \( \Delta \) and that \( \Delta \) are \Def{axioms} for \( \Gamma \).

  We are usually interested in \Def{minimal} systems of axioms, that is, sets of axioms which are minimal with respect to set inclusion.
\end{definition}

\begin{example}\label{ex:axiomatic_theory}
  Examples of \hyperref[def:propositional_theory]{axiomatic theories} include:
  \begin{itemize}
    \item The minimal propositional language (see \fullref{remark:minimal_propositional_language}).
    \item The first order equality axioms (see \fullref{remark:first_order_equality}).
    \item Peano arithmetic (see \fullref{def:peano_arithmetic}).
    \item Algebraic theories (see \fullref{def:algebraic_theory}).
  \end{itemize}
\end{example}

\begin{remark}\label{remark:minimal_propositional_language}
  By \fullref{ex:posts_completeness_theorem/nand}, it is actually sufficient for propositional logic to only have
  \begin{itemize}
    \item A set \( \Bold{Prop} \) of \hyperref[def:propositional_language/prop]{propositional variables}.
    \item Only a single connective - either \hyperref[def:propositional_language/connectives/pierces_arrow]{Pierce's arrow} \( \downarrow \) or \hyperref[def:propositional_language/sheffer_stroke]{Sheffer's stroke} \( \uparrow \). We will use Pierce's arrow \( \downarrow \).
    \item \hyperref[def:propositional_language/parentheses]{Parentheses}.
  \end{itemize}

  We are able to expand the language and define the constants, negation and all other connectives so that the truth functions in \fullref{def:truth_functions} are satisfied. This can simplify the study of propositional logic itself because it allows us to examine less cases in our inductive\IND proofs.

  Equivalence is a bit tricky to define since we use it to define the other logical operations. We choose\AOC two variables \( P \) and \( Q \) and define an auxiliary formula that expresses the equivalence \( P \iff Q \) through Pierce's arrow:
  \begin{equation*}
    \lambda \coloneqq (P \downarrow (Q \downarrow Q)) \downarrow (Q \downarrow (P \downarrow P)).
  \end{equation*}

  We can now introduce the following (infinite system of) axioms (see \fullref{def:propositional_theory}) for all formulas \( \varphi \) and \( \psi \):
  \begin{RefList}
    \IRef{def:propositional_language/connectives/equivalence} \( \lambda[P \to (\varphi \iff \psi), Q \to \lambda[P \to \varphi, Q \to \psi]] \)
    \IRef{def:propositional_language/negation} \( \neg \varphi \iff (\varphi \downarrow \varphi) \)
    \IRef{def:propositional_language/connectives/conjunction} \( (\varphi \wedge \psi) \iff ((\varphi \downarrow \varphi) \downarrow (\psi \downarrow \psi)) \)
    \IRef{def:propositional_language/connectives/disjunction} \( (\varphi \vee \psi) \iff \neg (\varphi \downarrow \psi) \)
    \IRef{def:propositional_language/constants/top} \( \top \iff (\varphi \wedge \varphi) \)
    \IRef{def:propositional_language/constants/bottom} \( \bot \iff \neg \top \)
    \IRef{def:propositional_language/connectives/implication} \( (\varphi \implies \psi) \iff (\neg \varphi \vee \psi) \)
    \IRef{def:propositional_language/sheffer_stroke} \( (\varphi \uparrow \psi) \iff \neg (\varphi \wedge \psi) \)
  \end{RefList}

  The proofs of these equivalences can be easily verified using the truth tables in \fullref{def:truth_functions}.
\end{remark}

\begin{proposition}\label{thm:boolean_equivalence_relation}
  The Boolean \hyperref[def:propositional_interpretation]{equivalence} \( \equiv_B \) is an equivalence relation on the set \( \CF_B \) of propositional formulas.
\end{proposition}

\begin{theorem}\label{thm:propositional_logic_boolean_algebra}
  The \hyperref[def:equivalence_relation]{quotient set} of propositional formulas \( \CF_B / \cong \) forms a boolean \hyperref[def:boolean_algebra]{algebra} with the following:
  \begin{ThmEnum}
    \IRef{def:binary_lattice_operations/join} Finite joins are given by \hyperref[def:propositional_language/connectives/disjunction]{disjunctions (\( \vee \))}.
    \IRef{def:binary_lattice_operations/meet} Finite meets are given by \hyperref[def:propositional_language/connectives/conjunction]{conjunctions (\( \wedge \))}.
    \IRef{def:lattice/top} The top is the equivalence class \( [\top] \) of \hyperref[def:propositional_interpretation/tautology]{tautologies}.
    \IRef{def:lattice/bottom} The bottom is the equivalence class \( [\bot] \) of \hyperref[def:propositional_interpretation/contradiction]{contradictions}.
    \IRef{def:boolean_algebra} The complement of the coset \( [\varphi] \) is the coset \hyperref[def:propositional_language/negation]{\( [\neg \varphi] \)}.
  \end{ThmEnum}
\end{theorem}
\begin{proof}
  See \fullref{remark:infinite_lattice_operations} about handling infinitary joins and meets. Once we prove \fullref{def:algebraic_theory/associativity}, \fullref{def:algebraic_theory/commutativity} and \fullref{def:binary_lattice_operations/absorption}, we can define a partial order on \( \CF_B / \cong \) that allows us to extend \( \vee \) and \( \wedge \) to handle an infinite amount of arguments.

  \SubProofOf{def:algebraic_theory/associativity} The \hyperref[def:truth_functions]{functions} \( H_\vee \) and \( H_\wedge \) are associative, hence the lattice operations are associative.
  \SubProofOf{def:algebraic_theory/commutativity} The proof is analogous to \fullref{def:algebraic_theory/associativity}.
  \SubProofOf{def:binary_lattice_operations/absorption} Let \( \varphi \) and \( \psi \) be propositional formulas and \( I \) be a propositional interpretation. Then
  \begin{equation*}
    \begin{tabular}{c c | c | c}
      \( \varphi[I] \) & \( \psi[I] \) & \( H_\wedge(\psi[I], \varphi[I]) \) & \( H_\vee(\varphi[I], H_\wedge(\psi[I], \varphi[I])) \) \\
      \hline
      \( T \)          & \( T \)       & \( T \)                             & \( T \)                                                 \\
      \( T \)          & \( F \)       & \( F \)                             & \( T \)                                                 \\
      \( F \)          & \( T \)       & \( F \)                             & \( F \)                                                 \\
      \( F \)          & \( F \)       & \( F \)                             & \( F \)
    \end{tabular}
  \end{equation*}
  hence \( \varphi[I] = H_\vee(\varphi[I], H_\wedge(\psi[I], \varphi[I])) \).

  The proof of the dual law is analogous.

  \SubProofOf{def:distributive_lattice} Let \( \varphi \), \( \psi \) and \( \theta \) be propositional formulas and \( I \) be a propositional interpretation. Then
  \begin{equation*}
    \begin{tabular}{c c c | c | c}
      \( \varphi[I] \) & \( \psi[I] \) & \( \theta[I] \) & \small{\( H_\wedge(\varphi[I], H_\vee(\psi[I], \theta[I])) \)} & \small{\( H_\vee(H_\wedge(\varphi[I], \psi[I]), H_\wedge(\varphi[I], \theta[I])) \)} \\
      \hline
      \( T \)          & \( T \)       & \( T \)         & \( T \)                                                        & \( T \)                                                                              \\
      \( T \)          & \( T \)       & \( F \)         & \( T \)                                                        & \( T \)                                                                              \\
      \( T \)          & \( F \)       & \( T \)         & \( T \)                                                        & \( T \)                                                                              \\
      \( T \)          & \( F \)       & \( F \)         & \( F \)                                                        & \( F \)                                                                              \\
      \( F \)          & \( T \)       & \( T \)         & \( F \)                                                        & \( F \)                                                                              \\
      \( F \)          & \( T \)       & \( F \)         & \( F \)                                                        & \( F \)                                                                              \\
      \( F \)          & \( F \)       & \( T \)         & \( F \)                                                        & \( F \)                                                                              \\
      \( F \)          & \( F \)       & \( F \)         & \( F \)                                                        & \( F \)
    \end{tabular}
  \end{equation*}

  The join and meet induce the partial order \( \varphi \leq \psi \iff \varphi \vee \psi \equiv \psi \).

  \SubProofOf{def:lattice/top} Fix an interpretation \( I \). A formula \( \omega \) should belong to the supremum \( \sup \CF_B \) if and only if \( \varphi \vee \omega \equiv \omega \) for any formula \( \varphi \in \CF_B \). If \( \varphi \) is a tautology, \( \varphi[I] = \top \) and thus
  \begin{equation*}
    (\varphi \vee \omega)[I] \coloneqq H_\vee(\varphi[I], \omega[I]) = \top.
  \end{equation*}

  It follows that \( \omega[I] = \top \). Since the interpretation \( I \) was arbitrary, \( \omega \) is also a tautology. Hence the top element is the equivalence class of all tautologies.

  \SubProofOf{def:lattice/bottom} The proof is analogous to \fullref{def:lattice/top}.
\end{proof}

\begin{proposition}\label{thm:boolean_equivalences}
  The following Boolean equivalences are often used:
  \begin{PropEnum}
    \ILabel{def:boolean_equivalences/commutativity} \hyperref[def:propositional_language/connectives/conjunction]{Conjunctions}, \hyperref[def:propositional_language/connectives/conjunction]{disjunctions} or \hyperref[def:propositional_language/connectives/implication]{implications} are \hyperref[def:algebraic_theory/commutativity]{commutative}:
    \begin{align}
      (P \wedge Q) & \cong_B (Q \wedge P) \label{def:boolean_equivalences/commutativity/and} \\
      (P \vee Q)   & \cong_B (Q \vee P) \label{def:boolean_equivalences/commutativity/or}    \\
      (P \iff Q)   & \cong_B (Q \iff P) \label{def:boolean_equivalences/commutativity/iff}
    \end{align}

    \ILabel{def:boolean_equivalences/de_morgan} \hyperref[thm:de_morgans_laws]{Finite De Morgan's laws} hold
    \begin{align}
      ((P \wedge Q) \vee R) & \cong_B ((P \vee R) \wedge (Q \vee R)) \label{def:boolean_equivalences/de_morgan/and_over_or}   \\
      ((P \vee Q) \wedge R) & \cong_B ((P \wedge R) \vee (Q \wedge R)) \label{def:boolean_equivalences/de_morgan/or_over_and}
    \end{align}

    \ILabel{def:boolean_equivalences/equivalence_via_implication} Equivalence is a conjunction of implications
    \begin{equation}\label{def:boolean_equivalences/equivalence_via_implication/axiom}
      (P \iff Q) \cong_B ((P \implies Q) \wedge (Q \implies P))
    \end{equation}

    \ILabel{def:boolean_equivalences/implication_iff_via_or} Implications and equivalences can be represented via conjunction, disjunction and negation:
    \begin{align}
      (P \implies Q) & \cong_B \neg (P \vee Q) \label{def:boolean_equivalences/implication_iff_via_or/implies}                        \\
      (P \iff Q)     & \cong_B ((P \wedge Q) \vee (\neg P \wedge \neg Q)) \label{def:boolean_equivalences/implication_iff_via_or/iff}
    \end{align}
  \end{PropEnum}
\end{proposition}
\begin{proof}
  All equivalences can be easily verified using the truth tables in \fullref{def:truth_functions}.
\end{proof}
