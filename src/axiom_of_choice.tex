\subsection{Axiom of choice}\label{subsec:axiom_of_choice}

\begin{remark}\label{remark:aoc}
  The axiom of choice (\cref{thm:aoc}) famously leads to counterintuitive results, which is a reason why it is frowned upon. It is insanely useful in the form of its many equivalent statements (see \cref{def:set_zfc/A8} and \cref{sec:index}).
\end{remark}

\begin{definition}\label{def:choice_function}
  Let \( X \) be a nonempty set. A \Def{choice function} is a function of type \( f: \Power(X) \to X \) such that, for all subsets \( A \subseteq X \),
  \begin{equation*}
    f(A) \in A.
  \end{equation*}

  That is, the function \( f \) \enquote{chooses} an element out of any subset of \( X \).
\end{definition}

\begin{theorem}\label{thm:aoc}
  The following are equivalent:

  \begin{thmenum}
    \DItem{thm:aoc/transversal}\cite[theorem 6M(4)]{Enderton1977} Every hypergraph has a minimal transversal (compare to \cref{thm:finite_hypergraphs_have_minimal_transversal} that only mentions finite sets).

    \DItem{thm:aoc/product}\cite[theorem 6M(2)]{Enderton1977} The Cartesian product\Tinyref{def:cartesian_product} of a family of nonempty sets is nonempty.

    \DItem{thm:aoc/choice}\cite[theorem 6M(3)]{Enderton1977} For any nonempty set \( X \) there exists a choice function\Tinyref{def:choice_function}.

    \DItem{thm:aoc/function}\cite[theorem 6M(1)]{Enderton1977} Every multivalued function\Tinyref{def:function} has a selection\Tinyref{def:function}.

    \DItem{thm:aoc/cardinals} Cardinal trichotomy (see \cref{thm:cardinal_trichotomy}).

    \DItem{thm:aoc/zorn} Zorn's lemma (see \cref{thm:zorns_lemma}).

    \DItem{thm:aoc/well_ordering_principle} The Well Ordering Principle (see \cref{thm:well_ordering_principle}).

    \DItem{thm:aoc/vector_space_bases} Every vector space has a basis (see \cref{thm:all_vector_spaces_are_free_left_modules}). Compare this to finite-dimensional vector spaces of order \( n \) over \( F \), which are all isomorphic to \( F^n \).

    \DItem{thm:aoc/tychonoff} Tychonoff's product theorem (see \cref{thm:tychonoffs_product_theorem}).
  \end{thmenum}
\end{theorem}
