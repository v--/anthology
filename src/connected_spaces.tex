\subsection{Connected spaces}\label{subsec:connected_sets}

Let \( (X, \T) \) be a topological space.

\begin{definition}\label{def:connected_space}\cite[theorem 6.1.1]{Engelking1989}
  We say that \( X \) is \Def{connected} if it satisfies any of the following equivalent conditions:
  \begin{defenum}
    \DItem{def:connected_space/open_union} If \( X = X_1 \cup X_2 \) and \( X_1, X_2 \) are disjoint open sets, either \( X_1 \) or \( X_2 \) is empty.

    \DItem{def:connected_space/closed_union} If \( X = X_1 \cup X_2 \) and \( X_1, X_2 \) are disjoint closed sets, either \( X_1 \) or \( X_2 \) is empty.

    \DItem{def:connected_space/separated_union} If \( X = X_1 \cup X_2 \) and \( X_1, X_2 \) are separated\Tinyref{def:topological_space_separation}, either \( X_1 \) or \( X_2 \) is empty.

    \DItem{def:connected_space/clopen} The only subsets of \( X \) that are both open and closed are \( \varnothing \) and \( X \).

    \DItem{def:connected_space/discrete_mapping} Every continuous mapping \( f: X \to \{ 0, 1 \} \) into the two-point discrete space is constant.
  \end{defenum}
\end{definition}

\begin{definition}\label{def:locally_connected}\cite[exercise 6.3.3]{Engelking1989}
  We say that \( X \) is \Def{locally connected} if for every point \( x \in X \) and every neighborhood \( U \) of \( x \) there exists a connected set \( C \subseteq U \) such that \( x \in \Int(C) \).
\end{definition}

\begin{definition}\label{def:path_connected}\cite[exercise 6.3.9]{Engelking1989}
  We say that a topological space is \Def{path connected} if every two points can be connected via a path\Tinyref{def:parametric_curve}.
\end{definition}

\begin{definition}\label{def:locally_path_connected}\cite[exercise 6.3.10]{Engelking1989}
  We say that \( X \) is \Def{locally path connected} if for every point \( x \in X \) and every neighborhood \( U \) of \( x \) there exists a neighborhood \( V \) of \( x \) such that for any \( y \in V \) there exists a path \( \gamma: [0, 1] \to U \) connecting \( x \) with \( y \).
\end{definition}

\begin{proposition}\label{thm:homomorphism_preserves_connectedness}
  If \( X \) is connected and \( f: X \to Y \) is a homeomorphism, then \( Y \) is also connected.
\end{proposition}
\begin{proof}
  Let \( Y = Y_1 \cup Y_2 \), where \( Y_1 \) and \( Y_2 \) are disjoint and open.

  Note that the preimages \( \gamma^{-1}(Y_1) \) and \( \gamma^{-1}(Y_2) \) are open and disjoint, hence \( X = \gamma^{-1}(Y_1) \cup \gamma^{-1}(Y_2) \). But \( X \) is connected and by \cref{def:connected_space/open_union}, either \( \gamma^{-1}(Y_1) \) or \( \gamma^{-1}(Y_2) \) is the null set. Thus either \( Y_1 \) and \( Y_2 \) is the null set and, again, by \cref{def:connected_space/open_union}, \( Y \) is connected.
\end{proof}

\begin{proposition}\label{thm:path_connected_implies_connected}
  Any path connected space is connected.
\end{proposition}
\begin{proof}
  Let \( X = X_1 \cup X_2 \), where \( X_1 \) and \( X_2 \) are disjoint and open.

  Assume\LEM that both are nonzero and take \( x_1 \in X_1, x_2 \in X_2 \). Then there exists a path \( \gamma: I \to X \) with endpoints \( x_1 \) and \( x_2 \). Note that the preimages \( \gamma^{-1}(X_1) \) and \( \gamma^{-1}(X_2) \) are nonempty and open, hence cannot be separated by \cref{def:connected_space/separated_union}. But this contradicts the disjointedness of \( X_1 \) and \( X_2 \).

  The obtained contradiction proves that \( X \) is connected.
\end{proof}
