\section{Order theory}\label{sec:order_theory}
\subsection{Orders}\label{subsec:orders}

We assume we have defined a notion of equality in our formal language\Tinyref{def:first_order_language}.

\begin{definition}\label{def:order}
  We will consider binary relations\Tinyref{def:relation} \( \sim\; \subseteq X \times X \) on a nonempty set \( X \).

  \begin{defenum}
    \DItem{def:order/preorder}\cite{nLab:preorder} The relation \( \sim \) is called a \textbf{preorder} if:
    \begin{description}
      \DItem{def:order/preorder/reflexivity}[reflexivity] For every \( x \in X \), \( x \sim x \).
      \DItem{def:order/preorder/transitivity}[transitivity] If \( x \sim y \) and \( y \sim z \), then \( x \sim z \).
    \end{description}

    The pair \( (X, \sim) \) is said to be a \textbf{preordered set}.

    \DItem{def:order/equivalence}\cite[56]{Enderton1977} The preorder \( \cong \) is called an \textbf{equivalence relation} if is is symmetric, i.e.
    \begin{description}
      \DItem{def:order/equivalence/reflexivity}[reflexivity] For every \( x \in X \), \( x \cong x \).
      \DItem{def:order/equivalence/symmetry}[symmetry] If \( x \cong y \), then \( y \cong x \).
      \DItem{def:order/equivalence/transitivity}[transitivity] If \( x \sim y \) and \( y \sim z \), then \( x \sim z \).
    \end{description}

    We define \textbf{equivalence classes} to be sets of the form
    \begin{equation*}
      [a] \coloneqq \{ b \in X \colon a \cong b \}.
    \end{equation*}
    and the \textbf{quotient set} of \( X \) by \( \cong \) to be the family
    \begin{equation*}
      X / \cong \coloneqq \{ [a] \colon a \in X \}.
    \end{equation*}

    We call the function
    \begin{align*}
      &\pi: X \to X / \cong \\
      &\pi(a) \coloneqq [a].
    \end{align*}
    the canonical projection. See~\cref{thm:equivalence_partition}.

    \DItem{def:order/directed}\cite[8]{Engelking1989} A preordered set\Tinyref{def:order/preorder} \( (X, \leq) \) is called a \textbf{directed set} (there is no ubiquitous name for the relation itself) if every two elements have an upper bound\Tinyref{def:poset/upper_lower_bound}, i.e.
    \begin{description}
      \DItem{def:order/directed/reflexivity}[reflexivity] For every \( x \in X \), \( x \leq x \).
      \DItem{def:order/directed/transitivity}[transitivity] If \( x \leq y \) and \( y \leq z \), then \( x \leq z \).
      \DItem{def:order/directed/upper_bound}[upper bound] For all \( x, y \in X \) there exists \( z \in X \) such that \( x \leq z \) and \( y \leq z \).
    \end{description}

    Since the set of all upper bounds of \( \{ a, b \} \) may not have a least smallest\Tinyref{def:poset/largest_smallest_element}, the upper bound condition is weaker than every two-element set having a supremum.

    Directed sets are used to define convergence in topological spaces, see \cref{subsec:convergence}.

    \DItem{def:order/partial}\cite[7]{Engelking1989} the preorder \( \leq \) is called a \textbf{partial order} if it is antisymmetric, i.e.
    \begin{description}
      \DItem{def:order/partial/reflexivity}[reflexivity] For every \( x \in X \), \( x \leq x \).
      \DItem{def:order/partial/antisymmetry}[antisymmetry] If \( x \leq y \) and \( y \leq x \), then \( x = y \).
      \DItem{def:order/partial/transitivity}[transitivity] If \( x \leq y \) and \( y \leq z \), then \( x \leq z \).
    \end{description}

    A set with a partial order is called a~\textbf{partially ordered set or poset}. See~\cref{def:poset}.

    We say that the partial order \( \leq \) is a \textbf{total order} or a \textbf{linear order} if it satisfies
    \begin{description}
      \DItem{def:order/partial/totality}[totality] For all \( x, y \in X \), either \( x \leq y \) or \( y \leq x \)
    \end{description}
    and a \textbf{well-order} if it satisfies
    \begin{description}
      \DItem{def:order/partial/well_order}[well-order] Every set \( A \subseteq X \) has a smallest element\Tinyref{def:poset/largest_smallest_element}.
    \end{description}

    \DItem{def:order/strict_partial}\cite[168]{Enderton1977} The relation \( < \) is called a \textbf{strict partial order} if the following hold:
    \begin{description}
      \DItem{def:order/strict_partial/irreflexivity}[antisymmetry] For every \( x \in X \), \( x < x \) does not hold.
      \DItem{def:order/strict_partial/transitivity}[transitivity] If \( x < y \) and \( y < z \), then \( x < z \).
    \end{description}

    We say that the strict partial order \( < \) is a \textbf{strict total order} or a \textbf{strict linear order} if it satisfies
    \begin{description}
      \DItem{def:order/strict_partial/trichotomy}[trichotomy] For all \( x, y \in X \), either \( x < y \) or \( x < y \) or \( x = y \)
    \end{description}
    and a \textbf{dense order} if it satisfies
    \begin{description}
      \DItem{def:order/strict_partial/density}[density] whenever \( x < z \), there exists \( z \in X \) such that \( x < y < z \).
    \end{description}

    See \cref{thm:strict_partial_order_conversion}.
  \end{defenum}
\end{definition}

\begin{proposition}\label{thm:equality_is_smallest_equivalence_relation}
  The equality relation\Tinyref{def:relation} \( = \) is the intersection of all equivalence relations.
\end{proposition}
\begin{proof}
  By \cref{thm:first_order_equality_equivalence_relation}, equality itself is an equivalence relation. It is equivalent to the \enquote{diagonal} relation\Tinyref{def:relation}
  \begin{equation*}
    \{ (x, x) \colon x \in X \}.
  \end{equation*}

  This relation is a subset of any \ref{def:order/equivalence/reflexivity} relation, hence it is contained in any equivalence relation and, thus, in the intersection of all equivalence relations. But it is itself an equivalence relation, hence the containment is not strict and \( = \) is the intersection of all equivalence relations.
\end{proof}

\begin{definition}\label{def:set_partition}
  Let \( X \) be a set. A \textbf{partition of \( X \)} is a disjoint family\Tinyref{note:family_of_sets} \( P \subseteq \Power(X) \) of nonempty sets such that \( X = \bigcup P \). In other words, each element of \( X \) belong to exactly one set in \( P \).
\end{definition}

\begin{proposition}\label{thm:equivalence_partition}
  Fix a set \( X \). The following three constructions are equivalent:
  \begin{defenum}
    \DItem{thm:equivalence_partition/partition} A partition\Tinyref{def:set_partition} \( P \) of \( X \)
    \DItem{thm:equivalence_partition/equivalence} An equivalence relation\Tinyref{def:order/equivalence} \( \cong \) on \( X \)
    \DItem{thm:equivalence_partition/function} A function \( f: X \to Y \) (where \( Y \) is arbitrary)
  \end{defenum}
\end{proposition}
\begin{proof}
  \Implies[thm:equivalence_partition/equivalence][thm:equivalence_partition/partition] Let \( \cong \) be an equivalence relation on \( X \). The quotient set \( X / \cong \) is a partition since
  \begin{itemize}
    \item Every element \( a \in X \) belongs to the equivalence class \( [a] \).
    \item Let \( [a] \cap [b] \neq \varnothing \) and \( c \in [a] \cap [b] \). Assume\LEM that \( a \not\cong b \). Then \( c \cong a \) and \( c \cong b \), thus \( a \cong c \cong b \) and \( a \cong b \), which is a contradiction. Thus either \( [a] = [b] \) or \( [a] \cap [b] = \varnothing \).
  \end{itemize}

  \Implies[thm:equivalence_partition/partition][thm:equivalence_partition/function] Let \( P \) be a partition of \( X \). Denote by \( P_x \) the set in \( P \) which contains \( x \) and define the function
  \begin{align*}
    &f: X \to P \\
    &f(x) = P_x.
  \end{align*}

  This function is well defined since since all sets in \( P \) are disjoint and thus \( x \) belongs to exactly one set in \( P \).

  \Implies[thm:equivalence_partition/function][thm:equivalence_partition/equivalence] Let \( f: X \to Y \) be a function. Define the relation
  \begin{equation*}
    a \cong b \iff f(a) = f(b).
  \end{equation*}

  It is an equivalence relation since it is induced by the equivalence relation \( = \)\Tinyref{thm:equality_is_smallest_equivalence_relation}.
\end{proof}

\begin{proposition}\label{thm:strict_partial_order_conversion}
  Fix a set \( X \). Then the relation \( \leq \) is a partial order\Tinyref{def:order/partial} on \( X \) if and only if the relation \( < \) is a strict partial order\Tinyref{def:order/strict_partial} on \( X \), where
  \begin{equation*}
    x < y \iff x \leq y \land x \neq y.
  \end{equation*}

  Additionally, \( \leq \) is total if and only if \( < \) total.
\end{proposition}
\begin{proof}
  \Implies Let \( \leq \) be a partial order.
  \begin{description}
    \RItem{def:order/strict_partial/transitivity} Let \( x < y \) and \( y < z \). Thus \( x \leq y \) and \( y \leq z \). By \ref{def:order/partial/transitivity}, \( x \leq z \).

    Additionally, \( x \neq y \) and \( y \neq z \). Assume\LEM that \( x = z \). By \ref{def:order/partial/reflexivity} we have \( z \leq x \) and, since \( y \leq z \), by \ref{def:order/partial/transitivity} we obtain \( y \leq x \). But since \( x \leq y \), by \ref{def:order/partial/antisymmetry}, we have \( x = y \), which contradicts the assumption that \( x < y \).

    Hence \( x < z \).

    \RItem{def:order/strict_partial/irreflexivity} Fix \( x \in X \). By \ref{def:order/partial/reflexivity}, \( x \leq x \) and by definition,
    \begin{equation*}
      x < x \iff x \leq x \land x \neq x,
    \end{equation*}
    thus
    \begin{equation*}
      x < x \iff x \neq x.
    \end{equation*}

    Since the right side is false, the left side \( x < x \) is also false.

    \RItem{def:order/strict_partial/trichotomy} If \( \leq \) is a total order, \( < \) is a strict total order. Fix two elements \( x, y \in X \). We have either \( x \leq y \) or \( y \leq x \). If \( x \leq y \) then either \( x = y \) or \( x < y \). Otherwise, if \( y \leq x \), then \( y = x \) or \( y < x \).
  \end{description}

  \ImpliedBy Let \( < \) be a linear order.
  \begin{description}
    \RItem{def:order/partial/reflexivity} Fix \( x \in X \) and assume\LEM that \( x \not\leq x \). Then \( x \neq x \) which contradicts the \ref{def:order/equivalence/reflexivity} of equality (see \cref{thm:equality_is_smallest_equivalence_relation}). Hence \( x \leq x \).

    \RItem{def:order/partial/antisymmetry} Let \( x \leq y \) and \( y \leq x \), that is, either \( x = y \) or both \( x < y \) and \( y < x \) hold. Assume\LEM the latter. By \ref{def:order/strict_partial/transitivity}, we have \( x < x \), which contradicts \ref{def:order/strict_partial/irreflexivity}. Hence \( x = y \).

    \RItem{def:order/partial/transitivity} Let \( x \leq y \) and \( y \leq z \). Then we have four cases depending on which of \( x \), \( y \) and \( z \) are equal. Since both relations \( < \) and \( = \) are transitive, it follows that in all four cases \( x \leq z \).

    \RItem{def:order/partial/totality} If \( < \) is a strict total order, then \( \leq \) is a total order. Fix \( x, y \in X \). By \ref{def:order/strict_partial/trichotomy}, either
    \begin{itemize}
      \item \( x < y \), which implies \( x \leq y \).
      \item \( y < x \), which implies \( y \leq x \).
      \item \( x = y \), which implies both \( x \leq y \) and \( y \leq x \).
    \end{itemize}
  \end{description}
\end{proof}

\begin{proposition}\label{thm:preorder_to_partial_order}
  Let \( (X, \sim) \) be a preordered set. Use the symmetric closure to define the equivalence relation
  \begin{equation*}
    \cong \coloneqq (\sim)^S.
  \end{equation*}

  The quotient set \( X / \sim \) along with the induced relation
  \begin{equation*}
    [x] \leq [y] \iff x \sim y
  \end{equation*}
  is then a partially ordered set\Tinyref{def:order/partial}.
\end{proposition}
