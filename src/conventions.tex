\section{Conventions}\label{sec:conventions}

This document requires some special conventions that are not necessary in general, when only a single branch of mathematics is covered in a single monograph. I tried following conventions as closely as possible to their respective fields, however this documents own consistency is sometimes preferred to established conventions. The document itself has a lot of remarks. Some conventions are discussed here.

For the sake of consistency, the following conventions are used through the document:
\begin{enumerate}
  \DItem{sec:conventions/natural_numbers} As explained in \cref{remark:peano_arithmetic_zero}, we avoid speaking about the set of natural numbers and instead refer to the set \( \Z^{\geq 0} \) of nonnegative integers and set \( \Z^{>0} \) of positive integers. This solves a stupid debate about whether \( 0 \) is a natural number by simply avoiding it.

  \DItem{sec:conventions/zero_one_index} What we cannot avoid, however, is a certain inconsistency between starting indices from \( 0 \) and from \( 1 \) in different situations. There is, unfortunately, no way around this. So we establish the following convention: indices start from \( 1 \) unless polynomials\Tinyref{def:polynomial} or series\Tinyref{def:convergent_series} are involved. To elaborate, consider the following situations:

  \begin{itemize}
    \item \enquote{Let \( e_1, e_2, \ldots, e_n \) be a basis of \( \R^n \)}.

    It would be inconvenient to start indexing from zero here, because the last index would then be \( n - 1 \). Besides, we refer to \( e_1 \) as the \enquote{first} basis vector, not the zeroth one.

    \item \enquote{Consider the polynomial \( p(x) = a_0 + a_1 x + a_2 x^2 + \cdots + a_n x^n = \sum_{k=0}^n a_k x^k \)}.

    There are different conventions in this situation, ranging from
    \begin{equation*}
      p(x) = a_0 x^n + a_1 x^{n-1} + \cdots + a_{n-1} x + a_n
    \end{equation*}
    used in e.g. \cite[3]{Боянов2008}, to never using the summation notation for polynomials.

    It is convenient, however, to write polynomials as
    \begin{equation*}
      p(x) = \sum_{k=0}^n a_k x^k
    \end{equation*}
    and this readily generalizes into notation for power series\Tinyref{def:convergent_power_series} by simply replacing \( n \) with \( \infty \):
    \begin{equation*}
      p(x) = \sum_{k=0}^\infty a_k x^k.
    \end{equation*}

    This is the convention used in \cite[9]{Knapp2016BAlg}, \cite[exercise 2.2]{Коцев2016} and \cite[\textnumero 276]{Фихтенгольц1968/2}.

    For the sake of consistency, we use the same notation for numeric series\Tinyref{def:convergent_series}.
  \end{itemize}

  \DItem{sec:conventions/n} The letter \( n \) is used in a plethora of situations and this occasionally causes confusion. For example, introductory books on real analysis usually deal with convergent sequences and series of real numbers (e.g. \cite[chapter 3]{Фихтенгольц1968/1} or \cite[chapter 15]{Фихтенгольц1968/2}). These books freely use the letter \( n \) as the index of a sequence or series, e.g. \fullref{ex:weierstrass_nowhere_differentiable_function}:
  \begin{equation*}
    \sum_{n=0}^\infty a^n \cos(b^n \pi x).
  \end{equation*}

  This is also applicable for single-variable complex analysis.

  When working in a finite-dimensional vector space, it is conventional to use \( n \) as the dimension of a vector space. Thus the notation for the sequence
  \begin{equation*}
    ( x_n )_{n=1}^\infty \subseteq \R^n
  \end{equation*}
  is confusing because \( n \) refers both to an index of the sequence and to the dimension of the vector space.

  Polynomial degrees are also commonly denoted by \( n \), for example in approximation theory (see \cref{subsec:lagrange_polynomials}), however when dealing with multivariate polynomials, we may reserve \( n \) for the dimension of the free module over which the polynomial functions take their values (see \fullref{thm:geometric_nullstellensatz}).

  Enumerative combinatorics\Tinyref{subsec:enumerative_combinatorics} usually involve no complex numbers and no ambient vector space, however the size of some ambient set is usually denoted by \( n \).

  We can use different letters for indices in different situations, like it is done in e.g. \cite{Rudin1991}. This, however, causes some cognitive overhead because of both inconsistency and having to choose an iteration variable in different places.

  We use \( n \) in the following situations:
  \begin{itemize}
    \item For denoting the cardinalities\Tinyref{def:cardinal} of finite sets. Cardinalities of infinite sets are usually denoted by Greek letters, e.g. \( \xi, \eta, \ldots \).

    \item For denoting the rank\Tinyref{def:free_left_module} of a free module\Tinyref{def:free_left_module} and the dimension\Tinyref{def:vector_space_dimension} of vector spaces\Tinyref{def:vector_space}.

    \item For denoting degrees of univariate polynomials (as previously discussed) or upper bounds of partial sums of series (e.g. \( \sum_{k=0}^n \frac 1 k \)). We reserve no letter for the degree of a multivariate polynomial.

    \item When no possibility of confusion is present, e.g. when not dealing with finite set cardinalities, finite-rank free modules or polynomials.
  \end{itemize}

  \DItem{sec:conventions/indexed_families} We use \( i \) as an index for general indexed families\Tinyref{def:indexed_family} and index categories. \enquote{General} in this context means that \( I \) is not a fixed set.

  It is logical to generally denote indices by \( i \). This, however, causes confusion when talking about complex numbers. For example, the series
  \begin{equation*}
    \sum_{k=0}^\infty i^k
  \end{equation*}
  cannot use \( i \) as an indexing variable. Besides that, the notation
  \begin{equation*}
    \sum_{k=0}^\infty \frac 1 k
  \end{equation*}
  is more aesthetically pleasing than
  \begin{equation*}
    \sum_{i=0}^\infty \frac 1 i.
  \end{equation*}

  \DItem{sec:conventions/k} We use \( k \) as the default general-purpose indexing variable. When a certain index plays a special role, we denote it by \( K \) (see e.g. \fullref{thm:cauchys_convergence_criterion}). This is also useful with other letters. Both \( k \) and \( K \), however, are established for indexing and are less ambiguous than \( n \) or \( i \).

  \DItem{sec:conventions/functions} The concept of a function is discussed in detail in \cref{remark:function_definition}. A quick recap of the entire \cref{remark:function_definition} is: unless this causes problems, we do not distinguish between a singleton set and its only element and this allows us to regard every function\Tinyref{def:function} \( f: X \to Y \) as a function \( f: \Power(X) \to \Power(Y) \) by simply putting
  \begin{equation*}
    f(A) \coloneqq \{ f(x) \colon x \in A \},
  \end{equation*}
  where \( A \) is a subset of \( X \). We do not use a special notation like \( f[A] \) for this.

  \DItem{sec:conventions/polynomials} Since we consider general polynomials over commutative unital rings\Tinyref{def:polynomial}, we use capital letters for their variables to highlight that they are not functions - see \cref{remark:polynomial_symbolic_expression}.

  \DItem{sec:conventions/x} When working in an ambient set, we usually denote it by \( X \). If the set has an additional structure, e.g. a topological vector space\Tinyref{def:topological_vector_space} \( (X, +, \cdot, \Cal{T}) \) or a partially ordered set\Tinyref{def:poset} \( (X, \leq) \), we usually only write \( X \) unless ambiguity is possible. For established notation like \( \Cal{T} \) for topologies, we often use subscripts like \( \Cal{T}_X \) to highlight whose topology is \( \Cal{T} \).

  Another convention is to denote magmas\Tinyref{def:magma/magma} and monoids\Tinyref{def:magma/monoid} by \( M, N \) and adjacent letters, groups by \( G, H \) and adjacent letters, (semi)rings\Tinyref{def:semiring} by \( R, T, S \), \ldots. This automatically highlights the kind of algebraic structure we are working with. General fields are usually denoted by \( \K \) or \( \k \).

  \DItem{sec:conventions/subspaces} Subspaces of \( X \) are denoted by \( X' \). So if \( Y \) is another space, then \( Y' \) is likely subspace of \( Y \).

  \DItem{sec:conventions/tuples} We do not distinguish between tuples\Tinyref{def:cartesian_product} and finite sequences\Tinyref{def:sequence}. It is conventional, for historical reasons, to use a different notation for both:

  \begin{itemize}
    \item Tuples are usually denoted as \( (x_1, x_2, \ldots, x_n) \) (see \cite[1]{Engelking1989}) or even \( \Gen{x_1, x_2, \ldots, x_n} \) (see \cite[42]{Enderton1977})
    \item Sequences are usually denoted as \( \{ x_i \}_{i=1}^n \) or even \( \{ x_i \} \) when the index \( i \) is assumed implicitly to range over \( \Z^{>0} \) (see \cite[3.1]{Rudin1991}).
  \end{itemize}

  It is established that \( \{ x_i \} \) does not refer to singleton sets in analysis, however it is not established that \( ( x_i ) \) does not refer to a single-element tuple. For this reason, we use curly braces for sequences and parentheses for tuples. Using angle brackets for tuples conflicts with duality pairings\Tinyref{def:duality_pairing}, group presentations\Tinyref{def:group_presentation} and generated ring ideals\Tinyref{def:generated_ring_ideal}.

  \DItem{sec:conventions/function_powers} It is customary, especially with regard to trigonometric functions, to write \( \sin^2(x) \) for \( \sin(x) \cdot \sin(x) \).

  This conflicts with the notation for iteration function application, i.e. \( \sin(\sin(x)) \).

  We will use \( \sin^2(x) \) to denote the latter and use \( \sin(x)^2 \) to denote the former.
\end{enumerate}
