\section{Real analysis}\label{sec:real_analysis}
\subsection{Topology of Euclidean spaces}\label{subsec:real_vector_space_geometry}

\begin{proposition}\label{thm:rn_bounded_iff_totally_bounded}
  A set in \( \R^n \) is totally bounded\Tinyref{def:totally_bounded_set} if and only if it is bounded\Tinyref{def:metric_space/bounded_set}.
\end{proposition}
\begin{proof}
  \begin{description}
    \Implies Follows from \cref{thm:totally_bounded_sets_are_bounded}.
    \ImpliedBy Let \( A \) be a bounded set in \( \R^n \) and let \( B(x, r) \) be a ball containing \( A \). Fix \( \varepsilon > 0 \).

    Denote by \( e_1, \ldots, e_n \) the basis\Tinyref{def:left_module_hamel_basis} of \( \R^n \). Denote by \( m \) the smallest integer such that \( m \varepsilon \geq r \).

    We can create a grid around \( B(x, r) \) as follows:

    Define the set
    \begin{equation*}
      \left\{ x + \sum_{i=1}^n [k_i \varepsilon] e_i \colon \forall i = 1, \ldots, n: k_i = 1, \ldots, m \right\}.
    \end{equation*}

    is finite. Furthermore, it is an \( \varepsilon \)-net of \( A \). Indeed, let \( y \in A \). Denote its coordinates along \( e_1, \ldots, e_n \) by \( y_1, \ldots, y_n \). Then \( y \) is contained in the ball
    \begin{equation*}
      B\left(x + \sum_{i=1}^n [\Ceil(y_i) \varepsilon] e_i, \varepsilon \right).
    \end{equation*}
  \end{description}
\end{proof}

\begin{theorem}[Heine-Borel theorem]\label{thm:heine_borel}
  A set in \( \R^n \) is compact in the sense of \cref{def:compact_set} if and only if it is closed and bounded.
\end{theorem}
\begin{proof}
  Follows from \cref{thm:rn_bounded_iff_totally_bounded} and \cref{thm:complete_metric_space_compact_conditions/closed_totally_bounded}.
\end{proof}

\begin{proposition}\label{thm:real_supremum_of_closure}
  The supremum (resp. infimum) of a set \( A \subseteq \R \), if it exists, is equal to the supremum (resp. infimum) of \( \Cl A \).
\end{proposition}
\begin{proof}
  \begin{description}
    \Implies Denote by \( M \) the supremum of \( A \). Assume\LEM that it is not a supremum of \( \Cl A \), that is, there exists an upper bound \( M' \) of \( \Cl A \) such that \( M' < M \). But this is impossible because \( A \subseteq \Cl A \).

    Therefore \( M \) is a supremum of \( \Cl A \).

    \ImpliedBy Denote by \( M \) the supremum of \( \Cl A \). Assume\LEM that it is not a supremum of \( A \), that is, there exists an upper bound \( M' \) of \( A \) such that \( M' < M \).

    Let \( \{ x_i \}_{i=1}^\infty \subseteq A \) be a sequence that converges to \( M \). Then
    \begin{equation*}
      x_i < M' < M.
    \end{equation*}

    By \cref{thm:squeeze_lemma/sequences}, we have \( M' = M \), which contradicts our choice of \( M' \). Thus \( M \) is the supremum of \( A \).
  \end{description}
\end{proof}

\begin{proposition}\label{thm:real_bounded_set_has_supremum}
  Every nonempty bounded set\Tinyref{def:metric_space/bounded_set} in \( \R \) has a supremum and infimum.
\end{proposition}
\begin{proof}
  Let \( A \subseteq \R \) be a nonempty bounded set. By \cref{thm:heine_borel}, the set \( \Cl A \) is compact. By \cref{thm:weierstrass_extreme_value_theorem}, the identity function \( \Id: \R \to \R \) attains its minimum \( m \) and maximum \( M \) on \( \Cl A \). Note that both \( m \) and \( M \) do not have to belong to \( A \), but \( m \) is a lower bound and \( M \) is an upper bound of the set \( A \).

  If we take any other upper bound \( M' \) of \( A \), then by \cref{thm:real_supremum_of_closure},
  \begin{equation*}
    M' \geq \sup A = \sup \Cl A = M.
  \end{equation*}

  Hence \( M \) is the least upper bound of \( A \).

  We can analogously prove that \( m \) is the greatest lower bound of \( A \).
\end{proof}
