\subsection{Progressions}\label{subsec:progressions}

\begin{remark}\label{remark:progressions}
  Progressions are an elementary concept that happens to be useful quite often. There is no definition of progression but rather the term \enquote{progression} refers to specific inductively defined\Tinyref{remark:induction} sequences\Tinyref{def:sequence}.

  We will restrict our attention to the fields \( \BK \in \{ \BR, \BC \} \), although some results hold more generally.
\end{remark}

\begin{definition}\label{def:arithmetic_progression}
  The \Def{arithmetic progression} with \Def{base} \( a_0 \) and \Def{difference} \( d \) is the sequence
  \begin{equation*}
    a_k \coloneqq \begin{cases}
      a_0, &k = 0, \\
      a_{k-1} + d, &k > 0.
    \end{cases}
  \end{equation*}

  \begin{defenum}
    \DItem{def:arithmetic_progression/closed_form} Obviously \( a_k = a_0 + kd \).

    \DItem{def:arithmetic_progression/finite_sum} The series\Tinyref{def:convergent_series} constructed from this sequence has partial sums
    \begin{equation*}
      \sum_{k=0}^n a_k = \frac {(n + 1) (a_n - a_0)} 2.
    \end{equation*}
  \end{defenum}
\end{definition}
\begin{proof}
  Note that
  \begin{align*}
    2 \sum_{k=0}^n a_k
    &=
    2 \sum_{k=0}^n (a_0 + kd)
    = \\ &=
    \sum_{k=0}^n (a_0 + kd) + \sum_{k=0}^n (a_0 + (n-k)d)
    = \\ &=
    \sum_{k=0}^n (2 a_0 + nd)
    = \\ &=
    (n + 1) (a_0 + a_n).
  \end{align*}
\end{proof}

\begin{definition}\label{def:geometric_progression}
  The \Def{geometric progression} with \Def{base} \( a_0 \) and \Def{denominator} \( q \) is the sequence
  \begin{equation*}
    a_k \coloneqq \begin{cases}
      a_0, &k = 0, \\
      a_{k-1} q, &k > 0.
    \end{cases}
  \end{equation*}

  \begin{defenum}
    \DItem{def:geometric_progression/closed_form} Obviously \( a_k = a_0 q^k \).

    \DItem{def:geometric_progression/series} The series
    \begin{equation*}
      \sum_{k=0}^\infty a_k = a_0 \sum_{k=0}^\infty q^k.
    \end{equation*}
    is called the \Def{geometric series} for \( q \). Without loss of generality, we will consider \( a_0 = 1 \).

    \DItem{def:geometric_progression/degenerate} When \( q = 1 \), the geometric progression turns into the constant sequence
    \begin{equation*}
      a_0, a_0, a_0, \ldots.
    \end{equation*}

    \DItem{def:geometric_progression/finite_sum} For all \( q \in \BC \setminus \{ 1 \} \), the geometric series has partial sums
    \begin{equation}\label{thm:geometric_progression/partial_sum}
      \sum_{k=0}^n q^k = \frac {1 - q^{n+1}} {1 - q}.
    \end{equation}

    \DItem{def:geometric_progression/series_sum_exterior} For \( \Abs{q} \geq 1 \), the geometric series diverges.

    \DItem{def:geometric_progression/series_sum_interior} For \( 0 < \Abs{q} < 1 \), the geometric series converges absolutely with sum
    \begin{equation}\label{thm:geometric_progression/series_sum}
      \sum_{k=0}^\infty q^k = \frac 1 {1 - q}.
    \end{equation}
  \end{defenum}
\end{definition}
\begin{proof}\mbox{}
  \begin{description}
    \RItem{def:geometric_progression/finite_sum} This is simply a restatement of \cref{thm:xn_minus_one_factorization}.

    \RItem{def:geometric_progression/series_sum_interior} Fix \( q \in B(0, 1) \). Since only \( q^n \) depends on \( n \) in \cref{thm:geometric_progression/partial_sum}, we obtain \cref{thm:geometric_progression/series_sum} by simply noting that \( q^n \to 0 \) when \( n \to \infty \).

    \RItem{def:geometric_progression/series_sum_exterior} For \( q = 1 \), \cref{def:geometric_progression/degenerate} implies that the series diverges because it grows indefinitely. If \( \Abs{q} = 1 \) and \( q \neq 1 \), the integer powers \( q^k \) are rotations around the unit circle, which do not tend to a limit. Hence the series diverges again.

    Since \( \Abs{q^k} \) grows indefinitely with \( k \) when \( \Abs{q} > 1 \), it follows that
    \begin{equation*}\label{thm:geometric_progression/cauchy_partial_sum}
      \sum_{k=m}^n q^k
      =
      q^m \sum_{k=0}^{n-m} q^k
      =
      q^m \frac {1 - q^{n-m+1}} {1 - q}
      =
      \frac {q^m - q^{n+1}} {1 - q}.
    \end{equation*}
    can get arbitrarily large. Therefore in this case the series also diverges.
  \end{description}
\end{proof}

\begin{example}\label{ex:series_of_reciprocal_powers_of_two}
  A simple example of a geometric series\Tinyref{def:geometric_progression/series} that occurs in practice is
  \begin{equation}\label{ex:series_of_reciprocal_powers_of_two/series}
    \sum_{k=0}^\infty \frac 1 {2^k} = \frac 1 {1 - \frac 1 2} = 2.
  \end{equation}

  Notice that if the series starts at \( k = 1 \), it sums to \( 1 \). This is actually analogous to \( 0.\Ol{9} = 0 \) but in binary except in decimal.
\end{example}

\begin{remark}\label{remark:progressions_and_interest}
  \Cref{def:arithmetic_progression/closed_form} and \cref{def:geometric_progression/closed_form} may seem obvious when formulated as is, however economists like to use them to highlight the difference between simple interest and compound interest. Note that in this example we exploit the equivalence between the the closed form of the progressions and the inductive definition.

  A savings account with \( 1000\$ \) with a simple monthly interest of \( 2\% \) will earn \( 240\$ \) over a year:
  \begin{equation*}
    1000 (1 + 12 \cdot \tfrac 2 {100}) = 1240.
  \end{equation*}

  The same account with a compound interest of \( 2\% \) will earn a bit more - about \( 268\$ \):
  \begin{equation*}
    1000 (1 + \tfrac 2 {100})^{12} \approx 1268.24.
  \end{equation*}

  Over the course of ten years, however, simple interest will earn a total of \( 2400\$ \), while compound interest will earn \( \approx 9765\$ \).

  The difference between linear growth and exponential growth appears staggering in a real world situation, although it may take quite some time for the difference to become noticeable.
\end{remark}

\begin{definition}\label{def:harmonic_progression}
  The \Def{harmonic progression} with \Def{base} \( a_0 \) and \Def{difference} \( d \) is the sequence
  \begin{equation*}
    a_k \coloneqq \frac 1 {a_0 + kd}.
  \end{equation*}

  \begin{defenum}
    \DItem{def:harmonic_progression/well_definedness} In order for this progression to be well defined, either
    \begin{itemize}
      \item \( d = 0 \) and \( a_0 \neq 0 \), which turns this into the constant sequence \( \{ \tfrac 1 {a_0} \} \).

      \item \( d \neq 0 \) and \( \frac {a_0} d \) is not a negative integer.
    \end{itemize}

    Furthermore, the series may only start at \( k = 0 \) if \( a_0 \neq 0 \). For example, \cref{def:harmonic_progression/harmonic_series} starts at \( k = 1 \).

    \DItem{def:harmonic_progression/inductive_form} Unlike \cref{def:arithmetic_progression} and \cref{def:geometric_progression}, we defined the sequence via a closed-form expression and now give an equivalent inductive definition. \begin{equation*}
      a_k \coloneqq \begin{dcases}
        \frac 1 {a_0}, &k = 0, \text{ only defined if } a_0 \neq 0, \\
        \frac 1 {a_0 + d}, &k = 1, \\
        \frac 1 {\frac 1 {a_{k-1}} + d}, &k > 0.
      \end{dcases}
    \end{equation*}

    \DItem{def:harmonic_progression/series} In the special case where \( a_0 = 0 \) and \( d = 1 \) (so that the progression starts at \( k = 1 \)), the series
    \begin{equation}\label{def:harmonic_progression/harmonic_series}
      \sum_{k=1}^\infty \frac 1 k = 1 + \frac 1 2 + \frac 1 3 + \frac 1 4 + \cdots.
    \end{equation}
    is called the \Def{harmonic series}.

    \DItem{def:harmonic_progression/hyperharmonic_series} For any \( s \in \BC \), the series
    \begin{equation*}
      \sum_{k=1}^\infty \frac 1 {k^s}.
    \end{equation*}
    is called the \Def{hyperharmonic series}. It converges when \( \Re(s) > 1 \).
  \end{defenum}
\end{definition}
\begin{proof}
  \begin{description}
    \RItem{def:harmonic_progression/series} We will first show that \cref{def:harmonic_progression/harmonic_series} diverges.

    Define the series
    \begin{equation}\label{def:harmonic_progression/powers_of_two}
      1 + \frac 1 2 + \underbrace{\frac 1 4 + \frac 1 4}_{\tfrac 1 2} + \underbrace{\frac 1 8 + \frac 1 8 + \frac 1 8 + \frac 1 8}_{\tfrac 1 2} + \underbrace{\frac 1 {16} + \cdots + \frac 1 {16}}_{\tfrac 1 2} + \cdots.
    \end{equation}

    It is divergent as the sum of infinitely many \( \frac 1 2 \).

    Furthermore, \cref{def:harmonic_progression/powers_of_two} is dominated by \cref{def:harmonic_progression/harmonic_series}:
    \begin{align*}
      &1 + \frac 1 2 + \frac 1 3 + \frac 1 4 + \frac 1 5 + \frac 1 6 + \frac 1 7 + \frac 1 8 + \frac 1 9 + \cdots + \frac 1 {16} + \cdots
      \\
      &1 + \frac 1 2 + \underbrace{\frac 1 4 + \frac 1 4}_{\tfrac 1 2} + \underbrace{\frac 1 8 + \frac 1 8 + \frac 1 8 + \frac 1 8}_{\tfrac 1 2} + \underbrace{\frac 1 {16} + \cdots + \frac 1 {16}}_{\tfrac 1 2} + \cdots
    \end{align*}

    By \cref{thm:positive_series_comparison}, the harmonic series \cref{def:harmonic_progression/harmonic_series} is also divergent.
  \end{description}
\end{proof}
