\subsection{Progressions}\label{subsec:progressions}

\begin{remark}\label{remark:progressions}
  Progressions are an elementary concept that happens to be useful quite often. There is no definition of progression but rather the term \enquote{progression} refers to specific inductively defined sequences\Tinyref{def:sequence}.

  We will restrict our attention to the fields \( \K \in \{ \R, \C \} \), although some results hold more generally.
\end{remark}

\begin{definition}\label{def:arithmetic_progression}
  The \Def{arithmetic progression} with \Def{base} \( a_0 \) and \Def{increment} \( d \) is the sequence
  \begin{equation*}
    a_k \coloneqq \begin{cases}
      a_0, &k = 0, \\
      a_{k-1} + d, &k > 0.
    \end{cases}
  \end{equation*}

  \begin{defenum}
    \DItem{def:arithmetic_progression/closed_form} Obviously \( a_k = a_0 + kd \).

    \DItem{def:arithmetic_progression/finite_sum} The series\Tinyref{def:convergent_series} constructed from this sequence has partial sums
    \begin{equation*}
      \sum_{k=0}^n a_k = \frac {(n + 1) (a_n - a_0)} 2.
    \end{equation*}
  \end{defenum}
\end{definition}
\begin{proof}
  Note that
  \begin{align*}
    2 \sum_{k=0}^n a_k
    &=
    2 \sum_{k=0}^n (a_0 + kd)
    = \\ &=
    \sum_{k=0}^n (a_0 + kd) + \sum_{k=0}^n (a_0 + (n-k)d)
    = \\ &=
    \sum_{k=0}^n (2 a_0 + nd)
    = \\ &=
    (n + 1) (a_0 + a_n).
  \end{align*}
\end{proof}

\begin{definition}\label{def:geometric_progression}
  The \Def{geometric progression} with \Def{base} \( a_0 \) and \Def{increment} \( q \) is the sequence
  \begin{equation*}
    a_k \coloneqq \begin{cases}
      a_0, &k = 0, \\
      a_{k-1} q, &k > 0.
    \end{cases}
  \end{equation*}

  \begin{defenum}
    \DItem{def:geometric_progression/closed_form} Obviously \( a_k = a_0 q^k \).

    \DItem{def:geometric_progression/finite_sum} The series\Tinyref{def:convergent_series} constructed from this sequence has partial sums
    \begin{equation*}
      \sum_{k=0}^n a_k = a_0 \frac {1 - q^n} {1 - q}
    \end{equation*}

    \DItem{def:geometric_progression/series} For \( \Abs{q} < 1 \), the \Def{geometric series}
    \begin{equation}\label{thm:geometric_progression/series}
      \sum_{k=0}^\infty a_k = a_0 \frac 1 {1 - q}
    \end{equation}
    converges.
  \end{defenum}
\end{definition}
\begin{proof}
  \begin{description}
    \RItem{def:geometric_progression/finite_sum} This is simply a restatement of \cref{thm:xn_minus_one_factorization}.
  \end{description}
\end{proof}

\begin{remark}\label{remark:progressions_and_interest}
  \Cref{def:arithmetic_progression/closed_form} and \cref{def:geometric_progression/closed_form} may seem obvious when formulated as is, however economists like to use them to highlight the difference between simple interest and compound interest.

  Indeed, the difference between linear growth and exponential growth appears staggering in a real world situation.
\end{remark}
