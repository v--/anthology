\section{Real analysis}\label{sec:real_analysis}

\begin{theorem}[Bolzano-Weierstrass]\label{def:bolzano_weierstrass}
  Every bounded sequence in $\BB{R}$ has a convergent\Tinyref{thm:metric_convergence_iff_metric_topology_convergence/limit_point} subsequence\Tinyref{def:sequence}.
\end{theorem}
\begin{proof}
  Let $\{ x_i \}_{i=1}^\infty$ be a bounded sequence in $\BB{R}$ and let $a \leq b$ be a lower and upper bound\Tinyref{def:poset/upper_lower_bound}, respectively. Construct the sequence $\{ F_i \}_{i=1}^\infty$ of closed intervals as follows: define $\alpha_1 \coloneqq a$ and $\beta_1 \coloneqq b$ and, at step $k = 1, 2, \ldots$, put
  \begin{align*}
    F_k \coloneqq \begin{cases}
      [\alpha_k, \tfrac{\alpha_k+\beta_k} 2], &[\alpha_k, \tfrac{\alpha_k+\beta_k} 2]\text{ contains infinitely many sequence members}, \\
      [\tfrac{\alpha_k+\beta_k} 2, \beta_k], &\text{otherwise}.
    \end{cases}
  \end{align*}

  Then put $\alpha_{k+1}$ and $\beta_{k+1}$ to be the endpoints of the interval $F_k$ and repeat with $k+1$ instead of $k$. Note that for any $i = 1, 2, \ldots$, $\Diam(F_i) = \tfrac 1 2 \Diam(F_{i-1})$, thus $\Diam(F_i) \xrightarrow[i \to \infty]{} 0$. As in \cref{thm:cantor_nested_compacts}, it follows that if we choose\AOC a sequence
  \begin{align*}
    x_i \in F_i, i = 1, 2, \ldots,
  \end{align*}
  it will be a fundamental sequence. Since the space is complete, this fundamental sequence necessarily converges.
\end{proof}

\begin{theorem}\label{def:real_numbers_complete_metric_space}
  The metric space $\BB{R}$ is complete.
\end{theorem}
\begin{proof}
  Let $\{ x_i \}_{i=1}^\infty$ be a fundamental sequence of real numbers. By \cref{thm:fundamental_sequence_is_bounded}, the sequence is bounded. By \cref{def:bolzano_weierstrass}, it has a convergent subsequence
  \begin{align*}
    \{ x_{i_k} \}_{k=1}^\infty \to x.
  \end{align*}

  By \cref{thm:fundamental_subsequence_convergence}, the sequence itself has the same limit $\lim_{i \to \infty} x_i = x$.
\end{proof}
