\subsection{Bilinear forms}\label{subsec:bilinear_forms}

\begin{definition}\label{def:bilinear_form}\cite[249]{Knapp2016BAlg}
  Let \( M \) and \( N \) be left \( R \)-modules and \( L: M \times N \to R \) be a multilinear\Tinyref{def:multilinear_function} function. We say that \( L \) is a \Def{bilinear form}.

  If \( M = N \), we have the following additional types of bilinear forms:
  \begin{defenum}
    \DItem{def:bilinear_form/symmetric} If \( L \) is a symmetric function\Tinyref{def:symmetric_function}, we say that is is a \Def{symmetric bilinear form}

    \DItem{def:bilinear_form/skew_symmetric} If for all \( x, y \in M \) instead of \( L(x, y) = L(y, x) \) we have \( L(x, y) = -L(y, x) \), we say that \( L \) is \Def{skew-symmetric}.

    \DItem{def:bilinear_form/alternating} If for all \( x \in M \) we have \( L(x, x) = 0 \), we say that \( L \) is \Def{alternating}.
  \end{defenum}
\end{definition}

\begin{proposition}\label{thm:skew_symmetric_iff_alternating}
  Let \( 1 + 1 = 2 \) be a unit in the ring \( R \). Let \( M \) be a left module over \( R \). Then the bilinear form \( L: M \times M \to R \) is alternating\Tinyref{def:bilinear_form/alternating} if and only if it is skew-symmetric\Tinyref{def:bilinear_form/skew_symmetric}.

  Alternating implies skew-symmetric even if \( 2 \) is not invertible.
\end{proposition}
\begin{proof}
  \begin{description}
    \Implies Let \( L \) be alternating. Then
    \begin{align*}
      L(x + y, x + y) &= L(x, x) + L(x, y) + L(y, x) + L(y, y) \\
      0 &= L(x, y) + L(y, x) \\
      L(x, y) = -L(y, x).
    \end{align*}
  \end{description}

  \ImpliedBy Let \( L \) be skew-symmetric. Then
  \begin{equation*}
    L(x, x) = -L(x, x),
  \end{equation*}
  which implies that \( 2L(x, y) = 0 \). Hence \( L \) is alternating if we are able to divide by \( 2 \).
\end{proof}

\begin{definition}\label{def:bilinear_form_radicals}\cite[250]{Knapp2016BAlg}
  Let \( L: M \times N \to R \) be a bilinear form. We define its \Def{left radical}
  \begin{equation*}
    \{ x \in M \colon \forall y \in N, \Prod x y = 0 \}
  \end{equation*}
  and \Def{right radical}
  \begin{equation*}
    \{ y \in N \colon \forall x \in M, \Prod x y = 0 \}.
  \end{equation*}

  Note that if \( L \) is symmetric or skew-symmetric (which also implies \( M = N \)), the two are identical and we speak simply of the \Def{radical} \( \sqrt L \).
\end{definition}

\begin{definition}\label{def:nondegenerate_bilinear_form}\cite[249]{Knapp2016BAlg}
  We say that a bilinear form \( L: M \times N \to R \) is \Def{nondegenerate} if both its left and right radicals\Tinyref{def:bilinear_form_radicals} are nontrivial.
\end{definition}

\begin{theorem}\label{thm:bilinear_form_matrix_presentation}
  Fix a commutative unital ring \( R \) and a bilinear form \( L: R^n \times R^m \to R \). Then there exists a matrix \( A \in R^{n \times m} \) such that
  \begin{equation*}
    L(x, y) \coloneqq x^T A y.
  \end{equation*}

  This matrix is called the generalized \Def{Gram matrix}.

  In particular, if \( L \) is symmetric\Tinyref{def:symmetric_function}, so it \( A \).
\end{theorem}
\begin{proof}
  Denote by \( e_1, \ldots, e_n \) the basis of \( R^n \) and by \( f_1, \ldots, f_m \) the basis of \( R^m \).

  Define the matrix \( A = \{ a_{i,j} \}_{i,j=1}^{n,m} \) by
  \begin{equation*}
    a_{i,j} \coloneqq L(e_i, f_j).
  \end{equation*}

  Note that if \( n = m \) and if \( L \) is symmetric, then the matrix \( A \) is obviously symmetric too.

  For any fixed basis vector \( e_i, i = 1, \ldots, n \) of \( R^n \), we have
  \begin{equation*}
    L(e_i, y)
    =
    \sum_{j=1}^m y_i L(e_i, f_j)
    =
    y_i a_{(i,-)},
  \end{equation*}
  where \( a_{(i,-)} \) is the \( i \)-th row of \( A \).

  Thus for an arbitrary \( x \in R^n \)
  \begin{equation*}
    L(x, y)
    =
    \sum_{i=1}^n x_i L(e_i, y)
    =
    \sum_{i=1}^n x_i (a_{(i,-)} y)
    =
    \left( \sum_{i=1}^n x_i a_{(i,-)} \right) y
    =
    x^T A y.
  \end{equation*}
\end{proof}

\begin{corollary}\label{thm:bilinear_forms_isomorphic_to_matrices}
  Fix a commutative unital ring \( R \). The vector space of bilinear forms of type \( R^n \times R^m \to R \) is isomorphic to the matrix space \( A \in R^{n \times m} \).
\end{corollary}

\begin{definition}\label{def:sesquilinear_form}\cite[258]{Knapp2016BAlg}
  Let \( V \) be a complex vector space and let \( \Ol V \) be its conjugate transpose\Tinyref{def:complex_conjucate_vector_space}. We call the bilinear form \( L: V \times \Ol V \to \BC \) a \Def{sesquilinear form} (we say that \( L \) is \enquote{semilinear} in its second argument and \enquote{sesqui} means \enquote{one and a half} is Latin).

  Similar to \fullref{def:bilinear_form}, we have
  \begin{defenum}
    \DItem{def:sesquilinear_form/hermitian} If for all \( x, y \in V \) we have \( L(x, y) = \Ol{L(y, x)} \), we say that \( L \) is \Def{Hermitian}.

    \DItem{def:sesquilinear_form/skew_hermitian} If for all \( x, y \in V \) we have \( L(x, y) = -\Ol{L(y, x)} \), we say that \( L \) is \Def{skew-Hermitian}.
  \end{defenum}
\end{definition}

\begin{definition}\label{def:duality_pairing}
  Let \( M \) and \( N \) be left \( R \)-modules. A \Def{duality pairing} \( \Prod \cdot \cdot: M \times N \to R \) is a nondegenerate\Tinyref{def:nondegenerate_bilinear_form} bilinear form.

  See \fullref{def:canonical_duality_pairing} and \fullref{def:locally_convex_duality_pairing}.
\end{definition}

\begin{definition}\label{def:quadratic_form}
  If \( L: M \times M \to R \) be a bilinear form\Tinyref{def:bilinear_form}, we call the function
  \begin{equation*}
    Q(x) \coloneqq L(x, x)
  \end{equation*}
  a \Def{quadratic form} over \( M \).
\end{definition}

\begin{definition}\label{def:quadratic_form_definiteness}
  We say that a bilinear form is positive or negative definite or semidefinite if the corresponding quadratic form has the same definiteness\Tinyref{def:function_definiteness} (as a function).
\end{definition}

\begin{proposition}\label{thm:bilinear_forms_vs_to_quadratic_forms}
  A \Def{quadratic form} \( Q: M \to R \) is a homogeneous function\Tinyref{def:homogenous_function} of degree \( 2 \). In particular, \( Q(x) = Q(-x) \).
\end{proposition}
\begin{proof}
  Let \( L: M \times M \to R \) be the corresponding bilinear form. Then, by \ref{def:linear_operator/homogeneity},
  \begin{equation*}
    Q(tx) = L(tx, tx) = t^2 L(x, x) = t^2 Q(x).
  \end{equation*}
\end{proof}

\begin{proposition}\label{thm:polarization_identity}\cite{nLab:polarization_identity}
  Let \( L: M \times M \to R \) be a bilinear form\Tinyref{def:bilinear_form} and \( Q: M \to R \) be its associated quadratic form\Tinyref{def:quadratic_form}. Then the \Def{polarization identity} holds:
  \begin{equation}\label{thm:polarization_identity/polarization_identity}
    2 L(x, y) + 2 L(y, x) = Q(x + y) - Q(x - y)
  \end{equation}

  The similar looking but slightly less useful parallelogram law also holds:
  \begin{equation}\label{thm:polarization_identity/parallelogram_law}
    2 Q(x) + 2 Q(y) = Q(x + y) + Q(x - y)
  \end{equation}

  If \( 2 = 1 + 1 \) is a unit in \( R \), we can \enquote{recover} from \( Q \) the bilinear form:
  \begin{equation}\label{thm:polarization_identity/symmetrization_definition}
    \hat L(x, y) \coloneqq \frac 1 2 \left[ Q(x + y) - Q(x) - Q(y) \right]
  \end{equation}

  The function \( \hat L \) is symmetric\Tinyref{def:symmetric_function} and is called the \Def{symmetrization} of \( L \). If \( L \) itself is symmetric, \( L = \hat L \).
\end{proposition}
\begin{proof}
  Identities \fullref{thm:polarization_identity/polarization_identity,thm:polarization_identity/parallelogram_law,thm:polarization_identity/symmetrization_definition} all follow from the bilinearity of \( L \), that is,
  \begin{equation*}
    Q(x \pm y)
    =
    L(x, x) \pm L(x, y) \pm L(y, x) + L(y, y)
    =
    [Q(x) + Q(y)] \pm [L(x, y) + L(y, x)].
  \end{equation*}
\end{proof}

\begin{definition}\label{def:orthogonality}
  Let \( R \) be a division ring\Tinyref{def:semiring/division_ring}, let \( M \) be a left \( R \)-module and let \( L: M \times N \to R \) be a nondegenerate bilinear form. We say that the vectors \( x \in M \) and \( y \in N \) are \Def{orthogonal} with respect to \( L \) if
  \begin{equation*}
    L(x, y) = 0.
  \end{equation*}

  For every submodule \( K \subseteq M \) we define its \Def{orthogonal complement} with respect to \( L \) as
  \begin{equation*}
    K^\perp \coloneqq \{ x \in M \colon \forall y, L(x, y) = 0 \}
  \end{equation*}
  and analogously for submodules of \( N \).

  Let \( I \) be an index set and \( \{ x_i \}_{i \in I} \subseteq M \), \( \{ y_i \}_{i \in I} \subseteq N \) be two families of vectors indexed by \( I \). We say that these families form a \Def{biorthogonal system} with respect to \( L \) if
  \begin{equation*}
    L(x_i, y_j) = 0 \iff i \neq j
  \end{equation*}

  If \( M = N \), we usually consider \Def{orthogonal systems} \( \{ x_i \}_{i \in I} \subseteq M \) where
  \begin{equation*}
    L(x_i, x_j) = 0 \iff i \neq j
  \end{equation*}
\end{definition}

\begin{definition}\label{def:inner_product_space}
  An \Def{inner product space} is a vector space \( V \) over \( F \) equipped with a positive definite\Tinyref{def:quadratic_form_definiteness} symmetric\Tinyref{def:bilinear_form/symmetric} bilinear form \( \Prod \cdot \cdot: V \times V \to F \).

  In the special case where \( F = \BC \), by convention we require \( V \) to instead be equipped with a positive definite Hermitian\Tinyref{def:sesquilinear_form/hermitian} sesquilinear form instead.
\end{definition}

\begin{definition}\label{def:symplectic_vector_space}
  A \Def{symplectic vector space} is a vector space \( V \) over \( F \) equipped with a nondegenerate\Tinyref{def:bilinear_form/symmetric}  alternating\Tinyref{def:bilinear_form/alternating} bilinear form \( \Prod \cdot \cdot: V \times V \to F \).
\end{definition}

\begin{lemma}\label{thm:inner_product_quadratic_form_is_positive_definite}
  Let \( V \) be a real or complex inner product space\Tinyref{def:inner_product_space} with product \( \Prod \cdot \cdot \). The function \( Q(x) \coloneqq \Prod x x \) (which is not a quadratic form in the complex case) is positive definite.
\end{lemma}
\begin{proof}
  The real case is trivial. Assume that \( V \) is a complex vector space and that \( \Prod \cdot \cdot \) is Hermitian. This implies that \( \Prod x x = \Ol{\Prod x x} \), thus \( \Prod x x \in \BR \). Furthermore, since the inner product is positive definite, we have \( Q(x) = \Prod x x \geq 0 \). Thus \( Q \) is nonnegative real valued.

  Since \( \Prod \cdot \cdot \) is positive definite, so is \( Q \).
\end{proof}

\begin{theorem}[Cauchy-Bunyakovsky-Schwarz inequality]\label{thm:cauchy_bunyakovsky_schwarz_inequality}
  Let \( V \) be a real or complex inner product space\Tinyref{def:inner_product_space} with product \( \Prod \cdot \cdot \). For every \( x, y \in V \) it holds that
  \begin{equation}\label{thm:cauchy_bunyakovsky_schwarz_inequality/inequality}
    {\Abs{\Prod x y}}^2 \leq \Prod x x \Prod y y.
  \end{equation}

  Furthermore, equality is achieved if and only if \( x \) and \( y \) are linearly dependent.
\end{theorem}
\begin{proof}
  Note that we use this theorem to prove that the induced norm is a norm so we cannot use the norm here. Associate with \( \Prod \cdot \cdot \) the function \( Q(x) \coloneqq \Prod x x \). By \fullref{thm:inner_product_quadratic_form_is_positive_definite}, \( Q \) is positive definite.

  Fix \( x, y \in V \) and \( t \in \BC \). If either vector is zero the statement is trivially true so let both be nonzero. We have
  \begin{align*}
    Q(x + ty)
    &=
    \Prod {x + ty} {x + ty}
    = \\ &=
    Q(x) + \Ol t \Prod x y + t \Prod y x + \Abs{t}^2 Q(y)
    = \\ &=
    Q(x) + 2\Re t \Ol{\Prod x y} + \Abs{t}^2 Q(y)
  \end{align*}

  Take \( t \coloneqq - \frac {\Prod x y} {Q(y)} \) so that
  \begin{equation*}
    Q(x + ty)
    =
    Q(x) - 2 \frac {\Abs{\Prod x y}^2} {Q(y)} + \frac {\Abs{\Prod x y}^2} {Q(y)}
    =
    Q(x) - \frac {\Abs{\Prod x y}^2} {Q(y)}
  \end{equation*}

  Since \( Q(x + ty) \geq 0 \), it follows that
  \begin{align*}
    Q(x) - \frac {\Abs{\Prod x y}^2} {Q(y)} &\geq 0 \\
    Q(x) Q(y) &\geq \Abs{\Prod x y}^2.
  \end{align*}

  If \( x \) and \( y \) are linearly dependent, equality obviously holds. Conversely, suppose that equality holds. This implies that
  \begin{equation*}
    Q(x + ty) = 0,
  \end{equation*}
  which by the positive definiteness of \( Q \) means that \( x = -ty \). Thus \( x \) and \( y \) are linearly dependent.
\end{proof}

\begin{definition}\label{def:bilinear_form_induced_norm}
  Let \( V \) be a real or complex inner product space\Tinyref{def:inner_product_space} with product \( \Prod \cdot \cdot \). We define its induced norm\Tinyref{def:norm} as
  \begin{align*}
    &\Norm \cdot : V \to \BR_{\geq 0} \\
    &\Norm x \coloneqq \sqrt{\Prod x x}.
  \end{align*}

  If \( V \) is a real inner product space, the induced norm is a square root of the induced quadratic form\Tinyref{def:quadratic_form} of \( \Prod \cdot \cdot \).
\end{definition}
\begin{proof}
  We will only prove the complex case because the real case is identical but slightly simpler.

  Note that \( \Norm \cdot \) is well-defined (that is, positive definite) by \fullref{thm:inner_product_quadratic_form_is_positive_definite}.

  Now we will show that it is a norm.
  \begin{description}
    \RItem{def:norm/identity} Follows from the positive definiteness of \( \Prod \cdot \cdot \)

    \RItem{def:norm/absolute_homogeneity} For \( t \in \BC \) and \( x \in V \) we have
    \begin{equation*}
      \Norm{tx} = \sqrt{\Prod{tx} {tx}} = \Abs{t} \sqrt{\Prod x x} = \Abs t \Norm x.
    \end{equation*}

    \RItem{def:norm/subadditivity} For \( x, y \in V \) we have
    \begin{align*}
      \Norm{x + y}^2
      &=
      \Prod{x + y} {x + y}
      = \\ &=
      \Prod x x + \Prod x y + \Prod y x + \Prod y y
      = \\ &=
      \Norm{x}^2 + 2 \Re \Prod x y + \Norm{y}^2
      \leq \\ &\leq
      \Norm{x}^2 + 2 \Abs{\Re \Prod x y} + \Norm{y}^2
      \overset {\ref{thm:cauchy_bunyakovsky_schwarz_inequality}} = \\ &=
      \Norm{x}^2 + 2 \Norm x \Norm y + \Norm{y}^2
      =
      (\Norm{x} + \Norm{y})^2
    \end{align*}

    Therefore
    \begin{equation*}
      \Norm{x + y} \leq \Norm x + \Norm y.
    \end{equation*}
  \end{description}
\end{proof}
