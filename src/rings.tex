\section{Rings}\label{sec:rings}

\begin{definition}\label{def:ring}
  A \ul{ring} is an (additive\Tinyref{note:additive_group}) abelian group\Tinyref{def:group} $(R, +)$ with an additional multiplication operation $\cdot: R \times R \to R$ (denoted by juxtaposition), such that for all $a, b, c \in R$ the following axioms hold
  \begin{description}
    \DItem{associativity}{def:ring/associativity} $(ab)c = a(bc)$
    \DItem{left distributivity}{def:ring/left_distributivity} $(a + b)c = ab + bc$
    \DItem{right distributivity}{def:ring/right_distributivity} $a(b + c) = ab + ac$
  \end{description}

  We say that
  \begin{itemize}
    \item\label{def:ring/trivial_group} the ring $\{ 0 \}$ is the \ul{zero ring} or \ul{trivial ring}.
    \item\label{def:ring/subring} the subset $S \subseteq R$ is a \ul{subring of $R$} if $S$ is closed under the ring operations.
    \item\label{def:ring/trivial_subgroup} the ring $\{ 0_R \}$ is the \ul{trivial subring} of $R$.
    \item\label{def:ring/proper_subring} all subrings except for $R$ itself are \ul{proper subrings}.
    \item\label{def:ring/zero_divisor} $a \neq 0$ is a \ul{(left) zero divisor} (resp. \ul{right zero divisor}) if there exists $b \neq 0$ such that $ab = 0$ (resp. $ba = 0$).
    \item\label{def:ring/unit} $a$ is a \ul{(left) unit} (resp. \ul{right unit}) if there exists $a^{-1}$ such that $a \cdot a^{-1} = 1$ (resp $a^{-1} \cdot a = 1$).
    \item\label{def:ring/nilpotent_element} $a$ is \ul{nilpotent} if $a^n$ for some nonnegative integer $n$.
    \item\label{def:ring/idempotent_element} $a$ is \ul{idempotent} if $aa = a$.
  \end{itemize}

  Additionally, the following axioms define different types of rings
  \begin{description}
    \DItem{identity}{def:ring/identity} If $(R, \cdot)$ is a monoid, that is if there exists a multiplicative identity $1_R$ such that $1_R a = a1_R = a$ for all $a \in R$, we say that $(R, \cdot)$ is a \ul{ring with identity} or \ul{unital ring}. It is unique by~\cref{def:group_properties/unique_identity}. This is sometimes taken to be part of the definition of a ring.
    \DItem{commutativity}{def:ring/commutativity} If $(R, \cdot)$ is a commutative semigroup, i.e. $ab = ba$ for all $a, b \in R$, we say that $(R, \cdot)$ is a \ul{commutative ring}.
    \DItem{no zero divisors}{def:ring/no_zero_divisors} If the ring is a commutative ring and there are no zero divisors in an, we say that $(R, \cdot)$ is an \ul{integral domain}.
    \DItem{divisibility}{def:ring/divisibility} If all nonzero elements are units, we say that $(R, \cdot)$ is a \ul{division ring}.
  \end{description}

  If we only require the ring to be a monoid under addition (i.e. no inverse elements), we say that $(R, +, \cdot)$ is a \ul{semiring}.
\end{definition}

\begin{proposition}\label{def:ring_properties}
  Any ring $R$ has the following basic properties:
  \begin{defenum}
    \item\label{def:ring_properties/zero_absorbing} Multiplication by $0$ is \ul{absorbing}\Tinyref{def:group/absorbing_element}, that is, $a0 = 0a = 0$ for any $a \in R$.
  \end{defenum}
\end{proposition}
\begin{proof}\mbox{}
  \begin{itemize}
    \item[\ref{def:ring_properties/zero_absorbing}] We have that $0a = (0 + 0)a = 0a + 0a$, thus $0a$ is an additive identity and $0a = 0$. We obtain $a0 = 0$ analogously.
  \end{itemize}
\end{proof}

\begin{definition}\label{def:ring_homomorphism}
  Let $R$ and $T$ be rings. We say that the function $f: R \to T$ is a \ul{ring homomorphism} if it is \ul{compatible with the ring structures on $S$ and $T$}, that is, it is a group homomorphism on $(R, +)$ and a semigroup homomorphism on $(R, \cdot)$.

  If $R$ is a ring with identity, we additionally require $f(1_R) = 1_T$.

  The \ul{kernel} of $f$ is defined as the preimage\Tinyref{def:function_preimage} $f^{-1}(0)$.

  The terminology from~\cref{def:morphism_invertibility} applies to ring homomorphisms because of the category $\Bold{Ring}$ of rings\Tinyref{def:category_of_rings}.
\end{definition}

\begin{definition}\label{def:ring_ideal}
  Let $R$ be a ring and $I$ be a subset of $R$ (not necessarily a subring). We say that $I$ is a \ul{(two-sided) ideal of $R$} and write $I \unlhd R$ if any of the following equivalent conditions hold:
  \begin{defenum}
    \item\label{def:ring_ideal/direct} $(I, +)$ is a subgroup of $(R, +)$ and the inclusions $RI \subseteq I$ and $IR \subseteq I$ hold.
    \item\label{def:ring_ideal/kernel} $I$ is the kernel\Tinyref{def:ring_homomorphism} of some ring homomorphism.
  \end{defenum}

  We can weaken the condition in \ref{def:ring_ideal/direct} to define \ul{left ideals} (resp \ul{right ideals}) if only $RI \subseteq I$ (resp. $IR \subseteq I$) holds. If $I$ is a subring, being either a left or right ideal is equivalent to $I$ being a two-sided ideal. If $R$ is a commutative ring, left and right ideals coincide with two-sided ideals.

  If $R$ is a ring without identity, all two-sided ideals are subrings. If $R$ has an identity element, however, ideals are not necessarily subrings (see~\cref{thm:proper_ideals_containing_identity}).

  As with subrings, the \ul{trivial ideal} of $R$ is the trivial subring and all ideals except for $R$ itself are called \ul{proper ideals}.
\end{definition}
\begin{proof}
  (\ref{def:ring_ideal/direct} $\iff$ \ref{def:ring_ideal/kernel}) \Cref{def:normal_subgroup} implies that $(I, +)$ is a normal subgroup if and only if there exists a group homomorphism $f: (R, +) \to (T, +)$, where $(T, +, \cdot)$ is some ring, such that $I = \Ker(f)$. Note that since the additive group is abelian, by~\cref{thm:abelian_normal_subgroups}, all subgroups are normal.

  Additionally, we have $f(RI) = f(R)f(I)$, thus $RI \subseteq I$ if and only if $I$ is the kernel of $f$.
\end{proof}

\begin{proposition}\label{thm:proper_ideals_containing_identity}
  If $R$ is a ring with identity, the ideal $I$ is proper if and only if $1 \not\in I$.
\end{proposition}
\begin{proof}
  We will prove that $1 \in I \iff I = R$.

  ($\implies$) Let $1 \in I$. Then $r1 = r$ for any $r \in R$, thus $RI = R$. Since $I$ is an ideal, we have that $I = R$.
  ($\impliedby$) If $I = R$, then obviously $1 \in I = R$.
\end{proof}

\begin{definition}\label{def:ring_direct_product}
  Let $\{ X_i \}_{i \in I}$ be a nonempty family of rings.

  Analogously to \cref{def:group_direct_product}, we define their \ul{direct product} as the ring $\prod_{i \in I} X_i$, the operations defined componentwise as
  \begin{align*}
    &\{ x_i \}_{i \in I} + \{ y_i \}_{i \in I}
    \coloneqq
    \{ x_i + y_i \}_{i \in I}, \\
    &\{ x_i \}_{i \in I} \cdot \{ y_i \}_{i \in I}
    \coloneqq
    \{ x_i \cdot y_i \}_{i \in I}.
  \end{align*}

  We define their \ul{direct sum} as the subring of $\prod_{i \in I} X_i$\Tinyref{def:ring_direct_product} where only finitely many components of any ring element are different from zero.
\end{definition}

\begin{definition}\label{def:category_of_ring}
  The class\Tinyref{def:set_zfc} of all rings forms the category\Tinyref{def:category} $\Bold{Ring}$, where for every two rings $X, Y \in \Bold{Ring}$, the morphisms $\Bold{Ring}(X, Y)$ are the homomorphisms\Tinyref{def:ring_homomorphism} from $X$ to $Y$ and composition is the usual function composition\Tinyref{def:function_composition}.

  Furthermore, $\Bold{Ring}$ is locally small\Tinyref{def:category_cardinality} and concrete\Tinyref{def:concrete_category}.
\end{definition}

\begin{proposition}\label{thm:ring_categorical_limits}
  We are interested in categorical limits\Tinyref{def:categorical_limit} and colimits\Tinyref{def:categorical_colimit} in $\Bold{Ring}$. If $\{ X_i \}_{i \in I}$ is an indexed family of rings, then
  \begin{defenum}
    \item\label{thm:ring_categorical_limits/product} their categorical product\Tinyref{def:categorical_product} is their direct product\Tinyref{def:ring_direct_product} $\prod_{i \in I} X_i$, the projection morphisms being inherited from \cref{thm:set_categorical_limits/product}.
  \end{defenum}
\end{proposition}
