\subsection{Rings}\label{subsec:rings}

\begin{definition}\label{def:ring}
  A \textbf{ring} is an (additive\Tinyref{remark:additive_group}) abelian group\Tinyref{def:magma} \( (R, +) \) with an additional multiplication operation \( \cdot: R \times R \to R \) (denoted by juxtaposition), such that for all \( a, b, c \in R \) the following axioms hold
  \begin{description}
    \DItem{def:ring/associativity}[associativity] \( (ab)c = a(bc) \)
    \DItem{def:ring/left_distributivity}[left distributivity] \( (a + b)c = ab + bc \)
    \DItem{def:ring/right_distributivity}[right distributivity] \( a(b + c) = ab + ac \)
  \end{description}

  We say that
  \begin{itemize}
    \DItem{def:ring/trivial_group} the ring \( \{ 0 \} \) is the \textbf{zero ring} or \textbf{trivial ring}.
    \DItem{def:ring/subring} the subset \( S \subseteq R \) is a \textbf{subring of \( R \)} if \( S \) is closed under the ring operations.
    \DItem{def:ring/trivial_subgroup} the ring \( \{ 0_R \} \) is the \textbf{trivial subring} of \( R \).
    \DItem{def:ring/proper_subring} all subrings except for \( R \) itself are \textbf{proper subrings}.
    \DItem{def:ring/zero_divisor} \( a \neq 0 \) is a \textbf{(left) zero divisor} (resp. \textbf{right zero divisor}) if there exists \( b \neq 0 \) such that \( ab = 0 \) (resp. \( ba = 0 \)).
    \DItem{def:ring/unit} \( a \) is a \textbf{(left) unit} (resp. \textbf{right unit}) if there exists \( a^{-1} \) such that \( a \cdot a^{-1} = 1 \) (resp \( a^{-1} \cdot a = 1 \)).
    \DItem{def:ring/nilpotent_element} \( a \) is \textbf{nilpotent} if \( a^n \) for some nonnegative integer \( n \).
    \DItem{def:ring/idempotent_element} \( a \) is \textbf{idempotent} if \( aa = a \).
  \end{itemize}

  Additionally, the following axioms define different types of rings
  \begin{description}
    \DItem{def:ring/identity}[identity] If \( (R, \cdot) \) is a monoid, that is if there exists a multiplicative identity \( 1_R \) such that \( 1_R a = a1_R = a \) for all \( a \in R \), we say that \( (R, \cdot) \) is a \textbf{ring with identity} or \textbf{unital ring}. It is unique by~\cref{def:group_properties/unique_identity}. This is sometimes taken to be part of the definition of a ring.
    \DItem{def:ring/commutativity}[commutativity] If \( (R, \cdot) \) is a commutative semigroup, i.e. \( ab = ba \) for all \( a, b \in R \), we say that \( (R, \cdot) \) is a \textbf{commutative ring}.
    \DItem{def:ring/no_zero_divisors}[no zero divisors] If the ring is a commutative ring and there are no zero divisors in an, we say that \( (R, \cdot) \) is an \textbf{integral domain}.
    \DItem{def:ring/divisibility}[divisibility] If all nonzero elements are units, we say that \( (R, \cdot) \) is a \textbf{division ring}.
  \end{description}

  If we only require the ring to be a monoid under addition (i.e. no inverse elements), we say that \( (R, +, \cdot) \) is a \textbf{semiring}.
\end{definition}

\begin{proposition}\label{def:ring_properties}
  Any ring \( R \) has the following basic properties:
  \begin{defenum}
    \DItem{def:ring_properties/zero_absorbing} Multiplication by \( 0 \) is \textbf{absorbing}\Tinyref{def:magma/absorbing_element}, that is, \( a0 = 0a = 0 \) for any \( a \in R \).
  \end{defenum}
\end{proposition}
\begin{proof}\mbox{}
  \begin{itemize}
    \RItem{def:ring_properties/zero_absorbing} We have that \( 0a = (0 + 0)a = 0a + 0a \), thus \( 0a \) is an additive identity and \( 0a = 0 \). We obtain \( a0 = 0 \) analogously.
  \end{itemize}
\end{proof}

\begin{definition}\label{def:ring_homomorphism}
  Let \( R \) and \( T \) be rings. We say that the function \( f: R \to T \) is a \textbf{ring homomorphism} if it is \textbf{compatible with the ring structures on \( S \) and \( T \)}, that is, it is a group homomorphism on \( (R, +) \) and a semigroup homomorphism on \( (R, \cdot) \).

  If \( R \) is a ring with identity, we additionally require \( f(1_R) = 1_T \).

  The \textbf{kernel} of \( f \) is defined as the preimage\Tinyref{def:function_preimage} \( f^{-1}(0) \).

  The terminology from~\cref{def:morphism_invertibility} applies to ring homomorphisms because of the category \( \Bold{Ring} \) of rings\Tinyref{def:category_of_rings}.
\end{definition}

\begin{definition}\label{def:ring_ideal}
  Let \( R \) be a ring and \( I \) be a subset of \( R \) (not necessarily a subring). We say that \( I \) is a \textbf{(two-sided) ideal of \( R \)} and write \( I \unlhd R \) if any of the following equivalent conditions hold:
  \begin{defenum}
    \DItem{def:ring_ideal/direct} \( (I, +) \) is a subgroup of \( (R, +) \) and the inclusions \( RI \subseteq I \) and \( IR \subseteq I \) hold.
    \DItem{def:ring_ideal/kernel} \( I \) is the kernel\Tinyref{def:ring_homomorphism} of some ring homomorphism.
  \end{defenum}

  We can weaken the condition in \ref{def:ring_ideal/direct} to define \textbf{left ideals} (resp \textbf{right ideals}) if only \( RI \subseteq I \) (resp. \( IR \subseteq I \)) holds. If \( I \) is a subring, being either a left or right ideal is equivalent to \( I \) being a two-sided ideal. If \( R \) is a commutative ring, left and right ideals coincide with two-sided ideals.

  If \( R \) is a ring without identity, all two-sided ideals are subrings. If \( R \) has an identity element, however, ideals are not necessarily subrings (see~\cref{thm:proper_ideals_containing_identity}).

  As with subrings, the \textbf{trivial ideal} of \( R \) is the trivial subring and all ideals except for \( R \) itself are called \textbf{proper ideals}.
\end{definition}
\begin{proof}
  (\ref{def:ring_ideal/direct} \( \iff \) \ref{def:ring_ideal/kernel}) \Cref{def:normal_subgroup} implies that \( (I, +) \) is a normal subgroup if and only if there exists a group homomorphism \( f: (R, +) \to (T, +) \), where \( (T, +, \cdot) \) is some ring, such that \( I = \ker(f) \). Note that since the additive group is abelian, by~\cref{thm:abelian_normal_subgroups}, all subgroups are normal.

  Additionally, we have \( f(RI) = f(R)f(I) \), thus \( RI \subseteq I \) if and only if \( I \) is the kernel of \( f \).
\end{proof}

\begin{proposition}\label{thm:proper_ideals_containing_identity}
  If \( R \) is a ring with identity, the ideal \( I \) is proper if and only if \( 1 \not\in I \).
\end{proposition}
\begin{proof}
  We will prove that \( 1 \in I \iff I = R \).

  \begin{description}
    \Implies Let \( 1 \in I \). Then \( r1 = r \) for any \( r \in R \), thus \( RI = R \). Since \( I \) is an ideal, we have that \( I = R \).
    \ImpliedBy If \( I = R \), then obviously \( 1 \in I = R \).
  \end{description}
\end{proof}

\begin{definition}\label{def:ring_direct_product}
  Let \( \{ X_i \}_{i \in I} \) be a nonempty family of rings.

  Analogously to \cref{def:group_direct_product}, we define their \textbf{direct product} as the ring \( \prod_{i \in I} X_i \), the operations defined componentwise as
  \begin{align*}
    &\{ x_i \}_{i \in I} + \{ y_i \}_{i \in I}
    \coloneqq
    \{ x_i + y_i \}_{i \in I}, \\
    &\{ x_i \}_{i \in I} \cdot \{ y_i \}_{i \in I}
    \coloneqq
    \{ x_i \cdot y_i \}_{i \in I}.
  \end{align*}

  We define their \textbf{direct sum} as the subring of \( \prod_{i \in I} X_i \)\Tinyref{def:ring_direct_product} where only finitely many components of any ring element are different from zero.
\end{definition}

\begin{definition}\label{def:category_of_rings}
  The class\Tinyref{def:set_zfc} of all rings along with all homomorphisms\Tinyref{def:ring_homomorphism} between them forms a category, which we denote by \( \Bold{Ring} \). Furthermore, \( \Bold{Ring} \) is locally small\Tinyref{def:category_cardinality} and concrete\Tinyref{def:concrete_category}.
\end{definition}

\begin{proposition}\label{thm:ring_categorical_limits}
  We are interested in categorical limits\Tinyref{def:categorical_limit} and colimits\Tinyref{def:categorical_colimit} in \( \Bold{Ring} \). If \( \{ X_i \}_{i \in I} \) is an indexed family of rings, then
  \begin{defenum}
    \DItem{thm:ring_categorical_limits/product} their categorical product\Tinyref{def:categorical_product} is their direct product\Tinyref{def:ring_direct_product} \( \prod_{i \in I} X_i \), the projection morphisms being inherited from \cref{thm:set_categorical_limits/product}.
  \end{defenum}
\end{proposition}

\begin{definition}\label{def:field}\cite[142]{Knapp2016BAlg}
  A \textbf{field} \( (F, +, \cdot) \) is a nontrivial commutative division ring with identity\Tinyref{def:ring}. Explicitly, it is a nonempty set \( F \) with two distinct distinguished elements \( 0 \) and \( 1 \) and two operations
  \begin{align*}
    +: &F \times F \to F, \\
    \cdot: &F \times F \to F,
  \end{align*}
  called \textbf{addition} and \textbf{multiplication}, such that
  \begin{itemize}
    \item \( (F, +) \) is an abelian group with identity \( 0 \).
    \item \( (F, \cdot) \) is an abelian group with identity \( 1 \).
    \item \( + \) distributes over \( \cdot \), that is, for any \( a, b, c \in F \) we have
    \begin{equation*}
      (a + b)c = ab + bc.
    \end{equation*}
  \end{itemize}
\end{definition}

\begin{definition}\label{def:field_extension}
  If \( F \) and \( G \) are fields and \( G \) is a subring\Tinyref{def:ring/subring} of \( F \), we say that \( G \) is a \textbf{subfield} of \( F \) and that \( F \) is a \textbf{field extension} of \( G \).

  Field extension are also denoted as \( F / G \) to highlight the roles of \( F \) and \( G \). This is not a quotient ring but simply a notation. See \cref{def:galois_group}.
\end{definition}

\begin{definition}\label{def:galois_group}\cite[124]{Knapp2016BAlg}
  Let \( F \) be a field extension\Tinyref{def:field_extension} of \( G \). The group \( \Gal{F / G} \) of automorphisms of \( F \) that leave \( G \) fixed is called the \textbf{Galois group} of the field extension \( F / G \).
\end{definition}
