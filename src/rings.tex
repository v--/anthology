\begin{definition}\label{def:ring}
  A \uline{ring} is an (additive\Tinyref{note:additive_group}) abelian group\Tinyref{def:group} $(R, +)$ with an additional multiplication operation $\cdot: R \times R \to R$ (denoted by juxtaposition), such that for all $a, b, c \in R$ the following axioms hold
  \begin{description}
    \DItem{Associativity}{def:ring/associativity} $(ab)c = a(bc)$
    \DItem{Left distributivity}{def:ring/left_distributivity} $(a + b)c = ab + bc$
    \DItem{Right distributivity}{def:ring/right_distributivity} $a(b + c) = ab + ac$
  \end{description}

  We say that
  \begin{itemize}
    \item the ring $\{ 0 \}$ is the \uline{zero ring} or \uline{trivial ring}.
    \item the subset $S \subseteq R$ is a \uline{subring of $R$} if $S$ is closed under the ring operations.
    \item the ring $\{ 0_R \}$ is the \uline{trivial subring} of $R$.
    \item all subrings except for $R$ itself are called \uline{proper subrings}.
    \item $a \neq 0$ is a \uline{(left) zero divisor} (resp. \uline{right zero divisor}) if there exists $b \neq 0$ such that $ab = 0$ (resp. $ba = 0$).
    \item $a$ is a \uline{(left) unit} (resp. \uline{right unit}) if there exists $a^{-1}$ such that $a \cdot a^{-1} = 1$ (resp $a^{-1} \cdot a = 1$).
    \item $a$ is \uline{nilpotent} if $a^n$ for some nonnegative integer $n$.
    \item $a$ is \uline{idempotent} if $aa = a$.
  \end{itemize}

  Additionally, the following axioms define different types of rings
  \begin{description}
    \DItem{Identity}{def:ring/identity} If $(R, \cdot)$ is a monoid, that is if there exists a multiplicative identity $1$ such that $1a = a1 = a$ for all $a \in R$, we say that $(R, \cdot)$ is a \uline{ring with identity}. This is sometimes taken to be part of the definition of a ring.
    \DItem{Commutativity}{def:ring/commutativity} If $(R, \cdot)$ is a commutative semigroup, i.e. $ab = ba$ for all $a, b \in R$, we say that $(R, \cdot)$ is a \uline{commutative ring}.
    \DItem{No zero divisors}{def:ring/no_zero_divisors} If the ring is a commutative ring and there are no zero divisors in an, we say that $(R, \cdot)$ is an \uline{integral domain}.
    \DItem{Divisibility}{def:ring/divisibility} If all nonzero elements are units, we say that $(R, \cdot)$ is a \uline{division ring}.
  \end{description}

  If we only require the ring to be a monoid under addition (i.e. no inverse elements), we say that $(R, +, \cdot)$ is a \uline{semiring}.
\end{definition}

\begin{definition}\label{def:field}
  If a ring is both an integral domain and a division ring, we say that it is a \uline{field}.
\end{definition}

\begin{definition}\label{def:ring_homomorphism}
  Let $R$ and $T$ be rings. We say that the function $f: R \to T$ is a \uline{ring homomorphism} if it is \uline{compatible with the ring structures on $S$ and $T$}, that is, it is a group homomorphism on $(R, +)$ and a semigroup homomorphism on $(R, \cdot)$.

  If $R$ is a ring with identity, we additionally require $f(1_R) = 1_T$.

  The \uline{kernel} of $f$ is defined as the preimage\Tinyref{def:function_invertibility} $f^{-1}(0)$.
\end{definition}

\begin{definition}\label{def:ring_ideal}
  Let $R$ be a ring and $I$ be a subset of $R$ (not necessarily a subring). We say that $I$ is a \uline{(two-sided) ideal of $R$} and write $I \unlhd R$ if any of the following equivalent conditions hold:
  \begin{defenum}
    \item\label{def:ring_ideal/direct} $(I, +)$ is a subgroup of $(R, +)$ and the inclusions $RI \subseteq I$ and $IR \subseteq I$ hold.
    \item\label{def:ring_ideal/kernel} $I$ is the kernel\Tinyref{def:ring_homomorphism} of some ring homomorphism.
  \end{defenum}

  We can weaken the condition in \ref{def:ring_ideal/direct} to define \uline{left ideals} (resp \uline{right ideals}) if only $RI \subseteq I$ (resp. $IR \subseteq I$) holds. If $I$ is a subring, being either a left or right ideal is equivalent to $I$ being a two-sided ideal. If $R$ is a commutative ring, left and right ideals coincide with two-sided ideals.

  If $R$ is a ring without identity, all two-sided ideals are subrings. If $R$ has an identity element, however, ideals are not necessarily subrings (see~\cref{thm:proper_ideals_containing_identity}).

  As with subrings, the \uline{trivial ideal} of $R$ is the trivial subring and all ideals except for $R$ itself are called \uline{proper ideals}.
\end{definition}
\begin{proof}
  As in~\cref{def:normal_subgroup}, the proof is a restatement of~\cref{thm:equivalence_partition} with additional regularity conditions.
\end{proof}

\begin{proposition}\label{thm:proper_ideals_containing_identity}
  If $R$ is a ring with identity, the ideal $I$ is proper if and only if $1 \not\in I$.
\end{proposition}
\begin{proof}
  We will prove that $1 \in I \iff I = R$.

  ($\implies$) Let $1 \in I$. Then $r1 = r$ for any $r \in R$, thus $RI = R$. Since $I$ is an ideal, we have that $I = R$.
  ($\impliedby$) If $I = R$, then obviously $1 \in I = R$.
\end{proof}
