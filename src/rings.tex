\section{Ring theory}\label{sec:ring_theory}
\subsection{Rings}\label{subsec:rings}

\begin{definition}\label{def:semiring}
  A \Def{semiring} \( (R, +, \cdot) \) is an associative algebraic structure\Tinyref{def:algebraic_theory} with two binary operations:
  \begin{description}
    \item[addition] \( + \) is associative\Tinyref{def:algebraic_theory/associativity}, unital\Tinyref{def:algebraic_theory/identity} and commutative\Tinyref{def:algebraic_theory/commutativity}, i.e. \( (R, +) \) is a commutative monoid\Tinyref{def:magma/monoid}. The additive identity is usually denoted by \( 0 \).

    \item[multiplication] \( \cdot \) (usually written using juxtaposition) is associative\Tinyref{def:algebraic_theory/associativity}, i.e. \( (R, \cdot) \) a semigroup\Tinyref{def:magma/semigroup}.
  \end{description}

  We require that \( \cdot \) distributes\Tinyref{def:algebraic_theory/distributivity} over \( + \).

  The following are special kinds of elements in \( R \):
  \begin{defenum}
    \DItem{def:semiring/zero_divisor} \( x \neq 0 \) is a \Def{left zero divisor} (resp. \Def{right zero divisor}) if there exists \( y \neq 0 \) such that \( xy = 0 \) (resp. \( yx = 0 \)) or simply a \Def{zero divisor} if it is both.

    \DItem{def:semiring/nilpotent_element} \( x \) is \Def{nilpotent} if \( x^n = 0 \) for some nonnegative integer \( n \).
  \end{defenum}

  The following are special kinds of semirings:
  \begin{defenum}
    \DItem{def:semiring/ring} \Def{Rings} are semirings with invertible addition, i.e. \( R, + \) forms an abelian group\Tinyref{def:magma/abelian_group}. The category of rings is denoted by \( \Cat{Ring} \).

    \DItem{def:semiring/unital_ring} \Def{Unital rings} are rings\Tinyref{def:semiring/ring} in which multiplication is unital, i.e. \( R, \cdot \) is a monoid\Tinyref{def:magma/monoid}. Some authors define all rings to be unital. Units are denoted by \( 1 \). Invertible elements\Tinyref{def:algebraic_theory/invertibile_element} under multiplication are called \Def{units} and the operation itself is called \Def{division}.

    \DItem{def:semiring/commutative_ring} \Def{Commutative rings} are rings\Tinyref{def:semiring/ring} in which multiplication is commutative\Tinyref{def:algebraic_theory/commutativity}, i.e. \( R, \cdot \) is a commutative semigroup.

    \DItem{def:semiring/integral_domains} \Def{Integral domains} are commutative rings\Tinyref{def:semiring/commutative_ring} with no zero divisors\Tinyref{def:semiring/zero_divisor}.

    \DItem{def:semiring/division_ring} \Def{Division rings} are nontrivial\Tinyref{def:first_order_structure/minimal} unital rings\Tinyref{def:semiring/unital_ring} in which all nonzero elements are units, i.e. \( (F \setminus \{ 0 \}, \cdot) \) is a group\Tinyref{def:magma/group}.

    In order to fit multiplicative invertibility as an axiom for \cref{def:algebraic_theory}, we can use the following formula:
    \begin{equation*}
      \forall \xi ((\xi \doteq 0) \lor \exists \eta (\xi \cdot \eta \doteq 1))
    \end{equation*}
    or add an additional operation \( (\cdot)^{-1} \) that inverts all nonzero elements and fixes zero, that is,
    \begin{equation*}
      \forall \xi (\xi \cdot \xi^{-1} \doteq 1),
    \end{equation*}
    where we define \( 0^{-1} = 0 \). This is only a formalism since \( 0 \) is not actually \enquote{invertible}, but it is required if we wish to avoid existential quantifiers.

    \DItem{def:semiring/field} \Def{Fields} are commutative\Tinyref{def:algebraic_theory/commutativity} division rings\Tinyref{def:semiring/division_ring}, i.e. \( (F \setminus \{ 0 \}, \cdot) \) is an abelian group\Tinyref{def:magma/abelian_group}. The category of fields is denoted by \( \Cat{Field} \).
  \end{defenum}
\end{definition}

\begin{proposition}\label{def:semiring_properties}
  Any semiring \( R \) has the following basic properties:
  \begin{defenum}
    \DItem{def:semiring_properties/zero_is_absorbing} Multiplication by \( 0 \) is absorbing\Tinyref{def:algebraic_theory/absorbing_element}, that is, \( x0 = 0x = 0 \) for any \( x \in R \).
  \end{defenum}
\end{proposition}
\begin{proof}\mbox{}
  \begin{itemize}
    \RItem{def:semiring_properties/zero_is_absorbing} Follows from \cref{def:left_module_properties/ring_zero_is_absorbing} and \cref{def:left_module_properties/module_zero_is_absorbing}.
  \end{itemize}
\end{proof}

\begin{definition}\label{def:semiring_kernel}
  Let \( X \) be an arbitrary set and let \( R \) be a unital ring\Tinyref{def:semiring/unital_ring}.

  The \Def{kernel} \( \ker(f) \) of a function \( f: X \to R \) is the preimage\Tinyref{def:function_preimage} \( f^{-1}(0_R) \).
\end{definition}

\begin{proposition}\label{thm:ring_homomorphism_simpler_conditions}
  A function \( f: R \to S \) between the rings \( R \) and \( S \) is a homomorphism in the sense of \cref{def:first_order_homomorphism} if and only if for any \( x, y \in R \) it satisfies
  \begin{equation}\label{thm:ring_homomorphism_simpler_conditions/condition}
    \begin{dcases}
      f(x + y) &= f(x) + f(y), \\
      f(xy) &= f(x) f(y), \\
      f(1_R) &= 1_S.
    \end{dcases}
  \end{equation}

  Note that the last condition is only for unital rings.

  In other words, if a function satisfies \cref{thm:ring_homomorphism_simpler_conditions/condition}, the following are automatically satisfied:
  \begin{itemize}
    \item \( f(0_R) = 0_S \)
    \item for all \( x \in R \), we have \( f(-x) = -f(x) \)
    \item for all units \( x \in R \), we have \( f(x^{-1}) = f(x)^{-1} \)
  \end{itemize}
\end{proposition}
\begin{proof}
  Since \( (R, +) \) and \( (S, +) \) are groups, the first two equalities from \cref{thm:group_homomorphism_single_condition}.

  The proof of \( f(x^{-1}) = f(x)^{-1} \) is analogous to \cref{thm:group_homomorphism_single_condition}.
\end{proof}

\begin{definition}\label{def:semiring_direct_product}
  Let \( \{ X_i \}_{i \in I} \) be a nonempty family of rings.

  Analogously to \cref{def:group_direct_product}, we define their \Def{direct product} as the ring \( \prod_{i \in I} X_i \), the operations defined componentwise as
  \begin{align*}
    &\{ x_i \}_{i \in I} + \{ y_i \}_{i \in I}
    \coloneqq
    \{ x_i + y_i \}_{i \in I}, \\
    &\{ x_i \}_{i \in I} \cdot \{ y_i \}_{i \in I}
    \coloneqq
    \{ x_i \cdot y_i \}_{i \in I}.
  \end{align*}

  We define their \Def{direct sum} as the subring of \( \prod_{i \in I} X_i \)\Tinyref{def:semiring_direct_product} where only finitely many components of any ring element are different from zero.
\end{definition}

\begin{proposition}\label{thm:ring_categorical_limits}
  We are interested in categorical limits\Tinyref{def:categorical_limit} and colimits\Tinyref{def:categorical_colimit} in \( \Cat{Ring} \). If \( \{ X_i \}_{i \in I} \) is an indexed family of rings, then
  \begin{defenum}
    \DItem{thm:ring_categorical_limits/product} their categorical product\Tinyref{def:categorical_product} is their direct product\Tinyref{def:semiring_direct_product} \( \prod_{i \in I} X_i \), the projection morphisms being inherited from \cref{thm:set_categorical_limits/product}.
  \end{defenum}
\end{proposition}

\begin{definition}\label{def:opposite_ring}\cite[555]{Knapp2016BAlg}
  The opposite ring \( R^{-1} \) of \( R \) is defined as the same abelian group with the order of multiplication reversed. They are obviously isomorphic for commutative rings.
\end{definition}

\begin{definition}\label{def:ring_commutator}
  Let \( R \) be a ring. The commutator of \( x, y \in R \) is defined as
  \begin{equation*}
    [x, y] \coloneqq xy - yx.
  \end{equation*}

  The commutator ideal of \( R \) is the ideal generated\Tinyref{def:ring_ideal_generators} by all the commutators in \( G \).
\end{definition}

\begin{proposition}\label{thm:quotient_by_commutator_ideal}
  The quotient \( R / I \) of any unital ring\Tinyref{def:semiring/unital_ring} \( R \) by its commutator ideal \( I \) is commutative\Tinyref{def:semiring/commutative_ring}.
\end{proposition}

\begin{definition}\label{def:endomorphism_ring}
  Let \( X \) be an abelian group and let \( \End(X) \) be set of endomorphism over \( X \). We define two operations:
  \begin{itemize}
    \item Pointwise addition \( [f + g](x) \coloneqq f(x) + g(x) \).
    \item Multiplication by composition \( [fg](x) \coloneqq f(g(x)) \).
  \end{itemize}

  These operations make \( \End(X) \) into a unital ring.
\end{definition}

\begin{definition}\label{def:field_extension}
  If \( F \) and \( G \) are fields and \( G \) is a subring\Tinyref{def:first_order_structure/substructure} of \( F \), we say that \( G \) is a \Def{subfield} of \( F \) and that \( F \) is a \Def{field extension} of \( G \).

  Field extension are also denoted as \( F / G \) to highlight the roles of \( F \) and \( G \). This is not a quotient ring but simply a notation. See \cref{def:galois_group}.
\end{definition}

\begin{definition}\label{def:galois_group}\cite[124]{Knapp2016BAlg}
  Let \( F \) be a field extension\Tinyref{def:field_extension} of \( G \). The group \( \Gal(F / G) \) of automorphisms of \( F \) that leave \( G \) fixed is called the \Def{Galois group} of the field extension \( F / G \).
\end{definition}

\begin{definition}\label{def:functions_vanish_nowhere}
  Let \( \Cal{F} \) be a family of functions (not necessarily homomorphisms) between the rings \( R \) and \( S \). We say that \( \Cal{F} \) \Def{vanishes nowhere} if for every \( x \in R \) there exists a function \( f \in \Cal{F} \) such that \( f(x) \neq 0_S \).
\end{definition}
