\subsection{Numbers}\label{subsec:numbers}

\begin{definition}\label{def:peano_arithmetic}\cite[1]{Peano1889}
  The Peano arithmetic (see \cref{remark:peano_arithmetic_zero}) is a first-order logic language\Tinyref{def:first_order_logic_language} with 
  \begin{itemize}
    \item formal equality \( \doteq \).
    \item a constant \( 0 \) (see \cref{remark:peano_arithmetic_zero}).
    \item a unary function \( s \).
  \end{itemize}
  and axioms (see \cref{def:propositional_theory})
  \begin{defenum}
    \DItem{def:peano_arithmetic/PA1}[PA1] \( \forall \xi \forall \eta (\xi \doteq \eta \iff s(\xi) \doteq s(\eta)) \)
    \DItem{def:peano_arithmetic/PA2}[PA2] \( \forall \xi \neg s(\xi) \)
    \DItem{def:peano_arithmetic/PA3}[PA3] (the \Def{axiom schema of induction}) for all formulas \( \varphi(\xi) \) (it is possible for \( \varphi \) to have other free variables), we have
    \begin{equation*}
      \varphi(0) \land \forall \xi (\varphi(\xi) \implies \varphi(s(\xi)))
    \end{equation*}
  \end{defenum}

  Any of the models of this theory is called \Def{the set of natural numbers \( \BB{N} \)}. See \cref{def:natural_numbers} for a list of some models.
\end{definition}

\begin{remark}\label{remark:peano_arithmetic_zero}
  It is common to consider the first natural numbers to be zero (e.g. \cite[67]{Enderton1977}). Peano himself, however, considered one to be the first - see \cite[1]{Peano1889}.

  Whether \( 0 \) is a natural number or not, i.e. whether \( \varnothing \) encodes \( 0 \) or \( 1 \), is a matter of convention and notation like \( \Z^{\geq 0} \) and \( \Z^{>0} \) or \( \BB{N}^0 \) and \( \BB{N}^+ \) is sometimes used.

  Peano's original axioms referred to sets rather than first-order logic languages, however this more general framework is more suitable nowadays.

  It is easier to construct models of natural numbers that semantically include zero (see \cref{def:natural_numbers}), so we will construct the natural numbers along with zero and later on mostly use the set \( \Z^{>0} \).
\end{remark}

\begin{definition}\label{def:peano_arithmetic_models}
  We can define the natural numbers as any model of Peano arithmetic (see \cref{def:peano_arithmetic}). A set-theoretic model of the Peano arithmetic consists of a triple \( N, S, e \), where
  \begin{itemize}
    \item \( N \) is the universum, i.e. the set of natural numbers.
    \item \( S: N \to N \) is the successor function that increments a number.
    \item \( e \) is the first number, taken to be one (see \cref{remark:peano_arithmetic_zero}).
  \end{itemize}

  Any of the following (and many more) can be taken as a definition for the natural numbers:
  \begin{defenum}
    \DItem{def:natural_numbers/omega} the number of nested sets in the model
    \begin{itemize}
      \item \( N = \omega \) is the smallest inductive set\Tinyref{def:smallest_inductive_set}.
      \item \( S \) is the set-theoretic successor function\Tinyref{def:successor_operator}.
      \item \( e = \varnothing \).
    \end{itemize}

    \DItem{def:natural_numbers/language} the word length in the model
    \begin{itemize}
      \item \( N = \{ a \}^{*} \) is the Kleene star\Tinyref{def:language} of the singleton alphabet \( \{ a \} \).
      \item \( S(w) \coloneqq w \cdot a \) concatenates the letter at the end of a word.
      \item \( e = \varepsilon \) is the empty word.
    \end{itemize}

    \DItem{def:natural_numbers/vector_spaces} the vector space dimension\Tinyref{def:vector_space_dimension} in the model
    \begin{itemize}
      \item \( N \) is the set of all finite-dimensional vector spaces\Tinyref{def:vector_space} over a field\Tinyref{def:semiring/field} \( F \).
      \item \( S(V) \coloneqq V \times F \) is the module direct sum\Tinyref{def:left_module_direct_product} of \( V \) and \( F \).
      \item \( e = 0_F \) is the field considered as a vector space.
    \end{itemize}
  \end{defenum}
\end{definition}

\begin{definition}\label{def:natural_number_operations}
  Fix any set-theoretic model \( \BB{N} \) of the natural numbers\Tinyref{def:natural_numbers}. We use structural induction to define addition and multiplication:
  \begin{align*}
    x + y &\coloneqq \begin{cases}
      x, &y = 0, \\
      S(x + y'), &y = S(y'),
    \end{cases}
    \\
    x \cdot y &\coloneqq \begin{cases}
      0, &y = 0, \\
      x \cdot y' + x, &y = S(y'),
    \end{cases}
  \end{align*}

  We also define a partial order relation as
  \begin{equation*}
    x \leq y \iff \exists z \in \BB{N}: y = x + z.
  \end{equation*}
\end{definition}

\begin{proposition}\label{def:natural_numbers}
  \begin{itemize}\mbox{}
    \item The set \( \BB{N} \) is a commutative monoid\Tinyref{def:magma} under addition with \( 0 \) as identity.
    \item The set \( \BB{N} \setminus \{ 0 \} \) is a commutative monoid under multiplication with \( 1 \) as identity.
    \item The set \( \BB{N} \) is well-ordered by \( \leq \).
  \end{itemize}
\end{proposition}
\begin{proof}
  \begin{itemize}\mbox{}
    \item Fix \( x + y \). We use simultaneous induction by \( y \):
    \begin{itemize}
      \item if \( y = 0 \), induction by \( x \):
      \begin{itemize}
        \item if \( x = y = 0 \),
        \begin{equation*}
          x + y = 0 + 0 = 0 = 0 + 0 = y + x
        \end{equation*}

        \item if \( x = S(x') \) and \( y \) and \( x' \) commute with each other, we have
        \begin{align*}
          x + y
          &=
          S(x') + 0
          = \\ &=
          S(x')
          = \\ &=
          S(x' + 0)
          = \\ &=
          S(x' + y)
          = \\ &=
          S(y + x')
          = \\ &=
          y + S(x')
          =
          y + x.
        \end{align*}
      \end{itemize}

      \item if \( y = S(y') \) and \( y' \) commutes with \( x \), we use induction by \( x \):
      \begin{itemize}
        \item if \( x = 0 \),
        \begin{equation*}
          x + y = 0 + S(y') = S(0 + y') = S(y' + 0) = S(y') = y = y + x.
        \end{equation*}

        \item if \( x = S(x') \) and \( y' \) and \( x' \) commute with each other, we have
        \begin{align*}
          x + y
          &=
          x + S(y')
          = \\ &=
          S(x + y')
          = \\ &=
          S(y' + x)
          = \\ &=
          S(y' + S(x'))
          = \\ &=
          S(S(y' + x'))
          = \\ &=
          S(S(x' + y'))
          = \\ &=
          S(x' + S(y'))
          = \\ &=
          S(S(y') + x')
          = \\ &=
          S(y') + S(x')
          =
          y + x.
        \end{align*}
      \end{itemize}
    \end{itemize}

    \item Analogous.

    \item See \cref{thm:natural_numbers_are_well_ordered}.
  \end{itemize}
\end{proof}

\begin{definition}\label{def:integers}
  The set \( \Z \) of \Def{integers} is defined as the Grothendieck completion\Tinyref{thm:monoid_completion_to_abelian_group} of the commutative monoid \( (\BB{N}, +) \).

  Let \( \oplus \), \( \odot \) and \( \leq_N \) be the operations in \( \BB{N} \)\Tinyref{def:natural_number_operations}. Since either \( x \in \Z \) or \( -x \in \Z \) is isomorphic to a natural number, we extend the operations to \( \Z \) as follows:
  \begin{itemize}
    \item addition is defined in the completion.
    \item multiplication is defined as follows:
    \begin{equation*}
      x \cdot y \coloneqq \begin{cases}
        x \odot y, & x \geq 0 \iff y \geq 0 \\
        (-x) \odot y, & x < 0 \text{ and } y \geq 0 \\
        x \odot (-y), & x \geq 0 \text{ and } y < 0
      \end{cases}
    \end{equation*}

    \item the order is inherited
    \item the additional absolute value\Tinyref{def:absolute_value} operation is defined as
    \begin{equation*}
      \Abs{x} \coloneqq \begin{cases}
        x, &x \geq 0, \\
        -x, &x < 0.
      \end{cases}
    \end{equation*}
  \end{itemize}
\end{definition}
\begin{proof}
  The proof that multiplication and absolute values are well defined can be done similarly to the proof of \cref{thm:monoid_completion_to_abelian_group}.

  Integer multiplication obviously generalizes natural number multiplication.
\end{proof}

\begin{proposition}\label{thm:integers_are_euclidean_domain}
  The domain\Tinyref{def:semiring/integral_domain} of integers \( \Z \) is Euclidean\Tinyref{def:semiring/euclidean_domain} with \( \delta(n) \coloneqq \Abs{n} \). Furthermore, the remainder and quotient are unique.
\end{proposition}
\begin{proof}
  Let \( a, b \in \Z \) and \( b \neq 0 \). Suppose that \( b > 0 \). Define
  \begin{align*}
    q &\coloneqq \max \{ q \in \Z \colon bq \leq \Abs{a} \} \\
    r &\coloneqq a - bq.
  \end{align*}

  It remains to show that either \( r = 0 \) or \( \delta(r) < \delta(b) \).

  Note that \( r = 0 \) if and only if \( b \) is a divisor\Tinyref{def:commutative_ring_division} of \( a \).

  Suppose that \( b \) is not a divisor of \( a \). Note that \( g \geq 0 \) and \( r > 0 \). If \( r \geq b \), this would imply\LEM \( r - b \geq 0 \) and
  \begin{equation*}
    a = bq + r = b(q + 1) + (r - b) \geq b(q + 1),
  \end{equation*}
  which would contradict the maximality of \( q \). Thus \( r < b \).

  It remains to show uniqueness. Suppose that \( a = bq + r = bq' + r' \). Then
  \begin{equation*}
    0 = b(q - q') + (r - r').
  \end{equation*}

  Thus \( b \mid r - r' \). But \( -b < r - r' < b \), which implies that \( r = r' \). Thus also implies that \( q = q' \) since \( b \neq 0 \).

  Now suppose that \( b < 0 \). Define
  \begin{align*}
    q &\coloneqq \min \{ q \in \Z \colon -\Abs{a} \leq bq \} \\
    r &\coloneqq a - bq.
  \end{align*}

  Suppose that \( b \) is not a divisor of \( a \). Note that \( q \geq 0 \) and \( r < 0 \). If \( r \leq b \), this would imply\LEM \( r - b \leq 0 \) and
  \begin{equation*}
    a = bq + r = b(q + 1) + (r - b) \leq b(q + 1),
  \end{equation*}
  which would contradict the minimality of \( q \). Thus \( r > b \) and, since both \( b \) and \( r \) are negative, \( \Abs{r} < \Abs{b} \).

  To see uniqueness, suppose that \( a = bq + r = bq' + r' \). Thus \( b \mid r - r' \). But \( -\Abs{b} = b < r - r' < -b = \Abs{b} \), which implies that \( r = r' \) and \( q = q' \).

  In both cases, we obtained unique integers \( q \) and \( r \) such that \( a = bq + r \) with \( \Abs{r} < \Abs{b} \).
\end{proof}

\begin{remark}\label{remark:units_in_rings_etymology}
  An integer \( a \) is divisible by \( b \neq 0 \) if there exists a number \( q \) such that
  \begin{equation*}
    a = qb.
  \end{equation*}

  Obviously \( q = (-q)(-1) \) so the following also holds:
  \begin{equation*}
    a = [(-q)(-1)]b = (-q)(-b),
  \end{equation*}
  hence \( a \) is also divisible by \( -b \).

  For any nonzero number \( b = 1 \cdot b \) that divides \( a \), the number \( -b = (-1) \cdot b \) also divides \( a \). Both \( 1 \) and \( -1 \) have unit norm (that is, \( \Abs{1} = \Abs{-1} = 1 \)) so it is reasonable to call them \enquote{units}. They are the only integers \( e \) with the property that if \( b | a \), then \( eb | a \). This is probably the reason why invertible elements in arbitrary rings are named \enquote{units}.

  Consider fields, in which all nonzero elements are units. It makes no sense to speak of divisibility whatsoever because any real number \( a \) is divisible by any nonzero real number \( b \). Putting \( q \coloneqq \frac a b \) satisfies the divisibility condition. Now if \( e \) is any unit in \( \R \), we have
  \begin{equation*}
    a = qb = q(e^{-1} e) b = (qe^{-1}) (eb),
  \end{equation*}
  hence \( eb \) also divides \( a \).
\end{remark}

\begin{lemma}[Euclid's lemma]\label{thm:euclids_lemma}
  An integer\Tinyref{def:integers} is prime\Tinyref{def:prime_ring_ideal} if and only if it is irreducible.
\end{lemma}
\begin{proof}
  Follows from \cref{thm:pid_prime_iff_irreducible}.
\end{proof}

\begin{definition}\label{def:prime_number}
  Despite negative integers being prime elements\Tinyref{thm:euclids_lemma} of the ring \( \Z \), we only call positive prime integers \Def{prime numbers}. That is, a positive integer is prime if it has no divisors except \( 1 \) and itself.

  Non-prime integers are called \Def{composite numbers}.

  Two integers \( n, m \) are called \Def{coprime} (see \cref{def:coprime_ring_ideal}) if \( \gcd(n, m) = 1 \).
\end{definition}

\begin{proposition}[Fermat's little theorem]\label{thm:fermats_little_theorem}
  If \( p \) is a prime number\Tinyref{def:prime_number}, for any integer \( x \) we have
  \begin{equation*}
    p \mid (x^p - x).
  \end{equation*}
\end{proposition}

\begin{definition}\label{def:rational_numbers}
  The \Def{rational numbers} \( \Q \) are the localization\Tinyref{def:ring_localization} of the integers\Tinyref{def:integers} with \( S = \Z \setminus \{ 0 \} \). Both operations from \( \Z \) are inherited in \( \Q \) and all nonzero elements in \( \Q \) are now invertible, which makes \( \Q \) a field.
\end{definition}

\begin{definition}\label{def:real_numbers}
  The \Def{real numbers} \( \R \) are the metric space completion\Tinyref{def:complete_metric_space} of \( \Q \) with respect to the absolute value.
\end{definition}

\begin{definition}\label{def:floor_ceiling_functions}
  Let \( x \in \R \) be a real number. We define its floor function
  \begin{equation*}
    \lfloor x \rfloor \coloneqq \max \{ n \in \Z : n \leq x \}
  \end{equation*}
  and its ceiling function
  \begin{equation*}
    \lceil x \rceil \coloneqq \min \{ n \in \Z : n \geq x \}.
  \end{equation*}
\end{definition}

\begin{definition}\label{def:double_index_maps}
  Let \( n, m \in \BB{N} \). We want to map single indices to double indices and vice versa. Define the mutually inverse operations
  \begin{align*}
    &\sharp_{n,m}: \{ 1, 2, \ldots, nm \} \to \{ 1, 2, \ldots, n \} \times \{ 1, 2, \ldots, n \} \\
    &\sharp_{n,m}(k) \coloneqq (\lfloor k / m \rfloor + 1, k \Mod m + 1)
    &\\
    &\\
    &\flat_{n,m}: \{ 1, 2, \ldots, n \} \times \{ 1, 2, \ldots, n \} \to \{ 1, 2, \ldots, nm \} \\
    &\flat_{n,m}(i, j) \coloneqq (i - 1) \cdot m + (j - 1).
  \end{align*}

  For \( p \)-arity indices, inductively define
  \begin{equation*}
    \sharp_{n_1, n_2, \ldots, n_p}(k) \coloneqq (\sharp_{n_1, n_2, \ldots, n_{p-1}}(a - 1), b),
  \end{equation*}
  where
  \begin{equation*}
    (a, b) = \sharp_{(n_1 n_2 \cdots n_{p-1}), n_p}.
  \end{equation*}
  and
  \begin{equation*}
    \flat_{n_1, n_2, \ldots, n_p}(i_1, \ldots, i_p) \coloneqq \flat_{n_1, n_2, \ldots, n_{p-2}, (n_{p-1} n_p)}(i_1, \ldots, i_{p-2}, \flat_{(n_{p-1} n_p), n_p}(i_{p-1}, i_p) + 1).
  \end{equation*}
\end{definition}
\begin{proof}
  \begin{align*}
    \sharp(\flat(i, j))
    &=
    \sharp((i - 1) \cdot m + (j - 1))
    = \\ &=
    \Big( \lfloor ((i - 1) \cdot m + (j - 1)) / m \rfloor + 1, ((i - 1) \cdot m + (j - 1)) \Mod m + 1\Big)
    = \\ &=
    \Big( i - 1 + \lfloor (j - 1) / m \rfloor + 1, (j - 1) \Mod m + 1 \Big)
    =
    (i, j).
  \end{align*}

  Thus \( \flat \) is left invertible and injective. It is also right invertible since
  \begin{align*}
    \flat(\sharp(k))
    &=
    \flat\left( \Big( \lfloor k / m \rfloor + 1, k \Mod m + 1 \Big) \right)
    = \\ &=
    (\lfloor k / m \rfloor) \cdot m + (k \Mod m)
    = \\ &=
    [k - (k \Mod m)] + (k \Mod m)
    =
    k.
  \end{align*}

  Hence \( \flat \) is bijective and \( \sharp \) is its inverse.
\end{proof}

\begin{definition}\label{def:complex_numbers}
  The \Def{complex numbers} \( \Co \) are the algebra\Tinyref{def:algebra_over_ring} obtained from the vector space \( \R^2 \) with the multiplication operation
  \begin{align*}
    &\cdot: \Co \times \Co \to \Co \\
    &(a, b) \cdot (c, d) \coloneqq (ac - bd, ad + bc).
  \end{align*}

  We define the unary \Def{conjugation} operation as \( \Ol{(a, b)} \coloneqq (a, -b) \).

  We also define the canonical embedding
  \begin{align*}
    &\iota: \R \to \Co \\
    &\iota(x) \coloneqq (x, 0)
  \end{align*}
  and the shorthand \( i \coloneqq (0, 1) \) so that for any \( a, b \in \R \) we have
  \begin{equation*}
    (a, b) = \iota(a) + bi
  \end{equation*}
  usually written as
  \begin{equation*}
    (a, b) = a + bi.
  \end{equation*}
\end{definition}

\begin{theorem}[Fundamental theorem of algebra]\label{thm:fundamental_theorem_of_algebra}
  Every nonconstant complex polynomial has at least one root.
\end{theorem}

\begin{corollary}\label{thm:complex_polynomials_have_n_roots}
  A complex polynomial of degree \( n \) has exactly \( n \) roots.
\end{corollary}
\begin{proof}
  Follows from \cref{thm:integral_domain_polynomial_root_limit} and \cref{thm:fundamental_theorem_of_algebra}.
\end{proof}
