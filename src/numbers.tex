\subsection{Numbers}\label{subsec:numbers}

\begin{definition}\label{def:peano_arithmetic}\cite[1]{Peano1889}
  The Peano arithmetic (see \cref{remark:peano_arithmetic_zero}) is a first-order language\Tinyref{def:first_order_language} with 
  \begin{itemize}
    \item formal equality \( \doteq \).
    \item a constant \( 0 \) (see \cref{remark:peano_arithmetic_zero}).
    \item a unary function \( s \).
  \end{itemize}
  and axioms (see \cref{def:propositional_theory})
  \begin{defenum}
    \DItem{def:peano_arithmetic/PA1}[PA1] \( \forall \xi \forall \eta (\xi \doteq \eta \iff s(\xi) \doteq s(\eta)) \)
    \DItem{def:peano_arithmetic/PA2}[PA2] \( \forall \xi \neg s(\xi) \)
    \DItem{def:peano_arithmetic/PA3}[PA3] (the \Def{axiom schema of induction}) for all formulas \( \varphi(\xi) \) (it is possible for \( \varphi \) to have other free variables), we have
    \begin{equation*}
      \varphi(0) \land \forall \xi (\varphi(\xi) \implies \varphi(s(\xi)))
    \end{equation*}
  \end{defenum}

  Any of the models of this theory is called \Def{the set of natural numbers \( \BB{N} \)}. See \cref{def:natural_numbers} for a list of some models.
\end{definition}

\begin{remark}\label{remark:peano_arithmetic_zero}
  Although it is common to start the natural numbers with zero (e.g. \cite[67]{Enderton1977}). Peano himself, however, started his natural numbers from one - see \cite[1]{Peano1889}.

  Whether \( 0 \) is a natural number or not, i.e. whether \( \varnothing \) encodes \( 0 \) or \( 1 \), is a matter of convention and notation like \( \Cal{Z}^{\geq 0} \) and \( \Cal{Z}^{>0} \) or \( \Cal{N}^0 \) and \( \Cal{N}^+ \) is sometimes used.

  Peano's original axioms referred to sets rather than first-order languages, however this more general framework is more suitable nowadays.

  It is easier to construct models of natural numbers that semantically include zero (see \cref{def:natural_numbers}), so we will construct the natural numbers along with zero and later on mostly use the set \( \Cal{Z}^{>0} \).
\end{remark}

\begin{definition}\label{def:natural_numbers}
  We can define the natural numbers as any model of Peano arithmetic (see \cref{def:peano_arithmetic}). A set-theoretic model of the Peano arithmetic consists of a triple \( N, S, e \), where
  \begin{itemize}
    \item \( N \) is the universum, i.e. the set of natural numbers.
    \item \( S: N \to N \) is the successor function that increments a number.
    \item \( e \) is the first number, taken to be one (see \cref{remark:peano_arithmetic_zero}).
  \end{itemize}

  Any of the following (and many more) can be taken as a definition for the natural numbers:
  \begin{defenum}
    \DItem{def:natural_numbers/omega} the number of nested sets in
    \begin{itemize}
      \item \( N = \omega \) is the smallest inductive set\Tinyref{def:smallest_inductive_set}.
      \item \( S \) is the set-theoretic successor function\Tinyref{def:successor_operator}.
      \item \( e = \varnothing \).
    \end{itemize}

    \DItem{def:natural_numbers/language} the word length in
    \begin{itemize}
      \item \( N = \{ a \}^{*} \) is the Kleene star\Tinyref{def:language} of the singleton alphabet \( \{ a \} \).
      \item \( S(w) \coloneqq w \cdot a \) concatenates the letter at the end of a word.
      \item \( e = \varepsilon \) is the empty word.
    \end{itemize}

    \DItem{def:natural_numbers/vector_spaces} the vector space dimension\Tinyref{def:vector_space_dimension} in
    \begin{itemize}
      \item \( N \) is the set of all finite-dimensional vector spaces\Tinyref{def:vector_space} over a field\Tinyref{def:semiring/field} \( F \).
      \item \( S(V) \coloneqq V \times F \) is the module direct sum\Tinyref{def:left_module_direct_product} of \( V \) and \( F \).
      \item \( e = 0_F \) is the field considered as a vector space.
    \end{itemize}
  \end{defenum}
\end{definition}

\begin{definition}\label{def:natural_number_operations}
  Fix any set-theoretic model \( \BB{N} \) of the natural numbers\Tinyref{def:natural_numbers}. We use structural induction to define addition and multiplication:
  \begin{align*}
    x + y &\coloneqq \begin{cases}
      x, &y = 0, \\
      S(x + y'), &y = S(y'),
    \end{cases}
    \\
    x \cdot y &\coloneqq \begin{cases}
      0, &y = 0, \\
      x \cdot y' + x, &y = S(y'),
    \end{cases}
  \end{align*}

  We also define a partial order relation as
  \begin{equation*}
    x \leq y \iff \exists z \in \BB{N}: y = x + z.
  \end{equation*}
\end{definition}

\begin{proposition}\label{def:natural_numbers_semiring}
  \begin{itemize}\mbox{}
    \item The set \( \BB{N} \) is a commutative monoid\Tinyref{def:magma} under addition with \( 0 \) as identity.
    \item The set \( \BB{N} \setminus \{ 0 \} \) is a commutative monoid under multiplication with \( 1 \) as identity.
    \item The set \( \BB{N} \) is well-ordered by \( \leq \).
  \end{itemize}
\end{proposition}
\begin{proof}
  \begin{itemize}\mbox{}
    \item Fix \( x + y \). We use simultaneous induction by \( y \):
    \begin{itemize}
      \item if \( y = 0 \), induction by \( x \):
      \begin{itemize}
        \item if \( x = y = 0 \),
        \begin{equation*}
          x + y = 0 + 0 = 0 = 0 + 0 = y + x
        \end{equation*}

        \item if \( x = S(x') \) and \( y \) and \( x' \) commute with each other, we have
        \begin{align*}
          x + y
          &=
          S(x') + 0
          = \\ &=
          S(x')
          = \\ &=
          S(x' + 0)
          = \\ &=
          S(x' + y)
          = \\ &=
          S(y + x')
          = \\ &=
          y + S(x')
          =
          y + x.
        \end{align*}
      \end{itemize}

      \item if \( y = S(y') \) and \( y' \) commutes with \( x \), we use induction by \( x \):
      \begin{itemize}
        \item if \( x = 0 \),
        \begin{equation*}
          x + y = 0 + S(y') = S(0 + y') = S(y' + 0) = S(y') = y = y + x.
        \end{equation*}

        \item if \( x = S(x') \) and \( y' \) and \( x' \) commute with each other, we have
        \begin{align*}
          x + y
          &=
          x + S(y')
          = \\ &=
          S(x + y')
          = \\ &=
          S(y' + x)
          = \\ &=
          S(y' + S(x'))
          = \\ &=
          S(S(y' + x'))
          = \\ &=
          S(S(x' + y'))
          = \\ &=
          S(x' + S(y'))
          = \\ &=
          S(S(y') + x')
          = \\ &=
          S(y') + S(x')
          =
          y + x.
        \end{align*}
      \end{itemize}
    \end{itemize}

    \item Analogous.

    \item See \cref{thm:natural_numbers_are_well_ordered}.
  \end{itemize}
\end{proof}

\begin{definition}\label{def:double_index_maps}
  Let \( n, m \in \BB{N} \). We want to map single indices to double indices and vice versa. Define the mutually inverse operations
  \begin{align*}
    &\sharp_{n,m}: \{ 1, 2, \ldots, nm \} \to \{ 1, 2, \ldots, n \} \times \{ 1, 2, \ldots, n \} \\
    &\sharp_{n,m}(k) \coloneqq (\lfloor k / m \rfloor + 1, k \Mod m + 1)
    &\\
    &\\
    &\flat_{n,m}: \{ 1, 2, \ldots, n \} \times \{ 1, 2, \ldots, n \} \to \{ 1, 2, \ldots, nm \} \\
    &\flat_{n,m}(i, j) \coloneqq (i - 1) \cdot m + (j - 1).
  \end{align*}

  For \( p \)-arity indices, inductively define
  \begin{equation*}
    \sharp_{n_1, n_2, \ldots, n_p}(k) \coloneqq (\sharp_{n_1, n_2, \ldots, n_{p-1}}(a - 1), b),
  \end{equation*}
  where
  \begin{equation*}
    (a, b) = \sharp_{(n_1 n_2 \cdots n_{p-1}), n_p}.
  \end{equation*}
  and
  \begin{equation*}
    \flat_{n_1, n_2, \ldots, n_p}(i_1, \ldots, i_p) \coloneqq \flat_{n_1, n_2, \ldots, n_{p-2}, (n_{p-1} n_p)}(i_1, \ldots, i_{p-2}, \flat_{(n_{p-1} n_p), n_p}(i_{p-1}, i_p) + 1).
  \end{equation*}
\end{definition}
\begin{proof}
  \begin{align*}
    \sharp(\flat(i, j))
    &=
    \sharp((i - 1) \cdot m + (j - 1))
    = \\ &=
    \Big( \lfloor ((i - 1) \cdot m + (j - 1)) / m \rfloor + 1, ((i - 1) \cdot m + (j - 1)) \Mod m + 1\Big)
    = \\ &=
    \Big( i - 1 + \lfloor (j - 1) / m \rfloor + 1, (j - 1) \Mod m + 1 \Big)
    =
    (i, j).
  \end{align*}

  Thus \( \flat \) is left invertible and injective. It is also right invertible since
  \begin{align*}
    \flat(\sharp(k))
    &=
    \flat\left( \Big( \lfloor k / m \rfloor + 1, k \Mod m + 1 \Big) \right)
    = \\ &=
    (\lfloor k / m \rfloor) \cdot m + (k \Mod m)
    = \\ &=
    [k - (k \Mod m)] + (k \Mod m)
    =
    k.
  \end{align*}

  Hence \( \flat \) is bijective and \( \sharp \) is its inverse.
\end{proof}

\begin{definition}\label{def:integers}
  The set \( \Z \) of \Def{integers} is defined as the Grothendieck completion\Tinyref{thm:monoid_completion_to_abelian_group} of the commutative monoid \( (\BB{N}, +) \).

  Let \( \oplus \), \( \odot \) and \( \leq_N \) be the operations in \( \BB{N} \)\Tinyref{def:natural_number_operations}. Since either \( x \in \Z \) or \( -x \in \Z \) is isomorphic to a natural number, we extend them to \( \Z \) as follows:
  \begin{itemize}
    \item addition is defined in the completion.
    \item multiplication is defined as follows:
    \begin{equation*}
      x \cdot y \coloneqq \begin{cases}
        x \odot y, & x \geq 0 \iff y \geq 0 \\
        (-x) \odot y, & x < 0 \text{ and } y \geq 0 \\
        x \odot (-y), & x \geq 0 \text{ and } y < 0
      \end{cases}
    \end{equation*}

    \item the order is inherited
    \item the additional absolute value\Tinyref{def:absolute_value} operation is defined as
    \begin{equation*}
      \Abs{x} \coloneqq \begin{cases}
        x, &x \geq 0, \\
        -x, &x < 0.
      \end{cases}
    \end{equation*}
  \end{itemize}
\end{definition}
