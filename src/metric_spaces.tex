\section{Metric spaces}\label{sec:metric_spaces}

\begin{definition}\label{def:metric_space}\cite[248]{Engelking1989}
  A \ul{metric space} is a set $X$ along with a function $\mu: X \times X \to \BB{R}^{\geq 0}$, called a \ul{metric} or \ul{distance function}, such that
  \begin{description}
    \DItem{identity}{def:metric_space/identity} $\mu(x, y) = 0 \iff x = y$
    \DItem{symmetry}{def:metric_space/symmetry} $\mu(x, y) = \mu(y, x)$
    \DItem{triangle inequality}{def:metric_space/triangle_inequality} $\mu(x, y) \leq \mu(x, z) + \mu(z, y)$
  \end{description}

  If instead of \ref{def:metric_space/identity} we have the weaker condition
  \begin{description}
    \DItem{pseudometric identity}{def:metric_space/pseudometric_identity} $\forall x \in X, \mu(x, x) = 0$,
  \end{description}
  we call $\mu$ a \ul{pseudometric} and $(X, \mu)$ a \ul{pseudometric space}.

  \begin{defenum}
    \item\label{def:metric_space/ball} Define the function
    \begin{align*}
      &B: X \times \BB{R}^{>0} \to \Power(X), \\
      &B(x, r) \coloneqq \{ y \in X \colon \mu(x, y) = r \}.
    \end{align*}

    The set $B(x, r)$ is called a \ul{ball with center $x$ and radius $r$}.

    \item\label{def:metric_space/bounded_set} A set $A \subseteq X$ is called \ul{bounded} if it is contained in some ball $B(x, r)$.

    \item\label{def:metric_space/bounded_metric} If every set is bounded, we say that the metric itself is bounded.

    \item\label{def:metric_space/diameter} Define the function
    \begin{align*}
      &\Diam: \Power(X) \to \BB{R}^{\geq 0}, \\
      &\Diam(A) \coloneqq \sup \{ \mu(x, y) \colon x, y \in A \}.
    \end{align*}

    We call the number $\Diam(A)$ the \ul{diameter of $A$}.

    \item\label{def:metric_space/distance} Define the function
    \begin{align*}
      &\Dist: X \times \Power(X) \to \BB{R}^{\geq 0}, \\
      &\Dist(x, A) \coloneqq \inf \{ \mu(x, a) \colon a \in A \}.
    \end{align*}

    We call the number $\Dist(x, A)$ the \ul{distance from the point $x$ to the set $A$}. We use the convention that the infimum of an empty set of real numbers is $+\infty$, hence $\Dist(x, \varnothing) = \infty$.
  \end{defenum}
\end{definition}

\begin{note}\label{note:bounded_set_metric_order_equivalence}
  A set $A$ in a metric space $(X, \mu)$ is bounded\Tinyref{def:metric_space/bounded_set} if and only if for any point $x \in X$, the set $\{ \mu(x, a) \colon a \in A \}$ is bounded as a poset\Tinyref{def:poset/bounded_set}.
\end{note}

\begin{definition}\label{def:metric_topology}\cite[249]{Engelking1989}
  Let $(X, \mu)$ be a metric space. We define the \ul{metric topology} or \ul{induced topology} $\Cal{T}$ as the topology\Tinyref{def:topological_space} generated by the neighborhood system\Tinyref{def:topological_local_base}
  \begin{align*}
    \Cal{B}(x) \coloneqq \{ B(x, n) \colon n = 1, 2, 3, \ldots \}.
  \end{align*}

  If for some topological space $(X, \Cal{T})$ there exists a metric such that $\Cal{T}$ is its induced topology, we say that the topology $\Cal{T}$ is \ul{metrizable}.
\end{definition}
\begin{proof}
  This is indeed a neighborhood system as it satisfies \ref{thm:topological_local_base_axioms/BP1}-\ref{thm:topological_local_base_axioms/BP3}.
\end{proof}

\begin{proposition}\label{thm:metric_topology_properties}
  The metric topology $\Cal{T}$ on $X$ induced by $\mu$ has the following properties:
  \begin{defenum}
    \item\label{thm:metric_topology_properties/first_countable} $\Cal{T}$ is first-countable.
    \item\label{thm:metric_topology_properties/hausdorff} $\Cal{T}$ is Hausdorff.
  \end{defenum}
\end{proposition}
