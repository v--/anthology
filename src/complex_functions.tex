\section{Complex analysis}\label{sec:complex_analysis}

A lot of results in this section hold in both \( \BR \) and \( \BC \), so \( \BK \) will refer to either \( \BR \) or \( \BC \).

\subsection{Complex functions}\label{subsec:complex_functions}

\begin{definition}\label{def:sequence_spaces}
  We will define multiple Banach spaces of sequences over \( \BC \).

  \begin{defenum}
    \DItem{def:sequence_spaces/c00} The simplest nontrivial sequence space is that of all sequences with only finitely many nonzero elements. It is denoted by \( c_{00} \). It can be defined as
    \begin{equation*}
      c_{00} \coloneqq \bigcup_{i=1}^\infty \BC^k,
    \end{equation*}
    where \( \BC^k \) is the corresponding tuple \hyperref[def:left_module_of_tuples]{space}.

    This space can be generalized to modules over \hyperref[def:left_module]{dioids}.
  \end{defenum}
\end{definition}

\begin{definition}\label{def:function_spaces}
  We will define multiple Banach spaces of functions over \( \BK \).

  \begin{defenum}
    \DItem{def:function_spaces/c0} Define the set of functions \Def{vanishing at infinity}:
    \begin{equation*}
      C_0(\BC) \coloneqq \{ f: \BC \to \BC \colon f(x) \xrightarrow[\Abs{x} \to \infty]{} 0 \}.
    \end{equation*}

    \DItem{def:function_spaces/c} Fix \hyperref[def:topological_space]{topological space} \( X \). The set \( C(X) = C(X, \BK) \) of all \( \BK \)-valued continuous functions on \( X \) in a Banach space over \( \BK \).
  \end{defenum}
\end{definition}

\begin{theorem}[Arzela-Ascoli]\label{thm:arzela_ascoli}\cite[corollary 10.49]{Knapp2016BAlg}
  Let \( X \) be a \hyperref[def:compact_space]{compact} \hyperref[def:separation_axioms/T2]{Hausdorff} space.

  A family \( \CF \subseteq C(X, \BR) \) of continuous real-valued functions is totally \hyperref[def:totally_bounded_set]{bounded} if and only if it is pointwise \hyperref[def:bounded_function/pointwise]{bounded} and \hyperref[def:function_set_continuity/equicontinuous]{equicontinuous}.
\end{theorem}
