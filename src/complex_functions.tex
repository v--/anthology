\section{Complex analysis}\label{sec:complex_analysis}

A lot of results in this section hold in both \( \R \) and \( \C \), so \( \K \) will refer to either \( \R \) or \( \C \).

\subsection{Complex functions}\label{subsec:complex_functions}

\begin{definition}\label{def:function_spaces}
  We will define multiple Banach spaces of functions over \( \K \).

  \begin{defenum}
    \DItem{def:function_spaces/c0} Define the set of functions \Def{vanishing at infinity}:
    \begin{equation*}
      C_0(\C) \coloneqq \{ f: \C \to \C \colon f(x) \xrightarrow[\Abs{x} \to \infty]{} 0 \}.
    \end{equation*}

    \DItem{def:function_spaces/c} Fix topological space\Tinyref{def:topological_space} \( X \). The set \( C(X) = C(X, \K) \) of all \( \K \)-valued continuous functions on \( X \) in a Banach space over \( \K \).
  \end{defenum}
\end{definition}

\begin{theorem}[Arzela-Ascoli]\label{thm:arzela_ascoli}\cite[corollary 10.49]{Knapp2016BAlg}
  Let \( X \) be a compact\Tinyref{def:compact_set} Hausdorff\Tinyref{def:separation_axioms/T2} space.

  A family \( \F \subseteq C(X, \R) \) of continuous real-valued functions is totally bounded\Tinyref{def:totally_bounded_set} if and only if it is pointwise bounded\Tinyref{def:bounded_function/pointwise} and equicontinuous\Tinyref{def:function_set_continuity/equicontinuous}.
\end{theorem}
