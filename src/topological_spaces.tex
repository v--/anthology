\section{General topology}\label{sec:general_topology}
\subsection{Topological spaces}\label{subsec:topological_spaces}

\begin{definition}\label{def:topological_space}\cite[11]{Engelking1989}
  Let \( X \) be any set and \( \CT \subseteq \Power(X) \) be a family of subsets of \( X \). \( \CT \) is called a \Def{topology on \( X \)} and the tuple \( (X, \CT) \) is said to be a \Def{topological space} if the following axioms are satisfied:
  \begin{description}
    \DItem{def:topological_space/O1}[O1] \( \varnothing, X \in \CT \)
    \DItem{def:topological_space/O2}[O2] \( U, V \in \CT \implies U \cap V \in \CT \)
    \DItem{def:topological_space/O3}[O3] \( \CT' \subseteq \CT \implies \bigcap \CT' \in \CT \)
  \end{description}

  If the topology is obvious from the context, we say that \( X \) is a topological space.

  Elements of the set \( X \) are called \Def{points of the topological space}, elements of \( \CT \) are called \Def{open sets} and set-theoretic complements of open sets are called \Def{closed sets}.

  If \( x \in U \in \CT \), we say that \( U \) is a \Def{neighborhood} of \( x \). Note that some authors (e.g. \cite[38]{Kelley1955}) alternatively define neighborhoods as arbitrary sets that contain an open set that contains \( x \). For simplicity, we define the subfamily
  \begin{equation*}
    \CT(x) \coloneqq \{ U \in \CT \colon x \in U \}.
  \end{equation*}

  We say that \( U \) is a \Def{punctured neighborhood} of \( x \) if \( U \cup \{ x \} \) is an open set and, consequently, a neighborhood of \( x \).

  Dually, we can define the family \( \Cal{F} \) of closed sets, where
  \begin{description}
    \DItem{def:topological_space/F1}[F1] \( \varnothing, X \in \Cal{F} \)
    \DItem{def:topological_space/F2}[F2] \( U, V \in \Cal{F} \implies U \cup V \in \Cal{F} \)
    \DItem{def:topological_space/F3}[F3] \( \Cal{F}' \subseteq \Cal{F} \implies \bigcup \Cal{F}' \in \Cal{F} \)
  \end{description}

  If \( (X, \CT) \) is a topological space, we denote the corresponding family of closed sets by
  \begin{equation*}
    \Cal{F}_{\CT} \coloneqq \{ X \setminus U \colon U \in \CT \}.
  \end{equation*}
\end{definition}

\begin{definition}\label{def:standard_topologies}
  On a space \( X \), we can explicitly define the following standard topologies:
  \begin{defenum}
    \DItem{def:standard_topologies/discrete} The \Def{discrete topology} \( \CT \coloneqq \Power(X) \).
    \DItem{def:standard_topologies/indiscrete} The \Def{indiscrete topology} \( \CT \coloneqq \{ \varnothing, X \} \).
    \DItem{def:standard_topologies/co_cardinal} For any cardinal\Tinyref{def:cardinal} \( \xi \), the \Def{co-\( \xi \) topology} \( \CT \coloneqq \{ A \subseteq X \colon \Card A < \xi \} \) and, in particular, \Def{cofinite} (\( \xi = \aleph_0 \)) and \Def{cocountable} (\( \xi = c \)) topologies.
  \end{defenum}

  For a deeper connection between discrete and indiscrete topologies, see \cref{ex:top_adjoint_functor}.
\end{definition}

\begin{proposition}\label{thm:set_open_iff_neighborhood_is_contained}
  A set \( A \) is a topological space \( X \) is open if and only if every point \( x \in A \) has a neighborhood \( U \) such that \( U \subseteq A \).
\end{proposition}
\begin{proof}
  This holds vacuously for empty sets. Assume that \( A \subseteq X \) is nonempty.

  \begin{description}
    \Implies Assume that \( A \) is open and let \( x \in A \). Then \( A \) is a neighborhood of \( x \) and the theorem holds trivially.
    \ImpliedBy Assume that every point \( x \in A \) has a neighborhood \( U_x \) such that \( U_x \subseteq A \). Take the union
    \begin{equation*}
      B \coloneqq \cup_{x \in A} U_x.
    \end{equation*}

    Obviously \( B \subseteq A \). Aiming at a contradiction, suppose\LEM that \( y \in A \setminus B \). Then \( y \) has a neighborhood \( U_y \) such that \( U_y \setminus B \) is nonempty. But this is impossible by the definition of \( B \). The obtained contradiction proves \( B = A \).
  \end{description}
\end{proof}

\begin{remark}\label{remark:abritrary_family_to_topology}
  It is sometimes easier to define a topology \( \CT \) via a subset of \( \CT \). We will gradually construct a topology from a bare family of sets in \( X \). First, we will give two definitions for a base, one on which does not require an existing topology.
\end{remark}

\begin{definition}\label{def:topological_base}\cite[12]{Engelking1989}
  Fix a topological space \( (X, \CT) \). We say that the family \( \Cal{B} \subseteq \CT \) is a \Def{base for the topology \( \CT \)} if \( \Cal{B} \) satisfies any of the equivalent conditions
  \begin{defenum}
    \DItem{def:topological_base/union} Every open set \( U \in \CT \) is the union \( U = \bigcup \Cal{B}' \) of some subset \( \Cal{B}' = \Cal{B} \)
    \DItem{def:topological_base/subset} For any point \( x \in X \) and for any neighborhood \( U \) of \( x \) there exists a set \( V \in \Cal{B} \) in the base such that \( x \in V \subseteq U \)
  \end{defenum}
\end{definition}
\begin{proof}
  \Implies[def:topological_base/union][def:topological_base/subset] Fix a point \( x \in X \) and a neighborhood \( U \in \CT \) of \( x \). Let \( \Cal{B}' \) be a subfamily of \( \Cal{B} \) such that
  \begin{equation*}
    U = \bigcup \Cal{B}'.
  \end{equation*}

  Then \( x \in V \) for at least one \( V \in \Cal{B}' \).

  \Implies[def:topological_base/subset][def:topological_base/union] Fix an open set \( U \in \CT \). Then for every \( x \in U \), there exists a set \( V_x \in \Cal{B} \) such that \( x \in V_x \subseteq U \). We have
  \begin{equation*}
    \bigcup_{x \in U} V_x \subseteq U \subseteq \bigcup_{x \in U} V_x,
  \end{equation*}
  thus
  \begin{equation*}
    U = \bigcup_{x \in U} V_x.
  \end{equation*}
\end{proof}

\begin{proposition}\label{thm:topological_base_axioms}\cite[12]{Engelking1989}
  Let \( X \) be an arbitrary set and let \( \Cal{B} \) be a family of subset that satisfies
  \begin{description}
    \DItem{thm:topological_base_axioms/B1}[B1] \( \bigcup \Cal{B} = X \)
    \DItem{thm:topological_base_axioms/B2}[B2] \( \forall U, V \in \Cal{B}, \forall x \in U \cap V, \exists W \in \Cal{B}: x \in W \subseteq U \cap V \)
  \end{description}

  Then the family
  \begin{align}\label{thm:topological_base_axioms/topology}
    \CT \coloneqq \left\{ \bigcup \Cal{B}' \colon \Cal{B}' \subseteq \Cal{B} \right\}
  \end{align}
  is a topology on \( X \). Furthermore, \( \Cal{B} \) is a base\Tinyref{def:topological_base} of \( \CT \).

  In particular, the base on any topology satisfies \cref{thm:topological_base_axioms/B1} -- \cref{thm:topological_base_axioms/B2}.
\end{proposition}
\begin{proof}
  We will first prove that \( \CT \) is indeed a topology.

  \begin{description}
    \RItem{def:topological_space/O1} \( \varnothing = \bigcup \varnothing \in \tau \) and \( X = \bigcup \Cal{B} \in \CT \) (by~\ref{thm:topological_base_axioms/B1})

    \RItem{def:topological_space/O3} Fix \( \CT' = \{ U_\alpha \colon \alpha \in A \} \subseteq \CT \). By \cref{def:topological_base/union}, every set \( U_\alpha \) has a corresponding subfamily \( \Cal{B}_\alpha \) of \( \Cal{B} \) such that \( U_\alpha = \bigcup \Cal{B}_\alpha \).

    Define \( \Cal{B}' \coloneqq \bigcup_{\alpha \in A} \Cal{B}_\alpha \). Obviously \( \Cal{B}' \subseteq \Cal{B} \) and thus, by~\ref{thm:topological_base_axioms/B1}, \( \bigcup \Cal{B} \in \CT \).

    \RItem{def:topological_space/O2} Fix \( U, V \in \CT \) and families \( \Cal{B}_U, \Cal{B}_V \subseteq \Cal{B} \) such that \( U = \bigcup \Cal{B}_U \) and \( V = \bigcup \Cal{B}_V \).

    Fix arbitrary \( U' \in \Cal{B}_U \) and \( V' \in \Cal{B}_V \). We will show that \( U' \cap V' \in \tau \).

    By~\ref{thm:topological_base_axioms/B2}, for every \( x \in U' \cap V' \) there exists a neighborhood \( W_x \) of \( x \) such that \( W \subseteq U' \cap V' \).

    The family \( \Cal{B}_{U',V'} \coloneqq \{ W_x \colon x \in U' \cap V' \} \)~\AOC is a subfamily of \( \Cal{B} \) and thus \( U' \cap V' = \bigcup \Cal{B}_{U',V'} \in \CT \).

    Hence, by~\ref{def:topological_space/O3}, \( U \cap V \in \tau \).
  \end{description}

  Now, for any \( U \in \CT \), by \cref{thm:topological_base_axioms/topology}, there exists a subfamily \( \Cal{B}' \subseteq \Cal{B} \) such that
  \begin{equation*}
    U = \bigcup \Cal{B}'.
  \end{equation*}

  Hence \( \Cal{B} \) is a base for \( \CT \).
\end{proof}

\begin{definition}\label{def:topological_space_weight}
  We define the \Def{weight of \( (X, \CT) \)} as the cardinal
  \begin{equation*}
    w((X, \CT)) \coloneqq \min \{ \Abs{\Cal{B}} \colon \Cal{B} \text{ is a base for } \CT \}.
  \end{equation*}

  We simply write \( w(X) \) when the topology is clear from the context.

  Spaces for which \( w(X) \leq \aleph_0 \) are said to be \Def{second-countable}.
\end{definition}
\begin{proof}
  The definition is correct because of \cref{thm:cardinals_well_ordered}.
\end{proof}

\begin{definition}\label{def:topological_subbase}\cite[12]{Engelking1989}
  Fix a topological space \( (X, \CT) \). We say that the family \( \Cal{P} \subseteq \CT \) is a \Def{subbase} for the topology \( \CT \) if the family
  \begin{equation*}
    \Cal{B} \coloneqq \left\{ \bigcap P' \colon P' \text{ is a nonempty finite\Tinyref{def:finite_set} subset of } P \right\}
  \end{equation*}
  of finite intersections of \( \Cal{P} \) is a base\Tinyref{def:topological_base} of \( \CT \).
\end{definition}

\begin{proposition}\label{thm:subbase_from_arbitrary_family}
  Fix a set \( X \) and a family of subsets \( \Cal{P} \subseteq \Power(X) \). The family \( \Cal{P}' \coloneqq \Cal{P} \cup X \) is then a subbase\Tinyref{def:topological_subbase} of some topology on \( X \).
\end{proposition}

\begin{definition}\label{def:topological_local_base}\cite[12]{Engelking1989}
  Fix a topological space \( (X, \CT) \) and a point \( x \in X \). We say that the family \( \Cal{B}(x) \) is a \Def{local base for \( \CT \) at \( x \)} if for every neighborhood \( U \) of \( x \) there exists a set \( V \in \Cal{B}(x) \) such that \( x \in V \subseteq U \).

  The indexed family of local bases \( \{ \Cal{B}(x) \colon x \in X \} \) is called a \Def{neighborhood system} of \( \CT \).
\end{definition}

\begin{proposition}\label{thm:topological_local_base_axioms}\cite[13]{Engelking1989}
  Let \( X \) be an arbitrary set and let \( \{ \Cal{B}(x) \subseteq \Power(X) \colon x \in X \} \) be an indexed family of families of subsets of \( X \) that satisfies
  \begin{description}
    \DItem{thm:topological_local_base_axioms/BP1}[BP1] For every \( x \in X \), \( \Cal{B}(x) \neq \varnothing \) and \( x \in U \) for every \( U \in \Cal{B}(x) \).
    \DItem{thm:topological_local_base_axioms/BP2}[BP2] For every \( x \in X \) and for all \( U, V \in \Cal{B}(x) \), \( \exists W \in \Cal{B}(x): W \subseteq U \cap V \).
    \DItem{thm:topological_local_base_axioms/BP3}[BP3] For all \( x, y \in X \), \( x \in U \in \Cal{B}(y) \) implies that there exists \( V \in \Cal{B}(x) \) such that \( U \subseteq V \).
  \end{description}

  Then the family
  \begin{equation*}
    \Cal{B} \coloneqq \bigcup_{x \in X} \Cal{B}(x)
  \end{equation*}
  is the base\Tinyref{thm:topological_base_axioms} of some topology \( \CT \) on \( X \). Furthermore, \( \{ \Cal{B}(x) \subseteq \Power(X) \colon x \in X \} \) is a neighborhood system\Tinyref{def:topological_local_base} for \( (X, \CT) \).

  In particular, the local base on any topology satisfies \cref{thm:topological_local_base_axioms/BP1} -- \cref{thm:topological_local_base_axioms/BP3}.
\end{proposition}

\begin{definition}\label{def:topological_space_character}
  We define the \Def{character of the point \( x \in X \)} as the cardinal
  \begin{equation*}
    \chi(x) \coloneqq \min \{ \Abs{\Cal{B(x)}} \colon \Cal{B(x)} \text{ is a local base for } \CT \text{ at } x \}.
  \end{equation*}

  We define the \Def{character of of \( (X, \CT) \)} as
  \begin{equation*}
    \chi((X, \CT)) \coloneqq \sup \{ \chi(x) \colon x \in X \}.
  \end{equation*}

  We simply write \( \chi(X) \) when the topology is clear from the context.

  Spaces for which \( \chi(X) \leq \aleph_0 \) are said to be \Def{first-countable}.
\end{definition}
\begin{proof}
  The character of a point is well defined by \cref{thm:cardinals_well_ordered}. The character of a topological space is also well defined since by \cref{thm:equinumerous_ordinal_existence} there is at least one upper bound for the characters of all points and by \cref{thm:cardinals_well_ordered} this set has a least element.
\end{proof}

\begin{definition}\label{def:closure_operator}\cite[33]{Engelking1989}
  Let \( (X, \CT) \) be a topological space. Define the \Def{closure operator}
  \begin{align*}
    &\Cl: \Power(X) \to \Power(X) \\
    &\Cl(A) \coloneqq \bigcap \{ F : F \in \Cal{F}_{\CT}, A \subseteq F \}.
  \end{align*}
\end{definition}

\begin{proposition}\label{thm:closure_operator_properties}
  The closure operator\Tinyref{def:closure_operator} has the following basic properties
  \begin{propenum}
    \DItem{thm:closure_operator_properties/closed} The set \( A \) is closed if and only if \( A = \Cl A \).
    \DItem{thm:closure_operator_properties/neighborhood_intersection} For any \( x \in X \), \( x \in \Cl A \) if and only if every neighborhood of \( x \) intersects \( A \).
  \end{propenum}
\end{proposition}
\begin{proof}
  \begin{propenum}
    \RItem{thm:closure_operator_properties/closed} The condition \( A = \Cl{A} \) is equivalent to \( A \) being a closed superset of itself, which is equivalent to \( A \) being closed.

    \RItem{thm:closure_operator_properties/neighborhood_intersection} Note that this proof relies on \cref{def:topological_boundary}, however we do not use this property when defining the boundary.

    \begin{description}
      \Implies Fix \( x \in \Cl{A} \) and let \( U \) be a neighborhood of \( x \). If \( x \in A \), then obviously \( x \in U \cap A \neq \varnothing \). If \( x \not\in A \), then \( U \cap A \neq \varnothing \) by \cref{def:topological_boundary/neighborhoods}. In both cases, we obtain \( U \cap A \neq \varnothing \), which proves the statement.

      \ImpliedBy Fix \( x \in X \) and assume that every neighborhood of \( x \) intersects \( A \). Since the case \( x \in A \) is trivial, suppose that \( x \not\in A \). By \cref{thm:set_open_iff_neighborhood_is_contained}, every neighborhood \( U \) of \( x \) does not entirely belong to \( A \). By \cref{def:topological_boundary/neighborhoods}, \( x \in \Bd A \subseteq \Cl A \).
    \end{description}
  \end{propenum}
\end{proof}

\begin{proposition}\label{thm:closure_operator_axioms}\cite[14]{Engelking1989}
  Let \( X \) be an arbitrary set and let \( \Cl: \Power(X) \to \Power(X) \) be a function that satisfies
  \begin{description}
    \DItem{thm:closure_operator_axioms/CO1}[CO1] \( \Cl(\varnothing) = \varnothing \)
    \DItem{thm:closure_operator_axioms/CO2}[CO2] \( \forall A \in \Power(X), A \subseteq \Cl(A) \)
    \DItem{thm:closure_operator_axioms/CO3}[CO3] \( \forall A, B \in \Power(X), \Cl(A \cup B) = \Cl(A) \cup \Cl(B) \)
    \DItem{thm:closure_operator_axioms/CO4}[CO4] \( \forall A \in \Power(X), \Cl(\Cl(A)) = \Cl(A) \)
  \end{description}

  Then the family
  \begin{equation*}
    \CT \coloneqq \{ X \setminus F \colon F = \Cl(F) \}
  \end{equation*}
  is a topology on \( X \). Furthermore, \( \Cl = \Cl_{\CT} \), where \( \Cl_{\CT} \) is the closure operator\Tinyref{def:closure_operator} on \( (X, \CT) \).

  In particular, the closure operator on any topology satisfies \cref{thm:closure_operator_axioms/CO1} -- \cref{thm:closure_operator_axioms/CO4}.
\end{proposition}

\begin{definition}\label{def:interior_operator}\cite[15]{Engelking1989}
  Let \( (X, \CT) \) be a topological space. Define the \Def{interior operator}
  \begin{align*}
    &\Int: \Power(X) \to \Power(X) \\
    &\Int(A) \coloneqq \bigcup \{ U : U \in \CT, U \subseteq A \}.
  \end{align*}
\end{definition}

\begin{proposition}\label{thm:interior_operator_properties}
  The interior operator\Tinyref{def:interior_operator} has the following basic properties
  \begin{propenum}
    \DItem{thm:interior_operator_properties/open} A set \( A \) is a topological space is open if and only if \( A = \Int A \).
  \end{propenum}
\end{proposition}
\begin{proof}
  \begin{description}
    \RItem{thm:interior_operator_properties/open} Follows from \cref{thm:closure_operator_properties/closed} and \cref{thm:interior_closure_complement}.
  \end{description}
\end{proof}

\begin{proposition}\label{thm:interior_closure_complement} For every set \( A \subseteq X \) we have
  \begin{itemize}
    \item \( X \setminus \Int(A) = \Cl(X \setminus A) \)
    \item \( X \setminus \Cl(A) = \Int(X \setminus A) \)
  \end{itemize}
\end{proposition}
\begin{proof}
  Any open subset \( U \subseteq A \) is a closed superset of \( X \setminus A \). A point belongs to \( \Int(A) \) if it belongs to at least one open subset of \( A \), which happens if and only if it belongs to at least one closed superset of \( X \setminus A \). Therefore
  \begin{align*}
    X \setminus \Int(A)
    &=
    X \setminus \bigcup \{ U : U \in \CT, U \subseteq A \}
    = \\ &=
    X \setminus \bigcup \{ F : F \in \Cal{F}_{\CT}, X \setminus A \subseteq F \}
    \overset {X \setminus (X \setminus A) = A} = \\ &=
    \bigcup \{ F : F \in \Cal{F}_{\CT}, F \subseteq A \}.
    = \\ &=
    \Cl(A).
  \end{align*}

  The other equality is obtained by noting that \( X \setminus \Cl(A) = X \setminus (X \setminus \Int(A)) = \Int(A) \).
\end{proof}

\begin{proposition}\label{thm:interior_operator_axioms}
  Let \( X \) be an arbitrary set and let \( \Int: \Power(X) \to \Power(X) \) be a function that satisfies
  \begin{description}
    \DItem{thm:interior_operator_axioms/IO1}[IO1] \( \Int(X) = X \)
    \DItem{thm:interior_operator_axioms/IO2}[IO2] \( \forall A \in \Power(X), \Int(A) \subseteq A \)
    \DItem{thm:interior_operator_axioms/IO3}[IO3] \( \forall A, B \in \Power(X), \Int(A \cap B) = \Int(A) \cap \Int(B) \)
    \DItem{thm:interior_operator_axioms/IO4}[IO4] \( \forall A \in \Power(X), \Int(\Int(A)) = \Int(A) \)
  \end{description}

  Then the family
  \begin{equation*}
    \CT \coloneqq \{ U \colon U = \Int(U) \}
  \end{equation*}
  is a topology on \( X \). Furthermore, \( \Int = \Int_{\CT} \), where \( \Int_{\CT} \) is the interior operator\Tinyref{def:interior_operator} on \( (X, \CT) \).

  In particular, the interior operator on any topology satisfies \cref{thm:interior_operator_axioms/IO1} -- \cref{thm:interior_operator_axioms/IO4}.
\end{proposition}

\begin{definition}\label{def:topological_boundary}
  For a subset \( A \) of a topological space we define its \Def{boundary} \( \Bd(A) \) equivalently as
  \begin{defenum}
    \DItem{def:topological_boundary/closure} \( \Bd(A) \coloneqq \Cl(A) \setminus \Int(A) \)
    \DItem{def:topological_boundary/neighborhoods} \( \Bd(A) \) is the set of all points \( x \in X \) such that every neighborhood of \( x \) intersects both \( A \) and \( X \setminus A \).
  \end{defenum}
\end{definition}
\begin{proof}
  The equivalence of the definitions is trivial when \( \Bd(A) = \varnothing \). We assume that \( \Bd(A) \neq \varnothing \).

  \begin{description}
    \Implies[def:topological_boundary/closure][def:topological_boundary/neighborhoods] Let \( x \in \Cl(A) \setminus \Int(A) \).

    Aiming for a contradiction, suppose\LEM that there is a neighborhood \( U \) of \( x \) that does not intersect \( A \). Then \( U \subseteq X \setminus A \). Hence \( A \subseteq X \setminus U \). Since \( X \setminus U \) is closed, it follows that \( \Cl(A) \subseteq X \setminus U \) as the intersection of all closed supersets of \( A \). But \( x \not\in X \setminus U \), therefore \( x \not\in \Cl(A) \), which contradicts our choice of \( x \in \Cl(A) \).

    This proves that every neighborhood of \( x \) intersects \( A \).

    By passing to complements, we can reuse this to prove that every neighborhood of \( x \) intersects \( X \setminus A \) using \cref{thm:interior_closure_complement}.

    \Implies[def:topological_boundary/neighborhoods][def:topological_boundary/closure] Suppose that every neighborhood of \( x \in \Bd(A) \) intersects both \( A \) and \( X \setminus A \). Therefore no neighborhood of \( x \) is contained in neither \( A \) not \( X \setminus A \) and \( x \) belongs to neither \( \Int(A) \) nor \( \Int(X \setminus A) \). Hence
    \begin{equation*}
      x \in (X \setminus \Int(X \setminus A)) \setminus \Int(A) \overset {\ref{thm:interior_closure_complement}} = \Cl(A) \setminus \Int(A).
    \end{equation*}
  \end{description}
\end{proof}

\begin{proposition}\label{thm:topological_boundary_properties}
  The topological boundary\Tinyref{def:topological_boundary} has the following basic properties
  \begin{propenum}
    \DItem{thm:topological_boundary_properties/closed} \( \Bd(A) \) is a closed set.
    \DItem{thm:topological_boundary_properties/not_open} If \( \Bd(A) \) is not empty, it is not an open set.
    \DItem{thm:topological_boundary_properties/complement} \( \Bd(A) = \Bd(X \setminus A) \).
  \end{propenum}
\end{proposition}
\begin{proof}
  \begin{description}
    \RItem{thm:topological_boundary_properties/closed} Note that
    \begin{equation*}
      \Bd(A) = \Cl(A) \setminus \Int(A) = \Cl(A) \cap (X \setminus \Int(A)),
    \end{equation*}
    which is the intersection of two closed sets. Hence \( \Bd(A) \) is itself a closed set.

    \RItem{thm:topological_boundary_properties/not_open} Note that \( \Bd(A) \) is either empty or is not open because \cref{def:topological_boundary/neighborhoods} is incompatible with \cref{thm:set_open_iff_neighborhood_is_contained}.

    \RItem{thm:topological_boundary_properties/complement} By \cref{thm:interior_closure_complement},
    \begin{align*}
      \Bd(A)
      &=
      \Cl(A) \setminus \Int(A)
      = \\ &=
      \Cl(A) \cap (X \setminus \Int(A))
      \overset {\ref{thm:interior_closure_complement}} = \\ &=
      (X \setminus \Int(X \setminus A)) \cap \Cl(X \setminus A)
      = \\ &=
      \Cl(X \setminus A) \setminus \Int(X \setminus A)
      = \\ &=
      \Bd(X \setminus A).
    \end{align*}
  \end{description}
\end{proof}

\begin{definition}\label{def:topological_derived_set}\cite[24]{Engelking1989}
  Let \( (X, \CT) \) be a topological space.

  \begin{defenum}
    \DItem{def:topological_derived_set/cluster_point} We say that the point \( x \in X \) is a \Def{cluster point} or an \Def{accumulation point} of the set \( A \subseteq X \) if \( x \in \Cl(A \setminus \{ x \}) \). It is not necessary for \( x \) to belong to \( A \).

    \DItem{def:topological_derived_set/derived_set} The set of all cluster points of \( A \) is called the \Def{derived set} of \( A \) and is denoted by \( \Der(A) \).

    \DItem{def:topological_derived_set/perfect_set} If a set equals its derived set, we call it a \Def{perfect set}.

    \DItem{def:topological_derived_set/isolated_point} Points in \( A \setminus \Der(A) \) are said to be \Def{isolated points} of \( A \).

    \DItem{def:topological_derived_set/discrete_set} If \( \Der(A) = \varnothing \), that is, if \( A \) consists of only discrete points, we say that \( A \) is a \Def{discrete set}.
  \end{defenum}
\end{definition}

\begin{proposition}\label{thm:derived_set_properties}
  Derived sets\Tinyref{def:topological_derived_set} have the following basic properties
  \begin{propenum}
    \DItem{thm:derived_set_properties/cluster_via_neighborhoods} \( x \) is a cluster point of \( A \) if and only if every neighborhood of \( x \) intersects \( A \setminus \{ x \} \)
    \DItem{thm:derived_set_properties/isolated_via_neighborhoods} \( x \) is an isolated point of \( A \) if and only if there exists a neighborhood of \( x \) that does not intersect \( A \setminus \{ x \} \)
    \DItem{thm:derived_set_properties/closed} \( \Der(A) \) is a closed set.
    \DItem{thm:derived_set_properties/closure} \( A \cup \Der(A) = \Cl(A) \).
    \DItem{thm:derived_set_properties/closed_iff_contains_all_cluster_points} A set is closed if and only if it contains all of its cluster points. Compare this result to \cref{thm:cluster_point_iff_in_closure}.
    \DItem{thm:derived_set_properties/closed_iff_only_isolated_and_cluster_points} A set if closed if and only if every point is either a cluster point or an isolated point.
  \end{propenum}
\end{proposition}
\begin{proof}
  \begin{description}
    \RItem{thm:derived_set_properties/cluster_via_neighborhoods}\mbox{}
    \begin{description}
      \Implies If \( x \) is a cluster point of \( A \), then \( x \in \Cl(A \setminus \{ x \}) \). By \cref{thm:derived_set_properties/isolated_via_neighborhoods}, for every neighborhood \( U \) of \( x \) we have \( U \cap (A \setminus \{ x \}) \neq \varnothing \).

      \ImpliedBy If every neighborhood \( U \) of \( x \in A \) intersects \( A \setminus \{ x \} \), by \cref{thm:closure_operator_properties/neighborhood_intersection}, \( x \in \Cl(A \setminus \{ x \}) \) and \( x \) is therefore a cluster point.
    \end{description}

    \RItem{thm:derived_set_properties/isolated_via_neighborhoods} Dual to \cref{thm:derived_set_properties/cluster_via_neighborhoods}.
    \RItem{thm:derived_set_properties/closed} Consider the complement of \( \Der(A) \). If it is empty, \( \Der(A) \) is trivially closed. Otherwise, let \( x \in X \setminus \Der(A) \).

    \begin{itemize}
      \item If \( x \) is an isolated point of \( A \), by \cref{thm:derived_set_properties/isolated_via_neighborhoods} there exists a neighborhood of \( x \) that does not intersect \( A \setminus \{ x \} \).
      \item If \( x \) is not a point of \( A \), aiming at a contradiction, assume\LEM that every neighborhood of \( x \) intersects \( A \). Then, by \cref{def:topological_boundary/neighborhoods}, \( x \in \Bd(A) \). But \( \Bd(A) \subseteq \Cl(A) \) and \( \Cl(A) = \Cl(A \setminus \{ x \}) \) because \( x \) does not belong to \( A \). Therefore, \( x \) is a cluster point of \( A \). This contradicts our assumption that \( x \not\in \Der(A) \), hence we can conclude that there exists a neighborhood of \( X \) that does not intersect \( A = A \setminus \{ x \} \).
    \end{itemize}

    In both cases, \cref{thm:set_open_iff_neighborhood_is_contained} allows us to conclude that \( X \setminus \Der(A) \) is open and, hence, \( \Der(A) \) is closed.

    \RItem{thm:derived_set_properties/closure} Clearly \( A \subseteq \Cl(A) \). Also
    \begin{equation*}
      \Der(A) \subseteq \bigcup_{x \in X} \Cl(A \setminus \{ x \}) \subseteq \Cl(A).
    \end{equation*}

    Now we will prove the reverse inclusion. Let \( x \in \Cl(A) \). Then either \( x \in A \) or \( x \in \Bd(A) \). Assume the latter. By \cref{def:topological_boundary/neighborhoods}, every neighborhood \( U \) of \( x \) has points both in \( A \) and outside of \( A \), therefore \( U \cap (A \setminus \{ x \}) \) is nonempty. By \cref{thm:closure_operator_properties/neighborhood_intersection}, \( x \in \Cl(A \setminus \{ x \}) \), that is, \( x \in \Der(A) \).

    \RItem{thm:derived_set_properties/closed_iff_contains_all_cluster_points}\mbox{}
    \begin{description}
      \Implies If \( A \) is closed, by \cref{thm:derived_set_properties/closure},
      \begin{equation*}
        A \cup \Der(A) = \Cl(A) = A,
      \end{equation*}
      hence \( \Der(A) \subseteq A \).

      \ImpliedBy Assume that \( \Der(A) \subseteq A \) and, aiming at a contradiction, suppose that \( A \) is not closed. Fix a point \( x \in \Cl(A) \setminus A \). By \cref{thm:derived_set_properties/closure}, this is a cluster point. By \cref{thm:derived_set_properties/cluster_via_neighborhoods}, every for neighborhood \( U \) of \( x \) the intersection \( U \cap (A \setminus \{ x \}) \subseteq U \cap A \) is nonempty. Since this holds for arbitrary neighborhoods, by \cref{thm:closure_operator_properties/neighborhood_intersection}, \( A \) is closed.
    \end{description}

    \RItem{thm:derived_set_properties/closed_iff_only_isolated_and_cluster_points}\mbox{}
    \begin{description}
      \Implies Special case of \cref{thm:derived_set_properties/closed_iff_contains_all_cluster_points}.
      \ImpliedBy We already know from \cref{thm:derived_set_properties/closed_iff_contains_all_cluster_points} that it is sufficient for \( \Der(A) \) to belong to \( A \) for \( A \) to be closed. But \( A \setminus \Der(A) \) consists of all isolated points, therefore every point in \( A \) is either a cluster point or an isolated point.
    \end{description}
  \end{description}
\end{proof}

\begin{definition}\label{def:topologically_dense_set}\cite[25]{Engelking1989}
  Let \( (X, \CT) \) be a topological space and \( A \subseteq X \) be any set. We say that \( A \) is

  \begin{defenum}
    \DItem{def:topologically_dense_set/dense} \Def{dense in \( X \)} if \( \Cl{A} = X \) (if \( X \) is assumed from the context, we simply say that \( A \) is dense).

    \DItem{def:topologically_dense_set/codense} \Def{codense in \( X \)} if \( X \setminus A \) is dense, i.e. \( \Cl{X \setminus A} = X \).

    \DItem{def:topologically_dense_set/nowhere_dense} \Def{nowhere dense in \( X \)} if \( \Cl{A} \) is codense, i.e. \( \Cl{X \setminus \Cl{A}} = X \).

    \DItem{def:topologically_dense_set/dense_in_itself} \Def{dense in itself} if \( A \subseteq \Der(A) \), i.e. if \( A \) has no isolated points.
  \end{defenum}

  We define the \Def{density} \( d(X) \) of \( X \) to be the minimum cardinality\Tinyref{def:cardinal} of all dense sets. If \( d(X) \leq \aleph_0 \), we say that the space is \Def{separable}.
\end{definition}
