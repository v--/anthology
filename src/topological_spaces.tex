\begin{definition}\label{def:topological_space}\cite[21]{Lectures:general_topology}
  Let $X$ be any set and $\Cal{T} \subseteq \Power(X)$ be a family of subsets of $X$. $\Cal{T}$ is called a \uline{topology on $X$} and the tuple $(X, \Cal{T})$ is said to be a \uline{topological space} if the following axioms are satisfied:
  \begin{enumerate}[label={\textbf{O\arabic*)}}]
    \item\label{def:topological_space/O1} $\varnothing, X \in \Cal{T}$
    \item\label{def:topological_space/O2} $U, V \in \Cal{T} \implies U \cap V \in \Cal{T}$
    \item\label{def:topological_space/O3} $\Cal{T}' \subseteq \Cal{T} \implies \bigcap \Cal{T}' \in \Cal{T}$
  \end{enumerate}

  If the topology is obvious from the context, we say that $X$ is a topological space.

  Elements of the set $X$ are called \uline{points of the topological space}, elements of $\Cal{T}$ are called \uline{open sets} and set-theoretic complements of open sets are called \uline{closed sets}.

  If $x \in U \in \Cal{T}$, we say that $U$ is a \uline{neighborhood of $x$}. Note that some authors (e.g.~\cite[38]{Kelley1955}) alternatively define neighborhoods as arbitrary sets that contain an open set that contains $x$.
\end{definition}

It is sometimes easier to define a topology $\Cal{T}$ via a subset of $\Cal{T}$. We will gradually construct a topology from a bare family of sets in $X$. First, we will give two definitions for a base, one which requires an existing topology and one which does not.

\begin{definition}\label{def:topological_space_base}\cite[23]{Lectures:general_topology}
  Fix a topological space $(X, \Cal{T})$. We say that the family $\Cal{B} \subseteq \Cal{T}$ is a \uline{base for the topology $\Cal{T}$} if $\Cal{B}$ satisfies any of the equivalent conditions
  \begin{defenum}
    \item\label{def:topological_space_base/union} Every open set $U \in \Cal{T}$ is the union $U = \bigcup \Cal{B}'$ of some subset $\Cal{B}' = \Cal{B}$
    \item\label{def:topological_space_base/subset} For any point $x \in X$ and for any neighborhood $U$ of $x$ there exists a set $V \in \Cal{B}$ in the base such that $x \in V \subseteq U$
  \end{defenum}
\end{definition}

\begin{definition}\label{def:topological_base}\cite[23]{Lectures:general_topology}
  Fix a set $X$. We say that the family $\Cal{B} \subseteq \Power{X}$ is a \uline{topological base} if $\Cal{B}$ satisfies
  \begin{enumerate}[label={\textbf{B\arabic*)}}]
    \item\label{def:topological_base/B1} $\bigcup \Cal{B} = X$
    \item\label{def:topological_base/B2} $\forall U, V \in \Cal{B}, \forall x \in U \cap V, \exists W \in \Cal{B}: x \in W \subseteq U \cap V$
  \end{enumerate}

  We can now define the \uline{topology induced by the base $\Cal{B}$}
  \begin{equation}\label{def:topological_base/topology}
    \Cal{T} \coloneqq \left\{ \bigcup \Cal{B}' \colon \Cal{B}' \subseteq \Cal{B} \right\}.
  \end{equation}
\end{definition}
\begin{proof}
  We will prove that $\Cal{T}$ is indeed a topology
  \begin{enumerate}
    \item[\ref{def:topological_space/O1}] $\varnothing = \bigcup \varnothing \in \tau$ and $X = \bigcup \Cal{B} \in \Cal{T}$ (by~\ref{def:topological_base/B1})

    \item[\ref{def:topological_space/O3}] Fix $\Cal{T}' = \{ U_\alpha \colon \alpha \in A \} \subseteq \Cal{T}$. By~\cref{def:topological_base/topology}, every set $U_\alpha$ has a corresponding subfamily $\Cal{B}_\alpha$ of $\Cal{B}$ such that $U_\alpha = \bigcup \Cal{B}_\alpha$.

    Define $\Cal{B}' \coloneqq \bigcup_{\alpha \in A} \Cal{B}_\alpha$. Obviously $\Cal{B}' \subseteq \Cal{B}$ and thus, by~\ref{def:topological_base/B1}, $\bigcup \Cal{B} \in \Cal{T}$.

    \item[\ref{def:topological_space/O2}] Fix $U, V \in \Cal{T}$ and families $\Cal{B}_U, \Cal{B}_V \subseteq \Cal{B}$ such that $U = \bigcup \Cal{B}_U$ and $V = \bigcup \Cal{B}_V$.

    Fix arbitrary $U' \in \Cal{B}_U$ and $V' \in \Cal{B}_V$. We will show that $U' \cap V' \in \tau$.

    By~\ref{def:topological_base/B2}, for every $x \in U' \cap V'$ there exists a neighborhood $W_x$ of $x$ such that $W \subseteq U' \cap V'$.

    The family $\Cal{B}_{U',V'} \coloneqq \{ W_x \colon x \in U' \cap V' \}$~\AOC is a subfamily of $\Cal{B}$ and thus $U' \cap V' = \bigcup \Cal{B}_{U',V'} \in \Cal{T}$.

    Hence, by~\ref{def:topological_space/O3}, $U \cap V \in \tau$.
  \end{enumerate}
\end{proof}

\begin{proposition}\label{thm:topological_base_is_topological_space_base}
  Let $\Cal{B}$ be a topological base (see \cref{def:topological_base}) for an arbitrary set $X$. The family $\Cal{B}$ is then a base (see \cref{def:topological_space_base}) for the topological space $(X, \Cal{T})$.
\end{proposition}
\begin{proof}
  \Cref{def:topological_space_base/subset} is obvious by \ref{def:topological_base/B2}.
\end{proof}

\begin{proposition}\label{thm:topological_space_base_is_topological_base}
  Let $(X, \Cal{T})$ be a topological space and let $\Cal{B}$ be a base (see \cref{def:topological_space_base}). Then $\Cal{B}$ satisfies \ref{def:topological_base/B1} and \ref{def:topological_base/B2}.
\end{proposition}
\begin{proof}
  \item[\ref{def:topological_base/B1}] Implied by \cref{def:topological_space_base/union} with $U = X$.
  \item[\ref{def:topological_base/B2}] Implied by \cref{def:topological_space_base/subset} with $U = U \cap V$.
\end{proof}
