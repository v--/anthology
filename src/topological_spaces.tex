\subsection{Topological spaces}\label{subsec:topological_spaces}

\begin{definition}\label{def:topological_space}\cite[11]{Engelking1989}
  Let \( X \) be any set and \( \Cal{T} \subseteq \Power(X) \) be a family of subsets of \( X \). \( \Cal{T} \) is called a \textbf{topology on \( X \)} and the tuple \( (X, \Cal{T}) \) is said to be a \textbf{topological space} if the following axioms are satisfied:
  \begin{description}
    \DItem{def:topological_space/O1}[O1] \( \varnothing, X \in \Cal{T} \)
    \DItem{def:topological_space/O2}[O2] \( U, V \in \Cal{T} \implies U \cap V \in \Cal{T} \)
    \DItem{def:topological_space/O3}[O3] \( \Cal{T}' \subseteq \Cal{T} \implies \bigcap \Cal{T}' \in \Cal{T} \)
  \end{description}

  If the topology is obvious from the context, we say that \( X \) is a topological space.

  Elements of the set \( X \) are called \textbf{points of the topological space}, elements of \( \Cal{T} \) are called \textbf{open sets} and set-theoretic complements of open sets are called \textbf{closed sets}.

  If \( x \in U \in \Cal{T} \), we say that \( U \) is a \textbf{neighborhood of \( x \)}. Note that some authors (e.g.~\cite[38]{Kelley1955}) alternatively define neighborhoods as arbitrary sets that contain an open set that contains \( x \).

  Dually, we can define the family \( \Cal{F} \) of closed sets, where
  \begin{description}
    \DItem{def:topological_space/F1}[F1] \( \varnothing, X \in \Cal{F} \)
    \DItem{def:topological_space/F2}[F2] \( U, V \in \Cal{F} \implies U \cup V \in \Cal{F} \)
    \DItem{def:topological_space/F3}[F3] \( \Cal{F}' \subseteq \Cal{F} \implies \bigcup \Cal{F}' \in \Cal{F} \)
  \end{description}

  If \( (X, \Cal{T}) \) is a topological space, we denote the corresponding family of closed sets by
  \begin{align*}
    \Cal{F}_{\Cal{T}} \coloneqq \{ X \setminus U \colon U \in \Cal{T} \}.
  \end{align*}
\end{definition}

\begin{definition}\label{def:standard_topologies}
  On a space \( X \), we can explicitly define the following standard topologies:
  \begin{defenum}
    \DItem{def:standard_topologies/discrete} The \textbf{discrete topology} \( \Cal{T} \coloneqq \Power(X) \).
    \DItem{def:standard_topologies/indiscrete} The \textbf{indiscrete topology} \( \Cal{T} \coloneqq \{ \varnothing, X \} \).
    \DItem{def:standard_topologies/co_cardinal} For any cardinal\Tinyref{def:cardinal} \( \xi \), the \textbf{co-\( \xi \) topology} \( \Cal{T} \coloneqq \{ A \subseteq X \colon \Card A < \xi \} \) and, in particular, \textbf{cofinite} (\( \xi = \aleph_0 \)) and \textbf{cocountable} (\( \xi = c \)) topologies.
  \end{defenum}

  For a deeper connection between discrete and indiscrete topologies, see \cref{ex:top_adjoint_functor}.
\end{definition}

\begin{note}\label{note:abritrary_family_to_topology}
  It is sometimes easier to define a topology \( \Cal{T} \) via a subset of \( \Cal{T} \). We will gradually construct a topology from a bare family of sets in \( X \). First, we will give two definitions for a base, one on which does not require an existing topology.
\end{note}

\begin{definition}\label{def:topological_base}\cite[12]{Engelking1989}
  Fix a topological space \( (X, \Cal{T}) \). We say that the family \( \Cal{B} \subseteq \Cal{T} \) is a \textbf{base for the topology \( \Cal{T} \)} if \( \Cal{B} \) satisfies any of the equivalent conditions
  \begin{defenum}
    \DItem{def:topological_base/union} Every open set \( U \in \Cal{T} \) is the union \( U = \bigcup \Cal{B}' \) of some subset \( \Cal{B}' = \Cal{B} \)
    \DItem{def:topological_base/subset} For any point \( x \in X \) and for any neighborhood \( U \) of \( x \) there exists a set \( V \in \Cal{B} \) in the base such that \( x \in V \subseteq U \)
  \end{defenum}
\end{definition}
\begin{proof}
  \Implies[def:topological_base/union][def:topological_base/subset] Fix a point \( x \in X \) and a neighborhood \( U \in \Cal{T} \) of \( x \). Let \( \Cal{B}' \) be a subfamily of \( \Cal{B} \) such that
  \begin{align*}
    U = \bigcup \Cal{B}'.
  \end{align*}

  Then \( x \in V \) for at least one \( V \in \Cal{B}' \).

  \Implies[def:topological_base/subset][def:topological_base/union] Fix an open set \( U \in \Cal{T} \). Then for every \( x \in U \), there exists a set \( V_x \in \Cal{B} \) such that \( x \in V_x \subseteq U \). We have
  \begin{align*}
    \bigcup_{x \in U} V_x \subseteq U \subseteq \bigcup_{x \in U} V_x,
  \end{align*}
  thus
  \begin{align*}
    U = \bigcup_{x \in U} V_x.
  \end{align*}
\end{proof}

\begin{proposition}\label{thm:topological_base_axioms}\cite[12]{Engelking1989}
  Let \( X \) be an arbitrary set and let \( \Cal{B} \) be a family of subset that satisfies
  \begin{description}
    \DItem{thm:topological_base_axioms/B1}[B1] \( \bigcup \Cal{B} = X \)
    \DItem{thm:topological_base_axioms/B2}[B2] \( \forall U, V \in \Cal{B}, \forall x \in U \cap V, \exists W \in \Cal{B}: x \in W \subseteq U \cap V \)
  \end{description}

  Then the family
  \begin{align}\label{thm:topological_base_axioms/topology}
    \Cal{T} \coloneqq \left\{ \bigcup \Cal{B}' \colon \Cal{B}' \subseteq \Cal{B} \right\}
  \end{align}
  is a topology on \( X \). Furthermore, \( \Cal{B} \) is a base\Tinyref{def:topological_base} of \( \Cal{T} \).
\end{proposition}
\begin{proof}
  We will first prove that \( \Cal{T} \) is indeed a topology.

  \begin{description}
    \RItem{def:topological_space/O1} \( \varnothing = \bigcup \varnothing \in \tau \) and \( X = \bigcup \Cal{B} \in \Cal{T} \) (by~\ref{thm:topological_base_axioms/B1})

    \RItem{def:topological_space/O3} Fix \( \Cal{T}' = \{ U_\alpha \colon \alpha \in A \} \subseteq \Cal{T} \). By~\cref{def:topological_base/union}, every set \( U_\alpha \) has a corresponding subfamily \( \Cal{B}_\alpha \) of \( \Cal{B} \) such that \( U_\alpha = \bigcup \Cal{B}_\alpha \).

    Define \( \Cal{B}' \coloneqq \bigcup_{\alpha \in A} \Cal{B}_\alpha \). Obviously \( \Cal{B}' \subseteq \Cal{B} \) and thus, by~\ref{thm:topological_base_axioms/B1}, \( \bigcup \Cal{B} \in \Cal{T} \).

    \RItem{def:topological_space/O2} Fix \( U, V \in \Cal{T} \) and families \( \Cal{B}_U, \Cal{B}_V \subseteq \Cal{B} \) such that \( U = \bigcup \Cal{B}_U \) and \( V = \bigcup \Cal{B}_V \).

    Fix arbitrary \( U' \in \Cal{B}_U \) and \( V' \in \Cal{B}_V \). We will show that \( U' \cap V' \in \tau \).

    By~\ref{thm:topological_base_axioms/B2}, for every \( x \in U' \cap V' \) there exists a neighborhood \( W_x \) of \( x \) such that \( W \subseteq U' \cap V' \).

    The family \( \Cal{B}_{U',V'} \coloneqq \{ W_x \colon x \in U' \cap V' \} \)~\AOC is a subfamily of \( \Cal{B} \) and thus \( U' \cap V' = \bigcup \Cal{B}_{U',V'} \in \Cal{T} \).

    Hence, by~\ref{def:topological_space/O3}, \( U \cap V \in \tau \).
  \end{description}

  Now, for any \( U \in \Cal{T} \), by~\cref{thm:topological_base_axioms/topology}, there exists a subfamily \( \Cal{B}' \subseteq \Cal{B} \) such that
  \begin{align*}
    U = \bigcup \Cal{B}'.
  \end{align*}

  Hence \( \Cal{B} \) is a base for \( \Cal{T} \).
\end{proof}

\begin{definition}\label{def:topological_space_weight}
  We define the \textbf{weight of \( (X, \Cal{T}) \)} as the cardinal
  \begin{align*}
    w((X, \Cal{T})) \coloneqq \min \{ \Abs{\Cal{B}} \colon \Cal{B} \text{ is a base for } \Cal{T} \}.
  \end{align*}

  We simply write \( w(X) \) when the topology is clear from the context.

  Spaces for which \( w(X) \leq \aleph_0 \) are said to be \textbf{second-countable}.
\end{definition}
\begin{proof}
  The definition is correct because of \cref{thm:cardinals_well_ordered}.
\end{proof}

\begin{definition}\label{def:topological_subbase}\cite[12]{Engelking1989}
  Fix a topological space \( (X, \Cal{T}) \). We say that the family \( \Cal{P} \subseteq \Cal{T} \) is a \textbf{subbase for the topology \( \Cal{T} \)} if \( \FI(\Cal{P}) \)\Tinyref{def:finite_intersection_operator}) is a base\Tinyref{def:topological_base} of \( \Cal{T} \).
\end{definition}

\begin{proposition}
  Fix a set \( X \) and a family of subsets \( \Cal{P} \subseteq \Power(X) \). The family \( \Cal{P}' \coloneqq \Cal{P} \cup X \) is then a subbase\Tinyref{def:topological_subbase} of some topology on \( X \).
\end{proposition}

\begin{definition}\label{def:topological_local_base}\cite[12]{Engelking1989}
  Fix a topological space \( (X, \Cal{T}) \) and a point \( x \in X \). We say that the family \( \Cal{B}(x) \) is a \textbf{local base for \( \Cal{T} \) at \( x \)} if for every neighborhood \( U \) of \( x \) there exists a set \( V \in \Cal{B}(x) \) such that \( x \in V \subseteq U \).

  The indexed family of local bases \( \{ \Cal{B}(x) \colon x \in X \} \) is called a \textbf{neighborhood system} of \( \Cal{T} \).
\end{definition}

\begin{proposition}\label{thm:topological_local_base_axioms}\cite[13]{Engelking1989}
  Let \( X \) be an arbitrary set and let \( \{ \Cal{B}(x) \subseteq \Power(X) \colon x \in X \} \) be an indexed family of families of subsets of \( X \) that satisfies
  \begin{description}
    \DItem{thm:topological_local_base_axioms/BP1}[BP1] For every \( x \in X \), \( \Cal{B}(x) \neq \varnothing \) and \( x \in U \) for every \( U \in \Cal{B}(x) \).
    \DItem{thm:topological_local_base_axioms/BP2}[BP2] For every \( x \in X \) and for all \( U, V \in \Cal{B}(x) \), \( \exists W \in \Cal{B}: W \subseteq U \cap V \).
    \DItem{thm:topological_local_base_axioms/BP3}[BP3] For all \( x, y \in X \), \( x \in U \in \Cal{B}(y) \) implies that there exists \( V \in \Cal{B}(x) \) such that \( U \subseteq V \).
  \end{description}

  Then the family
  \begin{align*}
    \Cal{B} \coloneqq \bigcup_{x \in X} \Cal{B}(x)
  \end{align*}
  is the base\Tinyref{thm:topological_base_axioms} of some topology \( \Cal{T} \) on \( X \). Furthermore, \( \{ \Cal{B}(x) \subseteq \Power(X) \colon x \in X \} \) is a neighborhood system\Tinyref{def:topological_local_base} for \( (X, \Cal{T}) \).
\end{proposition}

\begin{definition}\label{def:topological_space_character}
  We define the \textbf{character of the point \( x \in X \)} as the cardinal
  \begin{align*}
    \chi(x) \coloneqq \min \{ \Abs{\Cal{B(x)}} \colon \Cal{B(x)} \text{ is a local base for } \Cal{T} \text{ at } x \}.
  \end{align*}

  We define the \textbf{character of of \( (X, \Cal{T}) \)} as
  \begin{align*}
    \chi((X, \Cal{T})) \coloneqq \sup \{ \chi(x) \colon x \in X \}.
  \end{align*}

  We simply write \( \chi(X) \) when the topology is clear from the context.

  Spaces for which \( \chi(X) \leq \aleph_0 \) are said to be \textbf{first-countable}.
\end{definition}
\begin{proof}
  The character of a point is well defined by \cref{thm:cardinals_well_ordered}. The character of a topological space is also well defined since by \cref{thm:equinumerous_ordinal_existence} there is at least one upper bound for the characters of all points and by \cref{thm:cardinals_well_ordered} this set has a least element.
\end{proof}

\begin{definition}\label{def:closure_operator}\cite[33]{Engelking1989}
  Let \( (X, \Cal{T}) \) be a topological space. Define the \textbf{closure operator}
  \begin{align*}
    &\Cl: \Power(X) \to \Power(X) \\
    &\Cl(A) \coloneqq \bigcap \{ F : F \in \Cal{F}_{\Cal{T}}, A \subseteq F \}.
  \end{align*}
\end{definition}

\begin{proposition}\label{thm:set_closed_iff_matches_closure}
  A set \( A \) is a topological space is closed if and only if \( A = \Cl A \).
\end{proposition}

\begin{proposition}\label{thm:closure_operator_axioms}\cite[14]{Engelking1989}
  Let \( X \) be an arbitrary set and let \( \Cl: \Power(X) \to \Power(X) \) be a function that satisfies
  \begin{description}
    \DItem{thm:closure_operator_axioms/CO1}[CO1] \( \Cl(\varnothing) = \varnothing \)
    \DItem{thm:closure_operator_axioms/CO2}[CO2] \( \forall A \in \Power(X), A \subseteq \Cl(A) \)
    \DItem{thm:closure_operator_axioms/CO3}[CO3] \( \forall A, B \in \Power(X), \Cl(A \cup B) = \Cl(A) \cup \Cl(B) \)
    \DItem{thm:closure_operator_axioms/CO4}[CO4] \( \forall A \in \Power(X), \Cl(\Cl(A)) = \Cl(A) \)
  \end{description}

  Then the family
  \begin{align*}
    \Cal{T} \coloneqq \{ X \setminus F \colon F = \Cl(F) \}
  \end{align*}
  is a topology on \( X \). Furthermore, \( \Cl = \Cl_{\Cal{T}} \), where \( \Cl_{\Cal{T}} \) is the closure operator\Tinyref{def:closure_operator} on \( (X, \Cal{T}) \).
\end{proposition}

\begin{definition}\label{def:interior_operator}\cite[15]{Engelking1989}
  Let \( (X, \Cal{T}) \) be a topological space. Define the \textbf{interior operator}
  \begin{align*}
    &\Int: \Power(X) \to \Power(X) \\
    &\Int(A) \coloneqq \bigcup \{ U : U \in \Cal{T}, U \subseteq A \}.
  \end{align*}
\end{definition}

\begin{proposition}\label{thm:set_open_iff_matches_interior}
  A set \( A \) is a topological space is open if and only if \( A = \Int A \).
\end{proposition}

\begin{proposition}\label{thm:interior_operator_axioms}
  Let \( X \) be an arbitrary set and let \( \Int: \Power(X) \to \Power(X) \) be a function that satisfies
  \begin{description}
    \DItem{thm:interior_operator_axioms/IO1}[IO1] \( \Int(X) = X \)
    \DItem{thm:interior_operator_axioms/IO2}[IO2] \( \forall A \in \Power(X), \Int(A) \subseteq A \)
    \DItem{thm:interior_operator_axioms/IO3}[IO3] \( \forall A, B \in \Power(X), \Int(A \cap B) = \Int(A) \cap \Int(B) \)
    \DItem{thm:interior_operator_axioms/IO4}[IO4] \( \forall A \in \Power(X), \Int(\Int(A)) = \Int(A) \)
  \end{description}

  Then the family
  \begin{align*}
    \Cal{T} \coloneqq \{ U \colon U = \Int(U) \}
  \end{align*}
  is a topology on \( X \). Furthermore, \( \Int = \Int_{\Cal{T}} \), where \( \Int_{\Cal{T}} \) is the interior operator\Tinyref{def:interior_operator} on \( (X, \Cal{T}) \).
\end{proposition}

\begin{definition}\label{def:topological_boundary}\cite[24]{Engelking1989}
  For a subset \( A \) of a topological space we define the \textbf{boundary}
  \begin{align*}
    \partial A \coloneqq \Cl{A} \setminus \Int{A}.
  \end{align*}
\end{definition}

\begin{definition}\label{def:topologically_derived_set}\cite[24]{Engelking1989}
  Let \( (X, \Cal{T}) \) be a topological space.

  \begin{defenum}
    \DItem{def:topologically_derived_set/accumulation_point} We say that the point \( x \in X \) is an \textbf{accumulation point of the set \( A \subseteq X \)} if \( x \in \Cl(A \setminus \{ x \}) \). It is not necessary for \( x \) to belong to \( A \).

    \DItem{def:topologically_derived_set/derived_set} The set of all accumulation points of \( A \) is called the \textbf{derived set of \( A \)} and is denoted by \( A^d \).

    \DItem{def:topologically_derived_set/isolated_point} Points in \( A \setminus A^d \) are said to be \textbf{isolated points of \( A \)}.
  \end{defenum}
\end{definition}

\begin{definition}\label{def:topologically_dense_set}\cite[25]{Engelking1989}
  Let \( (X, \Cal{T}) \) be a topological space and \( A \subseteq X \) be any set. We say that \( A \) is

  \begin{defenum}
    \DItem{def:topologically_dense_set/dense} \textbf{dense in \( X \)} if \( \Cl{A} = X \) (if \( X \) is assumed from the context, we simply say that \( A \) is dense).

    \DItem{def:topologically_dense_set/codense} \textbf{codense in \( X \)} if \( X \setminus A \) is dense, i.e. \( \Cl{X \setminus A} = X \).

    \DItem{def:topologically_dense_set/nowhere_dense} \textbf{nowhere dense in \( X \)} if \( \Cl{A} \) is codense, i.e. \( \Cl{X \setminus \Cl{A}} = X \).

    \DItem{def:topologically_dense_set/dense_in_itself} \textbf{dense in itself} if \( A \subseteq A^d \), i.e. if \( A \) has no isolated points.
  \end{defenum}

  We define the \textbf{density \( d(X) \) of \( X \)} to be the minimum cardinality\Tinyref{def:cardinal} of all dense sets. If \( d(X) \leq \aleph_0 \), we say that the space is separable.
\end{definition}
