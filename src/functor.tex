\begin{definition}\label{def:functor}\cite[definitions 1.2.1, 1.2.10]{Leinster2014}
  Let $\Bold A$ and $\Bold B$ be categories. A \uline{(covariant) functor} $F: \Bold A \to \Bold B$ consists of:
  \begin{itemize}
    \item a function $\Obj \Bold A \to \Obj \Bold B$, written as $A \mapsto F(A)$.
    \item for each $A, B \in \Obj \Bold A$, a function
    \begin{align*}
      \Bold{A}(A, B) \to \Bold{B}(F(A), F(B)),
    \end{align*}
    written as $f \mapsto F(f)$.
  \end{itemize}
  such that
  \begin{defenum}
    \item\label{def:functor/composition_axiom} $A \overset f \mapsto B \overset g \mapsto C$ implies $F(g \circ f) = F(g) \circ F(f)$.
    \item\label{def:functor/identity_axiom} $A \in \Obj \Bold A$ implies $F(\Id_A) = \Id_{F(A)}$.
  \end{defenum}

  If we replace the axiom~\cref{def:functor/composition_axiom} with
  \begin{defenum}
    \item[b')]\label{def:functor/contravariant_composition_axiom} $A \overset f \mapsto B \overset g \mapsto C$ implies $F(g \circ f) = F(f) \circ F(g)$,
  \end{defenum}
  we call $F$ a \uline{contravariant functor}. Equivalently, $F: \Bold A \to \Bold B$ is contravariant if and only if $F: \Bold{A}^{\Op} \to \Bold B$ is covariant.
\end{definition}

\begin{definition}\label{def:forgetful_functor}\cite[example 1.2.3]{Leinster2014}
  An informal notion is that of the \uline{forgetful functor}. A functor $F: \Bold A \to \Bold B$ is called forgetful if the images $F(A)$ of objects $A \in \Obj \Bold A$ have \enquote{less structure} than $A$. For example, a functor which sends topological spaces to their underlying sets is forgetful since it \enquote{forgets} about the topological structure.
\end{definition}

\begin{definition}\label{def:free_functor}\cite[example 1.2.4]{Leinster2014}
  Another informal notion, which is dual to \cref{def:forgetful_functor}, is that of a \uline{free functor}. In contrast to forgetful functors which \enquote{remove structure}, free functors \enquote{add structure}. For example, a functor which sends a set to its corresponding discrete topological space is a free functor.
\end{definition}

\begin{definition}\label{def:presheaf}\cite[definition 1.2.15]{Leinster2014}
  A \uline{presheaf} on the category $\Bold A$ is a contravariant functor
  \begin{align*}
    F: \Bold A \to \Bold{Set}.
  \end{align*}
\end{definition}

\begin{example}\label{ex:topological_space_presheaf}\cite[24]{Leinster2014}
  Let $(X, \tau)$ be a topological space. Form the category $\Bold C$ from the poset $(\tau, \subseteq)$ as in \cref{def:standard_categories/ord}. Presheaves on $\Bold C$ are also called presheaves on the topological space $(X, \tau)$.

  Let $(Y, \rho)$ be another topological space. Then the map
  \begin{align*}
    &F: \tau \MultTo C(\tau, Y) \\
    &F(U) = C(U, Y) = \{ f: U \mapsto Y, f \text{ is continuous} \}
  \end{align*}
  is a presheaf.
\end{example}

\begin{definition}\label{def:faithful_full_functors}\cite[definition 1.2.16]{Leinster2014}
  A functor $F: \Bold A \to \Bold B$ is called \uline{faithful} (resp. \uline{full}) if the map
  \begin{align*}
    \Bold{A}(A, B) &\to \Bold{B}(F(A), F(B)) \\
    f &\mapsto F(f)
  \end{align*}
  is injective (resp. surjective) (see \cref{def:function_invertability}).
\end{definition}

\begin{example}\label{def:subcategory_functors}\cite[25]{Leinster2014}
  Let $\Bold B$ be a subcategory of $\Bold A$. We define the inclusion functor $I: \Bold B \to \Bold A$ by sending each object and each morphism of $\Bold B$ to itself within $\Bold A$.

  Then $I$ is faithful and, if the subcategory $\Bold B$ is full (see \cref{def:subcategory}), then $I$ is also full.
\end{example}

\begin{definition}\label{def:natural_transformation}\cite[definition 1.3.1]{Leinster2014}
  Let $\Bold A$ and $\Bold B$ be categories and let $F$ and $G$ be functors from $\Bold A$ to $\Bold B$.

  A \uline{natural transformation} $\alpha: F \to G$ is a family $\{ \alpha_A \}_{A \in \Bold A}$ of morphisms in $\Bold B$ such that for every morphism $f: A \to B$ in $\Bold A$, the diagram
  \begin{center}
    \begin{tikzcd}
      F(A) \arrow[r, "F(f)"] \arrow[d, "\alpha_A"] & F(B) \arrow[d, "\alpha_B"] \\
      G(A) \arrow[r, "G(f)"]                       & G(B)
    \end{tikzcd}
  \end{center}
  commutes.

  The morphisms $\alpha_A$ are called the components of $\alpha$. We denote natural transformations using
  \begin{center}
    \begin{tikzcd}[column sep=huge]
      \Bold A
        \arrow[r, bend left, "F"]{}[name=F]{}
        \arrow[r, bend right, "G"']{}[name=G]{} &
      \Bold B
        \arrow[shorten <= 0.5em, Rightarrow,to path={(F) -- node[label=right:$\alpha$] {} (G)}]{}
    \end{tikzcd}
  \end{center}

  Composition of natural transformations is defined in an obvious way and is sometimes called \uline{vertical composition}.
\end{definition}

\begin{definition}\label{def:functor_category}
  Given categories $\Bold A$ and $\Bold B$, we define their \uline{functor category} ${\Bold B}^{\Bold A}$ by
  \begin{itemize}
    \item $\Obj {\Bold B}^{\Bold A}$ are functors $F: \Bold A \to \Bold B$.
    \item ${\Bold B}^{\Bold A}(F, G)$ are the natural transformations from $F$ to $G$.
  \end{itemize}

  When $\Bold A$ is a discrete category, then ${\Bold B}^{\Bold A}$ is called a \uline{product category} indexed by $\Bold A$. In particular, if $\Bold A$ is finite of cardinality $n$, we can also use the notation
  \begin{align*}
    {\Bold B}^{\Bold A} = {\Bold B}^n = \Bold B \times \ldots \times \Bold B.
  \end{align*}

  If the natural transformation $\alpha$ is an isomorphism in ${\Bold B}^{\Bold A}$, we say that the categories $\Bold A$ and $\Bold B$ are \uline{naturally isomorphic} and write $\Bold A \cong \Bold B$.
\end{definition}

\begin{definition}\label{def:category_equivalence}\cite[definition 1.3.15]{Leinster2014}
  An \uline{equivalence} between the categories $\Bold A$ and $\Bold B$ consists of a pair of functors $F, G: \Bold A \to \Bold B$ and a pair of natural isomorphisms
  \begin{align*}
    \xi: \Id_{\Bold A} \to G \circ F,
    &&
    \eta: F \circ G \to \Id_{\Bold B}.
  \end{align*}

  If an equivalence between $\Bold A$ and $\Bold B$ exists, we say that \uline{the categories $\Bold A$ and $\Bold B$ are equivalent} and write $\Bold A \simeq \Bold B$.

  An equivalence of the form $\Bold{A}^{\Op} \simeq \Bold{B}$ is called a \uline{duality} between $\Bold A$ and $\Bold B$ and we say that \uline{$\Bold A$ is dual to $\Bold B$} \cite[example 1.3.22]{Leinster2014}.
\end{definition}

\begin{definition}\label{def:natural_transformation_horizontal_composition}\cite[remarks 1.3.24]{Leinster2014}
  Let $\Bold A$, $\Bold B$ and $\Bold C$ be categories, $F, G: \Bold A \to \Bold C$ and $H, K: \Bold B \to \Bold C$ be functors and $\beta: F \to G$ and $\gamma: H \to K$ be natural transformations.
  \begin{center}
    \begin{tikzcd}[column sep=huge]
      \Bold A
        \arrow[r, bend left, "F"]{}[name=F]{}
        \arrow[r, bend right, "G"']{}[name=G]{} &
      \Bold B
        \arrow[shorten <= 0.5em, Rightarrow,to path={(F) -- node[label=right:$\beta$] {} (G)}]{}
        \arrow[r, bend left, "H"]{}[name=H]{}
        \arrow[r, bend right, "K"']{}[name=K]{} &
      \Bold C
        \arrow[shorten <= 0.5em, Rightarrow,to path={(H) -- node[label=right:$\gamma$] {} (K)}]{}
    \end{tikzcd}
  \end{center}

  We define the natural transformation
  \begin{align*}
    \alpha \coloneqq \gamma * \beta: H \circ F \to K \circ G,
  \end{align*}
  called \uline{horizontal composition of $\beta$ and $\gamma$}, defined by
  \begin{align*}
    \alpha_A \coloneqq \gamma_{G(A)} \circ H(\beta_A) = K(\beta_A) \circ \gamma_{F(A)}.
  \end{align*}
\end{definition}

\begin{note}
  We restrict our attention to locally small categories because we need to define an isomorphism of morphism sets.
\end{note}
\begin{definition}\label{def:adjoint_functor}\cite[definition 2.1.1]{Leinster2014}
  Let $\Bold A$ and $\Bold B$ be locally small categories and $F: \Bold A \to \Bold B$ and $G: \Bold B \to \Bold A$ be functors. Further assume that for every $A \in \Obj \Bold A$ and $B \in \Obj \Bold B$ we have an isomorphism
  \begin{align*}
    \Bold{A}(A, G(B)) \overset {\varphi_{X,Y}} {\cong} \Bold{B}(F(A), B),
  \end{align*}
  where $\Bold{A}(A, G(B))$ and $\Bold{B}(F(A), B)$ are regarded as objects in $\Bold{Set}$.

  Given a morphism $f: A \to G(B)$, we define the \uline{transpose $\Ol f$ of $f$} as
  \begin{align*}
    &\Ol f: F(A) \to B \\
    &\Ol f \coloneqq \varphi_{A, B} (f).
  \end{align*}

  Dually, given a morphism $g: F(A) \to B$, we define
  \begin{align*}
    &\Ol g: A \to G(B) \\
    &\Ol g \coloneqq \varphi_{A, B}^{-1} (g).
  \end{align*}

  We say that the isomorphism $\varphi_{X,Y}$ is \uline{natural} if,  given $A' \in \Obj \Bold A$ and morphisms $f: A \to G(B)$ and $p: A' \to A$, we have
  \begin{align*}
    \Ol{f \circ p} = \Ol f \circ F(p),
  \end{align*}
  and, given $B' \in \Obj \Bold B$ and morphisms $g: F(A) \to B$ and $q: B \to B'$, we have
  \begin{align*}
    \Ol{q \circ g} = G(q) \circ \Ol g.
  \end{align*}

  In this case, we say that $F$ is \uline{left-adjoint} to $G$ and $G$ is \uline{right-adjoint} to $F$, and write $F \dashv G$.
\end{definition}

\begin{example}\label{ex:top_adjoint_functor}\cite[example 2.1.5]{Leinster2014}
  Consider the functors
  \begin{itemize}
    \item $U: \Bold{Top} \to \Bold{Set}$, which sends topological spaces to their underlying sets.
    \item $D: \Bold{Set} \to \Bold{Top}$, which sends sets to topological spaces equipped with the discrete topology.
    \item $I: \Bold{Set} \to \Bold{Top}$, which sends sets to topological spaces equipped with the indiscrete topology.
  \end{itemize}

  Let $T \in \Bold{Top}$ and $S \in \Bold{Set}$.

  Let $f: T \to I(S)$ be any continuous function and $g: U(T) \to S$ be any function.

  Denote by $\Ol f: U(T) \to S$ the function between sets, corresponding to $f$ and by $\Ol g: T \to I(S)$ the corresponding function between the topological spaces $T$ and $I(S)$. Since any function into an indiscrete topological space is $T$ is continuous, we have that $\Ol g$ is a morphism $T \to I(S)$.

  Thus $\Ol{\Ol f} = f$ and $\Ol{\Ol g} = g$ and we have a natural isomorphism between $\Bold{Set}(U(T), S)$ and $\Bold{Top}(T, I(S))$. This proves that $U \dashv I$.

  Similarly, since any function from a discrete space is continuous, we have that $D \dashv U$.

  Hence $D \dashv U \dashv I$.
\end{example}

\begin{definition}\label{def:representable_functor}\cite[example 4.1.5]{Leinster2014}
  Let $\Bold A$ be a locally small category and $A \in \Obj \Bold A$. Define
  \begin{align*}
    &H^A: \Bold{A} \to \Bold{Set}, \\
    &H^A(B) \coloneqq \Bold{A}(A, B), \\
    &H^A(f: B \to C) \coloneqq p \mapsto f \circ p.
  \end{align*}

  We say that the functor $F: \Bold{A} \to \Bold{Set}$ is \uline{representable} with \uline{representation} $H^A$ if $F \cong H^A$.
\end{definition}

\begin{example}\label{def:top_representable_functor}\cite[definition 4.1.4]{Leinster2014}
  Let $U: \Bold{Top} \to \Bold{Set}$ be the forgetful functor which sends a topological space to its underlying set.

  Let $1$ be the one-element topological space. There is a correspondence between points $x$ in $T$ and continuous functions $p_x: 1 \to T$. Thus the functor $H^1$ sends
  \begin{itemize}
    \item any topological space $T$ into the set of morphisms
    \begin{align*}
      H^1(T) = \Bold{Top}(1, T) = \{ p_x: 1 \to T \} \cong U(T).
    \end{align*}
    \item any continuous function $f: T \to S$ to
    \begin{align*}
      H^1(f) = p_x \mapsto f \circ p_x \cong x \mapsto f(x) = f.
    \end{align*}
  \end{itemize}

  Thus $U$ is representable with representation $H^1$.
\end{example}
