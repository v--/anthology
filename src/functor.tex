\begin{definition}\label{def:functor}(\cite[definitions 1.2.1, 1.2.10]{Leinster2014})
  Let $\Cal A$ and $\Cal B$ be categories. A \uline{(covariant) functor} $F: \Cal A \to \Cal B$ consists of:
  \begin{itemize}
    \item a function $\Obj \Cal A \to \Obj \Cal B$, written as $A \mapsto F(A)$.
    \item for each $A, B \in \Obj \Cal A$, a function
    \begin{align*}
      \Cal{A}(A, B) \to \Cal{B}(F(A), F(B)),
    \end{align*}
    written as $f \mapsto F(f)$.
  \end{itemize}
  such that
  \begin{defenum}
    \item\label{def:functor/composition_axiom} $A \overset f \mapsto B \overset g \mapsto C$ implies $F(g \circ f) = F(g) \circ F(f)$.
    \item\label{def:functor/identity_axiom} $A \in \Obj \Cal A$ implies $F(\Ind_A) = \Ind_{F(A)}$.
  \end{defenum}

  If we replace the axiom~\cref{def:functor/composition_axiom} with
  \begin{defenum}
    \item[b')]\label{def:functor/contravariant_composition_axiom} $A \overset f \mapsto B \overset g \mapsto C$ implies $F(g \circ f) = F(f) \circ F(g)$,
  \end{defenum}
  we call $F$ a \uline{contravariant functor}. Equivalently, $F: \Cal A \to \Cal B$ is contravariant if and only if $F: \Cal{A}^{\Op} \to \Cal B$ is covariant.
\end{definition}

\begin{definition}\label{def:forgetful_functor}(\cite[example 1.2.3]{Leinster2014})
  An informal notion is that of the \uline{forgetful functor}. A functor $F: \Cal A \to \Cal B$ is called forgetful if the images $F(A)$ of objects $A \in \Obj \Cal A$ have \enquote{less structure} than $A$. For example, a functor which sends topological spaces to their underlying sets is forgetful since it \enquote{forgets} about the topological structure.
\end{definition}

\begin{definition}\label{def:free_functor}(\cite[example 1.2.4]{Leinster2014})
  Another informal notion, which is dual to \cref{def:forgetful_functor}, is that of a \uline{free functor}. In contrast to forgetful functors which \enquote{remove structure}, free functors \enquote{add structure}. For example, a functor which sends a set to its corresponding discrete topological space is a free functor.
\end{definition}

\begin{definition}\label{def:presheaf}(\cite[definition 1.2.15]{Leinster2014})
  A \uline{presheaf} on the category $\Cal A$ is a contravariant functor
  \begin{align*}
    F: \Cal A \to \Cat{Set}.
  \end{align*}
\end{definition}

\begin{example}\label{ex:topological_space_presheaf}(\cite[24]{Leinster2014})
  Let $(X, \tau)$ be a topological space. Form the category $\Cal C$ from the poset $(\tau, \subseteq)$ as in \cref{def:standard_categories/ord}. Presheaves on $\Cal C$ are also called presheaves on the topological space $(X, \tau)$.

  Let $(Y, \rho)$ be another topological space. Then the map
  \begin{align*}
    &F: \tau \to C(\tau, Y) \\
    &F(U) = C(U, Y) = \{ f: U \mapsto Y, f \text{ is continuous} \}
  \end{align*}
  is a presheaf.
\end{example}

\begin{definition}\label{def:faithful_full_functors}(\cite[definition 1.2.16]{Leinster2014})
  A functor $F: \Cal A \to \Cal B$ is called \uline{faithful} (resp. \uline{full}) if the map
  \begin{align*}
    \Cal{A}(A, B) &\to \Cal{B}(F(A), F(B)) \\
    f &\mapsto F(f)
  \end{align*}
  is injective (resp. surjective) (see \cref{def:function_invertability}).
\end{definition}

\begin{example}\label{def:subcategory_functors}(\cite[25]{Leinster2014})
  Let $\Cal B$ be a subcategory of $\Cal A$. We define the inclusion functor $I: \Cal B \to \Cal A$ by sending each object and each morphism of $\Cal B$ to itself within $\Cal A$.

  Then $I$ is faithful and, if the subcategory $\Cal B$ is full (see \cref{def:subcategory}), then $I$ is also full.
\end{example}
