\subsection{Noncompactness measures}\label{subsec:noncompactness_measures}

\begin{definition}\label{def:noncompactness_measures}\MarginCite[def. 7.1]{Deimling1985}
  Let \( (X, \rho) \) be a \hyperref[def:metric_space]{metric space} and let \( \CB \) be the family of bounded \hyperref[def:metric_space/bounded_set]{sets} in \( X \). We define the following functions
  \begin{DefEnum}
    \ILabel{def:noncompactness_measures/sets} The \Def{Kuratowski measure of noncompactness},
    \begin{align*}
       & \alpha: \CB \to [0, \infty)                                                                                                            \\
       & \alpha(A) \coloneqq \inf \{d > 0 \colon \exists U_1, \ldots, U_n \subseteq X: \Diam {U_k} < d \Tand A \subseteq \bigcup_{k=1}^n U_k \}
    \end{align*}

    \ILabel{def:noncompactness_measures/balls} The \Def{ball measure of noncompactness},
    \begin{align*}
       & \beta: \CB \to [0, \infty)                                                                                   \\
       & \beta(A) \coloneqq \inf \{r > 0 \colon \exists x_1, \ldots, x_2 \in X: A \subseteq \cup_{k=1}^n B(x_k, r) \}
    \end{align*}
  \end{DefEnum}
\end{definition}

\begin{example}\label{ex:noncompactness_measures}\MarginCite[exer. 7.3]{Deimling1985}
  Consider the subsets \( B_2 \subseteq B_3 \subseteq B_1 \subseteq C([0, 1]) \), defined by
  \begin{align*}
    B_1 & \coloneqq \left\{
    x \in C([0, 1]) \colon \begin{aligned}
      0 \leq t \leq 1 \implies 0 \leq x(t) \leq 1 \\
      x(0) = 0, x(1) = 1                          \\
    \end{aligned}
    \right\}
    \\
    B_2 & \coloneqq \left\{
    x \in B_1 \colon \begin{aligned}
      0 \leq t \leq \frac 1 2 \implies 0 \leq x(t) \leq \frac 1 2 \\
      \frac 1 2 \leq t \leq 1 \implies \frac 1 2 \leq x(t) \leq 1
    \end{aligned}
    \right\}
    \\
    B_3 & \coloneqq \left\{
    x \in B_1 \colon \begin{aligned}
      0 \leq t \leq \frac 1 2 \implies 0 \leq x(t) \leq \frac 2 3 \\
      \frac 1 2 \leq t \leq 1 \implies \frac 1 3 \leq x(t) \leq 1
    \end{aligned}
    \right\}
  \end{align*}

  Then \( \alpha(B_1) = 1, \alpha(B_2) = \frac 1 2, \alpha(B_3) = \frac 1 3 \) and \( \beta(B_1) = \beta(B_2) = \beta(B_3) = \frac 1 2 \).
\end{example}
\begin{proof}
  Since the distance between any two functions from \( B_1 \) is at most 1, we have that \( \Diam B_1 = 1 \) and \( \alpha(B_1) \leq 1 \).

  Fix \( \varepsilon > 0 \). For any function \( f \in B_1 \), continuity of \( f \) gives us a radius \( \delta_f > 0 \) such that
  \begin{equation*}
    x < 2 \delta_f \implies f(x) < \varepsilon.
  \end{equation*}

  \begin{figure}
    \centering
    \begin{mplibcode}
      input metapost/plotting;
      u := 4cm;
      e := 0.16; % epsilon
      d := sqrt(e); % delta

      vardef f(expr x) =
      x ** 2
      enddef;

      vardef tf(expr x) =
      if x < d / 2:
      (2 / d) * x
      elseif x < d:
      f(d) + (1 - f(d)) * ((2 / d) * (d - x))
      else:
      f(x)
      fi
      enddef;

      beginfig(1)
      drawarrow (0, 0) -- (u,  0);
      drawarrow (0, 0) -- (0, u);

      drawoptions(dashed withdots scaled 0.3);
      draw (0, e) scaled u -- (1, e) scaled u;
      label.lft("$\varepsilon$", (0, e) scaled u);

      draw (d, 0) scaled u -- (d, e) scaled u;
      label.bot("$\delta$", (d, 0) scaled u);
      drawoptions();

      draw path_of_plot(f, 0, 1, 0.01, u);
      label.rt("$f$", (0.75, 0.5) scaled u);

      drawoptions(dashed evenly);
      draw path_of_plot(tf, 0, d, 0.01, u);
      drawoptions();
      label.rt("$T_\varepsilon(f)$", (0.3, 0.7) scaled u);
      endfig;
    \end{mplibcode}
    \Caption{ex:noncompactness_measures/spike_plot}{The operator $T_\varepsilon$ adds \enquote{spikes} to functions.}
  \end{figure}

  Define
  \begin{align*}
    T_\varepsilon(f)(x) \coloneqq \begin{cases}
      \frac x {\delta_f},                                       & 0 \leq x < \delta_f          \\
      f(\delta_f) + [1 - f(\delta_f)] (2 - \frac x {\delta_f}), & \delta_f \leq x < 2 \delta_f \\
      f(x),                                                     & x \geq 2 \delta_f,
    \end{cases}
  \end{align*}
  so that
  \begin{align*}
    \Norm{T_\varepsilon(f) - f}
    \geq
    T_\varepsilon(f) (\delta_f) - f(\delta_f)
    =
    1 - f(\delta_f)
    >
    1 - \varepsilon.
  \end{align*}

  Additionally, because \( \delta_{T_\varepsilon(f)} < \delta_f \), we have that \( f(\delta_{T_\varepsilon(f)}) < \varepsilon \) and
  \begin{align*}
    \Norm{T_\varepsilon(T_\varepsilon(f)) - f}
    \geq
    T_\varepsilon(T_\varepsilon(f)) (\delta_{T_\varepsilon(f)}) - f(\delta_{T_\varepsilon(f)})
    =
    1 - f(\delta_{T_\varepsilon(f)})
    >
    1 - \varepsilon.
  \end{align*}

  Thus, proceeding by induction\IND, we see that for any \( m = 1, 2, \ldots \)
  \begin{equation*}
    \Norm{T_\varepsilon^m(f) - f} > 1 - \varepsilon,
  \end{equation*}
  where \( T_\varepsilon^m \) denotes repeated application of \( T_\varepsilon \).

  Consider the sequence
  \begin{equation*}
    \{ T_\varepsilon^k(f) \}_{k=0}^\infty = \{ f, T_\varepsilon(f), T_\varepsilon(T_\varepsilon(f)), \ldots \}.
  \end{equation*}

  We can easily see that the distance between any two elements of the sequence, say \( T_\varepsilon^k(f) \) and \( T_\varepsilon^{k+m}(f) \), is strictly greater that \( 1 - \varepsilon \), i.e.
  \begin{align*}
    \Norm{T_\varepsilon^k(f) - T_\varepsilon^{k+m}(f)}
    =
    \Norm{T_\varepsilon^k(f) - T_\varepsilon^m(T_\varepsilon^k(f))}
    >
    1 - \varepsilon.
  \end{align*}

  Hence \( B_1 \) cannot be covered by a finite \( (1-\varepsilon) \)-net and \( \alpha(B_1) \geq 1 - \varepsilon \). Since \( \varepsilon > 0 \) can be made arbitrarily small, this implies that \( \alpha(B_1) \geq 1 \) and, because we already have the reverse inequality, \( \alpha(B_1) = 1 \).

  In the set \( B_2 \), the maximum distance between two functions is \( \frac 1 2 \), thus \( \Diam(B_2) = \frac 1 2 \) and \( \alpha(B_2) \leq \frac 1 2 \). We can then define an operator similar to \( T_\varepsilon \) that creates \enquote{spikes} of height \( \frac 1 2 \) to prove the reverse inequality, obtaining
  \begin{equation*}
    \alpha(B_2) = \frac 1 2.
  \end{equation*}

  Finally, the set \( B_3 \) has diameter \( \frac 2 3 \) and hence \( \alpha(B_3) = \frac 2 3 \).

  The ball measure for \( B_1 \) satisfies the inequalities
  \begin{equation*}
    \frac 1 2 \leq \beta(B_1) \leq 1.
  \end{equation*}

  Additionally, \( B_1 \) is strictly contained in the ball centered in the constant function \( \frac 1 2 \) with radius \( \frac 1 2 \), which implies that \( \beta(B_1) \leq \frac 1 2 \), hence \( \beta(B_1) = \frac 1 2 \).

  For \( B_2 \) we have
  \begin{equation*}
    \frac 1 4 \leq \beta(B_2) \leq \frac 1 2.
  \end{equation*}

  Assume\LEM that for some \( \varepsilon > 0 \) the set \( B_2 \) can be covered by a finite set of balls with centers \( \{ f_1, \ldots, f_n \} \subsetneq C([0, 1]) \) and radius \( \frac 1 2 - \varepsilon \).

  Because of continuity, we can find a radius \( \delta > 0 \) such that for all \( f_k, k = 1, \ldots, n \) we have
  \begin{equation*}
    x \in \left[\tfrac {1 - \delta} 2, \tfrac {1 + \delta} 2 \right] \implies \Abs{f_k(x) - f_k(\tfrac 1 2)} < \varepsilon.
  \end{equation*}

  Consider the function
  \begin{align*}
    g(x) \coloneqq \begin{cases}
      0,                                & 0 \leq x < \frac {1 - \delta} 2,                       \\
      \frac{2x + \delta - 1} {2\delta}, & \frac {1 - \delta} 2 \leq x \leq \frac {1 + \delta} 2, \\
      1,                                & \frac {1 + \delta} 2 < x \leq 1.
    \end{cases}
  \end{align*}

  \begin{figure}
    \centering
    \begin{mplibcode}
      input metapost/plotting;

      u := 6cm;
      e := 0.1; % epsilon

      vardef f_k(expr x) =
      1 - 1 / (1 + exp(2 - 3 * x))
      enddef;

      vardef g(expr x) =
      if x < (1 - d) / 2:
      0
      elseif x < (1 + d) / 2:
      (2x + d - 1) / 2d
      else:
      1
      fi
      enddef;

      f_minus := f_k(0.5) - e;
      f_plus := f_k(0.5) + e;
      d := (ln(1 / (1 - f_plus) - 1) - ln(1 / (1 - f_minus) - 1)) / 3; % delta
      d_minus := (1 - d) / 2;
      d_plus := (1 + d) / 2;

      beginfig(1)
      drawarrow (-0.1, 0) -- (u,  0);
      drawarrow (0, 0) -- (0, u);

      drawoptions(dashed withdots scaled 0.3);
      draw (0, f_minus) scaled u -- (1, f_minus) scaled u;
      label.lft("$f_k(\frac 1 2) - \varepsilon$", (0, f_minus) scaled u);

      draw (0, f_plus) scaled u -- (1, f_plus) scaled u;
      label.lft("$f_k(\frac 1 2) + \varepsilon$", (0, f_plus) scaled u);

      draw (0.5, 0) scaled u -- (0.5, 1) scaled u;
      label.bot("$\frac 1 2$", (0.5, 0) scaled u);

      draw (d_minus, 0) scaled u -- (d_minus, 1) scaled u;
      label.bot("$\frac {1 - \delta} 2$", (d_minus, 0) scaled u);

      draw (d_plus, 0) scaled u -- (d_plus, 1) scaled u;
      label.bot("$\frac {1 + \delta} 2$", (d_plus, 0) scaled u);
      drawoptions();

      drawoptions(dashed evenly);
      draw path_of_plot(f_k, -0.1, 1, 0.01, u);
      label.rt("$f_k$", (1, f_k(1)) scaled u);
      drawoptions();

      draw path_of_plot(g, 0, 1, 0.005, u);
      label.lft("$g$", (d_minus, 0.1) scaled u);
      endfig;
    \end{mplibcode}
    \Caption{ex:noncompactness_measures/sigmoid_plot}{The function $g$ always has points that are far enough from all $f_k, k = 1, \ldots, n$.}
  \end{figure}

  If \( f_k(\tfrac 1 2) \geq \frac 1 2 \), then \( f_k(\tfrac {1 - \delta} 2) > \tfrac 1 2 - \varepsilon \) and
  \begin{equation*}
    \Norm{f_k - g} \geq f_k(\tfrac {1 - \delta} 2) - g(\tfrac {1 - \delta} 2) = f_k(\tfrac {1 - \delta} 2) > \tfrac 1 2 - \varepsilon.
  \end{equation*}

  Analogously, if \( f_k(\tfrac 1 2) < \frac 1 2 \), then \( f_k(\tfrac {1 + \delta} 2) < \tfrac 1 2 + \varepsilon \) and
  \begin{equation*}
    \Norm{g - f_k} \geq g(\tfrac {1 + \delta} 2) - f_k(\tfrac {1 + \delta} 2) = 1 - f_k(\tfrac {1 + \delta} 2) > \tfrac 1 2 - \varepsilon.
  \end{equation*}

  Thus, for every \( k = 1, \ldots, n \) we have
  \begin{equation*}
    \Norm{g - f_k} > \frac 1 2 - \varepsilon,
  \end{equation*}
  i.e. \( g \) in not contained in a ball of radius \( \frac 1 2 - \varepsilon \) around any of the centers \( f_1, \ldots, f_n \).

  Hence \( \beta(B_2) \geq \frac 1 2 \), which implies \( \beta(B_2) = \frac 1 2 \). Because of the inclusion \( B_2 \subsetneq B_3 \subsetneq B_1 \), we have
  \begin{equation*}
    \frac 1 2 = \beta(B_2) \leq \beta(B_3) \leq \beta(B_1) = \frac 1 2,
  \end{equation*}
  hence \( \beta(B_3) = \frac 1 2 \).
\end{proof}

\begin{theorem}[Kuratowski's noncompactness lemma]\label{thm:noncompact_kuratowskis_lemma}\MarginCite[exer. 7.4]{Deimling1985}
  Let \( X \) be a Banach space and \( \{ A_n \}_n \) be a decreasing sequence of nonempty closed subsets such that \( \alpha(A_n) \to 0 \). Then \( A \coloneqq \bigcap_n A_n \) is nonempty and compact.
\end{theorem}
\begin{proof}
  The set \( A \) is compact because it is closed as the intersection of closed sets and \( \alpha(A) \leq \alpha(A_n) \to 0 \), hence \( \alpha(A) = 0 \).

  It remains to show that \( A \) is nonempty.
  Choose\AOC any sequence \( \{ x_n \}_n \) where \( x_n \in A_n \). Since any finite set is compact, we have that for any \( k \geq 1 \)
  \begin{align*}
    \alpha(\{ x_n \}_{n \geq 1})
    =
    \max\{ \alpha(\{ x_n \}_{n < k}), \alpha(\{ x_n \}_{n \geq k}) \}
    =
    \alpha(\{ x_n \}_{n \geq k})
    \leq
    \alpha(A_k) \to 0,
  \end{align*}
  hence the set \( \{ x_n \colon n \geq 1 \} \) is compact and thus sequentially compact. We can choose a convergent subsequence \( \{ x_{n_k} \}_k \) of \( \{ x_n \}_n \) whose limit lies in every \( A_n \) (since they are closed) and, consequently, in their intersection \( A \). So \( A \) is nonempty.
\end{proof}
