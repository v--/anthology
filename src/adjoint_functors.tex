\subsection{Adjoint functors}\label{subsec:adjoint_functors}

\begin{remark}\label{def:adjoint_functors}
  Adjoint functors are a generalization of invertible functors. When the functor \( F: \Cat{A} \to \Bold B \) is left adjoint to \( G: \Cat{A} \to \Bold B \) and \( G \) is not invertible, then \( F \) finds a \enquote{generalized inverse} for every object in \( \Bold B \) that tries to preserve its morphisms.

  Jean-Pierre Marquis in \cite{StanfordPlato:category_theory} refers to adjoint functors as \enquote{conceptual inverses} that reverse the idea rather than the realization.
\end{remark}

\begin{definition}\label{def:adjoint_functor}\MarginCite[exer. 4.1.32]{Leinster2014}
  Let \( \Cat{A} \) and \( \Bold B \) be locally small categories and \( F: \Cat{A} \to \Bold B \) and \( G: \Bold B \to \Cat{A} \) be functors.

  We say that \( F \) is \Def{left-adjoint} to \( G \) and \( G \) is \Def{right-adjoint} to \( F \) and write \( F \dashv G \) if there is a natural \hyperref[def:natural_isomorpism]{isomorphism} between the functors \( \Cat{A}(-, G(-)) \) and \( \Cat{B}(F(-), -) \).
\end{definition}

\begin{example}\label{ex:top_adjoint_functor}\MarginCite[ex. 2.1.5]{Leinster2014}
  Consider the functors
  \begin{itemize}
    \item \( U: \Cat{Top} \to \Cat{Set} \), which maps topological spaces to their underlying sets.
    \item \( D: \Cat{Set} \to \Cat{Top} \), which maps sets to topological spaces equipped with the discrete \hyperref[def:standard_topologies/discrete]{topology}.
    \item \( I: \Cat{Set} \to \Cat{Top} \), which maps sets to topological spaces equipped with the indiscrete \hyperref[def:standard_topologies/indiscrete]{topology}.
  \end{itemize}

  Let \( T \in \Cat{Top} \) and \( S \in \Cat{Set} \). Every function from a discrete topological space is continuous, therefore
  \begin{equation*}
    \Cat{Top}(D(S), T) = \Cat{Set}(S, U(T)),
  \end{equation*}
  where equality means that all the underlying functions are equal.

  Similarly, every function into an indiscrete topological space is continuous, therefore
  \begin{equation*}
    \Cat{Top}(T, I(S)) = \Cat{Set}(U(T), S).
  \end{equation*}
\end{example}
