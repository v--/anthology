\section{First order logic}\label{sec:first_order_logic}

The idea of first-order logic (FOL) is to create a formal language whose semantics (given by structures) support boolean operations and can quantify over all elements of an ambient universe. Unlike in propositional logic\Tinyref{sec:propositional_logic}, there are many FOL languages.

\begin{definition}\label{def:first_order_language}\cite[definition 2.1]{Nerode2012}
  The alphabet for a \ul{first-order predicate language}\Tinyref{def:language} $\Cal{L}$ consists of:
  \begin{description}
    \item[Logical symbols]
    \mbox{}
    \begin{enumerate}
      \item A countable alphabet of variables $\Bold{Var}_{\Cal{L}}$, usually denoted by $x_0, x_1, \ldots$ or $x, y, z$.

      \item Certain propositional operations:
      \begin{description}
        \RItem{def:propositional_logic_language/constants} $\top$ and $\bot$ - zero-arity operations
        \RItem{def:propositional_logic_language/negation} $\neg$ - unary operation
        \RItem{def:propositional_logic_language/connectives} $\Sigma = \{ \land, \lor, \implies, \iff, \downarrow, \uparrow \}$ - binary operations
      \end{description}

      \item Quantifiers
      \begin{itemize}
        \item $\forall$ (universal quantifier)
        \item $\exists$ (existential quantifier)
      \end{itemize}

      \item Parentheses $($ and $)$ for defining the order of operations unambiguously (see \cref{note:propositional_formula_parentheses}).

      \item Optionally, an equality symbol $\doteq$.
    \end{enumerate}

    \item[Non-logical symbols]
    \mbox{}
    \begin{enumerate}
      \item A set of functional symbols, $\Bold{Func}_{\Cal{L}}$, whose elements are usually denoted by $f_0, f_1, \ldots$ or $f, g, h$. Each functional symbol has an associated natural number called its \ul{arity}, denoted by $\#_{\Cal{L}} f$. Functional symbols with a zero arity are called \ul{constants}.

      \item A set of predicate symbols, $\Bold{Pred}_{\Cal{L}}$, whose elements are usually denoted by $p_0, p_1, \ldots$ or by symbols like $\oplus$ or $\circ$. Predicate symbols also have an associated arity. Predicate symbols with zero arity are called \ul{propositional variables}.
    \end{enumerate}
  \end{description}
\end{definition}

\begin{example}\label{ex:algebraic_theory_language}\cite[remark 2.1.4]{Leinster2014}
  Most algebraic structures (with the notable exception of fields) can be defined as first-order languages with equality, no predicates and a set of functional symbols called \ul{algebraic operations}.
  \begin{itemize}
    \item Group theory\Tinyref{sec:groups} has
    \begin{itemize}
      \item one zero-arity operation called its identity element $e$
      \item one unitary operation $(\cdot)^{-1}$ called the inverse element
      \item one binary operation $\oplus$ called the group operation
    \end{itemize}

    \item Linear algebra\Tinyref{sec:vector_spaces} has
    \begin{itemize}
      \item one zero-arity operation called its zero element $0$
      \item one binary operation $+$ called vector sum
      \item for every scalar $\lambda$ in the underlying field, a unitary operation $\lambda \cdot$ called scalar multiplication by $\lambda$
    \end{itemize}
  \end{itemize}
\end{example}

\begin{definition}\label{def:first_order_term}\cite[definition 2.2]{Nerode2012}
  Given a FOL language $\Cal{L}$, the set $\Cal{T}_{\Cal{L}}$ of terms is defined by structural induction as
  \begin{itemize}
    \item Each variable is a term
    \item If $\tau_1, \ldots, \tau_n$ are terms and $f$ is a functional symbol with arity $n$, then the following word is also a term:
    \begin{align*}
      f(\tau_1, \ldots, \tau_n)
    \end{align*}
  \end{itemize}

  In particular, constants are also terms.

  Furthermore, the grammar of first-order terms is unambiguous (see \cref{thm:first_order_formulas_are_unambiguous}).

  For each term $\tau$, we define its variables as
  \begin{align*}
    \Bold{Free}(\tau) \coloneqq \begin{cases}
      x,                                                        &\tau = x \in \Bold{Var}_{\Cal{L}}, \\
      \Bold{Free}(\tau_1) \cup \ldots \cup \Bold{Free}(\tau_n), &\tau = f(\tau_1, \ldots, \tau_n).
    \end{cases}
  \end{align*}
\end{definition}

\begin{definition}\label{def:first_order_formula}\cite[definition 2.5]{Nerode2012}
  Given a FOL language $\Cal{L}$, we define the set of atomic formulas inductively as
  \begin{itemize}
    \item Both $\top$ and $\bot$ are atomic formulas.
    \item If $p$ is an n-ary predicate symbol and if $\tau_1, \ldots, \tau_n$ are terms, then $p(\tau_1, \ldots, \tau_n)$ is an atomic formula.
    \item If $\Cal{L}$ has an equality symbol and if $\tau_1, \tau_2$ are terms, then $(\tau_1 \doteq \tau_2)$ is an atomic formula.
  \end{itemize}

  The set $\Cal{F}_{\Cal{L}}$ of predicate formulas is then defined as
  \begin{itemize}
    \item All atomic formulas are formulas
    \item If $\varphi$ is a formula, its negation $\neg \varphi$ is also a formula
    \item If $\varphi$ and $\psi$ are formulas, then $(\varphi \circ \psi), \circ \in \Sigma$\Tinyref{def:propositional_logic_language}, are also formulas
    \item If $\varphi$ is a formula and $x$ is a variable, then the following are also formulas:
    \begin{itemize}
      \item $\forall x \varphi$
      \item $\exists x \varphi$
    \end{itemize}
  \end{itemize}

  Furthermore, the grammar of first-order formulas is unambiguous (see \cref{thm:first_order_formulas_are_unambiguous}).

  For each formula $\varphi$, we define its free and bound variables as
  \begin{align*}
    \Bold{Free}(\varphi) \coloneqq \begin{cases}
      \varnothing,                                              &\varphi \in \{ \top, \bot \} \\
      \Bold{Free}(\tau_1) \cup \ldots \cup \Bold{Free}(\tau_n), &\varphi = p(\tau_1, \ldots, \tau_n) \\
      \Bold{Free}(\tau_1) \cup \Bold{Free}(\tau_2),             &\varphi = (\tau_1 \doteq \tau_2), \\
      \Bold{Free}(\psi),                                        &\varphi = \neg \psi, \\
      \Bold{Free}(\psi_1) \cup \Bold{Free}(\psi_2),             &\varphi = \psi_1 \circ \psi_2, \circ \in \Sigma, \\
      \Bold{Free}(\psi) \setminus \{ x \},                      &\varphi = Q x \psi, Q \in \{ \forall, \exists \}
    \end{cases}
  \end{align*}
  and
  \begin{align*}
    \Bold{Bound}(\varphi) \coloneqq \begin{cases}
      \varnothing,                                              &\varphi \in \{ \top, \bot \} \\
      \varnothing,                                              &\varphi = p(\tau_1, \ldots, \tau_n) \\
      \varnothing,                                              &\varphi = (\tau_1 \doteq \tau_2), \\
      \Bold{Bound}(\psi),                                       &\varphi = \neg \psi, \\
      \Bold{Bound}(\psi_1) \cup \Bold{Bound}(\psi_2),           &\varphi = \psi_1 \circ \psi_2, \circ \in \Sigma, \\
      \Bold{Bound}(\psi) \cup \{ x \},                          &\varphi = Q x \psi, Q \in \{ \forall, \exists \}.
    \end{cases}
  \end{align*}

  A formula is called \ul{closed} if it has no bound variables.

  If a formula $\varphi$ has free variables $\Bold{Free} = \{ x_1, \ldots, x_n \}$, a common convention is to write it as
  \begin{align*}
    \varphi(x_1, \ldots, x_n).
  \end{align*}

  This highlights that formulas with free variables can act as predicates, however their semantics are completely determined, unlike the semantics of predicates.

  Analogously to \cref{def:propositional_theory}, we define a \ul{first-order theory} to be a set of formulas along with any additional axioms, like the ones from \cref{note:minimal_first_order_language} and \cref{note:first_order_equality}.
\end{definition}

\begin{proposition}\label{thm:first_order_formulas_are_unambiguous}
  The grammar\Tinyref{def:grammar}
  \begin{displaymath}
    \begin{aligned}
      &\Theta \to v,                                          && v \in \Bold{Var} \\
      &\tau \to \Theta,                                       && \\
      &\tau \to f(\tau, \ldots, \tau),                        && f \in \Bold{Func} \text{ is an } n-\text{ary functional symbol} \\
      &\Phi \to \top \;|\; \bot                               && \\
      &\Phi \to p(\tau, \ldots, \tau),                        && p \in \Bold{Pred} \text{ is an } n-\text{ary predicate symbol} \\
      &\Phi \to (\tau \doteq \tau)                            && \\
      &\Phi \to \neg \Phi                                     && \\
      &\Phi \to (\Phi \circ \Phi),                            && \circ \in \Sigma \\
      &\Phi \to \forall \Theta \Phi \;|\; \exists \Theta \Phi && \\
    \end{aligned}
  \end{displaymath}
  of first order formulas\Tinyref{def:first_order_formula} is unambiguous\Tinyref{def:ambiguous_grammar}.
\end{proposition}
\begin{proof}
  The proof is more complicated but similar to \cref{thm:propositional_formulas_are_unambiguous}.
\end{proof}

\begin{note}\label{note:minimal_first_order_language}
  As in \cref{note:minimal_propositional_language}, to avoid redundancy in definitions and proofs, we can use the Pierce arrow $\downarrow$ to define the constants, negation and all other connectives by adding additional axioms to every theory.
\end{note}

\begin{note}\label{note:first_order_equality}\cite[definition 5.1]{Nerode2012}
  Equality is a concept that implies that two objects are completely indistinguishable. Let $\Cal{L}$ be a first-order language with an equality symbol. In order to make equality behave as expected, we want the following formulas to be added implicitly to any theory:

  \begin{defenum}
    \item\label{note:first_order_equality/reflexivity} for any $x \in \Bold{Var}_{\Cal{L}}$, add the formula $(x \doteq x)$.
    \item\label{note:first_order_equality/equality} for any four variables $x_1, x_2, y_1, y_2$, add
    \begin{align*}
      ((x_1 \doteq y_1) \land (x_2 \doteq y_2)) \implies ((x_1 \doteq x_2) \iff (y_1 \doteq y_2)).
    \end{align*}

    \item\label{note:first_order_equality/functions} for any $n$-ary function $f$ and any set $\{ x_1, \ldots, x_n, y_1, \ldots, y_n \} \subseteq \Bold{Var}$, add
    \begin{align*}
      ((x_1 \doteq y_1) \land \ldots \land (x_n \doteq y_n)) \implies (f(x_1, \ldots, x_n) \doteq f(y_1, \ldots, y_n)).
    \end{align*}

    \item\label{note:first_order_equality/predicates} analogously, for any $n$-ary predicate $p$, add
    \begin{align*}
      ((x_1 \doteq y_1) \land \ldots \land (x_n \doteq y_n)) \implies (p(x_1, \ldots, x_n) \iff p(y_1, \ldots, y_n)).
    \end{align*}
  \end{defenum}

  In particular, this ensures that equality is an equivalence relation (see \cref{thm:first_order_equality_equivalence_relation}).
\end{note}

\begin{definition}\label{def:first_order_substition}
  Let $\varphi$ be a first-order formula with a free variable $y$ and $\rho$ be a term. We define the \ul{substitions}
  \begin{align*}
    \tau[y \to \rho] &\coloneqq \begin{cases}
      \rho,                                              &\tau = y, \\
      x,                                                 &\tau = x \in \Bold{Var}_{\Cal{L}} \setminus \{ y \}, \\
      f(\tau_1[y \to \rho], \ldots, \tau_n[y \to \rho]), &\tau = f(\tau_1, \ldots, \tau_n).
    \end{cases}
    \\
    \varphi[y \to \rho] &\coloneqq \begin{cases}
      \varphi,                                           &\varphi \in \{ \top, \bot \} \\
      p(\tau_1[y \to \rho], \ldots, \tau_n[y \to \rho]), &\varphi = p(\tau_1, \ldots, \tau_n) \\
      (\tau_1[y \to \rho] \doteq \tau_2[y \to \rho]),    &\varphi = (\tau_1 \doteq \tau_2), \\
      \neg \psi[y \to \rho],                             &\varphi = \neg \psi, \\
      \psi_1[y \to \rho] \circ \psi_2[x \to \rho],       &\varphi = \psi_1 \circ \psi_2, \circ \in \Sigma, \\
      Q x \psi[y \to \rho],                              &\varphi = Q x \psi, Q \in \{ \forall, \exists \}, x \not\in \Bold{Free}(\rho), \\
      Q x \psi[y \to \rho[x \to z]],                     &\varphi = Q x \psi, Q \in \{ \forall, \exists \}, x \in \Bold{Free}(\rho)
    \end{cases}
  \end{align*}
  where in the last step $z \in \Bold{Var} \setminus \Bold{Free}(\rho)$.

  We define \ul{simultaneous substition of $y_1, \ldots, y_n$ with $\rho_1, \ldots, \rho_n$} analogously to \cref{def:propositional_substition}.
\end{definition}

\begin{definition}\label{def:first_order_structure}\cite[definition 4.1]{Nerode2012}
  Fix a FOL language $\Cal{L}$. A \ul{structure} for $\Cal{L}$ consists of:
  \begin{enumerate}
    \item A nonempty set $A$.
    \item A binary relation\Tinyref{def:relation} $I(\doteq) subseteq A^2$ called the \ul{interpretation of the equality}.
    \item For every $n$-ary function symbol $f$, a function\Tinyref{def:function} $I(f): A^n \to A$ called the \ul{interpretation of $f$}.
    \item For every $n$-ary predicate $p$, an n-ary relation\Tinyref{def:relation} $I(p) \subseteq A^n$ called the \ul{interpretation of $p$}, i.e. all tuples of values that satisfy the predicate within the structure.
  \end{enumerate}
\end{definition}

\begin{definition}\label{def:first_order_evaluation}
  Fix a structure $\Cal{A} = (A, I)$ for a FOL language $\Cal{L}$. An \ul{evaluation for the variables of $\Cal{L}$} is any function $v: \Bold{Var}_{\Cal{L}} \to A$.

  For every variable $x$ and every universe element $a \in A$ we also define the \ul{modified at $x$ with $a$ evaluation}
  \begin{align*}
    v_a^x(y) \coloneqq \begin{cases}
      a,    &y = x, \\
      v(y), &y \neq x.
    \end{cases}
  \end{align*}

  Inductively,
  \begin{align*}
    v_{a_1, \ldots, a_n}^{x_1, \ldots, x_n}(y) \coloneqq ((v_{a_1}^{x_1})_{a_2}^{x_2})\cdots_{a_n}^{x_n}.
  \end{align*}

  This allows us to define semantics for all terms:
  \begin{align*}
    \tau[v] \coloneqq \begin{cases}
      v(x),                               &\tau = x \in \Bold{Var}_{\Cal{L}}, \\
      I(f)(\tau_1[v], \ldots, \tau_n[v]), &\tau = f(\tau_1, \ldots, \tau_n).
    \end{cases}
  \end{align*}
  and all formulas:
  \begin{align*}
    \varphi[v] \coloneqq \begin{cases}
      T,                                                &\varphi = \top, \\
      F,                                                &\varphi = \bot, \\
      (\tau_1[v], \tau_2[v]) \in I(\doteq),             &\varphi = (\tau_1 \doteq \tau_2), \\
      (\tau_1[v], \ldots, \tau_n[v]) \in I(p),          &\varphi = p(\tau_1, \ldots, \tau_n), \\
      H_\neg(\psi[v]),                                  &\varphi = \neg \psi, \\
      H_\circ(\psi_1[v], \psi_2[v]),                    &\varphi = \psi_1 \circ \psi_2, \circ \in \Sigma, \\
      \text{for all } a \in A, \psi[v_a^x] = T,         &\varphi = \forall x \psi, \\
      \text{there exists } a \in A: \psi[v_a^x] = T,    &\varphi = \exists x \psi.
    \end{cases}
  \end{align*}

  If $\varphi[v] = T$, we say that \ul{$\varphi$ is true in $\Cal{A}$ under the evaluation $v$} and we write $\Cal{A} \models_v \varphi$. If $\varphi$ is true in $\Cal{A}$ under every evaluation, we say that \ul{$\varphi$ is true or valid in $\Cal{A}$} and we write $\Cal{A} \models \varphi$.

  Given a formula $\varphi(x_1, \ldots, x_n)$, we write
  \begin{align*}
    \varphi(a_1, \ldots, a_n) \coloneqq \varphi(x_1, \ldots, x_n)[v_{a_1, \ldots, a_n}^{x_1, \ldots, x_n}].
  \end{align*}
\end{definition}

\begin{definition}\label{def:first_order_model}\cite[definition 4.4]{Nerode2012}
  A \ul{model} for a first-order theory $\Gamma$ in the FOL language $\Cal{L}$ is a structure $\Cal{A}$ such that there exists a single evaluation $v$ so that for every formula $\gamma \in \Gamma$, we have $\Cal{A} \models_v \gamma$. We write $\Cal{A} \models_v \Gamma$ or simply $\Cal{A} \models \Gamma$.
\end{definition}

\begin{definition}\label{def:first_order_consistent}
  A first-order theory is \ul{consistent} if, under any evaluation in any structure, every formula is either true or false.
\end{definition}

\begin{proposition}\label{thm:first_order_equality_equivalence_relation}
  In a FOL language with equality, the equality is an equivalence relation\Tinyref{def:order/equivalence}, that is, for any structure\Tinyref{def:first_order_structure} $\Cal{A}$, we have
  \begin{description}
    \DItem{reflexivity}{thm:first_order_equality_equivalence_relation/reflexivity} $\Cal{A} \models \forall x (x \doteq x)$
    \DItem{symmetry}{thm:first_order_equality_equivalence_relation/symmetry} $\Cal{A} \models \forall x \forall y ((x \doteq y) \iff (y \doteq x))$
    \DItem{transitivity}{thm:first_order_equality_equivalence_relation/transitivity} $\Cal{A} \models \forall x \forall y \forall z (((x \doteq y) \land (y \doteq x)) \implies (x \doteq z))$
  \end{description}
\end{proposition}
\begin{proof}
  Let $\Cal{A} = (A, I)$ be a structure and let $v: A \to \{ T, F \}$ be an evaluation function\Tinyref{def:first_order_evaluation}. Then

  \begin{description}
    \RItem{thm:first_order_equality_equivalence_relation/reflexivity} The evaluation $(\forall x (x \doteq x))[v]$ is true if and only if for every $a \in A$, we have
    \begin{align*}
      (x \doteq x)[v_x^a] = T.
    \end{align*}

    By \cref{note:first_order_equality/reflexivity}, $(y \doteq y)$ is an axiom for every $y \in \Bold{Var}_{\Cal{L}}$, hence \mbox{$(x \doteq x)[v_x^a] = T$} for all $a \in A$.We conclude that
    \begin{align*}
      \Cal{A} \models_v \forall x (x \doteq x).
    \end{align*}

    \RItem{def:order/equivalence/symmetry} Let $a, b \in A$ be arbitrary. Since $(x \doteq x)$ is an axiom for every $x \in \Bold{Var}$, from \cref{note:first_order_equality/equality} we obtain
    \begin{align*}
      T &=
      (((x \doteq x) \land (x \doteq y)) \implies ((x \doteq x) \iff (y \doteq x)))[v_{x,y}^{a,b}]
      = \\ &=
      H_\Rightarrow(H_\land((x \doteq x)[v_{x,y}^{a,b}], (x \doteq y)[v_{x,y}^{a,b}]), H_\Leftrightarrow((x \doteq x)[v_{x,y}^{a,b}], (y \doteq x)[v_{x,y}^{a,b}]))
      = \\ &=
      H_\Rightarrow(H_\land(T, (x \doteq y)[v_{x,y}^{a,b}]), H_\Leftrightarrow(T, (y \doteq x)[v_{x,y}^{a,b}]))
      = \\ &=
      H_\Leftrightarrow((x \doteq y)[v_{x,y}^{a,b}], (y \doteq x)[v_{x,y}^{a,b}])
      = \\ &=
      ((x \doteq y) \iff (y \doteq x))[v_{x,y}^{a,b}].
    \end{align*}

    Both $a$ and $b$ were arbitrary, hence
    \begin{align*}
      \Cal{A} \models_v \forall x \forall y ((x \doteq y) \iff (y \doteq x)).
    \end{align*}

    \RItem{def:order/equivalence/transitivity} Analogously to \ref{def:order/equivalence/symmetry}, let $a, b, c \in A$. From \cref{note:first_order_equality/equality} we obtain
    \begin{align*}
      T &=
      (((x \doteq y) \land (z \doteq y)) \implies ((x \doteq z) \iff (y \doteq y)))[v_{x,y,z}^{a,b,c}]
      = \\ &=
      H_\Rightarrow(H_\land((x \doteq y)[v_{x,y,z}^{a,b,c}], (z \doteq y)[v_{x,y,z}^{a,b,c}]), H_\Leftrightarrow((x \doteq z)[v_{x,y,z}^{a,b,c}], (y \doteq y)[v_{x,y,z}^{a,b,c}]))
      = \\ &=
      H_\Rightarrow(H_\land((x \doteq y)[v_{x,y,z}^{a,b,c}], (z \doteq y)[v_{x,y,z}^{a,b,c}]), H_\Leftrightarrow((x \doteq z)[v_{x,y,z}^{a,b,c}], T))
      = \\ &=
      H_\Rightarrow(H_\land((x \doteq y)[v_{x,y,z}^{a,b,c}], (z \doteq y)[v_{x,y,z}^{a,b,c}]), (x \doteq z)[v_{x,y,z}^{a,b,c}]))
      = \\ &=
      (((x \doteq y) \land (z \doteq y)) \implies (x \doteq z))[v_{x,y,z}^{a,b,c}].
    \end{align*}

    The values $a$, $b$ and $c$ were arbitrary, hence
    \begin{align*}
      \Cal{A} \models_v \forall x \forall y \forall z (((x \doteq y) \land (z \doteq y)) \implies (x \doteq z)).
    \end{align*}
  \end{description}
\end{proof}
