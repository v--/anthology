\subsection{First-order logic}\label{subsec:first_order_logic}

\begin{definition}\label{def:first_order_logic_language}\cite[187]{OpenLogic20201202}
  The idea of first-order predicate logic is to create a formal language whose semantics (given by structures) support boolean operations and can quantify over all elements of an ambient universum. Unlike in propositional logic\Tinyref{subsec:propositional_logic}, there are many first-order logic languages.

  The alphabet for a \Def{first-order predicate logic language}\Tinyref{def:language} \( \CL \) consists of:
  \begin{description}
    \item[Logical symbols]
    \mbox{}
    \begin{enumerate}
      \item A countable alphabet of variables \( \Bold{Var} \), usually denoted by \( \xi_0, \xi_1, \ldots \) or \( \xi, \eta, \zeta \).

      \item Certain propositional operations:
      \begin{description}
        \RItem{def:propositional_logic_language/constants} \( \top \) and \( \bot \) - zero-arity operations
        \RItem{def:propositional_logic_language/negation} \( \neg \) - unary operation
        \RItem{def:propositional_logic_language/connectives} \( \Sigma = \{ \land, \lor, \implies, \iff, \downarrow, \uparrow \} \) - binary operations
      \end{description}

      \item Quantifiers \( Q = ( \forall, \exists ) \)
      \begin{description}
        \DItem{def:propositional_logic_language/universal_quantifier}(universal quantifier) \( \forall \)
        \DItem{def:propositional_logic_language/existential_quantifier}(existential quantifier) \( \exists \)
      \end{description}

      \item Parentheses \( ( \) and \( ) \) for defining the order of operations unambiguously (see \fullref{remark:propositional_formula_parentheses}).

      \item Optionally, an equality symbol \( \doteq \). See \fullref{remark:first_order_equality}.
    \end{enumerate}

    \item[Non-logical symbols]
    \mbox{}
    \begin{enumerate}
      \item A set of functional symbols, \( \Cat{Func} \), whose elements are usually denoted by \( f_0, f_1, \ldots \) or \( f, g, h \). Each functional symbol has an associated natural number called its \Def{arity}, denoted by \( \# f \). Functional symbols with a zero arity are called \Def{constants}.

      \item A set of predicate symbols, \( \Cat{Pred} \), whose elements are usually denoted by \( p_0, p_1, \ldots \) or by symbols like \( \oplus \) or \( \circ \). Predicate symbols also have an associated arity. Predicate symbols with zero arity are called \Def{propositional variables}.
    \end{enumerate}
  \end{description}
\end{definition}

\begin{example}\label{ex:first_order_languages}
  Examples of first-order logic languages include
  \begin{itemize}
    \item \Fullref{def:peano_arithmetic} defines the Peano arithmetic
    \item \Fullref{def:algebraic_theory} defines different algebraic theories
  \end{itemize}
\end{example}

\begin{definition}\label{def:first_order_term}\cite[189]{OpenLogic20201202}
  Given a first-order logic language \( \CL \), the set \( \Cal{T} \) of terms is defined inductively as
  \begin{itemize}
    \item Each variable is a term
    \item If \( \tau_1, \ldots, \tau_n \) are terms and \( f \) is a functional symbol with arity \( n \), then the following word is also a term:
    \begin{equation*}
      f(\tau_1, \ldots, \tau_n)
    \end{equation*}
  \end{itemize}

  In particular, constants are terms.

  Furthermore, the grammar of first-order terms is unambiguous (see \fullref{thm:first_order_formulas_are_unambiguous}).

  For each term \( \tau \), we define its variables as
  \begin{align*}
    \Bold{Free}(\tau) \coloneqq \begin{cases}
      \xi,                                                      &\tau = \xi \in \Bold{Var}, \\
      \Bold{Free}(\tau_1) \cup \ldots \cup \Bold{Free}(\tau_n), &\tau = f(\tau_1, \ldots, \tau_n).
    \end{cases}
  \end{align*}
\end{definition}

\begin{definition}\label{def:first_order_formula}\cite[189]{OpenLogic20201202}
  Given a first-order logic language \( \CL \), we define the set of atomic formulas inductively as
  \begin{itemize}
    \item Both \( \top \) and \( \bot \) are atomic formulas.
    \item If \( p \) is an n-ary predicate symbol and if \( \tau_1, \ldots, \tau_n \) are terms, then \( p(\tau_1, \ldots, \tau_n) \) is an atomic formula.
    \item If \( \CL \) has an equality symbol and if \( \tau_1, \tau_2 \) are terms, then \( (\tau_1 \doteq \tau_2) \) is an atomic formula.
  \end{itemize}

  The set \( \Cal{F} \) of first-order formulas is then defined as
  \begin{itemize}
    \item All atomic formulas are formulas
    \item If \( \varphi \) is a formula, its negation \( \neg \varphi \) is also a formula
    \item If \( \varphi \) and \( \psi \) are formulas, then \( (\varphi \circ \psi), \circ \in \Sigma \)\Tinyref{def:propositional_logic_language}, are also formulas
    \item If \( \varphi \) is a formula and \( \xi \) is a variable, then the following are also formulas:
    \begin{itemize}
      \item \( \forall \xi \varphi \)
      \item \( \exists \xi \varphi \)
    \end{itemize}
  \end{itemize}

  Furthermore, the grammar of first-order formulas is unambiguous (see \fullref{thm:first_order_formulas_are_unambiguous}).

  For each formula \( \varphi \), we define its free and bound variables as
  \begin{align*}
    \Bold{Free}(\varphi) \coloneqq \begin{cases}
      \varnothing,                                             &\varphi \in \{ \top, \bot \} \\
      \Bold{Free}(\tau_1) \cup \ldots \cup \Bold{Free}(\tau_n), &\varphi = p(\tau_1, \ldots, \tau_n) \\
      \Bold{Free}(\tau_1) \cup \Bold{Free}(\tau_2),             &\varphi = (\tau_1 \doteq \tau_2), \\
      \Bold{Free}(\psi),                                        &\varphi = \neg \psi, \\
      \Bold{Free}(\psi_1) \cup \Bold{Free}(\psi_2),             &\varphi = \psi_1 \circ \psi_2, \circ \in \Sigma, \\
      \Bold{Free}(\psi) \setminus \{ \xi \},                    &\varphi = Q \xi \psi, Q \in \{ \forall, \exists \}
    \end{cases}
  \end{align*}
  and
  \begin{align*}
    \Cat{Bound}(\varphi) \coloneqq \begin{cases}
      \varnothing,                                             &\varphi \in \{ \top, \bot \} \\
      \varnothing,                                             &\varphi = p(\tau_1, \ldots, \tau_n) \\
      \varnothing,                                             &\varphi = (\tau_1 \doteq \tau_2), \\
      \Cat{Bound}(\psi),                                       &\varphi = \neg \psi, \\
      \Cat{Bound}(\psi_1) \cup \Bold{Bound}(\psi_2),           &\varphi = \psi_1 \circ \psi_2, \circ \in \Sigma, \\
      \Cat{Bound}(\psi) \cup \{ \xi \},                        &\varphi = Q \xi \psi, Q \in \{ \forall, \exists \}.
    \end{cases}
  \end{align*}

  A formula is called \Def{closed} if it has no bound variables.

  If a formula \( \varphi \) has free variables \( \Bold{Free}(\varphi) = \{ \xi_1, \ldots, \xi_n \} \), we write it as
  \begin{equation*}
    \varphi(\xi_1, \ldots, \xi_n).
  \end{equation*}

  This highlights that formulas with free variables can act as predicates, however their semantics\Tinyref{def:first_order_model} are completely determined by the actual predicates.
\end{definition}

\begin{proposition}\label{thm:first_order_formulas_are_unambiguous}
  The grammar\Tinyref{def:grammar}
  \begin{equation*}
    \begin{aligned}
      &\Theta \to v,                                          && v \in \Bold{Var} \\
      &\tau \to \Theta,                                       && \\
      &\tau \to f(\tau, \ldots, \tau),                        && f \in \Cat{Func} \text{ is an } n-\text{ary functional symbol} \\
      &\Phi \to \top \;|\; \bot                               && \\
      &\Phi \to p(\tau, \ldots, \tau),                        && p \in \Cat{Pred} \text{ is an } n-\text{ary predicate symbol} \\
      &\Phi \to (\tau \doteq \tau)                            && \\
      &\Phi \to \neg \Phi                                     && \\
      &\Phi \to (\Phi \circ \Phi),                            && \circ \in \Sigma \\
      &\Phi \to \forall \Theta \Phi \;|\; \exists \Theta \Phi && \\
    \end{aligned}
  \end{equation*}
  of first-order terms\Tinyref{def:first_order_term} and formulas\Tinyref{def:first_order_formula} is unambiguous\Tinyref{def:ambiguous_grammar}.
\end{proposition}
\begin{proof}
  The proof is more complicated but similar to \fullref{thm:propositional_formulas_are_unambiguous}.
\end{proof}

\begin{definition}\label{def:first_order_substition}
  Let \( \varphi \) be a first-order formula with a free variable \( \eta \) and \( \rho \) be a term. We define the \Def{substitions}
  \begin{align*}
    \tau[\eta \to \rho] &\coloneqq \begin{cases}
      \rho,                                                    &\tau = \eta, \\
      \xi,                                                     &\tau = \xi \in \Bold{Var} \setminus \{ \eta \}, \\
      f(\tau_1[\eta \to \rho], \ldots, \tau_n[\eta \to \rho]), &\tau = f(\tau_1, \ldots, \tau_n).
    \end{cases}
    \\
    \varphi[\eta \to \rho] &\coloneqq \begin{cases}
      \varphi,                                                 &\varphi \in \{ \top, \bot \} \\
      p(\tau_1[\eta \to \rho], \ldots, \tau_n[\eta \to \rho]), &\varphi = p(\tau_1, \ldots, \tau_n) \\
      (\tau_1[\eta \to \rho] \doteq \tau_2[\eta \to \rho]),    &\varphi = (\tau_1 \doteq \tau_2), \\
      \neg \psi[\eta \to \rho],                                &\varphi = \neg \psi, \\
      \psi_1[\eta \to \rho] \circ \psi_2[\xi \to \rho],          &\varphi = \psi_1 \circ \psi_2, \circ \in \Sigma, \\
      Q \xi \psi[\eta \to \rho],                               &\varphi = Q \xi \psi, Q \in \{ \forall, \exists \}, \xi \not\in \Bold{Free}(\rho), \\
      Q \xi \psi[\eta \to \rho[\xi \to \zeta]],                &\varphi = Q \xi \psi, Q \in \{ \forall, \exists \}, \xi \in \Bold{Free}(\rho)
    \end{cases}
  \end{align*}
  where in the last step \( \zeta \in \Bold{Var} \setminus \Bold{Free}(\rho) \).

  We define \Def{simultaneous substition of \( \eta_1, \ldots, \eta_n \) with \( \rho_1, \ldots, \rho_n \)} analogously to \fullref{def:propositional_substition}.
\end{definition}

\begin{definition}\label{def:first_order_structure}\cite[definition 14.27]{OpenLogic20201202}
  Fix a first-order logic language \( \CL \). A \Def{structure} \( \CA \) for \( \CL \) is a pair \( (A, I) \), where
  \begin{defenum}
    \DItem{def:first_order_structure/set} \( A \) is a nonempty set called the \Def{universum} of \( \CA \).

    \DItem{def:first_order_structure/interpretation} \( I \) is a map\Tinyref{def:function} called the \Def{interpretation} of \( \CA \) that we define partially as
    \begin{defenum}
      \DItem{def:first_order_structure/interpretation/equality} The interpretation \( I(\doteq) \) of the equality is a binary relation\Tinyref{def:binary_relation} on \( A \).

      \DItem{def:first_order_structure/interpretation/function} For every \( n \)-ary function symbol \( f \), its interpretation is a function\Tinyref{def:function} of type \( I(f): A^n \to A \).

      \DItem{def:first_order_structure/interpretation/predicate} For every \( n \)-ary predicate \( p \), its interpretation is a an n-ary relation\Tinyref{def:relation} \( I(p) \subseteq A^n \). This relation corresponds to all tuples of values that satisfy the predicate within the structure.
    \end{defenum}
  \end{defenum}
\end{definition}

\begin{definition}\label{def:first_order_substructure}
  If \( \CA = (A, I) \) is a structure and \( A' \subseteq A \), then we say that \( \CA' = (A', I) \) is a \Def{substructure} of \( \CA \) if it is closed under function application, that is, for any function symbol \( f \) with arity \( n \), we have \( I(f)(A'^n) \subseteq A' \).

  If \( A' \neq A \), we say that is is a \Def{proper substructure}.

  The family of all subalgebras of \( A \) can be ordered under set inclusion\Tinyref{def:subset}.

  A structure \( \Cal{O} \) is called a \Def{trivial} or \Def{zero} structure if it can be embedded homomorphically \Tinyref{def:first_order_homomorphism/embedding} into any structure for \( \CL \).
\end{definition}

\begin{definition}\label{def:first_order_variable_assignment}\cite[definition 14.32]{OpenLogic20201202}
  Fix a structure \( \CA = (A, I) \) for a first-order logic language \( \CL \). A \Def{variable assignment} for the variables of \( \CL \) is any function \( v: \Bold{Var} \to A \) (loosely similar to \fullref{def:propositional_interpretation}).

  For every variable \( \xi \) and every universum element \( x \in A \) we also define the \Def{modified assignment} at \( \xi \) with \( x \):
  \begin{align*}
    v_a^\xi(\eta) \coloneqq \begin{cases}
      a,    &\eta = \xi, \\
      v(\eta), &\eta \neq \xi.
    \end{cases}
  \end{align*}

  Inductively,
  \begin{equation*}
    v_{x_1, \ldots, x_n}^{\xi_1, \ldots, \xi_n}(\eta) \coloneqq ((v_{x_1}^{\xi_1})_{x_2}^{\xi_2})\cdots_{x_n}^{\xi_n}.
  \end{equation*}

  This allows us to define semantics for all terms:
  \begin{align*}
    \tau[v] \coloneqq \begin{cases}
      v(\xi),                               &\tau = \xi \in \Bold{Var}, \\
      I(f)(\tau_1[v], \ldots, \tau_n[v]),   &\tau = f(\tau_1, \ldots, \tau_n).
    \end{cases}
  \end{align*}
  and all formulas:
  \begin{align*}
    \varphi[v] \coloneqq \begin{cases}
      T,                                                &\varphi = \top, \\
      F,                                                &\varphi = \bot, \\
      (\tau_1[v], \tau_2[v]) \in I(\doteq),             &\varphi = (\tau_1 \doteq \tau_2), \\
      (\tau_1[v], \ldots, \tau_n[v]) \in I(p),          &\varphi = p(\tau_1, \ldots, \tau_n), \\
      H_\neg(\psi[v]),                                  &\varphi = \neg \psi, \\
      H_\circ(\psi_1[v], \psi_2[v]),                    &\varphi = \psi_1 \circ \psi_2, \circ \in \Sigma, \\
      \text{for all } x \in A, \psi[v_x^\xi] = T,       &\varphi = \forall \xi \psi, \\
      \text{there exists } x \in A: \psi[v_x^\xi] = T,  &\varphi = \exists \xi \psi.
    \end{cases}
  \end{align*}

  If \( \varphi[v] = T \), we say that \( \varphi \) is \Def{true under the assignment} \( v \) in \( \CA \) and we write \( \CA \models_v \varphi \). If \( \varphi \) is true in \( \CA \) under every variable assignment, we say that \( \varphi \) is \Def{true} or \Def{valid} in \( \CA \) and we write \( \CA \models \varphi \).

  Given a formula \( \varphi(\xi_1, \ldots, \xi_n) \), we write
  \begin{equation*}
    \varphi[x_1, \ldots, x_n] \coloneqq \varphi(\xi_1, \ldots, \xi_n)[v_{x_1, \ldots, x_n}^{\xi_1, \ldots, \xi_n}].
  \end{equation*}

  We also apply this notation for terms.
\end{definition}

\begin{definition}\label{def:first_order_model}\cite[14.35]{OpenLogic20201202}
  A \Def{model} for a first-order theory \( \Gamma \) in the first-order logic language \( \CL \) is a structure \( \CA \) such that there exists a single variable assignment \( v \) so that for every formula \( \gamma \in \Gamma \), we have \( \CA \models_v \gamma \). We write \( \CA \models_v \Gamma \) or simply \( \CA \models \Gamma \).

  If \( \varphi \) is any formula, we say that \( \Gamma \) \Def{entails} \( \varphi \) and write \( \Gamma \models \varphi \) if any model for \( \Gamma \) is a model for \( \varphi \).
\end{definition}

\begin{definition}\label{def:first_order_consistency}
  A first-order theory is \Def{consistent} if, under any variable assignment in any structure, every formula is either true or false.
\end{definition}

\begin{definition}\label{def:first_order_definability}
  Fix a first-order logic language and structure \( \CA = (A, I) \). We say that the set \( B \subseteq A^n \) is \Def{definable} using the formula \( \varphi(\xi_1, \ldots, \xi_n) \) if
  \begin{equation*}
    \varphi[x_1, \ldots, x_n] = T \text{ if and only if } (x_1, \ldots, x_n) \in B
  \end{equation*}
\end{definition}

\begin{definition}\label{def:first_order_equation}
  A first-order \Def{equation} is an equality proposition, i.e. a proposition of the form
  \begin{equation*}
    \tau(\xi_1, \ldots, \xi_n) \doteq \rho(\xi_1, \ldots, \xi_n),
  \end{equation*}
  where both \( \tau(\xi_1, \ldots, \xi_n) \) and \( \rho(\xi_1, \ldots, \xi_n) \) are terms with the same free variables.

  Given a structure \( \CA = (A, I) \), we call the elements of the set defined by this formula \Def{solutions}. That is, we say that the tuple \( (x_1, \ldots, x_n) \) is a solution to the equation if
  \begin{equation*}
    \tau[x_1, \ldots, x_n] = \rho[x_1, \ldots, x_n]
  \end{equation*}
\end{definition}

\begin{remark}\label{remark:equations}
  A remarkable portion of mathematics concerns the study of different types of equations (even though they are not usually restricted to equations in first-order logic):

  \begin{itemize}
    \item Matrix theory (see \fullref{subsec:matrices}) can be regarded as the study of linear equations.
    \item Differential equations (see \fullref{sec:diffeq}) is aptly named since it studies equations in functional spaces concerning functions and their derivatives.
    \item Roots of generalized derivatives (see \fullref{sec:nonsmooth_analysis}) are studied in optimization.
    \item Diophantine equations are studied in number theory (see \fullref{subsec:integers}).
    \item Fixed points of functions are studied in different branches of mathematics.
    \item Affine varieties (see \fullref{subsec:affine_varieties}) are studied in algebraic geometry.
  \end{itemize}
\end{remark}

\begin{definition}\label{def:first_order_homomorphism}\cite[definition 23.8]{OpenLogic20201202}
  Let \( \CA = (A, I_{\CA}) \) and \( \Cal{B} = (B, I_{\Cal{B}}) \) be structures over a common language. We say that the function\Tinyref{def:function} \( \varphi: A \to B \) is a \Def{homomorphism} between \( \CA \) and \( \Cal{B} \) if it preserves all functions and relations, that is, for any \( f \in \Cat{Func} \) and any \( x_1, \ldots, x_{\#f} \in A \),
  \begin{equation*}
    \varphi(f_{\CA}[x_1, \ldots, x_{\#f}]) = f_{\Cal{B}}[\varphi(x_1), \ldots, \varphi(x_{\#f})]
  \end{equation*}
  and for any \( p \in \Cat{Pred} \) and any \( x_1, \ldots, x_{\#p} \in A \),
  \begin{equation*}
    (x_1, \ldots, x_{\#p}) \in p_{\CA} \text{ if and only if } (\varphi(x_1), \ldots, \varphi(x_{\#p})) \in p_{\Cal{B}}.
  \end{equation*}

  We say that the homomorphism \( \varphi: A \to B \) is
  \begin{defenum}
    \DItem{def:first_order_homomorphism/embedding} an \Def{embedding} or \Def{monomorphism} if \( \varphi \) is an injective function\Tinyref{def:function_invertibility/injection}.

    \DItem{def:first_order_homomorphism/projection} a \Def{projection} or \Def{epimorphism} if \( \varphi \) is a surjective function\Tinyref{def:function_invertibility/surjection}.

    \DItem{def:first_order_homomorphism/isomorphism} an \Def{isomorphism} if \( \varphi \) is a bijective function\Tinyref{def:function_invertibility/bijection}.

    \DItem{def:first_order_homomorphism/endomorphism} an \Def{endomorphism} if \( A = B \) and an \Def{automorphism} if \( \varphi \) is a bijective endomorphism.
  \end{defenum}

  Homomorphism between first-order structures are a direct generalization of homomorphisms in algebra (see \fullref{def:algebraic_theory}) and the terminology in \fullref{def:morphism_invertibility} is inspired by homomorphisms.
\end{definition}

\begin{proposition}\label{thm:first_order_homomorphism_properties}
  The following are some basic properties of structure homomorphisms\Tinyref{def:first_order_homomorphism}:
  \begin{defenum}
    \DItem{thm:first_order_homomorphism_properties/substructure} If \( \CA = (A, I) \) is a structure and \( \Cal{A'} = (A', I) \) is a substructure\Tinyref{def:first_order_substructure} of \( \CA \), then the projection \( \pi: A' \to A \) is an epimorphism\Tinyref{def:first_order_homomorphism/projection}.

    \DItem{thm:first_order_homomorphism_properties/preserves_substructures} If \( \CA = (A, I_{\CA}) \) and \( \Cal{B} = (B, I_{\Cal{B}}) \) are structures and \( \varphi: A \to B \) is a homomorphism, then the image\Tinyref{def:function} \( \varphi(\CA) = (\varphi(A), I_{\Cal{B}}) \) is a substructure of \( \Cal{B} \).

    \DItem{thm:first_order_homomorphism_properties/composition} The composition\Tinyref{def:function/composition} of two homomorphisms is again a homomorphism.
  \end{defenum}
\end{proposition}

\begin{proposition}\label{thm:first_order_homomorphism_preserves_models}\cite[definition 23.8]{OpenLogic20201202}
  Let \( \CA = (A, I_{\CA}) \) and \( \Cal{B} = (B, I_{\Cal{B}}) \) be structures over a common language. Let \( \varphi: A \to B \) be a homomorphism between them. Let \( \Gamma \) be a set of formulas.

  If \( \CA \models \Gamma \), then its image \( \varphi(\CA) \models \Gamma \).
\end{proposition}

\begin{definition}\label{def:first_order_theory}
  We define closure of sets of formulas and axiomatic theories analogously to \fullref{def:propositional_theory}.
\end{definition}

\begin{remark}\label{remark:minimal_first_order_language}
  As in \fullref{remark:minimal_propositional_language}, to avoid redundancy in definitions and proofs, we can use the Pierce arrow \( \downarrow \) to define the constants, negation and binary connectives by adding additional predicates as axioms.
\end{remark}

\begin{remark}\label{remark:first_order_equality}
  Equality is a concept that implies that two objects are completely indistinguishable. It is usually assumed to be part of the language. Examples of languages without formal equality are obscure. The usual example is \fullref{def:set_zfc}, where instead of having equality as part of the language, we can define equality as a formula schema based on the membership predicate and \fullref{def:set_zfc/A1}.

  Let \( \CL \) be a first-order logic language with an equality symbol. In order to make equality behave as expected, we want the following formulas to be added implicitly to every axiomatic theory (see \fullref{def:first_order_theory}):

  \begin{defenum}
    \DItem{remark:first_order_equality/reflexive} for any \( \xi \in \Bold{Var} \), add the formula \( (\xi \doteq \xi) \).
    \DItem{remark:first_order_equality/equality} for any four variables \( \xi_1, \xi_2, \eta_1, \eta_2 \), add
    \begin{equation*}
      ((\xi_1 \doteq \eta_1) \land (\xi_2 \doteq \eta_2)) \implies ((\xi_1 \doteq \xi_2) \iff (\eta_1 \doteq \eta_2)).
    \end{equation*}

    \DItem{remark:first_order_equality/functions} for any \( n \)-ary function \( f \) and any set \( \{ \xi_1, \ldots, \xi_n, \eta_1, \ldots, \eta_n \} \subseteq \Bold{Var} \), add
    \begin{equation*}
      ((\xi_1 \doteq \eta_1) \land \ldots \land (\xi_n \doteq \eta_n)) \implies (f(\xi_1, \ldots, \xi_n) \doteq f(\eta_1, \ldots, \eta_n)).
    \end{equation*}

    \DItem{remark:first_order_equality/predicates} analogously, for any \( n \)-ary predicate \( p \), add
    \begin{equation*}
      ((\xi_1 \doteq \eta_1) \land \ldots \land (\xi_n \doteq \eta_n)) \implies (p(\xi_1, \ldots, \xi_n) \iff p(\eta_1, \ldots, \eta_n)).
    \end{equation*}
  \end{defenum}

  In particular, this ensures that equality is an equivalence relation (see \fullref{thm:first_order_equality_is_equivalence_relation}).
\end{remark}

\begin{proposition}\label{thm:first_order_equality_is_equivalence_relation}
  In a first-order logic language with equality, the equality is an equivalence relation\Tinyref{def:equivalence_relation}, that is, for any structure\Tinyref{def:first_order_structure} \( \CA \), we have
  \begin{description}
    \DItem{thm:first_order_equality_is_equivalence_relation/reflexive}(reflexivity) \( \CA \models \forall \xi (\xi \doteq \xi) \)
    \DItem{thm:first_order_equality_is_equivalence_relation/symmetric}(symmetry) \( \CA \models \forall \xi \forall \eta ((\xi \doteq \eta) \iff (\eta \doteq \xi)) \)
    \DItem{thm:first_order_equality_is_equivalence_relation/transitive}(transitivity) \( \CA \models \forall \xi \forall \eta \forall \zeta (((\xi \doteq \eta) \land (\eta \doteq \xi)) \implies (\xi \doteq \zeta)) \)
  \end{description}
\end{proposition}
\begin{proof}
  Let \( \CA = (A, I) \) be a structure and let \( v: A \to \{ T, F \} \) be an variable assignment function\Tinyref{def:first_order_variable_assignment}. Then

  \begin{description}
    \RItem{thm:first_order_equality_is_equivalence_relation/reflexive} The variable assignment \( (\forall \xi (\xi \doteq \xi))[v] \) is true if and only if for every \( a \in A \), we have
    \begin{equation*}
      (\xi \doteq \xi)[v_x^a] = T.
    \end{equation*}

    By \fullref{remark:first_order_equality/reflexive}, \( (\eta \doteq \eta) \) is an axiom for every \( \eta \in \Bold{Var} \), hence \mbox{\( (\xi \doteq \xi)[v_x^a] = T \)} for all \( a \in A \).We conclude that
    \begin{equation*}
      \CA \models_v \forall \xi (\xi \doteq \xi).
    \end{equation*}

    \RItem{thm:first_order_equality_is_equivalence_relation/symmetric} Let \( a, b \in A \) be arbitrary. Since \( (\xi \doteq \xi) \) is an axiom for every \( \xi \in \Bold{Var} \), from \fullref{remark:first_order_equality/equality} we obtain
    \begin{align*}
      T &=
      (((\xi \doteq \xi) \land (\xi \doteq \eta)) \implies ((\xi \doteq \xi) \iff (\eta \doteq \xi)))[v_{\xi,\eta}^{a,b}]
      = \\ &=
      H_\Rightarrow(H_\land((\xi \doteq \xi)[v_{\xi,\eta}^{a,b}], (\xi \doteq \eta)[v_{\xi,\eta}^{a,b}]), H_\Leftrightarrow((\xi \doteq \xi)[v_{\xi,\eta}^{a,b}], (\eta \doteq \xi)[v_{\xi,\eta}^{a,b}]))
      = \\ &=
      H_\Rightarrow(H_\land(T, (\xi \doteq \eta)[v_{\xi,\eta}^{a,b}]), H_\Leftrightarrow(T, (\eta \doteq \xi)[v_{\xi,\eta}^{a,b}]))
      = \\ &=
      H_\Leftrightarrow((\xi \doteq \eta)[v_{\xi,\eta}^{a,b}], (\eta \doteq \xi)[v_{\xi,\eta}^{a,b}])
      = \\ &=
      ((\xi \doteq \eta) \iff (\eta \doteq \xi))[v_{\xi,\eta}^{a,b}].
    \end{align*}

    Both \( a \) and \( b \) were arbitrary, hence
    \begin{equation*}
      \CA \models_v \forall \xi \forall \eta ((\xi \doteq \eta) \iff (\eta \doteq \xi)).
    \end{equation*}

    \RItem{thm:first_order_equality_is_equivalence_relation/transitive} Analogously to \fullref{def:binary_relation/symmetric}, let \( a, b, c \in A \). From \fullref{remark:first_order_equality/equality} we obtain
    \begin{align*}
      T &=
      (((\xi \doteq \eta) \land (\zeta \doteq \eta)) \implies ((\xi \doteq \zeta) \iff (\eta \doteq \eta)))[v_{\xi,\eta,\zeta}^{a,b,c}]
      = \\ &=
      H_\Rightarrow(H_\land((\xi \doteq \eta)[v_{\xi,\eta,\zeta}^{a,b,c}], (\zeta \doteq \eta)[v_{\xi,\eta,\zeta}^{a,b,c}]), H_\Leftrightarrow((\xi \doteq \zeta)[v_{\xi,\eta,\zeta}^{a,b,c}], (\eta \doteq \eta)[v_{\xi,\eta,\zeta}^{a,b,c}]))
      = \\ &=
      H_\Rightarrow(H_\land((\xi \doteq \eta)[v_{\xi,\eta,\zeta}^{a,b,c}], (\zeta \doteq \eta)[v_{\xi,\eta,\zeta}^{a,b,c}]), H_\Leftrightarrow((\xi \doteq \zeta)[v_{\xi,\eta,\zeta}^{a,b,c}], T))
      = \\ &=
      H_\Rightarrow(H_\land((\xi \doteq \eta)[v_{\xi,\eta,\zeta}^{a,b,c}], (\zeta \doteq \eta)[v_{\xi,\eta,\zeta}^{a,b,c}]), (\xi \doteq \zeta)[v_{\xi,\eta,\zeta}^{a,b,c}]))
      = \\ &=
      (((\xi \doteq \eta) \land (\zeta \doteq \eta)) \implies (\xi \doteq \zeta))[v_{\xi,\eta,\zeta}^{a,b,c}].
    \end{align*}

    The values \( a \), \( b \) and \( c \) were arbitrary, hence
    \begin{equation*}
      \CA \models_v \forall \xi \forall \eta \forall \zeta (((\xi \doteq \eta) \land (\zeta \doteq \eta)) \implies (\xi \doteq \zeta)).
    \end{equation*}
  \end{description}
\end{proof}

\begin{definition}\label{def:first_order_model_category}
  Let \( \CL \) be a first-order logic language\Tinyref{def:first_order_logic_language} and let \( \Gamma \) be a closed set of formulas\Tinyref{def:first_order_theory}. We denote by \( \Cat{Model}_\Gamma \) the category\Tinyref{def:category} where
  \begin{itemize}
    \item the class\Tinyref{def:set_zfc} of objects is the class of all models for \( \Gamma \)\Tinyref{def:first_order_model}.
    \item the morphisms between two models are the homomorphisms between them.
  \end{itemize}

  We define the forgetful functor
  \begin{align*}
    &U: \Cat{Model}_\Gamma \to \Cat{Set} \\
    &U((A, I)) \coloneqq A \\
    &U(f: A \to B) \coloneqq f.
  \end{align*}

  The image \( U(\Cat{Model}_\Gamma) \) is a subcategory of \( \Cat{Set} \), which implies that \( \Cat{Model}_\Gamma \) is locally small\Tinyref{def:category_cardinality}.
\end{definition}
\begin{proof}
  \Fullref{thm:first_order_homomorphism_properties/composition} shows that the composition of two homomorphisms is indeed a homomorphism and \fullref{thm:first_order_homomorphism_preserves_models} shows that the image under a homomorphism of a model for \( \Gamma \) is again a model for \( \Gamma \).
\end{proof}

\begin{remark}\label{remark:induction}
  Assume that we need to prove a statement about natural numbers\Tinyref{def:natural_numbers} or, more generally, a statement of the form \( \forall x \varphi(x) \) in a structure \( \CA = (A, I) \).

  We can infer \( \varphi(x) \) for every \( x \in A \) directly by using some property of \( A \). For example, when proving the equivalences in \fullref{def:totally_bounded_set}, we assume that every \( x \in A \) has certain properties and without further assumptions we proceed to prove that each \( x \in A \) has property \( \varphi(x) \). This inference is called \Def{deduction} because we deduce \( \varphi(x) \) from some properties that hold for the entirety of \( A \).

  Another type of inference is called \Def{induction}, where we prove the statement \( \varphi(x) \) by inspecting every \( x \in A \) and proving the statement separately. This becomes cumbersome when \( A \) is a large enough finite set and impossible when \( A \) is an infinite set.

  There are, however, special cases where \( A \) can be \enquote{exhausted} using a simple procedure. This method of exhausting is called an \Def{inductive definition}.

  For example, \fullref{def:peano_arithmetic} shows us how the set of natural numbers can be exhausted by starting from some element \( 0 \in \BN \) and then defining \( 1 \coloneqq s(0), 2 \coloneqq s(1) \) etc. We cannot prove a statement about the natural numbers by inspecting every natural number, however we can exhaust the entire set \( \BN \) as follows:
  \begin{itemize}
    \item We prove that the formula \( \varphi(0) \) holds in \( \CA \).
    \item We assume that \( \varphi(n) \) holds for some \( n \in \BN \) and prove \( \varphi(n + 1) \). Equivalently, we prove that the formula
    \begin{equation*}
      \forall n (\varphi(n) \implies \varphi(n + 1))
    \end{equation*}
    holds in \( \BN \).
  \end{itemize}

  The axiom schema of induction, \fullref{def:peano_arithmetic/PA3}, guarantees that we have now proved \( \varphi(n) \) for all \( n \in \BN \). Thus \fullref{def:peano_arithmetic} gives us an inductive definition for \( \BN \) and we can use induction to prove properties of \( \BN \).

  Another form of induction on \( \BN \) is
  \begin{equation*}
    \forall n (\forall k (k < n \implies \varphi(k)) \implies \varphi(n)).
  \end{equation*}

  This is called \Def{strong induction}. Here we only have a single clause that says \enquote{for each natural number \( n \), if \( \varphi \) holds for all \( k < n \), then it holds for \( \varphi \)}.

  We can \enquote{exhaust} more complicated sets. An inductive definition usually concern expressions in formal languages (strings of symbols in the sense of \fullref{def:language}). Every clause in an inductive definition for the set \( X \) is either
  \begin{itemize}
    \item A subset clause that says \( Y \subseteq X \) for some fixed subset \( Y \subseteq A \) of the universum. This is the \Def{base} of the induction.
    \item An implication\Tinyref{def:material_implication} clause which says \enquote{if \( x_1, \ldots, x_n \in X \), then \( f(x_1, \ldots, x_n) \in X \)}, where \( f \) is some function of type \( f: A^n \to A \).
  \end{itemize}

  It is conventional to add a clause \enquote{there are no other members of \( X \)} to an inductive definition of the set \( X \). We assume, however, that an inductive definition of \( X \) exhausts all of \( X \) so we avoid this practice.

  To prove a property \( \varphi(x) \) for \( X \),
  \begin{itemize}
    \item For every subset clause \( Y \subseteq X \), we prove that \( \varphi(y) \) for all \( y \in Y \).
    \item We then prove every implication clause directly, that is, we prove the formula
    \begin{equation*}
      \forall x_1 \ldots \forall x_n (\varphi(x_1) \land \ldots \land \varphi(x_n) \implies \varphi(f(x_1, \ldots, x_n))).
    \end{equation*}
  \end{itemize}

  For example, consider the following definition for the set \( E \) of arithmetic expressions over \( \BN \):
  \begin{itemize}
    \item If \( n \in \BN \), then \( n \in E \).
    \item If \( \xi, \eta \in E \), then \( (\xi + \eta) \in E \)
    \item If \( \xi, \eta \in E \), then \( (\xi \cdot \eta) \in E \)
  \end{itemize}

  If we want to prove that every arithmetic expression has an odd number of symbols, we proceed as follows:
  \begin{itemize}
    \item If \( n \in \BN \), the expression has length \( 1 \).
    \item Assume that \( \zeta = (\xi + \eta) \) and also that \( \xi \) and \( \eta \) have odd length. Since \( \Len(\zeta) = \Len(\xi) + \Len(\eta) + 3 \), we conclude that \( \zeta \) also has an odd number of symbols.
    \item Analogous.
  \end{itemize}

  We say that the proof uses \Def{mathematical induction on \( X \)} and that it is an \Def{inductive proof}. In the case where \( X \neq \BN \) we also call it a \Def{proof by structural induction}.

  We can also inductively define relations\Tinyref{def:relation} and functions\Tinyref{def:function}, for example \fullref{def:propositional_interpretation}. We can define operations in \( E \) based on \fullref{def:natural_number_operations}.

  Note that in order for an inductively defined function to be well-defined, in analogy to unambiguous grammars\Tinyref{def:ambiguous_grammar}, we usually want to be able to generate every element of \( A \) in only one way. This is the reason for theorems like \fullref{thm:propositional_formulas_are_unambiguous} and \fullref{thm:first_order_formulas_are_unambiguous}. \Fullref{ex:context_free_grammar/real_arithmetic} demonstrates how removing parenthesis in arithmetic expressions can introduce ambiguity.

  An infamous example of the consequences of ambiguity is the expression \( 6 / 2 \cdot 3 \) that can be interpreted as either \( \tfrac 6 2 \cdot 3 = 9 \) or \( \tfrac 6 {2 \cdot 3} = 1 \).

  See \fullref{sec:index} for a list of inductive definitions and proofs.
\end{remark}
