\subsection{Continuous functions}\label{subsec:continuous_functions}

\begin{definition}\label{def:continuous_function}
  Let \( (X, \Cal{T}) \) and \( (Y, \Cal{O}) \) be topological spaces\Tinyref{def:topological_space}. We say that the function\Tinyref{def:function} \( f: X \to Y \) is continuous if any of the equivalent conditions hold:
  \begin{defenum}
    \DItem{def:continuous_function/direct} For every open set \( V \in \Cal{O} \), the preimage\Tinyref{def:function_preimage} \( f^{-1}(V) \) is open.
    \DItem{def:continuous_function/closed} For every closed set \( V \in \Cal{F}_{\Cal{O}} \), the preimage \( f^{-1}(V) \) is closed.
    \DItem{def:continuous_function/base} There exists a base\Tinyref{def:topological_base} \( \Cal{B}_{\Cal{O}} \subseteq \Cal{O} \), such that for every \( V \in \Cal{B}_{\Cal{O}} \), the preimage \( f^{-1}(V) \) is open.
    \DItem{def:continuous_function/subbase} There exists a subbase\Tinyref{def:topological_subbase} \( \Cal{P}_{\Cal{O}} \subseteq \Cal{O} \), such that for every \( V \in \Cal{P}_{\Cal{O}} \), the preimage \( f^{-1}(V) \) is open.
    \DItem{def:continuous_function/local_base} There exist neighborhood systems\Tinyref{def:topological_local_base} \( \{ \Cal{B}_{\Cal{T}}(x) \colon x \in X \} \) and \( \{ \Cal{B}_{\Cal{O}}(y) \colon y \in Y \} \), such that for every point \( x \in X \) and for any \( V \in \Cal{B}_{\Cal{O}}(f(x)) \), there exists a set \( U \in \Cal{B}_{\Cal{O}}(x) \) such that \( f(U) \subseteq V \).
    \DItem{def:continuous_function/closure} For every set \( A \subseteq X \), \( f(\Cl(A)) \subseteq \Cl(f(A)) \).
    \DItem{def:continuous_function/limits} For every net\Tinyref{def:topological_net} \( \{ x_i \}_{i \in I} \), we have
    \begin{equation*}
      f\left(\lim_{i \in I} x_i \right) \subseteq \lim_{i \in I} f(x_i).
    \end{equation*}
  \end{defenum}
\end{definition}

\begin{definition}\label{def:homeomorphism}
  We say that the continuous function \( f: (X, \Cal{T}) \to (Y, \Cal{O}) \) is a \textbf{open} (resp. \textbf{closed}), if the image \( f(U) \) of an open set (resp. closed set) in \( \Cal{T} \) is open (resp. closed) in \( \Cal{O} \).

  If \( f \) is an open bijection, we say that \( f \) is a \textbf{homeomorphism}. If \( f \) is only an open injection, we say that \( f \) is a \textbf{homeomorphic embedding}.
\end{definition}
