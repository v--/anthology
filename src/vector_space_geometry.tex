\subsection{Vector space geometry}\label{subsec:vector_space_geometry}

\begin{remark}\label{remark:real_field_extensions}
  When speaking about vector spaces, we usually restrict ourselves to vector spaces over \( \BR \) or, at most, \( \BC \). This restriction may seem arbitrary, however important concepts like \hyperref[def:geometric_ray]{rays} or \hyperref[def:convex_set]{convexity} requires the field to be an extension of \( \BR \) and it just so happens that, by \fullref{thm:fundamental_theorem_of_algebra} and \fullref{thm:no_finite_extensions_of_closed_fields}, the only finite extension \hyperref[def:field_extension]{fields} of \( \BR \) are \( \BR \) and \( \BC \). It is technically possible to work with infinite extension fields, however in practice vector spaces over \( \BC \) are isoteric enough. A benefit of considering only \( \BC \) is given in \fullref{remark:linear_functionals_over_c}.
\end{remark}

\begin{definition}\label{def:geometric_shape}
  A \Def{geometric shape} is an informal notion that refers to certain special subsets of a vector space, usually defined in a coordinate-independent manner. Shapes in two-dimensional spaces are called \Def{figures} and shapes in three dimensions are called \Def{surfaces}.

  When two geometric shapes shapes intersect, we say that they are \Def{incident}.
\end{definition}

\begin{remark}\label{remark:vector_space_set_operations}
  It is customary to perform addition and scalar multiplication with sets in vector spaces. That is, if \( A \) and \( B \) are sets, it is customary to write
  \begin{align*}
    A + B \coloneqq \{ a + b \colon a \in A, b \in B \},
  \end{align*}
  and if \( \alpha \) is a scalar,
  \begin{equation*}
    \alpha A \coloneqq \{ \alpha a \colon a \in A \}.
  \end{equation*}

  If we use the convention in \fullref{remark:singleton_sets}, this turns sets in vector spaces into an ill-defined algebraic structure, but it is useful when no metric is available, e.g. in \hyperref[def:topological_vector_space]{topological vector spaces} or set-valued analysis.
\end{remark}

\begin{definition}\label{def:point}
  A \Def{point} is a simple geometric \hyperref[def:geometric_shape]{shape} comprising a singleton subset of any set (usually a vector space or a topological space). We use the convention \fullref{remark:singleton_sets} and, unless the distinction is important, we do not distinguish between singleton sets and their only element - e.g. in \fullref{def:simplex/point}.

  Points are also called vectors, which is justified by \fullref{def:euclidean_plane_coordinate_system}.
\end{definition}

\begin{definition}\label{def:zero_locus}
  Let \( S \) be an arbitrary set and let \( M \) be a \hyperref[def:algebraic_theory/identity]{unital} \hyperref[def:magma/magma]{magma} with identity \( e \).

  The \Def{zero locus} or \Def{set of zeros} of a function \( f: S \to M \) is the preimage
  \begin{equation*}
    f^{-1}(e) = \{ x \in X \colon f(x) = e \}.
  \end{equation*}

  In practice, \( M \) is usually a \hyperref[def:semiring/ring]{ring} or a \hyperref[def:left_module]{module}, in which case the zero locus is defined for the additive group, i.e.
  \begin{equation*}
    f^{-1}(0_M) = \{ x \in X \colon f(x) = 0_M \}.
  \end{equation*}

  See also \fullref{def:unital_magma_kernel}, \fullref{def:semiring_kernel} and \fullref{def:left_module_kernel}.
\end{definition}

\begin{definition}\label{def:hypersurface}
  A \Def{hypersurfaces} can have different meanings depending on the context. We are interested in

  \begin{defenum}
    \DItem{def:hypersurface/parametric} A parametric hypersurface (\fullref{def:parametric_hypersurface}) is a purely topological definition.
    \DItem{def:hypersurface/algebraic} An affine variety (\fullref{def:affine_variety}) is a purely algebraic definition.
    \DItem{def:hypersurface/geometric} A manifold (\fullref{def:topological_manifold}) can be regarded as a geometric definition.
  \end{defenum}

  Note that all of the enumerated hypersurfaces have a concept of dimension. Hypersurfaces of dimension \( 2 \) are simply called \Def{surfaces} (see \fullref{def:affine_variety/algebraic_surface}) and hypersurfaces of dimension \( 1 \) are called \Def{curves} (see also \fullref{def:affine_variety/algebraic_curve} and \fullref{def:affine_variety/algebraic_curve}).
\end{definition}

\begin{definition}\label{def:geometric_line}
  A particularly important \hyperref[def:hypersurface]{curve} is a \Def{line} in a vector space \( X \) over any field \( \BK \), which can be defined equivalently as

  \begin{defenum}
    \DItem{def:geometric_line/subspace} A subspace of \( X \) of \hyperref[def:vector_space_dimension]{dimension} one. Note that this is not consistent with the other definitions because this defines only lines through the origin \( 0_X \). Hence if \( L \subseteq X \) is a line (a subspace of dimension one) and if \( a \in X \) is any point, we define a line with origin \( a \) to be the translation \( a + L \).

    \DItem{def:geometric_line/algebraic} An \hyperref[def:affine_variety/algebraic_curve]{algebraic curve} in \( X \) given by a polynomial of degree one.

    \DItem{def:geometric_line/parametric} If the field \( \BK \) is ordered (usually when \( \BK = \BR \)), we can define a line with \Def{directional vector} \( x \) and \Def{origin} \( a \) as the parametric curve
    \begin{align*}
      &l: \BK \to X \\
      &l(t) = tx + a.
    \end{align*}
  \end{defenum}
\end{definition}

\begin{definition}\label{def:geometric_ray}
  If \( X \) is a vector space over \( \BK \in \{ \BR, \BC \} \), we define \Def{closed rays} with a vertex \( a \) as the parametric curves
  \begin{align*}
    &l^+: [0, \infty) \to X \\
    &l^+(t) = tx + a
  \end{align*}
  and
  \begin{align*}
    &l^-: (-\infty, 0] \to X \\
    &l^-(t) = tx + a
  \end{align*}

  If the inequalities are strict, we instead obtain \Def{open rays}.

  Unless explicitly noted otherwise, we assume that the vertex of the ray is \( 0 \) because every ray is a translation of a ray centered at \( 0 \).
\end{definition}

\begin{definition}\label{def:geometric_cone}
  An open (resp. closed) \Def{cone} in a vector space over \( \BK \in \{ \BR, \BC \} \) is a union of open (resp. closed) \hyperref[def:geometric_ray]{rays} with a common vertex, called the \Def{vertex} of the cone.
\end{definition}

\begin{definition}\label{def:hyperplane}
  Dually to \hyperref[def:geometric_line]{lines}, another particularly important \hyperref[def:hypersurface]{hypersurface} is a \Def{hyperplane} in a vector space \( X \) over any field.

  \begin{defenum}
    \DItem{def:hyperplane/subspace} A \Def{linear hyperplane} is simply a subspace of \( X \) of \hyperref[def:vector_space_dimension]{codimension} one. As in \fullref{def:geometric_line/subspace}, we define a \Def{affine hyperplane} to be a translation of a linear hyperplane.

    \DItem{def:hyperplane/kernel} Linear hyperplanes (as defined in \fullref{def:hyperplane/subspace}) are simply zero \hyperref[def:zero_locus]{loci} (\hyperref[def:semiring_kernel]{kernels}) of linear \hyperref[def:linear_operator]{functionals}. and affine hyperplanes are zero loci of \hyperref[def:affine_operator]{affine functionals}.
  \end{defenum}
\end{definition}

\begin{example}\label{ex:hyperplanes}
  Affine \hyperref[def:hyperplane]{hyperplanes} in \( \BR^2 \) are \hyperref[def:geometric_line]{lines} and affine hyperplanes in \( \BR^3 \) are planes.

  Linear \hyperref[def:hyperplane]{hyperplanes} in \( \BR^2 \) are the lines passing through the origin \( (0, 0) \) and linear hyperplanes in \( \BR^3 \) are the planes incident to \( (0, 0, 0) \).
\end{example}

\begin{definition}\label{def:half_space}
  Vector spaces over \( \BR \) have the concept of \Def{half-spaces}. Given a \hyperref[def:hyperplane]{hyperplane} \( H \) of the real vector space \( X \), defined by the affine functional \( l(x) = \Prod {x^*} x + a \), its closed half-spaces are defined as
  \begin{equation*}
    H^+ \coloneqq \{ l(x) \geq 0 \} = \{ \Prod {x^*} x \geq -a \}
  \end{equation*}
  and
  \begin{equation*}
    H^- \coloneqq \{ l(x) \leq 0 \} = \{ \Prod {x^*} x \leq -a \}.
  \end{equation*}

  If the inequalities are strict, we instead obtain \Def{open half-spaces}.
\end{definition}

\begin{definition}\label{def:polyhedron}
  A \Def{polyhedron} in a real vector space is an intersection of half-\hyperref[def:half_space]{spaces}.
\end{definition}

\begin{definition}\label{def:hyperplane_separation}
  Although \hyperref[def:hyperplane]{hyperplanes} are defined for vector spaces over an arbitrary field \( \BK \), we define \Def{hyperplane separation} only for \( \BK \in \{ \BR, \BC \} \) (see \fullref{remark:real_field_extensions}).

  We say that the sets \( A, B \subseteq X \) are \Def{separated} by the linear functional \( l \in X^* \) if there exists a real number \( c \in \BR \) such that
  \begin{equation}\label{def:hyperplane_separation/normal}
    \Re l(x) < c \leq \Re l(y) \quad\forall x \in A, y \in B.
  \end{equation}

  See \fullref{remark:linear_functionals_over_c} for a justification of only considering the real part of \( l \).

  The asymmetry in the inequalities \fullref{def:hyperplane_separation/normal} can be inverted by considering \( -l(x) \) and \( -c \).

  We say that \( A \) and \( B \) are \Def{strongly separated} by \( l \) if both inequalities in \fullref{def:hyperplane_separation/normal} are strict:
  \begin{equation}\label{def:hyperplane_separation/strong}
    \Re l(x) < c < \Re l(y) \quad\forall x \in A, y \in B.
  \end{equation}

  We can regard \( l \) as a hyperplane as in \fullref{def:hyperplane/kernel}, which justifies the terminology \enquote{hyperplane separation}. It is more correct, however, especially if \( \BK = \BR \), to say that they are separated by the affine hyperplane \( l(x) + c \).

  If \( \BK = \BR \) \fullref{def:hyperplane_separation/normal} is equivalent to requiring that \( A \) is contained in an open half-\hyperref[def:half_space]{space} relative to \( l(x) + c \) and that \( B \) is contained in the complementing closed half-space (or vice-versa). \Fullref{def:hyperplane_separation/strong} then states that both \( A \) and \( B \) are contained in opposite open half-spaces.
\end{definition}

\begin{definition}\label{def:convex_set}\mbox{}
  \begin{defenum}
    \DItem{def:convex_set/line_segment} Given two points \( x, y \in X \), we define the \Def{line segment} between \( x \) and \( y \) as the parametric curve \( t \mapsto tx + (1-t)y, t \in [0, 1] \). The image
    \begin{equation*}
      [x, y] \coloneqq \{ tx + (1-t)y \colon t \in [0, 1] \}
    \end{equation*}
    of this parametric curve is called the \Def{convex hull} of \( x \) and \( y \). We usually use the term \enquote{line segment} to refer to the convex hull itself.

    \DItem{def:convex_set/hull} We define the convex hull \( \Conv A \) of a set \( A \subseteq X \) as the union of all line segments with endpoints in \( A \).

    \DItem{def:convex_set/set} We call a set \Def{convex} if it coincides with its convex hull, that is, if it contains the line segment between any two of its points.
  \end{defenum}
\end{definition}

\begin{proposition}\label{thm:convex_set_properties}
  Convex \hyperref[def:convex_set]{sets} have the following basic properties:

  \begin{propenum}
    \DItem{thm:convex_set_properties/closed_under_combinations} A convex set is closed under convex \hyperref[def:linear_combination/convex]{combinations}.
    \DItem{thm:convex_set_properties/cone_closed_under_combinations} A closed \hyperref[def:convex_set]{convex} \hyperref[def:geometric_cone]{cone} is closed under conic \hyperref[def:linear_combination/conic]{combinations}.
    \DItem{thm:convex_set_properties/closed_under_intersections} Any intersection of convex sets is convex.
  \end{propenum}
\end{proposition}
\begin{proof}\mbox{}
  \begin{description}
    \RItem{thm:convex_set_properties/closed_under_combinations} Fix a convex set \( C \). Let \( \sum_{k=1}^n t_k x_k \) be a convex combination of elements of \( C \).

    We will use induction\IND on \( n \). If \( n = 1 \), this is obvious. If \( n = 2 \), this is given by definition. Assume that it is true for \( n - 1 \). Denote \( s \coloneqq \sum_{k=1}^{n-1} t_k \). If \( s = 0 \), take another convex combination in order to handle all the possible cases of the induction\IND. Suppose \( s \neq 0 \). Then
    \begin{equation*}
      \sum_{k=1}^n t_k x_k
      =
      s \sum_{k=1}^n \frac {t_k} s x_k
      =
      s \underbrace{\sum_{k=1}^{n-1} \frac {t_k} s x_k}_{\eqqcolon y} + t_n x_n.
    \end{equation*}

    By the inductive hypothesis, \( y \in C \). Note that \( s \in [0, 1] \) and that \( s + t_n = 1 \) by definitions of \( s \). Then \( s y + t_n x_n \) is a binary convex combination that we know is contained in \( C \) by definition.

    \RItem{thm:convex_set_properties/cone_closed_under_combinations} Fix a cone \( C \). Let \( \sum_{k=1}^n t_k x_k \) be a conic combination of elements of \( C \). Each vector \( x_k \) lies on a closed ray, say \( r_k \), thus \( t_k x_k \) also lies on \( r_k \).

    Therefore we only need to show that the sum of two elements \( x_1, x_2 \in C \) is again in \( C \). This is true because \( x_1 + x_2 \) is a convex combination of \( 2x_1 \in r_1 \) and \( 2x_2 \in r_2 \).

    \RItem{thm:convex_set_properties/closed_under_intersections} Let \( X = \cap_{\alpha \in \CA} X_\alpha \) be an intersection of convex sets. Take \( x, y \in X \) and \( t \in [0, 1] \). Then \( tx + (1-t)y \in X_\alpha \) for all \( \alpha \in \CA \), hence \( tx + (1-t)y \in X \). Therefore \( X \) is convex.
  \end{description}
\end{proof}

\begin{definition}\label{def:simplex}
  A \( k \)-\Def{simplex} is the convex \hyperref[def:convex_set/hull]{hull} of \( k + 1 \) affinely \hyperref[affine_independence]{independent} vectors called the \Def{vertices} of the simplex. The convex hull of any subset of the vertices is called a \Def{face} of the simplex.

  \begin{defenum}
    \DItem{def:simplex/point} A \( 0 \)-simplex is a point as defined in \fullref{def:point}.
    \DItem{def:simplex/line_segment} A \( 1 \)-simplex is a line segment as defined in \fullref{def:convex_set/line_segment}.
    \DItem{def:simplex/triangle} A \( 2 \)-simplex is called a \Def{triangle}.
    \DItem{def:simplex/tetrahedron} A \( 3 \)-simplex is called a \Def{tetrahedron}.
  \end{defenum}
\end{definition}

\begin{definition}\label{def:neighborhood_set_types}
  The following topology-independent definitions are often used for neighborhoods in a topological vector space \( X \):

  \begin{defenum}
    \DItem{def:neighborhood_set_types/absorbing} \( A \) is \Def{absorbing} if \( \bigcup_{k=0}^\infty kA = X \).
    \DItem{def:neighborhood_set_types/symmetric} \( A \) is \Def{symmetric} if \( -A = A \).
    \DItem{def:neighborhood_set_types/balanced} \( A \) is \Def{balanced} if \( tA \subseteq A \) for any \( t \in [0, 1] \).
  \end{defenum}
\end{definition}
