\section{Abstract simplicial complexes}\label{sec:abstract_simplicial_complexes}

\begin{definition}\label{def:abstract_simplicial_complex}\cite[definition 2.1]{Carlsson2009}
  An \uline{abstract simplicial complex} is a pair $(V, \Sigma)$, where
  \begin{itemize}
    \item $V$ is a finite set
    \item $\Sigma \subseteq \Power(V)$ such that $\sigma \in \Sigma$ and $\tau \subseteq \sigma$ implies $\tau \in \Sigma$.
  \end{itemize}

  Due to the equivalence with families of simplices (see \cref{thm:abstract_simplicial_complex_iff_simplicial_complex}), elements of $V$ are called \uline{vertices} and elements $\Sigma$ are called \uline{simplices}.

  Denote by $\Sigma_k$ the family of all simplices $S$ with $\Abs{S} = k + 1$, that is, all \uline{$k$-simplices}.
\end{definition}

\begin{definition}\label{def:simplicial_complex}
  A \uline{simplicial complex} in $\BB{R}^n$ is a set $K$ of simplices\Tinyref{def:simplex}, such that
  \begin{itemize}
    \item For any simplex $S \in K$, any face of $S$ is also in $K$.
    \item The intersection of any two simplices $S_1$ and $S_2$ of $K$ is either empty or is a face of both $S_1$ and $S_2$.
  \end{itemize}

  Denote by $K_k$ the family of all $k$-simplices in $K$.
\end{definition}

\begin{proposition}\label{thm:abstract_simplicial_complex_iff_simplicial_complex}
  Let $(V, \Sigma)$ be an abstract simplicial complex\Tinyref{def:abstract_simplicial_complex} and let $v_1 < \ldots < v_n$ be an ordering of elements of $V$. Define the map $E(v_k) \coloneqq e_k, k = 1, \ldots, n$, where $e_k$ are the corresponding basis vectors in $\BB{R^n}$. Then the set
  \begin{align*}
    K \coloneqq \{ \Conv E(S) \colon S \in \Sigma \}
  \end{align*}
  is a simplicial complex\Tinyref{def:simplicial_complex}.

  Conversely, if $K$ is a simplicial complex in $\BB{R^n}$, denote by $V$ all $0$-simplices (vertices) in $K$ and
  \begin{align*}
    \Sigma \coloneqq \{ U \subseteq V \colon \Conv U \in K \}.
  \end{align*}

  Then $(V, \Sigma)$ is an abstract simplicial complex.
\end{proposition}

\begin{definition}\label{def:group_of_chains}\cite[262]{Carlsson2009}
  Let $X = (V, \Sigma)$ be an abstract simplicial complex. For each nonnegative integer $k$, define the corresponding \uline{group of $k$-chains} $C_k(X)$ as the free abelian group\Tinyref{def:free_abelian_group} generated by the $k$-simplices $\Sigma_k$.

  Let $v_1 < \ldots < v_n$ be a total order\Tinyref{def:poset/total_order} on the vertex set $V$. We define the functions
  \begin{align*}
    &d_i: \Sigma \to \Sigma \\
    &d_i(S) \coloneqq S \setminus \{ v_i \}.
  \end{align*}
  and the homomorphisms
  \begin{align*}
    &\partial_k: C_k(X) \to C_{k-1}(X) \\
    &\partial_k(S) \coloneqq \sum_{i=1}^k (-1)^i d_i(S)
  \end{align*}

  We can use the induced ordering to represent the operators $\partial_k$ via matrices.
\end{definition}

\begin{proposition}\label{def:abstract_simplicial_chain_complex}
  In an abstract simplicial complex $X = (V, \Sigma)$, the homomorphisms $\partial_k: C_k(X) \to C_{k-1}(X)$ form a chain complex\Tinyref{def:chain_complex}.
\end{proposition}
