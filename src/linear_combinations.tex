Let $X$ be any vector space over $\R$ (or an extension field of $\R$).

\begin{definition}\label{def:analysis/linear_combinations}
  Let $x_1, \ldots, x_n \in X$ and $t_1, \ldots, t_n \in \R$ be finite sequences and let $A \subseteq X$ be any subset.

  \begin{defenum}
    \item Any element of the form $\sum_{k=1}^n t_k x_k$ is called a \uline{linear combination of $x_1, \ldots, x_n$ with coefficients $t_1, \ldots, t_n$}.
    \item A linear combination is called an \uline{affine combination} if we additionally have $\sum_{k=1}^n t_k = 1$.
    \item A linear combination is called a \uline{conic combination} if all of the coefficients are nonnegative.
    \item A linear combination is called a \uline{convex combination} if it is both affine and conic. A convex combination of two elements $x, y \in X$ is usually written as $tx + (1-t)y$ for some scalar $t \in [0, 1]$.
  \end{defenum}
\end{definition}

\begin{definition}\label{def:analysis:fund:hulls}
  Let $A$ be any subset of $X$. We define the \uline{linear (resp. affine, conic or convex) hull} of $A$ by any of the following equivalent definitions:
  \begin{defenum}
    \item The set of all combinations of the corresponding type of finite subsets of $A$.
    \item The smallest superset of $A$ of the corresponding type (defined in~\cref{def:analysis/linear_combination_sets}).
  \end{defenum}

  We denote this hull by $\Span A$ (resp. $\Aff A$, $\Conn A$ or $\Co A$).

  The linear hull is usually called the \uline{linear span of $A$}.

  The convex hull of two elements $x, y \in X$ is often called the segment between $x$ and $y$ and is denoted as
  \begin{align*}
    [x, y] \coloneqq \{ tx + (1-t)y \colon t \in [0, 1] \}.
  \end{align*}
\end{definition}

\begin{definition}\label{def:analysis/linear_combination_sets}
  A set $A \subseteq X$ is called \uline{linear (resp. affine, conic or convex)} if any of the following equivalent conditions hold:

  \begin{defenum}
    \item It contains all combinations of the corresponding type of any finite subset of itself.
    \item It contains the hull of any of its finite subsets.
    \item It coincides with its own hull.
  \end{defenum}

  Since linear sets are themselves linear spaces, they are usually called \uline{linear subspaces}. The term \uline{linear set} is usually refers to chains in partial orders (see~\cref{def:orders:partial}).

  Conic sets are usually called \uline{cones}.
\end{definition}
