\subsection{Uniform spaces}\label{subsec:uniform_spaces}

\begin{definition}\label{def:entourage}\cite[section 8.1]{Engelking1989}
  Let \( X \) be a set. For two binary relations\Tinyref{def:relation} \( A \) and \( B \) on \( X \) we define their sum as
  \begin{equation*}
    A + B \coloneqq \{ (x, z) \colon \exists y \in X: (x, y) \in A, (y, z) \in B \}
  \end{equation*}
  and \( nA \) by \( n \)-fold iterative addition.

  For any relation \( A \), we denote by \( -A \) the converse relation\Tinyref{def:derived_relations/converse}.

  A relation \( V \) on \( X \) is called an \Def{entourage} if \( V \) is reflexive and \( V = -V \).

  In analogy to metric spaces\Tinyref{def:metric_space}, we define
  \begin{defenum}
    \DItem{def:entourage/distance} We say that the \Def{distance} between \( x, y \in X \) is less than \( V \) and write \( \Abs{x - y} < V \) if \( (x, y) \in V \). Otherwise, we write \( \Abs{x - y} > V \).

    \DItem{def:entourage/diameter} We say that the \Def{diameter} of the set \( A \subseteq X \) is less than \( V \) and write \( \Diam(A) < V \) if \( (x, y) \in V \) for every pair \( (x, y) \in A \times A \).

    \DItem{def:entourage/ball} We define the ball with \Def{center} \( x \) and \Def{radius} \( V \) to be the set
    \begin{equation*}
      B(x, V) \coloneqq \{ y \in X \colon \Abs{y - x} < V \}.
    \end{equation*}

    \DItem{def:entourage/bounded_set} We say that the set \( S \subseteq X \) is \Def{bounded} if it is contained in some ball.
  \end{defenum}
\end{definition}

\begin{proposition}\label{thm:entourage_simulates_metric}\cite[section 8.1]{Engelking1989}
  Using the notation of \cref{def:entourage}, we obtain properties similar to those of metrics:
  \begin{description}
    \RItem{def:metric_space/identity} \( \Abs{x - x} < V \)
    \RItem{def:metric_space/symmetry} \( \Abs{x - y} < V \) if and only if \( \Abs{y - x} < V \)
    \RItem{def:metric_space/triangle_inequality} If \( \Abs{x - y} < U \) and \( \Abs{y - z} < V \), then \( \Abs{x - y} < U + V \).
  \end{description}
\end{proposition}

\begin{definition}\label{def:uniform_space}\cite[section 8.1]{Engelking1989}
  A \Def{uniform space} is a set \( X \) with a family \( \CV \) of entourages\Tinyref{def:entourage} on \( X \) such that
  \begin{defenum}
    \DItem{def:uniform_space/U1}[U1] If \( V \in \CV \) and \( V \subseteq W \) for some entourage \( W \) on \( X \), then \( W \in \CV \).
    \DItem{def:uniform_space/U2}[U2] If \( V_1, V_2 \in \CV \), then \( V_1 \cap V_2 \in \CV \).
    \DItem{def:uniform_space/U3}[U3] For every \( V \in \CV \) there exists \( W \in \CV \) such that \( W + W \subseteq V \).
    \DItem{def:uniform_space/U4}[U4] \( \bigcap \CV = \Delta \), where \( \Delta = \{ (x, x) \colon x \in X \} \).
  \end{defenum}
\end{definition}

\begin{proposition}\label{thm:metric_spaces_are_uniform}
  Metric spaces\Tinyref{def:metric_space} are uniform\Tinyref{def:uniform_space}.
\end{proposition}
\begin{proof}
  Fix a metric space \( (X, \rho) \). For each \( \varepsilon \) define the entourage\Tinyref{def:entourage}
  \begin{equation*}
    V_\varepsilon \coloneqq \{ (x, y) \in X \times X \colon \rho(x, y) < \varepsilon \}.
  \end{equation*}

  That is, we have \( \Abs{x - y} < V_\varepsilon \iff \rho(x, y) < \varepsilon \). This is indeed an entourage because of the symmetry\Tinyref{def:metric_space/symmetry} of \( \rho \) and because \( \rho(x, x) = 0 \quad\forall x \in X \).

  We will show that the family
  \begin{equation*}
    \CV \coloneqq \{ U \text{ is an entourage on } X \mid \exists \varepsilon > 0: V_\varepsilon \subseteq U \}
  \end{equation*}
  is a uniform structure on \( X \).

  \begin{description}
    \RItem{def:uniform_space/U1} Fix \( V_\varepsilon \in \CV \) and let \( W \) be some entourage on \( X \) such that \( V_\varepsilon \subseteq W \). Then \( W \in \CV \) by the definition of \( \CV \).

    \RItem{def:uniform_space/U2} If \( V_\varepsilon, V_\alpha \in \CV \), then
    \begin{equation*}
      V_\varepsilon \cap V_\alpha
      =
      \{ (x, y) \in X \times X \colon \rho(x, y) < \varepsilon \text{ and } \rho(x, y) < \alpha \}
      =
      V_{\min\{\varepsilon, V_\alpha \}}.
    \end{equation*}

    \RItem{def:uniform_space/U3} Fix \( V_\varepsilon \in \CV \). By the triangle inequality we have that if \( \rho(x, y) < \frac \varepsilon 2 \) and \( \rho(y, z) < \frac \varepsilon 2 \), then
    \begin{equation*}
       \rho(x, z) \leq \rho(x, y) + \rho(y, z) < \frac \varepsilon 2 + \frac \varepsilon 2 = \varepsilon.
    \end{equation*}

    Thus
    \begin{equation*}
      V_{\frac \varepsilon 2} + V_{\frac \varepsilon 2}
      =
      \left\{ (x, z) \colon \exists y \in X: \rho(x, y) < \frac \varepsilon 2 \text{ and } \rho(y, z) < \frac \varepsilon 2 \right\}
      \subseteq
      V_\varepsilon
    \end{equation*}

    \RItem{def:uniform_space/U4} Because \( \rho(x, y) = 0 \iff x = y \), we have \( \bigcap \CV = \Delta \).
  \end{description}
\end{proof}

\begin{proposition}\label{thm:uniform_spaces_are_topological}
  Uniform spaces\Tinyref{def:uniform_space} are topological spaces\Tinyref{def:topological_space}.
\end{proposition}
\begin{proof}
  This proof does not actually rely on the uniform structure but rather on the properties of entourages.

  Fix a uniform space \( (X, \CV) \) and define the neighborhood system\Tinyref{def:topological_local_base}
  \begin{equation*}
    \Cal{B}(x) \coloneqq \{ B(x, V) \colon V \in \CV \}.
  \end{equation*}

  It is indeed a neighborhood system because
  \begin{description}
    \RItem{thm:topological_local_base_axioms/BP1} Every entourage\Tinyref{def:entourage} is reflexive, hence \( x \) is contained in every ball in \( \Cal{B}(x) \).

    \RItem{thm:topological_local_base_axioms/BP2} For \( B(x, U) \) and \( B(x, V) \) we have
    \begin{align*}
      B(x, U \cap V)
      &=
      \{ y \in X \colon \Abs{x - y} < U \cap V \}
      = \\ &=
      \{ y \in X \colon \Abs{x - y} < U \text{ and } \Abs{x - y} < V \}
      = \\ &=
      B(x, U) \cap B(x, V).
    \end{align*}

    \RItem{thm:topological_local_base_axioms/BP3} Fix \( x, y \in X \) and a ball \( B(y, V) \in \Cal{B}(y) \) that contains \( x \). We will show that \( B(y, V) \subseteq B(x, 2V) \).

    Fix \( z \in B(y, V) \). We have \( \Abs{z - y} < V \). Then \( \Abs{z - x} < V + V = 2V \). Since \( z \in B(y, V) \) was arbitrary, we conclude that \( B(y, V) \subseteq B(x, 2V) \).
  \end{description}
\end{proof}

\begin{proposition}\label{thm:uniform_space_continuity}
  Let \( (X, \CT) \) be a topological space and \( (Y, \CU) \) be a uniform space. A function \( f: X \to Y \) is continuous at \( \Ol{x} \in X \) in the sense of \cref{def:continuous_function} if and only if
  \begin{equation}\label{thm:uniform_space_continuity/topological_source}
    \forall V \in \CV \ \exists A \in \CT(\Ol x) : x \in A \implies \Abs{f(\Ol{x}) - f(x)} < V.
  \end{equation}

  If instead, \( (X, \CU) \) is a uniform space, then continuity of \( f: X \to Y \) is equivalent to
  \begin{equation}\label{thm:uniform_space_continuity/uniform_source}
    \forall V \in \CV \ \exists U \in \CU : \Abs{\Ol{x} - x} < U \implies \Abs{f(\Ol{x}) - f(x)} < V.
  \end{equation}
\end{proposition}
\begin{proof}
  We will only prove \cref{thm:uniform_space_continuity/uniform_source} because the proof contains a proof of \cref{thm:uniform_space_continuity/topological_source}.

  \begin{description}
    \Implies Suppose that \( f \) is continuous at \( \Ol{x} \) and fix a neighborhood \( B \) of \( f(\Ol{x}) \). Then there exists a neighborhood \( A \) of \( \Ol{x} \) such that \( f(A) \subseteq B \).

    Fix an entourage \( V \in \CV \). Then \( B(f(\Ol{x}), V) \) is also a neighborhood of \( f(\Ol{x}) \) and thus there exists a smaller entourage \( V' \subseteq \CV \) such that \( B(f(x), V') \subseteq B \cap B(f(\Ol{x}), V) \).

    Fix an entourage \( U \in \CU \) such that \( B(\Ol{x}, U) \subseteq A \). Then for any \( x \in X \), if \( \Abs{\Ol{x} - x} < U \) then \( \Abs{f(\Ol{x}) - f(x)} < V' \). But \( V' \subseteq V \), thus
    \begin{equation*}
      \Abs{\Ol{x} - x} < U \implies \Abs{f(\Ol{x}) - f(x)} < V.
    \end{equation*}

    This concludes the proof.

    \ImpliedBy Fix a neighborhood \( B \) of \( f(\Ol{x}) \). Fix an entourage \( V \in \CV \) such that \( B(\Ol{x}, V) \subseteq B \). Then there exists \( U \in \CU \) such that
    \begin{equation*}
      \Abs{\Ol{x} - x} < U \implies \Abs{f(\Ol{x}) - f(x)} < V.
    \end{equation*}

    Thus \( A \coloneqq B(\Ol{x}, U) \) is a neighborhood of \( \Ol{x} \) such that \( f(A) \subseteq B \).
  \end{description}
\end{proof}

\begin{definition}\label{def:uniform_space_arbitrarily_small_sets}\cite[446]{Engelking1989}
  Fix a uniform space \( (X, \CV) \). The family \( \Cal{S} \) of subsets of \( X \) is said to \Def{contain arbitrarily small sets} if for every \( V \in \CV \) there exists a set \( S \in \Cal{S} \) such that \( \Diam(S) < V \).
\end{definition}

\begin{definition}\label{def:complete_uniform_space}\cite[446]{Engelking1989}
  A uniform space \( (X, \CV) \) is called \Def{complete} if every centered family\Tinyref{def:centered_family} \( \Cal{S} \) of subsets of \( X \) that contains arbitrarily small sets\Tinyref{def:uniform_space_arbitrarily_small_sets} has a nonempty intersection.
\end{definition}

\begin{proposition}\label{thm:uniform_space_complete_iff_metric_space_complete}
  A metric space is complete in the sense of \cref{def:complete_metric_space} if and only if it is complete as a uniform space in the sense of \cref{def:complete_uniform_space}.
\end{proposition}

\begin{theorem}[Uniform space completion]\label{thm:uniform_space_completion_existence}\cite[theorem 8.3.12]{Engelking1989}
  Every uniform space has a completion\Tinyref{def:complete_uniform_space}.
\end{theorem}

\begin{definition}\label{def:bounded_function}
  Fix a set \( X \) and a uniform space\Tinyref{def:uniform_space} \( (Y, \CV) \). Fix a function \( f: X \to Y \).

  \begin{defenum}
    \DItem{def:bounded_function/bounded} We say that the function \( f: X \to Y \) is \Def{bounded} if \( f(X) \) is a bounded set, that is, if there exists a ball\Tinyref{def:entourage/ball} \( B(y, V) \) such that \( f(X) \subseteq B(y, V) \).

    \DItem{def:bounded_function/bounded_family} We say that the family of functions \( \CF \) from \( X \) to \( Y \) is \Def{bounded} at \( \Ol{x} \) if there exists a ball \( B(y, V) \) such that the set \( \CF(\Ol x) \coloneqq \{ f(\Ol x) \colon f \in \CF \} \) is contained in \( B(y, V) \).

    \DItem{def:bounded_function/pointwise} We say that \( \CF \) is \Def{pointwise bounded} on the set \( S \subseteq X \) if
    \begin{equation*}
      \forall x \in S \ \exists B(y, V) : \CF(x) \subseteq B(y, V).
    \end{equation*}

    \DItem{def:bounded_function/uniform} We say that \( \CF \) is \Def{uniformly bounded} on \( S \subseteq X \) if
    \begin{equation*}
      \exists B(y, V) \ \forall x \in S : \CF(x) \subseteq B(y, V).
    \end{equation*}

    \DItem{def:bounded_function/locally_bounded} If there is a topology \( \CT \) on \( X \), we say that the function \( f: X \to Y \) is \Def{locally bounded} if there exists an entourage \( V \in \CV \) such that for each neighborhood \( A \in \CT(x) \) we have \( \Diam{f(A)} < V \).
  \end{defenum}
\end{definition}

\begin{proposition}\label{thm:continuous_implies_locally_bounded}
  Let \( (X, \CT) \) be a topological space and \( (Y, \CV) \) be a uniform space. Any continuous function\Tinyref{thm:uniform_space_continuity/topological_source} from \( X \) to \( Y \) is locally bounded\Tinyref{def:bounded_function/locally_bounded}.
\end{proposition}
\begin{proof}
  Trivial.
\end{proof}

\begin{definition}\label{def:function_net_convergence}
  Fix a set \( X \) and a uniform space\Tinyref{def:uniform_space} \( (Y, \CV) \). Let \( \{ f_i \}_{i \in I} \) be a net\Tinyref{def:topological_net} of functions from \( X \) to \( Y \). We say that \( \{ f_i \}_{i \in I} \) \Def{converges pointwise} to the function \( f \) and write \( f_i \to f \) if
  \begin{equation}\label{def:function_net_convergence/pointwise}
    \forall V \in \CV \ \underbrace{\forall x \in X \ \exists i_0 \in I} : i \geq i_0 \implies \Abs{f_i(x) - f(x)} < V
  \end{equation}
  and that \( \{ f_i \}_{i \in I} \) \Def{converges uniformly} to \( f \) and write \( f_i \MultTo f \) if
  \begin{equation}\label{def:function_net_convergence/uniform}
    \forall V \in \CV \ \underbrace{\exists i_0 \in I \ \forall x \in X} : i \geq i_0 \implies \Abs{f_i(x) - f(x)} < V
  \end{equation}

  In the special case where \( X \) is a topological space with topology \( \CT \), we call the sequence \( \{ f_i \}_{i \in I} \) \Def{locally uniformly convergent} (see \cite{ProofWiki:locally_uniform_convergence}) if every point in \( S \) has a neighborhood in which the sequence converges uniformly. Symbolically,
  \begin{equation}\label{def:function_net_convergence/locally_uniform}
    \forall V \in \CV \ \forall \Ol{x} \in S \ \exists A \in \CT(\Ol{x}) \ \exists i_0 \in I \ \forall x \in A : i \geq i_0 \implies \Abs{f_i(x) - f(x)} < V.
  \end{equation}

  If the index \( i_0 \) does not depend on the neighborhood \( A \) and the point \( \Ol{x} \), then this is equivalent to uniform convergence. It is still more powerful than pointwise convergence. For example, power series\Tinyref{def:convergent_power_series} are locally uniformly convergent on the interior of their domain of convergence - see \cref{thm:power_series_are_locally_uniform_convergent}.

  A slightly weaker notion is that of \Def{compact convergence} (see \cite{ProofWiki:compact_convergence}), which is defined as uniform convergence on any compact subset. Symbolically,
  \begin{equation}\label{def:function_net_convergence/compact}
    \forall V \in \CV \ \forall \text{ compact } C \subseteq S \ \exists i_0 \in I \ \forall x \in C : i \geq i_0 \implies \Abs{f_i(x) - f(x)} < V.
  \end{equation}
\end{definition}

\begin{definition}\label{def:uniform_continuity}\cite[435]{Engelking1989}
  Fix two uniform spaces\Tinyref{def:uniform_space} \( (X, \CU) \) and \( (Y, \CV) \). We say that the function \( f: X \to Y \) if is \Def{uniformly continuous} on the set \( S \subseteq X \) if
  \begin{equation}\label{def:uniform_continuity/uniform}
    \forall V \in \CV \ \underbrace{\exists U \in \CU \ \forall x_1, x_2 \in S} : \Abs{x_1 - x_2} < U \implies \Abs{f(x_1) - f(x_2)} < V.
  \end{equation}

  Compare this to \Def{pointwise continuity} on \( S \), which is defined by \cref{thm:uniform_space_continuity/uniform_source} as convergence for any \( x_1 \in X \):
  \begin{equation}\label{def:uniform_continuity/pointwise}
    \forall V \in \CV \ \underbrace{\forall x_1, x_2 \in S \ \exists U \in \CU} : \Abs{x_1 - x_2} < U \implies \Abs{f(x_1) - f(x_2)} < V.
  \end{equation}
\end{definition}

\begin{definition}\label{def:function_set_continuity}\cite[285]{Bouziad2004}
  Fix a topological space \( (X, \CT) \) and a uniform space\Tinyref{def:uniform_space} \( (Y, \CV) \). We say that the family \( \CF \) of functions from \( X \) to \( Y \) is \Def{functionwise continuous} at \( \Ol{x} \in X \) if
  \begin{equation}\label{def:function_set_continuity/functionwise}
    \forall V \in \CV \ \underbrace{\forall f \in \CF \ \exists A \in \CT(\Ol{x})} : f(A) \subseteq B(f(\Ol{x}), V),
  \end{equation}
  and \Def{equicontinuous} at \( \Ol{x} \in X \) if
  \begin{equation}\label{def:function_set_continuity/equicontinuous}
    \forall V \in \CV \ \underbrace{\exists A \in \CT(\Ol{x}) \ \forall f \in \CF} : f(A) \subseteq B(f(\Ol{x}), V).
  \end{equation}

  If, instead, \( (X, \CU) \) is a uniform space, then we can define \Def{uniform equicontinuity} of the family \( \CF \) on the set \( S \subseteq X \) as
  \begin{equation}\label{def:function_set_continuity/uniform_equicontinuous}
    \forall V \in \CV \ \underbrace{\exists U \in \CU \ \forall f \in \CF \ \forall x_1, x_2 \in S} : \Abs{x_1 - x_2} < U \implies \Abs{f(x_1) - f(x_2)} < V
  \end{equation}

  Compare this to \Def{pointwise equicontinuity} of \( \CF \) on \( S \), as defined by \cref{def:function_set_continuity/equicontinuous} for all \( x_1, x_2 \in S \),
  \begin{equation}\label{def:function_set_continuity/pointwise_equicontinuous}
    \forall V \in \CV \ \underbrace{\forall x_1, x_2 \in S \ \exists U \in \CU \ \forall f \in \CF} : \Abs{x_1 - x_2} < U \implies \Abs{f(x_1) - f(x_2)} < V
  \end{equation}
  to \Def{functionwise uniform continuity} of \( \CF \) on \( S \), which is defined by \cref{def:uniform_continuity/uniform} for all \( f \in \CF \),
  \begin{equation}\label{def:function_set_continuity/uniform_functionwise}
    \forall V \in \CV \ \underbrace{\forall f \in \CF \ \exists U \in \CU \ \forall x_1, x_2 \in S} : \Abs{x_1 - x_2} < U \implies \Abs{f(x_1) - f(x_2)} < V
  \end{equation}
  and to \Def{functionwise pointwise continuity} of \( \CF \) on \( S \), i.e. regular continuity as defined by \cref{thm:uniform_space_continuity} for all \( x_1, x_2 \in S \) and all \( f \in \CF \),
  \begin{equation}\label{def:function_set_continuity/functionwise_pointwise}
    \forall V \in \CV \ \underbrace{\forall f \in \CF \ \forall x_1, x_2 \in S \ \exists U \in \CU} : \Abs{x_1 - x_2} < U \implies \Abs{f(x_1) - f(x_2)} < V
  \end{equation}
\end{definition}

\begin{proposition}\label{def:uniform_limit_of_continuous_functions}
  \mbox{}
  \begin{propenum}
    \DItem{def:uniform_limit_of_continuous_functions/continuous} A locally uniform limit\Tinyref{def:function_net_convergence} of functions continuous at a point\Tinyref{thm:uniform_space_continuity} is continuous at that point.
    \DItem{def:uniform_limit_of_continuous_functions/uniform} A uniform limit of functions uniformly continuous\Tinyref{def:uniform_continuity} on a set is uniformly continuous on the set.
  \end{propenum}
\end{proposition}
\begin{proof}
  \begin{description}
    The two proofs are similar but have a lot of subtle differences.

    Fix uniform spaces \( (X, \CU) \) and \( (Y, \CV) \). Let \( \{ f_i \}_{i \in I} \) be a net\Tinyref{def:topological_net} of functions from \( S \subseteq X \) to \( (Y, \CV) \).

    \begin{description}
      \RItem{def:uniform_limit_of_continuous_functions/continuous} Assume that the functions \( f_i, i \in I \) are continuous and that they converge to the function \( f \) locally uniformly\Tinyref{def:function_net_convergence/locally_uniform}.

      Fix an entourage \( W \in \CV \) and use \cref{def:uniform_space/U3} to obtain \( V \subseteq W \) such that \( 3V \subseteq W \).

      . and a point \( \Ol{x} \in S \). Let \( A \) be a neighborhood of \( \Ol{x} \). From locally uniform convergence\Tinyref{def:function_net_convergence}, there exists an index \( i_0 \in I \) such that
      \begin{equation*}
        \forall i > i_0 \ \forall x \in A : \Abs{f_i(x) - f(x)} < V.
      \end{equation*}

      Fix \( i > i_0 \). From uniform continuity\Tinyref{def:function_net_convergence/locally_uniform}, there exists an entourage \( U \in \CU \) such that
      \begin{equation*}
        \forall x \in S : \Abs{\Ol{x} - x} < U \implies \Abs{f_i(\Ol{x}) - f_i(x)} < V.
      \end{equation*}

      Combining the last two inequalities, we note that for any \( x \in A \),
      \begin{itemize}
        \item \( \Abs{f(\Ol{x}) - f(x)} < V \),
        \item \( \Abs{f_i(\Ol{x}) - f(\Ol{x})} < V \),
        \item \( \Abs{f_i(x) - f(x)} < V \),
      \end{itemize}
      thus by applying the triangle inequality in \cref{thm:entourage_simulates_metric} twice, we obtain
      \begin{equation*}
        \Abs{f(\Ol{x}) - f(x)} < 3V \subseteq W \quad\forall x \in A \cap B(\Ol{x}, U).
      \end{equation*}

      Given an entourage \( W \in \CV \), we found a neighborhood \( A \cap B(\Ol{x}, U) \) of \( \Ol{x} \) such that \cref{thm:uniform_space_continuity/topological_source} is satisfied. Thus \( f \) is continuous at \( \Ol{x} \).

      \RItem{def:uniform_limit_of_continuous_functions/continuous} Assume that the functions \( f_i, i \in I \) are uniformly continuous and that they converge to \( f \) uniformly\Tinyref{def:function_net_convergence/locally_uniform}.

      As in \cref{def:uniform_limit_of_continuous_functions/continuous}, fix entourages \( V, W \in \CV \) such that \( 3V \subseteq W \). From uniform continuity\Tinyref{def:uniform_continuity},
      \begin{equation*}
        \forall i \in I \ \exists U \in \CU \ \forall x_1, x_2 \in S : \Abs{x_1 - x_2} < U \implies \Abs{f_i(x_1) - f_i(x_2)} < V.
      \end{equation*}

      From uniform convergence\Tinyref{def:function_net_convergence}, there exists an index \( i_0 \in I \) such that
      \begin{equation*}
        \forall i > i_0 \ \forall x \in S : \Abs{f_i(x) - f(x)} < V.
      \end{equation*}

      Fix an index \( i > i_0 \) and let \( U \in \CU \) be such that
      \begin{equation}\label{def:uniform_limit_of_continuous_functions/uniform/continuity}
        \forall x_1, x_2 \in S : \Abs{x_1 - x_2} < U \implies \Abs{f_i(x_1) - f_i(x_2)} < V.
      \end{equation}

      For any two points \( x_1, x_2 \in S \), we also have that
      \begin{equation}\label{def:uniform_limit_of_continuous_functions/uniform/convergence}
        \Abs{f(x_i) - f_i(x_i)} < V, i = 1, 2.
      \end{equation}

      Analogously to \cref{def:uniform_limit_of_continuous_functions/continuous}, from \cref{def:uniform_limit_of_continuous_functions/uniform/continuity} and \cref{def:uniform_limit_of_continuous_functions/uniform/convergence}, we obtain
      \begin{equation*}
        \forall x_1, x_2 \in S : \Abs{x_1 - x_2} < U \implies \Abs{f(x_1) - f(x_2)} < 3V \subseteq W.
      \end{equation*}

      Thus the entourage \( U \) depends on \( W \) and not on \( x_1 \) and \( x_2 \). Technically, it also depends on \( i_0 \), however we are only concerned with existence and not uniqueness. Hence \( f \) is uniformly continuous.
    \end{description}
  \end{description}
\end{proof}
