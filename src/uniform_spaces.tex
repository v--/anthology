\subsection{Uniform spaces}\label{subsec:uniform_spaces}

\begin{definition}\label{def:entourage}\cite[section 8.1]{Engelking1989}
  Let \( X \) be a set. For two binary relations\Tinyref{def:relation} \( A \) and \( B \) on \( X \) we define their sum as
  \begin{equation*}
    A + B \coloneqq \{ (x, z) \colon \exists y \in X: (x, y) \in A, (y, z) \in B \}
  \end{equation*}
  and \( nA \) by \( n \)-fold iterative addition.

  For any relation \( A \), we denote by \( -A \) the converse relation\Tinyref{def:derived_relations/converse}.

  A relation \( V \) on \( X \) is called an \Def{entourage} if \( V \) is reflexive and \( V = -V \).

  In analogy to metric spaces\Tinyref{def:metric_space}, we define
  \begin{defenum}
    \DItem{def:entourage/distance} We say that the \Def{distance} between \( x, y \in X \) is less than \( V \) and write \( \Abs{x - y} < V \) if \( (x, y) \in V \).
    \DItem{def:entourage/diameter} We say that the \Def{diameter} of the set \( A \subseteq X \) is less than \( V \) and write \( \Diam(A) < V \) if \( (x, y) \in V \) for every pair \( (x, y) \in A \times A \).
    \DItem{def:entourage/ball} We define the ball with \Def{center} \( x \) and \Def{radius} \( V \) to be the set
    \begin{equation*}
      \B(x, V) \coloneqq \{ y \in X \colon \Abs{y - x} < V \}.
    \end{equation*}
  \end{defenum}
\end{definition}

\begin{definition}\label{def:uniform_space}\cite[section 8.1]{Engelking1989}
  A \Def{uniform space} is a set \( X \) with a family \( \U \) of entourages\Tinyref{def:entourage} on \( X \) such that
  \begin{defenum}
    \DItem{def:uniform_space/U1}[U1] If \( V \in \U \) and \( V \subseteq W \) for some entourage \( W \) on \( X \), then \( W \in U \).
    \DItem{def:uniform_space/U2}[U2] If \( V_1, V_2 \in \U \), then \( V_1 \cap V_2 \in \U \).
    \DItem{def:uniform_space/U3}[U3] For every \( V \in \U \) there exists \( W \in \U \) such that \( W + W \subseteq V \).
    \DItem{def:uniform_space/U4}[U4] \( \bigcap \U = \Delta \), where \( \Delta = \{ (x, x) \colon x \in X \} \).
  \end{defenum}
\end{definition}

\begin{definition}\label{def:uniform_continuity}
  Fix two uniform spaces\Tinyref{def:uniform_space} \( (X, \U) \) and \( (Y, \V) \). Let \( \Cal{F} \) be a family of functions from \( X \) to \( Y \). We say that the family \( \Cal{F} \) if is \Def{pointwise continuous} if
  \begin{equation}\label{def:uniform_continuity/pointwise}
    \forall V \in \V \ \underbrace{\forall f \in \Cal{F} \ \exists U \in \U_V} \ (\Abs{x - y} < U \implies \Abs{f(x) - f(y)} < V)
  \end{equation}
  and that \( \Cal{F} \) is \Def{uniformly continuous} if
  \begin{equation}\label{def:uniform_continuity/uniform}
    \forall V \in \V \ \underbrace{\exists U \in \U_V \ \forall f \in \Cal{F}} \ (\Abs{x - y} < U \implies \Abs{f(x) - f(y)} < V)
  \end{equation}
\end{definition}

\begin{definition}\label{def:uniform_convergence}
  Fix a set \( X \) and a uniform space\Tinyref{def:uniform_space} \( (Y, \V) \). Let \( \{ f_i \}_{i \in I} \) be a net\Tinyref{def:topological_net} of functions from \( X \) to \( Y \). We say that \( \{ f_i \}_{i \in I} \) \Def{converges pointwise} to the function \( f \) and write \( f_i \to f \) if
  \begin{equation}\label{def:uniform_convergence/pointwise}
    \forall V \in \V \ \underbrace{\forall x \in X \ \exists i_0 \in I} \ (i \geq i_0 \implies \Abs{f_i(x) - f(x)} < V)
  \end{equation}
  and that \( \{ f_i \}_{i \in I} \) \Def{converges uniformly} to \( f \) and write \( f_i \MultTo f \) if
  \begin{equation}\label{def:uniform_convergence/uniform}
    \forall V \in \V \ \underbrace{\exists i_0 \in I \ \forall x \in X} \ (i \geq i_0 \implies \Abs{f_i(x) - f(x)} < V)
  \end{equation}
\end{definition}
