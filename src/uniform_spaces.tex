\subsection{Uniform spaces}\label{subsec:uniform_spaces}

\begin{definition}\label{def:entourage}\cite[section 8.1]{Engelking1989}
  Let \( X \) be a set. For two binary relations\Tinyref{def:relation} \( A \) and \( B \) on \( X \) we define their sum as
  \begin{equation*}
    A + B \coloneqq \{ (x, z) \colon \exists y \in X: (x, y) \in A, (y, z) \in B \}
  \end{equation*}
  and \( nA \) by \( n \)-fold iterative addition.

  For any relation \( A \), we denote by \( -A \) the converse relation\Tinyref{def:derived_relations/converse}.

  A relation \( V \) on \( X \) is called an \Def{entourage} if \( V \) is reflexive and \( V = -V \).

  In analogy to metric spaces\Tinyref{def:metric_space}, we define
  \begin{defenum}
    \DItem{def:entourage/distance} We say that the \Def{distance} between \( x, y \in X \) is less than \( V \) and write \( \Abs{x - y} < V \) if \( (x, y) \in V \). Otherwise, we write \( \Abs{x - y} > V \).

    \DItem{def:entourage/diameter} We say that the \Def{diameter} of the set \( A \subseteq X \) is less than \( V \) and write \( \Diam(A) < V \) if \( (x, y) \in V \) for every pair \( (x, y) \in A \times A \).

    \DItem{def:entourage/ball} We define the ball with \Def{center} \( x \) and \Def{radius} \( V \) to be the set
    \begin{equation*}
      B(x, V) \coloneqq \{ y \in X \colon \Abs{y - x} < V \}.
    \end{equation*}

    \DItem{def:entourage/bounded_set} We say that the set \( S \subseteq X \) is \Def{bounded} if it is contained in some ball.
  \end{defenum}
\end{definition}

\begin{proposition}\label{thm:entourage_simulates_metric}\cite[section 8.1]{Engelking1989}
  Using the notation of \fullref{def:entourage}, we obtain properties similar to those of metrics:
  \begin{description}
    \RItem{def:metric_space/identity} \( \Abs{x - x} < V \)
    \RItem{def:metric_space/symmetry} \( \Abs{x - y} < V \) if and only if \( \Abs{y - x} < V \)
    \RItem{def:metric_space/triangle_inequality} If \( \Abs{x - y} < U \) and \( \Abs{y - z} < V \), then \( \Abs{x - y} < U + V \).
  \end{description}
\end{proposition}

\begin{definition}\label{def:uniform_space}\cite[section 8.1]{Engelking1989}
  A \Def{uniform space} is a set \( X \) with a family \( \CV \) of entourages\Tinyref{def:entourage} on \( X \) such that
  \begin{defenum}
    \DItem{def:uniform_space/U1}[U1] If \( V \in \CV \) and \( V \subseteq W \) for some entourage \( W \) on \( X \), then \( W \in \CV \).
    \DItem{def:uniform_space/U2}[U2] If \( V_1, V_2 \in \CV \), then \( V_1 \cap V_2 \in \CV \).
    \DItem{def:uniform_space/U3}[U3] For every \( V \in \CV \) there exists \( W \in \CV \) such that \( W + W \subseteq V \).
    \DItem{def:uniform_space/U4}[U4] \( \bigcap \CV = \Delta \), where \( \Delta = \{ (x, x) \colon x \in X \} \).
  \end{defenum}
\end{definition}

\begin{proposition}\label{thm:uniform_spaces_are_topological}
  Uniform spaces\Tinyref{def:uniform_space} are topological spaces\Tinyref{def:topological_space}.
\end{proposition}
\begin{proof}
  This proof does not actually rely on the uniform structure but rather on the properties of entourages.

  Fix a uniform space \( (X, \CV) \) and define the neighborhood system\Tinyref{def:topological_local_base}
  \begin{equation*}
    \Cal{B}(x) \coloneqq \{ B(x, V) \colon V \in \CV \}.
  \end{equation*}

  It is indeed a neighborhood system because
  \begin{description}
    \RItem{thm:topological_local_base_axioms/BP1} Every entourage\Tinyref{def:entourage} is reflexive, hence \( x \) is contained in every ball in \( \Cal{B}(x) \).

    \RItem{thm:topological_local_base_axioms/BP2} For \( B(x, U) \) and \( B(x, V) \) we have
    \begin{align*}
      B(x, U \cap V)
      &=
      \{ y \in X \colon \Abs{x - y} < U \cap V \}
      = \\ &=
      \{ y \in X \colon \Abs{x - y} < U \text{ and } \Abs{x - y} < V \}
      = \\ &=
      B(x, U) \cap B(x, V).
    \end{align*}

    \RItem{thm:topological_local_base_axioms/BP3} Fix \( x, y \in X \) and a ball \( B(y, V) \in \Cal{B}(y) \) that contains \( x \). We will show that \( B(y, V) \subseteq B(x, 2V) \).

    Fix \( z \in B(y, V) \). We have \( \Abs{z - y} < V \). Then \( \Abs{z - x} < V + V = 2V \). Since \( z \in B(y, V) \) was arbitrary, we conclude that \( B(y, V) \subseteq B(x, 2V) \).
  \end{description}
\end{proof}

\begin{lemma}\label{thm:uniform_space_neighborhood_contains_ball}
  In a uniform space \( (X, \CV) \), for every neighborhood \( A \) (in the topology) of a point \( x_0 \in X \) there exists an entourage \( V \in \CV \) such that \( B(x_0, V) \subseteq A \).
\end{lemma}
\begin{proof}
  By \fullref{thm:uniform_spaces_are_topological} and \fullref{def:topological_base/union}, \( A \) is a union of balls centered at \( x_0 \). For any ball \( B(x_0, V) \) of this union, we have \( B(x_0, V) \subseteq A \).
\end{proof}

\begin{definition}\label{}

\end{definition}

\begin{proposition}\label{thm:uniform_space_local_convergence}
  Fix a topological space \( (X, \CT) \) and a uniform space \( (Y, \CU) \). Let \( A \subseteq X \) be a nonempty set and let \( f: A \to Y \) be a function. Then \( y_0 \) is a limit point of \( f \) at \( x_0 \in X \) in the sense of \fullref{def:local_continuity} if and only if
  \begin{equation}\label{thm:uniform_space_local_continuity/topological_source}
    \forall V \in \CV \ \exists A \in \CT(x_0) : x \in A \implies \Abs{f(x) - y_0} < V.
  \end{equation}

  If instead, \( (X, \CU) \) is a uniform space, then \( y_0 \) is a limit point of \( f \) at \( x_0 \in X \) if and only if
  \begin{equation}\label{thm:uniform_space_local_continuity/uniform_source}
    \forall V \in \CV \ \exists U \in \CU : \Abs{x - x_0} < U \implies \Abs{f(x) - y_0} < V.
  \end{equation}

  Note that the limit point may not be unique because uniform spaces are not Hausdorff\Tinyref{def:separation_axioms/T2} in general.
\end{proposition}
\begin{proof}
  We will only prove \fullref{thm:uniform_space_local_continuity/uniform_source} because the proof of \fullref{thm:uniform_space_local_continuity/topological_source} is a special case.

  \begin{description}
    \Implies Suppose that \( y_0 \) is a limit point of \( f \) at \( x_0 \) and fix a neighborhood \( B \) of \( y_0 \). Then there exists a neighborhood \( A \) of \( x_0 \) such that \( f(A) \subseteq B \).

    Fix an entourage \( V \in \CV \). Then \( B(y_0, V) \) is also a neighborhood of \( y_0 \). By \fullref{thm:uniform_space_neighborhood_contains_ball} and \fullref{def:uniform_space/U2}, there exists an entourage \( V' \subseteq \CV \) such that \( B(f(x), V') \subseteq B \cap B(y_0, V) \).

    Fix an entourage \( U \in \CU \) such that \( B(x_0, U) \subseteq A \). Then for any \( x \in X \), if \( \Abs{x - x_0} < U \), we have \( \Abs{f(x) - y_0} < V' \). But \( V' \subseteq V \), therefore
    \begin{equation*}
      \Abs{x - x_0} < U \implies \Abs{f(x) - y_0} < V.
    \end{equation*}

    This concludes the proof.

    \ImpliedBy Fix a neighborhood \( B \) of \( y_0 \) and an entourage \( V \in \CV \) such that \( B(x_0, V) \subseteq B \) (see \fullref{thm:uniform_space_neighborhood_contains_ball} for a justification). Then there exists \( U \in \CU \) such that
    \begin{equation*}
      \Abs{x - x_0} < U \implies \Abs{f(x) - y_0} < V.
    \end{equation*}

    Therefore \( A \coloneqq B(x_0, U) \) is a neighborhood of \( x_0 \) such that \( f(A) \subseteq B \).
  \end{description}
\end{proof}

\begin{corollary}\label{thm:uniform_space_local_continuity}
  A function \( f: (X, \CV) \to (Y, \CU) \) between uniform spaces is continuous at \( x_0 \in X \) if and only if
  \begin{equation*}
    \forall V \in \CV \ \exists U \in \CU : \Abs{x - x_0} < U \implies \Abs{f(x) - f(x_0)} < V.
  \end{equation*}
\end{corollary}

\begin{definition}\label{def:bounded_function}
  Fix a set \( X \) and a uniform space\Tinyref{def:uniform_space} \( (Y, \CV) \). Fix a function \( f: X \to Y \).

  \begin{defenum}
    \DItem{def:bounded_function/bounded} We say that the function \( f: X \to Y \) is \Def{bounded} if \( f(X) \) is a bounded set, that is, if there exists a ball\Tinyref{def:entourage/ball} \( B(y, V) \) such that \( f(X) \subseteq B(y, V) \).

    \DItem{def:bounded_function/bounded_family} We say that the family of functions \( \CF \) from \( X \) to \( Y \) is \Def{bounded} at \( x_0 \) if there exists a ball \( B(y, V) \) such that the set \( \CF(x_0) \coloneqq \{ f(x_0) \colon f \in \CF \} \) is contained in \( B(y, V) \).

    \DItem{def:bounded_function/pointwise} We say that \( \CF \) is \Def{pointwise bounded} on the set \( S \subseteq X \) if
    \begin{equation*}
      \forall x \in S \ \exists B(y, V) : \CF(x) \subseteq B(y, V).
    \end{equation*}

    \DItem{def:bounded_function/uniform} We say that \( \CF \) is \Def{uniformly bounded} on \( S \subseteq X \) if
    \begin{equation*}
      \exists B(y, V) \ \forall x \in S : \CF(x) \subseteq B(y, V).
    \end{equation*}

    \DItem{def:bounded_function/locally_bounded} If there is a topology \( \CT \) on \( X \), we say that the function \( f: X \to Y \) is \Def{locally bounded} if there exists an entourage \( V \in \CV \) such that for each neighborhood \( A \in \CT(x) \) we have \( \Diam{f(A)} < V \).
  \end{defenum}
\end{definition}

\begin{proposition}\label{thm:continuous_implies_locally_bounded}
  Let \( (X, \CT) \) be a topological space and \( (Y, \CV) \) be a uniform space. Any continuous function\Tinyref{thm:uniform_space_local_continuity/topological_source} from \( X \) to \( Y \) is locally bounded\Tinyref{def:bounded_function/locally_bounded}.
\end{proposition}
\begin{proof}
  Trivial.
\end{proof}

\begin{definition}\label{def:function_net_convergence}
  Fix a set \( X \) and a uniform space\Tinyref{def:uniform_space} \( (Y, \CV) \). Let \( \{ f_\al \}_{\al \in \CA} \) be a net\Tinyref{def:topological_net} of functions from \( X \) to \( Y \). We say that \( \{ f_\al \}_{\al \in \CA} \) \Def{converges pointwise} to the function \( f \) and write \( f_\al \to f \) if
  \begin{equation}\label{def:function_net_convergence/pointwise}
    \forall V \in \CV \ \underbrace{\forall x \in X \ \exists \al_0 \in \CA} : \al \geq \al_0 \implies \Abs{f_\al(x) - f(x)} < V
  \end{equation}
  and that \( \{ f_\al \}_{\al \in \CA} \) \Def{converges uniformly} to \( f \) and write \( f_\al \MultTo f \) if
  \begin{equation}\label{def:function_net_convergence/uniform}
    \forall V \in \CV \ \underbrace{\exists \al_0 \in \CA \ \forall x \in X} : \al \geq \al_0 \implies \Abs{f_\al(x) - f(x)} < V
  \end{equation}

  In the special case where \( X \) is a topological space with topology \( \CT \), we call the sequence \( \{ f_\al \}_{\al \in \CA} \) \Def{locally uniformly convergent} (see \cite{ProofWiki:locally_uniform_convergence}) if every point in \( S \) has a neighborhood in which the sequence converges uniformly. Symbolically,
  \begin{equation}\label{def:function_net_convergence/locally_uniform}
    \forall V \in \CV \ \forall x_0 \in S \ \exists A \in \CT(x_0) \ \exists \al_0 \in \CA \ \forall x \in A : \al \geq \al_0 \implies \Abs{f_\al(x) - f(x)} < V.
  \end{equation}

  If the index \( \al_0 \) does not depend on the neighborhood \( A \) and the point \( x_0 \), then this is equivalent to uniform convergence. It is still more powerful than pointwise convergence. For example, power series\Tinyref{def:convergent_power_series} are locally uniformly convergent on the interior of their domain of convergence - see \fullref{thm:power_series_are_locally_uniform_convergent}.

  A slightly weaker notion is that of \Def{compact convergence} (see \cite{ProofWiki:compact_convergence}), which is defined as uniform convergence on any compact subset. Symbolically,
  \begin{equation}\label{def:function_net_convergence/compact}
    \forall V \in \CV \ \forall \text{ compact } C \subseteq S \ \exists \al_0 \in \CA \ \forall x \in C : \al \geq \al_0 \implies \Abs{f_\al(x) - f(x)} < V.
  \end{equation}
\end{definition}

\begin{definition}\label{def:uniform_continuity}\cite[435]{Engelking1989}
  Fix two uniform spaces\Tinyref{def:uniform_space} \( (X, \CU) \) and \( (Y, \CV) \). We say that the function \( f: X \to Y \) if is \Def{uniformly continuous} on the set \( S \subseteq X \) if
  \begin{equation}\label{def:uniform_continuity/uniform}
    \forall V \in \CV \ \underbrace{\exists U \in \CU \ \forall x_1, x_2 \in S} : \Abs{x_1 - x_2} < U \implies \Abs{f(x_1) - f(x_2)} < V.
  \end{equation}

  Compare this to \Def{pointwise continuity} on \( S \), which is defined by \fullref{thm:uniform_space_local_continuity/uniform_source} as convergence for any \( x_1 \in X \):
  \begin{equation}\label{def:uniform_continuity/pointwise}
    \forall V \in \CV \ \underbrace{\forall x_1, x_2 \in S \ \exists U \in \CU} : \Abs{x_1 - x_2} < U \implies \Abs{f(x_1) - f(x_2)} < V.
  \end{equation}
\end{definition}

\begin{definition}\label{def:function_set_continuity}\cite[285]{Bouziad2004}
  Fix a topological space \( (X, \CT) \) and a uniform space\Tinyref{def:uniform_space} \( (Y, \CV) \). We say that the family \( \CF \) of functions from \( X \) to \( Y \) is \Def{functionwise continuous} at \( x_0 \in X \) if
  \begin{equation}\label{def:function_set_continuity/functionwise}
    \forall V \in \CV \ \underbrace{\forall f \in \CF \ \exists A \in \CT(x_0)} : f(A) \subseteq B(f(x_0), V),
  \end{equation}
  and \Def{equicontinuous} at \( x_0 \in X \) if
  \begin{equation}\label{def:function_set_continuity/equicontinuous}
    \forall V \in \CV \ \underbrace{\exists A \in \CT(x_0) \ \forall f \in \CF} : f(A) \subseteq B(f(x_0), V).
  \end{equation}

  If, instead, \( (X, \CU) \) is a uniform space, then we can define \Def{uniform equicontinuity} of the family \( \CF \) on the set \( S \subseteq X \) as
  \begin{equation}\label{def:function_set_continuity/uniform_equicontinuous}
    \forall V \in \CV \ \underbrace{\exists U \in \CU \ \forall f \in \CF \ \forall x_1, x_2 \in S} : \Abs{x_1 - x_2} < U \implies \Abs{f(x_1) - f(x_2)} < V
  \end{equation}

  Compare this to \Def{pointwise equicontinuity} of \( \CF \) on \( S \), as defined by \fullref{def:function_set_continuity/equicontinuous} for all \( x_1, x_2 \in S \),
  \begin{equation}\label{def:function_set_continuity/pointwise_equicontinuous}
    \forall V \in \CV \ \underbrace{\forall x_1, x_2 \in S \ \exists U \in \CU \ \forall f \in \CF} : \Abs{x_1 - x_2} < U \implies \Abs{f(x_1) - f(x_2)} < V
  \end{equation}
  to \Def{functionwise uniform continuity} of \( \CF \) on \( S \), which is defined by \fullref{def:uniform_continuity/uniform} for all \( f \in \CF \),
  \begin{equation}\label{def:function_set_continuity/uniform_functionwise}
    \forall V \in \CV \ \underbrace{\forall f \in \CF \ \exists U \in \CU \ \forall x_1, x_2 \in S} : \Abs{x_1 - x_2} < U \implies \Abs{f(x_1) - f(x_2)} < V
  \end{equation}
  and to \Def{functionwise pointwise continuity} of \( \CF \) on \( S \), i.e. regular continuity as defined by \fullref{thm:uniform_space_local_continuity} for all \( x_1, x_2 \in S \) and all \( f \in \CF \),
  \begin{equation}\label{def:function_set_continuity/functionwise_pointwise}
    \forall V \in \CV \ \underbrace{\forall f \in \CF \ \forall x_1, x_2 \in S \ \exists U \in \CU} : \Abs{x_1 - x_2} < U \implies \Abs{f(x_1) - f(x_2)} < V
  \end{equation}
\end{definition}

\begin{proposition}\label{def:uniform_limit_of_continuous_functions}
  \mbox{}
  \begin{propenum}
    \DItem{def:uniform_limit_of_continuous_functions/continuous} A locally uniform limit\Tinyref{def:function_net_convergence} of functions continuous at a point\Tinyref{thm:uniform_space_local_continuity} is continuous at that point.
    \DItem{def:uniform_limit_of_continuous_functions/uniform} A uniform limit of functions uniformly continuous\Tinyref{def:uniform_continuity} on a set is uniformly continuous on the set.
  \end{propenum}
\end{proposition}
\begin{proof}\mbox{}
  The two proofs are similar but have a lot of subtle differences.

  Fix uniform spaces \( (X, \CU) \) and \( (Y, \CV) \). Let \( \{ f_\al \}_{\al \in \CA} \) be a net\Tinyref{def:topological_net} of functions from \( S \subseteq X \) to \( (Y, \CV) \).

  \begin{description}
    \RItem{def:uniform_limit_of_continuous_functions/continuous} Assume that the functions \( f_\al, \al \in \CA \) are continuous and that they converge to the function \( f \) locally uniformly\Tinyref{def:function_net_convergence/locally_uniform}.

    Fix an entourage \( W \in \CV \) and use \fullref{def:uniform_space/U3} to obtain \( V \subseteq W \) such that \( 3V \subseteq W \).

    . and a point \( x_0 \in S \). Let \( A \) be a neighborhood of \( x_0 \). From locally uniform convergence\Tinyref{def:function_net_convergence}, there exists an index \( \al_0 \in \CA \) such that
    \begin{equation*}
      \forall \al > \al_0 \ \forall x \in A : \Abs{f_\al(x) - f(x)} < V.
    \end{equation*}

    Fix \( \al > \al_0 \). From uniform continuity\Tinyref{def:function_net_convergence/locally_uniform}, there exists an entourage \( U \in \CU \) such that
    \begin{equation*}
      \forall x \in S : \Abs{x - x_0} < U \implies \Abs{f_\al(x_0) - f_\al(x)} < V.
    \end{equation*}

    Combining the last two inequalities, we note that for any \( x \in A \),
    \begin{itemize}
      \item \( \Abs{f(x_0) - f(x)} < V \),
      \item \( \Abs{f_\al(x_0) - f(x_0)} < V \),
      \item \( \Abs{f_\al(x) - f(x)} < V \),
    \end{itemize}
    thus by applying the triangle inequality in \fullref{thm:entourage_simulates_metric} twice, we obtain
    \begin{equation*}
      \Abs{f(x_0) - f(x)} < 3V \subseteq W \quad\forall x \in A \cap B(x_0, U).
    \end{equation*}

    Given an entourage \( W \in \CV \), we found a neighborhood \( A \cap B(x_0, U) \) of \( x_0 \) such that \fullref{thm:uniform_space_local_continuity/topological_source} is satisfied. Thus \( f \) is continuous at \( x_0 \).

    \RItem{def:uniform_limit_of_continuous_functions/continuous} Assume that the functions \( f_\al, \al \in \CA \) are uniformly continuous and that they converge to \( f \) uniformly\Tinyref{def:function_net_convergence/locally_uniform}.

    As in \fullref{def:uniform_limit_of_continuous_functions/continuous}, fix entourages \( V, W \in \CV \) such that \( 3V \subseteq W \). From uniform continuity\Tinyref{def:uniform_continuity},
    \begin{equation*}
      \forall \al \in \CA \ \exists U \in \CU \ \forall x_1, x_2 \in S : \Abs{x_1 - x_2} < U \implies \Abs{f_\al(x_1) - f_\al(x_2)} < V.
    \end{equation*}

    From uniform convergence\Tinyref{def:function_net_convergence}, there exists an index \( \al_0 \in \CA \) such that
    \begin{equation*}
      \forall \al > \al_0 \ \forall x \in S : \Abs{f_\al(x) - f(x)} < V.
    \end{equation*}

    Fix an index \( \al > \al_0 \) and let \( U \in \CU \) be such that
    \begin{equation}\label{def:uniform_limit_of_continuous_functions/uniform/continuity}
      \forall x_1, x_2 \in S : \Abs{x_1 - x_2} < U \implies \Abs{f_\al(x_1) - f_\al(x_2)} < V.
    \end{equation}

    For any two points \( x_1, x_2 \in S \), we also have that
    \begin{equation}\label{def:uniform_limit_of_continuous_functions/uniform/convergence}
      \Abs{f(x_i) - f_\al(x_i)} < V, i = 1, 2.
    \end{equation}

    Analogously to \fullref{def:uniform_limit_of_continuous_functions/continuous}, from \fullref{def:uniform_limit_of_continuous_functions/uniform/continuity} and \fullref{def:uniform_limit_of_continuous_functions/uniform/convergence}, we obtain
    \begin{equation*}
      \forall x_1, x_2 \in S : \Abs{x_1 - x_2} < U \implies \Abs{f(x_1) - f(x_2)} < 3V \subseteq W.
    \end{equation*}

    Thus the entourage \( U \) depends on \( W \) and not on \( x_1 \) and \( x_2 \). Technically, it also depends on \( \al_0 \), however we are only concerned with existence and not uniqueness. Hence \( f \) is uniformly continuous.
  \end{description}
\end{proof}

\begin{definition}\label{def:category_of_uniform_spaces}
  We denote by \( \Cat{Unif} \) the subcategory\Tinyref{def:subcategory} of \( \Bold{Top} \)\Tinyref{def:category_of_topological_spaces} where
  \begin{itemize}
    \item the class\Tinyref{def:set_zfc} of objects is the class of all uniform spaces\Tinyref{def:uniform_space}.
    \item the morphisms between two uniform spaces are the globally uniformly continuous functions\Tinyref{def:uniform_continuity} between them.
  \end{itemize}
\end{definition}

\begin{definition}\label{def:fundamental_net}
  A net\Tinyref{def:topological_net} \( \{ x_\al \}_{\al \in \CA} \) in a uniform space \( (X, \CV) \) is called a \Def{fundamental net} or \Def{Cauchy net} if
  \begin{equation*}
    \forall V \in \CV \ \exists \al_0 \in \CA \ \forall \al, \beta \geq \al_0 : \Abs{x_\al - x_\beta} < V.
  \end{equation*}
\end{definition}

\begin{definition}\label{def:complete_uniform_space}\cite[446]{Engelking1989}
  A uniform space is called \Def{complete} if it is Hausdorff\Tinyref{def:separation_axioms/T2} and if every fundamental net\Tinyref{def:fundamental_net} converges\Tinyref{def:net_convergence/limit}.

  The \Def{completion} of uniform space is a uniformly continuous embedding into a complete uniform space.
\end{definition}

\begin{theorem}[Uniform space completion]\label{thm:uniform_space_completion_existence}\cite[theorem 8.3.12]{Engelking1989}
  Every uniform space has a unique (up to an isomorphism) completion\Tinyref{def:complete_uniform_space}.
\end{theorem}
