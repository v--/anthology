\subsection{Riemann integration}\label{subsec:riemann_integration}

\begin{definition}\label{def:partition}\MarginCite[def. 6.1]{Rudin1976}
  A \Def{partition} of a nonempty \hyperref[def:total_order_interval/closed]{closed interval} \( [a, b] \) is a set \( \{ x_0, \ldots, x_n \} \subseteq [a, b] \) that satisfies
  \begin{equation}\label{eq:def:partition}
    a = x_0 < x_1 < \ldots < x_{n-1} < x_n = b.
  \end{equation}
\end{definition}

\begin{remark}\label{remark:set_and_interval_partitions}
  Note that \eqref{eq:def:partition} is not a partition in the sense of \fullref{def:set_partition}, however the set of intervals
  \begin{equation*}
    \left\{ [x_0, x_1), [x_1, x_2), \ldots, [x_{n-2}, x_{n-1}), [x_{n-1}, x_n] \right\}
  \end{equation*}
  is a set-theoretic partition of \( [a, b] \). Conversely, every finite set-theoretic partition of \( [a, b] \) gives rise to a partition in the sense of \fullref{def:partition}.
\end{remark}

\begin{definition}\label{def:partition_refinement}\MarginCite[def. 6.1]{Rudin1976}
  The \hyperref[def:partition]{partition}
  \begin{equation*}
    a = y_0 < y_1 < \ldots < y_{m-1} < y_m = b
  \end{equation*}
  is called a \Def{refinement} of the partition \eqref{eq:def:partition} if we have the \hyperref[def:subset]{set inclusion}
  \begin{equation*}
    \{ x_0, x_1, \ldots, x_n \} \subseteq \{ y_0, y_1, \ldots, y_m \}.
  \end{equation*}
\end{definition}

\begin{definition}\label{def:tagged_partition}\MarginCite[def. 2]{Chernysh2018}
  A \Def{tagged partition} of \( [a, b] \) is a \hyperref[def:partition]{partition} \eqref{eq:def:partition} of \( [a, b] \) along with a choice of \( \xi_k \) for each closed interval \( [x_{k-1}, x_k], k = 1, \ldots, n \). That is, a tagged partition is a tuple \( (\{ x_k \}_{k=0}^n, \{ \xi_k \}_{k=1}^n) \).
\end{definition}
