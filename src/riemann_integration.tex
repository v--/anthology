\subsection{Riemann integration}\label{subsec:riemann_integration}

\begin{definition}\label{def:riemann_partition}\cite[def. 1]{Gordon1991}
  The concept of a partition of a nonempty \hyperref[def:real_numbers]{real} \hyperref[def:total_order_interval/closed]{closed interval} \( [a, b] \) is the base for defining Riemann-style integrals.

  \begin{DefEnum}
    \ILabel{def:riemann_partition/partition} A \Def{Riemann partition} of \( [a, b] \) is a set \( \{ x_0, \ldots, x_n \} \subseteq [a, b] \) that satisfies
    \begin{equation}\label{eq:def:riemann_partition/partition}
      a = x_0 < x_1 < \ldots < x_{n-1} < x_n = b.
    \end{equation}

    \ILabel{def:diameter} The \Def{diameter} of the partition \eqref{eq:def:riemann_partition/partition} is defined as
    \begin{equation}\label{eq:def:riemann_partition/diameter}
      \Diam(\{ x_k \}_{k=0}^n) \coloneqq \max_{1 \leq k \leq n} (x_k - x_{k-1}).
    \end{equation}

    \ILabel{def:riemann_partition/refinement} The partition
    \begin{equation*}
      a = y_0 < y_1 < \ldots < y_{m-1} < y_m = b
    \end{equation*}
    is called a \Def{refinement} of the partition \eqref{eq:def:riemann_partition/partition} if we have the \hyperref[def:subset]{set inclusion}
    \begin{equation}\label{eq:def:riemann_partition/refinement/inclusion}
      \{ x_0, x_1, \ldots, x_n \} \subseteq \{ y_0, y_1, \ldots, y_m \}.
    \end{equation}

    \ILabel{def:riemann_partition/tagged} A \Def{tagged partition} of \( [a, b] \) is a partition \eqref{eq:def:riemann_partition/partition} of \( [a, b] \) along with a choice of \( \xi_k \) for each closed interval \( [x_{k-1}, x_k], k = 1, \ldots, n \). That is, a tagged partition is a tuple
    \begin{equation}\label{eq:def:riemann_partition/tagged}
      \Delta \coloneqq \Big( \{ x_k \}_{k=0}^n, \{ \xi_k \}_{k=1}^n \Big).
    \end{equation}

    We call \( x_0, \ldots, x_n \) the \Def{points} of the partition and \( \xi_1, \ldots, \xi_n \) the \Def{tags} of the partition.

    We define the diameter \( \Diam(\Delta) \) of \( \Delta \) to be the diameter \eqref{eq:def:riemann_partition/diameter} and say that the tagged partition \( \Gamma = ( \{ y_k \}_{k=0}^m, \{ \eta_k \}_{k=1}^m ) \) is a refinement of \( \Delta \) if \eqref{eq:def:riemann_partition/refinement/inclusion} holds. Note that we do not require tags to be preserved by refinements.
  \end{DefEnum}
\end{definition}

\begin{remark}\label{remark:set_and_riemann_partitions}
  Note that \eqref{eq:def:riemann_partition/partition} is not a partition in the sense of \fullref{def:set_partition}, however the set of intervals
  \begin{equation*}
    \Big\{ [x_0, x_1), [x_1, x_2), \ldots, [x_{n-2}, x_{n-1}), [x_{n-1}, x_n] \Big\}
  \end{equation*}
  is a set-theoretic partition of \( [a, b] \). Conversely, every finite set-theoretic partition of \( [a, b] \) gives rise to a Riemann partition in the sense of \fullref{def:riemann_partition/partition}.
\end{remark}

\begin{definition}\label{def:riemann_sum}\MarginCite[def. 2]{Gordon1991}
  Let \( \CX \) be a real \hyperref[def:vector_space]{vector space}. Fix a \hyperref[def:function/single_valued]{function} \( f: [a, b] \to \CX \) and a \hyperref[def:riemann_partition/tagged]{tagged partition} \( \Delta = ( \{ x_k \}_{k=0}^n, \{ \xi_k \}_{k=1}^n ) \).

  The \Def{Riemann sum} of \( f \) corresponding to the partition \( \Delta \) is defined as
  \begin{equation*}
    S(f, \Delta) \coloneqq \sum_{k=1}^n f(\xi_k) (x_k - x_{k-1}).
  \end{equation*}
\end{definition}

\begin{definition}\label{def:riemann_integral}\MarginCite[def. 2]{Gordon1991}
  We can make the set \( \Op{Part}([a, b]) \) of all \hyperref[def:riemann_partition/tagged]{tagged partitions} of \( [a, b] \) into a \hyperref[def:directed_set]{directed set} using two common approaches:
  \begin{DefEnum}
    \ILabel{def:riemann_integral/refinement} Put \( \Delta \preceq_R \Gamma \) if and only if \( \Gamma \) is a \hyperref[def:riemann_partition/refinement]{refinement} of \( \Delta \). This actually makes \( (\Op{Part}([a, b]), \preceq_R) \) a \hyperref[def:poset]{poset}.
    \ILabel{def:riemann_integral/diameter} Put \( \Delta \preceq_D \Gamma \) if and only if \( \Diam(\Gamma) \leq \Diam(\Delta) \).
  \end{DefEnum}

  Let \( \CX \) be a real \hyperref[def:separation_axioms/T2]{Hausdorff} \hyperref[def:topological_vector_space]{topological vector space}. Consider the net
  \begin{equation}\label{eq:def:riemann_integral/net}
    \{ S(f, \Delta) \}_{\Delta \in \Op{Part}([a, b])}
  \end{equation}
  of \hyperref[def:riemann_sum]{Riemann sums}. The order on \( \Op{Part}([a, b]) \) does not matter because both give equivalent convergence. If the limit exists, \( f \) is said to be \Def{Riemann integrable} in \( [a, b] \).

  If this net has a \hyperref[def:net_convergence/limit]{limit}, we call it the \Def{Riemann integral} of \( f \) and denote it by
  \begin{equation}\label{eq:def:riemann_integral}
    \int_a^b f(x) dx.
  \end{equation}
\end{definition}
\begin{proof}
  \SubProofImplication{def:riemann_integral/refinement}{def:riemann_integral/diameter} Let \( I \) be the limit \eqref{eq:def:riemann_integral} with respect to \( \preceq_R \). Fix a neighborhood \( U \) of \( 0 \). Since \eqref{eq:def:riemann_integral/net} is eventually in \( I + U \), there exists a tagged partition \( \Delta_0 \) such that \( S(f, \Gamma) \in I + U \) if \( \Gamma \) is a refinement of \( \Delta_0 \).

  Note that \( f \) is \hyperref[def:bounded_function/bounded]{bounded}. Indeed, if\LEM it is unbounded, then there exist points \( x, y \in [a, b] \) such that \( f(x) - f(y) \not\in U \). Therefore any refinement of \( \Delta_0 \) that contains \( x \) and \( y \) as tags will not belong to \( I + U \). This contradicts our choice of \( \Delta_0 \).

  Since \( f \) is bounded, there exists a neighborhood \( V_0 \) of \( 0 \) such that \( f([a, b]) \subseteq V_0 \) and hence \( f(x) - f(y) \in V \coloneqq V_0 - V_0 \) for all \( x, y \in [a, b] \).

  Let \( \Delta \) be a tagger partition such that \( \Diam(\Delta) \leq \Diam(\Delta_0) \). Let \eqref{eq:def:riemann_partition/tagged} be the points and tags of \( \Delta \) and let
  \begin{equation*}
    \Delta_0 = \Big( \{ x_j^{(0)} \}_{j=0}^m, \{ \xi_j^{(0)} \}_{j=1}^m \Big).
  \end{equation*}

  We introduce another partition \( \Gamma \) as follows:
  \begin{enumerate}
    \item Let the points of \( \Gamma \) be the union of the points \( \{ x_k \}_{k=0}^n \) of \( \Delta \) and the points \( \{ x_j^{(0)} \}_{j=0}^m \) of \( \Delta_0 \).

    \item For every \( k = 1, \ldots, n \), denote by \( p_k \) the number of points of \( \Delta_0 \) in the half-open interval \( [x_{k-1}, x_k) \) and put
    \begin{equation*}
      y_{k,j} \coloneqq \begin{cases}
        x_{k-1},                                                                          &j = 0, \\
        x_k,                                                                              &j = p_k, \\
        j\text{-th point of } \{ x_0^{(0)}, \ldots, x_m^{(0)} \} \cap [x_{k-1}, x_k], &0 < j < p_k.
      \end{cases}
    \end{equation*}

    \item For every \( k = 1, \ldots, n \) and every \( j = 0, \ldots, p_k \), choose\LEM an arbitrary tag
    \begin{equation*}
      \eta_{k,j} \in [y_{k,j-1}, y_{k,j}].
    \end{equation*}
  \end{enumerate}

  Then we have
  \begin{BreakableAlign*}
    S(f, \Delta) - I
    &=
    S(f, \Delta) - S(f, \Gamma) + \underbrace{S(f, \Gamma) - I}_{\in U}
    \in \\ &\in
    \sum_{k=1}^n \sum_{j=1}^{p_k} [ \underbrace{f(\xi_k) - f(\eta_{k,j})}_{\in V} ] (y_{k,j} - y_{k,j-1}) + U
    \subseteq \\ &\subseteq
    V \cdot \underbrace{\sum_{k=1}^n \sum_{j=1}^{p_k} (y_{k,j} - y_{k,j-1})}_{b-a} + U
    \subseteq \\ &\subseteq
    (b - a) V + U.
  \end{BreakableAlign*}

  In this result, \( V \) only depends on \( f \) and \( U \) only depends on \( \Delta_0 \). Therefore, given any neighborhood \( W \) of \( 0 \), we can choose \( \Delta_0 \) such that \( U + (b - a) \cdot V \subseteq W \) and hence \( S(f, \Delta) - I \in W \) holds whenever \( \Diam(\Delta) \leq \Diam(\Delta_0) \).

  This finishes the proof.

  \SubProofImplication{def:riemann_integral/diameter}{def:riemann_integral/refinement} Note that if \( \Gamma \) is a refinement of \( \Delta \), clearly \( \Diam(\Gamma) \leq \Diam(\Delta) \). Therefore if the net \eqref{eq:def:riemann_integral/net} with respect to \( \preceq_D \) is eventually in some open set \( U \), the corresponding net with respect to \( \preceq_R \) is also eventually in \( U \). This finishes the proof.
\end{proof}

\begin{corollary}\label{thm:riemann_integrable_implies_bounded}
  A Riemann-integrable function is bounded.
\end{corollary}
\begin{proof}
  Proven in \fullref{def:riemann_integral}.
\end{proof}
