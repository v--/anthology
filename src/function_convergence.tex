\subsection{Function convergence}\label{subsec:function_convergence}

\begin{definition}\label{def:convergence_of_function_at_point}
  Fix two topological spaces \( X \) and \( Y \). Let \( U \in \CT_X \) be a nonempty open set and let \( f: U \to Y \) be a function. We give two equivalent conditions for the \Def{convergence} of \( f \) to \( y_0 \in Y \) at \( x_0 \in \Cl(D) \). In both cases, we write
  \begin{equation*}
    \lim_{x \to x_0} f(x) = y_0.
  \end{equation*}

  \begin{defenum}
    \DItem{def:convergence_of_function_at_point/neighborhoods}(Cauchy-style condition) For every neighborhood \( V \) of \( y_0 \) there exists a neighborhood \( U \) of \( x_0 \) such that \( f(U) \subseteq V \).

    \DItem{def:convergence_of_function_at_point/nets}(Heine-style condition) For every net\Tinyref{def:topological_net} \( \{ x_\al \}_{\al \in \CA} \subseteq U \), for which \( x_0 \) is a limit point\Tinyref{def:topological_net_convergence/limit}, the corresponding net \( \{ f(x_\al) \}_{\al \in \CA} \) has \( y_0 \) as a limit point.
  \end{defenum}
\end{definition}
\begin{proof}
  \begin{description}
    \Implies[def:convergence_of_function_at_point/neighborhoods][def:convergence_of_function_at_point/nets] Let \( \{ x_\al \}_{\al \in \CA} \subseteq U \) be a net  with limit point \( x_0 \). Consider the net \( \{ f(x_\al) \}_{\al \in \CA} \). Fix a neighborhood \( V \) of \( y_0 \). We need to show that \( \{ f(x_\al) \}_{\al \in \CA} \) is eventually in \( V \).

    By \cref{def:convergence_of_function_at_point/neighborhoods}, there exists a neighborhood \( U \) of \( x_0 \) such that \( f(U) \subseteq V \). Since \( x_0 \) is a limit point of \( \{ x_\al \}_{\al \in \CA} \), there exists an index \( \al_0 \) such that for all \( \al \geq \al_0 \), \( x_\al \in U \) and therefore \( f(x_\al) \in V \). Hence \( \{ f(x_\al) \}_{\al \in \CA} \) is eventually in \( V \).

    We conclude that \( y_0 \) is a limit point of the net \( \{ f(x_\al) \}_{\al \in \CA} \) and that the Heine-style condition is satisfied.

    \Implies[def:convergence_of_function_at_point/nets][def:convergence_of_function_at_point/neighborhoods] Suppose that \cref{def:convergence_of_function_at_point/nets} holds while \cref{def:convergence_of_function_at_point/neighborhoods} does not\LEM. Let \( V \) be a neighborhood of \( y_0 \). Then there exists no neighborhood \( U \) of \( x_0 \) such that \( f(U) \subseteq V \).

    For any neighborhood \( U \) of \( x_0 \) and let \( y_U \in f(U) \setminus V \) and \( x_U \in f^{-1} (U) \) so that \( f(x_U) = y_U \). Consider the families
    \begin{align*}
      \{ x_U \}_{U \in \CT(x_0)},
      &&
      \{ f(x_U) \}_{U \in \CT(x_0)},
    \end{align*}
    ordered by reverse inclusion \Tinyref{ex:reverse_inclusion_net} of the neighborhoods \( \CT(x_0) \) of \( x_0 \).

    Note that \( x_0 \) is a limit point of \( \{ x_U \}_{U \in \CT(x_0)} \). By \cref{def:convergence_of_function_at_point/nets}, \( y_0 \) is a limit point of \( \{ f(x_U) \}_{U \in \CT(x_0)} \). But this contradicts our choice of the nets because \( f(x_U) \not\in V \) for any \( U \in \CT(x) \).

    The obtained contradiction demonstrates that \cref{def:convergence_of_function_at_point/nets} implies \cref{def:convergence_of_function_at_point/neighborhoods}.
  \end{description}
\end{proof}
