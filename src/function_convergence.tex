\subsection{Function convergence}\label{subsec:function_convergence}

\begin{definition}\label{def:local_convergence}
  Fix two topological spaces \( X \) and \( Y \). Let \( A \subseteq X \) be a nonempty set and let \( f: A \to Y \) be a function. We give two equivalent definitions for \( y_0 \in Y \) being a \Def{limit point} of \( f \) at \( x_0 \in \Cl(A) \). If \( y_0 \) is the unique limit point (e.g. in Hausdorff spaces\Tinyref{thm:t2_iff_singleton_limits}), we write
  \begin{equation*}
    \lim_{x \to x_0} f(x) = y_0.
  \end{equation*}

  \begin{defenum}
    \DItem{def:local_convergence/neighborhoods}(Cauchy-style condition) For every neighborhood \( V \) of \( y_0 \) there exists a neighborhood \( U \) of \( x_0 \) such that \( f(U \cap A) \subseteq V \).

    \DItem{def:local_convergence/nets}(Heine-style condition) For every net\Tinyref{def:topological_net} \( \{ x_\al \}_{\al \in \CA} \subseteq A \), for which \( x_0 \) is a limit point\Tinyref{def:net_convergence/limit}, the corresponding net \( \{ f(x_\al) \}_{\al \in \CA} \) has \( y_0 \) as a limit point.
  \end{defenum}
\end{definition}
\begin{proof}\mbox{}
  \begin{description}
    \Implies[def:local_convergence/neighborhoods][def:local_convergence/nets] Let \( \{ x_\al \}_{\al \in \CA} \subseteq U \) be a net  with limit point \( x_0 \). Consider the net \( \{ f(x_\al) \}_{\al \in \CA} \). Fix a neighborhood \( V \) of \( y_0 \). We need to show that \( \{ f(x_\al) \}_{\al \in \CA} \) is eventually in \( V \).

    By \fullref{def:local_convergence/neighborhoods}, there exists a neighborhood \( U \) of \( x_0 \) such that \( f(U) \subseteq V \). Since \( x_0 \) is a limit point of \( \{ x_\al \}_{\al \in \CA} \), there exists an index \( \al_0 \) such that for all \( \al \geq \al_0 \), \( x_\al \in U \) and therefore \( f(x_\al) \in V \). Hence \( \{ f(x_\al) \}_{\al \in \CA} \) is eventually in \( V \).

    We conclude that \( y_0 \) is a limit point of the net \( \{ f(x_\al) \}_{\al \in \CA} \) and that the Heine-style condition is satisfied.

    \Implies[def:local_convergence/nets][def:local_convergence/neighborhoods] Suppose that \fullref{def:local_convergence/nets} holds while \fullref{def:local_convergence/neighborhoods} does not\LEM. Let \( V \) be a neighborhood of \( y_0 \). Then there exists no neighborhood \( U \) of \( x_0 \) such that \( f(U) \subseteq V \).

    For any neighborhood \( U \) of \( x_0 \) and let \( y_U \in f(U) \setminus V \) and \( x_U \in f^{-1} (U) \) so that \( f(x_U) = y_U \). Consider the families
    \begin{align*}
      \{ x_U \}_{U \in \CT(x_0)},
      &&
      \{ f(x_U) \}_{U \in \CT(x_0)},
    \end{align*}
    ordered by reverse inclusion \Tinyref{ex:reverse_inclusion_net} of the neighborhoods \( \CT(x_0) \) of \( x_0 \).

    Note that \( x_0 \) is a limit point of \( \{ x_U \}_{U \in \CT(x_0)} \). By \fullref{def:local_convergence/nets}, \( y_0 \) is a limit point of \( \{ f(x_U) \}_{U \in \CT(x_0)} \). But this contradicts our choice of the nets because \( f(x_U) \not\in V \) for any \( U \in \CT(x) \).

    The obtained contradiction demonstrates that \fullref{def:local_convergence/nets} implies \fullref{def:local_convergence/neighborhoods}.
  \end{description}
\end{proof}

\begin{proposition}\label{thm:cauchy_function_convergence_via_subbases}
  Fix two topological spaces \( X \) and \( Y \) and two points \( x_0 \in X \) and \( y_0 \in Y \). Let \( \CP(x_0) \) and \( \CP(y_0) \) be local subbases\Tinyref{def:topological_local_subbase} for the corresponding points. Then the function \( f: X \to Y \) converges\Tinyref{def:local_convergence} to \( y_0 \) at \( x_0 \) if and only if every \( V_P \in \CP(y_0) \) there exists \( U_P \in \CB(x_0) \) such that \( f(U_P) \subseteq V_P \).

  Compare this result to \fullref{thm:net_convergence_via_subbases}.
\end{proposition}
\begin{proof}\mbox{}
  \begin{description}
    \Implies Obvious consequence of \fullref{def:local_convergence/neighborhoods}.
    \ImpliedBy Fix a neighborhood \( V \) of \( x \). We will show that \fullref{def:local_convergence/neighborhoods} holds.

    Let \( \{ V_k \}_{k=1}^n \subseteq \CP(y_0) \) be a family such that \( \bigcap_{k=1}^n V_k \subseteq V \) (such a family exists by definition of a local subbase). By the antecedent of the implication we are proving, for every \( k = 1, \ldots, n \) there exists an \( U_k \in \CP(x_0) \) such that \( f(U_k) \subseteq V_k \). Then \( U \coloneqq \bigcap_{k=1}^n U_k \) is a neighborhood of \( x_0 \) and, furthermore,
    \begin{equation*}
      f(U)
      =
      f\left(\bigcap_{k=1}^n U_k \right)
      \subseteq
      \bigcap_{k=1}^n f(U_k)
      \subseteq
      \bigcap_{k=1}^n V_k
      \subseteq
      V.
    \end{equation*}

    Therefore \fullref{def:local_convergence/neighborhoods} holds.
  \end{description}
\end{proof}
