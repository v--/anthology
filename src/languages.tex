\subsection{Languages}\label{sec:languages}

Languages are used to define formulas for expressing the axioms of set theory\Tinyref{def:set_zfc}. Here, sets are used to formally define languages. This vicious cycle is left to logicians.

\begin{definition}\label{def:language}
  Given a set \( \Cal{A} \), called an \textbf{alphabet}, whose elements are called \textbf{symbols}, we define a \textbf{word} or \textbf{string} over \( \Cal{A} \) to be any tuple\Tinyref{def:cartesian_product} of symbols. Words are written simply as strings of symbols, that is, \( abc \) instead of \( (a, b, c) \). The empty word with no symbols is usually denoted by \( \varepsilon \).

  The set of all (finite) words over \( \Cal{A} \) is denoted by \( \Cal{A}^{*} \). The operation \( * \) is called the \textbf{Kleene star}. A \textbf{language} \( \Cal{L} \) is any subset of \( \Cal{A}^{*} \).

  We define two functions:
  \begin{align*}
    &\Len: \Cal{A}^{*} \to \BB{Z}^{\geq 0}
    &&\cdot: \Cal{A}^{*} \to \BB{Z}^{\geq 0}
    \\
    &\Len(w) \coloneqq \text{length of the tuple } w
    &&v \cdot w \coloneqq (v_1, \ldots, v_{\Len(v)}, w_1, \ldots, w_{\Len(w)}).
  \end{align*}

  The function \( v \cdot w \) is called \textbf{concatenation} and is usually denoted by juxtaposition. It is obviously associative.

  We say that \( p \) is a \textbf{prefix} of \( w \) if the first \( \Len(p) \) symbols of \( w \) are identical to those of \( p \), that is,
  \begin{align*}
    w = (p_1, \ldots, p_{\Len(p)}, w_{\Len(p) + 1}, \ldots, w_{\Len(w)}).
  \end{align*}

  \textbf{Suffixes} are defined analogously. We say that \( v \) is a \textbf{subword} of \( w \) if there exists a prefix \( p \) and a suffix \( s \) such that \( w = pvs \). We define the partial order\Tinyref{def:order/partial} \( v \leq w \iff v \) is a subword of \( w \).

  Evidently both prefixes and suffixes are subwords and \( v \leq w \iff \Len(v) \leq \Len(w) \).

  For convenience, we denote \textbf{runs of length \( n \)} of some letter \( a \) as \( a^n \), that is,
  \begin{align*}
    a^n \coloneqq \begin{cases}
      \varepsilon, &n = 0, \\
      a a^{n-1}, &n > 1.
    \end{cases}
  \end{align*}

  Thus we do not distinguish between the words \( aaabbaa \) and \( a^3 b^2 a^2 \).
\end{definition}

\begin{proposition}\label{thm:set_of_all_words_is_monoid}
  For any alphabet \( \Cal{A} \), the language \( (\Cal{A}^{*}, \cdot) \) is a monoid.
\end{proposition}
