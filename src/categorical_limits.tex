\subsection{Limits}\label{subsec:categorical_limits}

\begin{remark}\label{def:categorical_limit_examples}
  Examples of limits and colimits can be found in \fullref{thm:set_categorical_limits}, \fullref{thm:group_categorical_limits} and \fullref{subsec:initial_final_topologies}.
\end{remark}

\begin{definition}\label{def:diagonal_functor}\cite[143]{Leinster2014}
  Let \( \Bold I \) be a small index category and let \( \Bold A \) be any category. For each object \( A \in \Bold A \), we define the functor \( \Delta A: \Bold I \to \Bold A \) as
  \begin{itemize}
    \item For every object \( i \in \Bold I \), define \( \Delta A(i) = A \)
    \item For every morphism \( u: i \to j \), define \( \Delta A(u) = \Id_A \)
  \end{itemize}

  We combine these functors for every object \( A \in \Bold A \) to obtain the functor \( \Delta: \Bold A \to [\Bold I, \Bold A] \).
\end{definition}

\begin{definition}\label{def:categorical_cone}\cite[definition 5.1.19(a)]{Leinster2014}
  Let \( \Bold A \) be a category and \( \Bold I \) be a category\Tinyref{def:categorical_diagram} which we shall call an \Def{index} category. Let \( D: \Bold I \to \Bold A \) be a diagram. A \Def{cone} on \( D \) can be defined equivalently as:

  \begin{defenum}
    \DItem{def:categorical_cone/explicit} a family of \Def{projection} morphisms \( \{ \pi_i: A \to D(i) \}_{i \in \Bold I} \) from the \Def{vertex} \( A \) such that for all morphisms \( u: i \to j \) in \( \Bold I \), the following diagram commutes:
    \begin{AlignedEquation}\label{def:categorical_cone/universal_property}
      \begin{mplibcode}
      	beginfig(1);
          input metapost/graphs;

          v1 := thelabel("$A$", origin);
          v2 := thelabel("$D(i)$", (-1, -1) scaled u);
          v3 := thelabel("$D(j)$", (1, -1) scaled u);

          a1 := straight_arc(v1, v2);
          a2 := straight_arc(v1, v3);
          a3 := straight_arc(v2, v3);

          draw_vertices(v);
          draw_arcs(a);

          label.ulft("$\pi_i$", straight_arc_midpoint of a1);
          label.urt("$\pi_j$", straight_arc_midpoint of a2);
          label.bot("$D(u)$", straight_arc_midpoint of a3);
        endfig;
      \end{mplibcode}
    \end{AlignedEquation}

    \DItem{def:categorical_cone/natural} a natural transformation in \( [\Bold I, \Bold A](\Delta A, D) \).

    \DItem{def:categorical_cone/comma} an object of the comma category \( (\Delta \downarrow D) \) (see the equivalence proof for details).
  \end{defenum}
\end{definition}
\begin{proof}
  (\ref{def:categorical_cone/explicit} \( \iff \) \ref{def:categorical_cone/natural}) Let \( i, j \in \Bold I \) and \( u: i \to j \). Then a natural transformation \( f \) in  satisfies the following commutative diagram:
  \begin{equation*}
    \begin{mplibcode}
    	beginfig(1);
        input metapost/graphs;

        v1 := thelabel("$\Delta A(i)$", (-1, 0) scaled u);
        v2 := thelabel("$\Delta A(j)$", (1, 0) scaled u);
        v3 := thelabel("$D(i)$", (-1, -1) scaled u);
        v4 := thelabel("$D(j)$", (1, -1) scaled u);

        a1 := straight_arc(v1, v2);
        a2 := straight_arc(v1, v3);
        a3 := straight_arc(v2, v4);
        a4 := straight_arc(v3, v4);

        draw_vertices(v);
        draw_arcs(a);

        label.top("$\Delta A(u)$", straight_arc_midpoint of a1);
        label.lft("$\pi_i$", straight_arc_midpoint of a2);
        label.rt("$\pi_j$", straight_arc_midpoint of a3);
        label.bot("$D(u)$", straight_arc_midpoint of a4);
      endfig;
    \end{mplibcode}
  \end{equation*}

  Since \( \Delta A(i) = \Delta A(j) = A \), the above diagram is the same as \fullref{def:categorical_cone/universal_property}.

  (\ref{def:categorical_cone/natural} \( \iff \) \ref{def:categorical_cone/comma}) We can regard \( D: \Bold I \to \Bold A \) as an object in the functor category \( [\Bold I, \Bold A] \). Since \( \Delta: \Bold A \to [\Bold I, \Bold A] \), an object \( (A, h) \) in \( (\Delta \downarrow D) \) consists of an object \( A \) of \( \Bold A \) and a natural transformation from \( \Delta A \) to \( D \). The converse also applies.
\end{proof}

\begin{definition}\label{def:categorical_limit}\cite[definitions 5.1.19(b), definition 6.3.6]{Leinster2014}
  Let \( \Bold A \) be a category and \( \Bold I \) be an index category. The (unique up to an isomorphism, if it exists) \Def{limit} or \Def{limit cone} \( \varprojlim D \) of \( D \) is a cone 
  \begin{equation*}
    \{ L \overset {\pi_i} \to D(i) \}_{i \in \Bold I}
  \end{equation*}
  such that for every cone
  \begin{equation*}
    \{ L' \overset {\pi_i'} \to D(i) \}_{i \in \Bold I}
  \end{equation*}
  there exists exactly one morphism \( f: L' \to L \) such that \( f \circ \pi_i' = \pi_i, i \in \Bold I \), i.e. the following diagram commutes:
  \begin{equation*}
    \begin{mplibcode}
    	beginfig(1);
        input metapost/graphs;

        v1 := thelabel("$D(i)$", origin);
        v2 := thelabel("$L'$", (-1, 1) scaled u);
        v3 := thelabel("$L$", (1, 1) scaled u);

        a1 := straight_arc(v2, v1);
        a2 := straight_arc(v3, v1);

        d1 := straight_arc(v2, v3);

        draw_vertices(v);
        draw_arcs(a);

        drawarrow d1 dotted;

        label.llft("$\pi_i$", straight_arc_midpoint of a1);
        label.lrt("$\pi_i'$", straight_arc_midpoint of a2);
        label.top("$f$", straight_arc_midpoint of d1);
      endfig;
    \end{mplibcode}
  \end{equation*}

  If the diagram \( \Bold I \) is small, its limit is called a \Def{small limit}. If a category \( \Bold A \) has all small limits, it is called \Def{complete}.
\end{definition}

\begin{definition}\label{def:categorical_product}\cite[definition 5.1.1, 5.1.7]{Leinster2014}
  If the index category \( \Bold I \) is discrete, then any diagram \( D: \Bold I \to \Bold A \) is simply an indexed family \( \{ X_i \}_{i \in \Bold I} \) of objects of \( \Bold A \). In this case, the limit \( L \) does not depend on the functor \( D \). We call it the \Def{product in \( \Bold A \) indexed by \( \Bold I \)} and denote it by \( \prod_{i \in \Bold I} X_i \).

  Explicitly, the \Def{product of \( \{ X \}_{i \in \Bold I} \)} is an object \( P \coloneqq \prod_{i \in \Bold I} X_i \) with associated \Def{projection morphisms} \( \{ \pi_i: P \to X_i \}_{i \in \Bold I} \), that satisfy the following universal property: for any object \( P' \) and any family of morphisms \( \{ \pi_i': P' \to {X_i} \}_{i \in \Bold I} \) there exists exactly one morphism \( f: P' \to P \) such that for every \( i \in \Bold I \) we have \( f \circ \pi_i = \pi_i' \), i.e. the following diagram commutes:
  \begin{equation*}
    \begin{mplibcode}
    	beginfig(1);
        input metapost/graphs;

        v1 := thelabel("$X_i$", origin);
        v2 := thelabel("$P'$", (-1, 1) scaled u);
        v3 := thelabel("$P$", (1, 1) scaled u);

        a1 := straight_arc(v2, v1);
        a2 := straight_arc(v3, v1);

        d1 := straight_arc(v2, v3);

        draw_vertices(v);
        draw_arcs(a);

        drawarrow d1 dotted;

        label.llft("$\pi_i'$", straight_arc_midpoint of a1);
        label.lrt("$\pi_i$", straight_arc_midpoint of a2);
        label.top("$f$", straight_arc_midpoint of d1);
      endfig;
    \end{mplibcode}
  \end{equation*}

  The function \( f \) is also denoted as \( \{ f_i \}_{i \in \Bold I} \).

  In particular, for two objects \( X, Y \in \Bold A \) (i.e. when \( \Bold I \) is a two-object discrete category), the product is an object \( X \times Y \) with projections \( \pi_X: X \times Y \to X \) and \( \pi_Y: X \times Y \to Y \) such that for each object $P'$ and morphisms $\pi_X': P' \to X$ and $\pi_Y': P' \to Y$ the following diagram commutes:
  \begin{equation*}
    \begin{mplibcode}
    	beginfig(1);
        input metapost/graphs;

        v1 := thelabel("$P'$", (-1, 1) scaled u);
        v2 := thelabel("$X \times Y$", origin);
        v3 := thelabel("$X$", (0, -1) scaled u);
        v4 := thelabel("$Y$", (1, 0) scaled u);

        a1 := straight_arc(v1, v3);
        a2 := straight_arc(v1, v4);
        a3 := straight_arc(v2, v3);
        a4 := straight_arc(v2, v4);

        d1 := straight_arc(v1, v2);

        draw_vertices(v);
        draw_arcs(a);

        drawarrow d1 dotted;

        label.llft("$\pi_X'$", straight_arc_midpoint of a1);
        label.urt("$\pi_Y'$", straight_arc_midpoint of a2);
        label.rt("$\pi_X$", straight_arc_midpoint of a3);
        label.bot("$\pi_Y$", straight_arc_midpoint of a4);

        fill fullcircle scaled 0.25u shifted (center d1) withcolor white;
        label("$f$", straight_arc_midpoint of d1);
      endfig;
    \end{mplibcode}
  \end{equation*}
\end{definition}

\begin{remark}\label{remark:small_categorical_product}
  If the discrete category \( \Bold I \) is small, denote the set of its objects by \( I \). This allows us to talk about products of families \( \{ X_i \}_{i \in I} \) indexed by the set \( I \) rather than the category \( \Bold I \).
\end{remark}

\begin{remark}\label{remark:empty_categorical_product}
  The product \( \prod_{i \in \varnothing} X_i \) of an empty family of objects is the terminal object of the category.
\end{remark}

\begin{definition}\label{def:categorical_fork}\cite[112]{Leinster2014}
  A \Def{fork} in the category \( \Bold A \) is a commutative diagram of the form
  \begin{equation*}
    \begin{mplibcode}
    	beginfig(1);
        input metapost/graphs;

        v1 := thelabel("$A$", origin);
        v2 := thelabel("$X$", (1, 0) scaled u);
        v3 := thelabel("$Y$", (2, 0) scaled u);

        a1 := straight_arc(v1, v2);
        a2 := straight_arc_shifted(v2, v3, (0, safe_arc_spacing));
        a3 := straight_arc_shifted(v2, v3, (0, -safe_arc_spacing));

        draw_vertices(v);
        draw_arcs(a);

        label.top("$f$", straight_arc_midpoint of a1);
        label.top("$s$", straight_arc_midpoint of a2);
        label.bot("$t$", straight_arc_midpoint of a3);
      endfig;
    \end{mplibcode}
  \end{equation*}

  Commutativity simply means that \( s \circ f = t \circ f \).
\end{definition}

\begin{definition}\label{def:categorical_equalizer}\cite[definition 5.1.11]{Leinster2014}
  Assume that the index category \( \Bold I \) consists of two objects and two unidirectional morphisms:
  \begin{equation*}
    \begin{mplibcode}
    	beginfig(1);
        input metapost/graphs;

        v1 := thelabel("$\bullet$", (1, 0) scaled u);
        v2 := thelabel("$\bullet$", (2, 0) scaled u);

        a1 := straight_arc_shifted(v1, v2, (0, safe_arc_spacing));
        a2 := straight_arc_shifted(v1, v2, (0, -safe_arc_spacing));

        draw_vertices(v);
        draw_arcs(a);
      endfig;
    \end{mplibcode}
  \end{equation*}

  Diagrams \( D \) of shape \( \Bold I \) are simply subcategories of \( \Bold A \) of the shape
  \begin{equation*}
    \begin{mplibcode}
    	beginfig(1);
        input metapost/graphs;

        v1 := thelabel("$X$", (1, 0) scaled u);
        v2 := thelabel("$Y$", (2, 0) scaled u);

        a1 := straight_arc_shifted(v1, v2, (0, safe_arc_spacing));
        a2 := straight_arc_shifted(v1, v2, (0, -safe_arc_spacing));

        draw_vertices(v);
        draw_arcs(a);

        label.top("$s$", straight_arc_midpoint of a1);
        label.bot("$t$", straight_arc_midpoint of a2);
      endfig;
    \end{mplibcode}
  \end{equation*}

  Cones with vertex \( A \) are then given by commutative diagrams of shape
  \begin{equation*}
    \begin{mplibcode}
    	beginfig(1);
        input metapost/graphs;

        v1 := thelabel("$A$", origin);
        v2 := thelabel("$X$", (-1, -1) scaled u);
        v3 := thelabel("$Y$", (1, -1) scaled u);

        a1 := straight_arc(v1, v2);
        a2 := straight_arc(v1, v3);
        a3 := straight_arc_shifted(v2, v3, (0, safe_arc_spacing));
        a4 := straight_arc_shifted(v2, v3, (0, -safe_arc_spacing));

        draw_vertices(v);
        draw_arcs(a);

        label.ulft("$f$", straight_arc_midpoint of a1);
        label.urt("$g$", straight_arc_midpoint of a2);
        label.top("$s$", straight_arc_midpoint of a3);
        label.bot("$t$", straight_arc_midpoint of a4);
      endfig;
    \end{mplibcode}
  \end{equation*}

  Since the morphism \( g: A \to Y \) is determined uniquely by \( f \) and \( s \), the cones are actually forks:
  \begin{equation*}
    \begin{mplibcode}
    	beginfig(1);
        input metapost/graphs;

        v1 := thelabel("$A$", origin);
        v2 := thelabel("$X$", (1, 0) scaled u);
        v3 := thelabel("$Y$", (2, 0) scaled u);

        a1 := straight_arc(v1, v2);
        a2 := straight_arc_shifted(v2, v3, (0, safe_arc_spacing));
        a3 := straight_arc_shifted(v2, v3, (0, -safe_arc_spacing));

        draw_vertices(v);
        draw_arcs(a);

        label.top("$f$", straight_arc_midpoint of a1);
        label.top("$s$", straight_arc_midpoint of a2);
        label.bot("$t$", straight_arc_midpoint of a3);
      endfig;
    \end{mplibcode}
  \end{equation*}

  The limit \( (L, l) \) of \( D \) then satisfies the universal property: for any fork \( (L', l') \), there exists a unique morphism \( f: L' \to L \) such that the following diagram commutes:
  \begin{equation*}
    \begin{mplibcode}
    	beginfig(1);
        input metapost/graphs;

        v1 := thelabel("$X$", origin);
        v2 := thelabel("$Y$", (1, 0) scaled u);
        v3 := thelabel("$L'$", (-1, 1) scaled u);
        v4 := thelabel("$L$", (-1, -1) scaled u);

        a1 := straight_arc_shifted(v1, v2, (0, safe_arc_spacing));
        a2 := straight_arc_shifted(v1, v2, (0, -safe_arc_spacing));
        a3 := straight_arc(v3, v1);
        a4 := straight_arc(v4, v1);

        d1 := straight_arc(v3, v4);

        draw_vertices(v);
        draw_arcs(a);

        drawarrow d1 dotted;

        label.top("$s$", straight_arc_midpoint of a1);
        label.bot("$t$", straight_arc_midpoint of a2);
        label.urt("$l'$", straight_arc_midpoint of a3);
        label.lrt("$l$", straight_arc_midpoint of a4);
        label.rt("$f$", straight_arc_midpoint of d1);
      endfig;
    \end{mplibcode}
  \end{equation*}

  This limit is called the \Def{equalizer} of \( s \) and \( t \).
\end{definition}

\begin{definition}\label{def:categorical_pullback}\cite[definition 5.1.16]{Leinster2014}
  Assume that the index category \( \Bold I \) has the shape
  \begin{equation*}
    \bullet \longrightarrow \bullet \longleftarrow \bullet
  \end{equation*}

  Cones of shape \( \Bold I \) with vertex \( A \) are then given by commutative diagrams of shape
  \begin{equation*}
    \begin{mplibcode}
    	beginfig(1);
        input metapost/graphs;

        v1 := thelabel("$A$", origin);
        v2 := thelabel("$X$", (0, -1) scaled u);
        v3 := thelabel("$Y$", (1, 0) scaled u);
        v4 := thelabel("$Z$", (1, -1) scaled u);

        a1 := straight_arc(v1, v2);
        a2 := straight_arc(v1, v3);
        a3 := straight_arc(v2, v4);
        a4 := straight_arc(v3, v4);

        draw_vertices(v);
        draw_arcs(a);

        label.lft("$\pi_X$", straight_arc_midpoint of a1);
        label.top("$\pi_Y$", straight_arc_midpoint of a2);
        label.bot("$s$", straight_arc_midpoint of a3);
        label.rt("$t$", straight_arc_midpoint of a4);
      endfig;
    \end{mplibcode}
  \end{equation*}

  The limit \( (L, \pi_X, \pi_Y) \) then satisfies the universal property: for any \( \Bold I \)-cone \( (L', \pi_X', \pi_Y') \), there exists a unique morphism \( f: L' \to L \) such that the following diagram commutes:
  \begin{equation*}
    \begin{mplibcode}
    	beginfig(1);
        input metapost/graphs;

        v1 := thelabel("$L$", origin);
        v2 := thelabel("$X$", (0, -1) scaled u);
        v3 := thelabel("$Y$", (1, 0) scaled u);
        v4 := thelabel("$Z$", (1, -1) scaled u);
        v5 := thelabel("$L'$", (-1, 1) scaled u);

        a1 := straight_arc(v1, v2);
        a2 := straight_arc(v1, v3);
        a3 := straight_arc(v2, v4);
        a4 := straight_arc(v3, v4);
        a5 := straight_arc(v5, v2);
        a6 := straight_arc(v5, v3);

        d1 := straight_arc(v5, v1);

        draw_vertices(v);
        draw_arcs(a);

        drawarrow d1 dotted;

        label.rt("$\pi_X$", straight_arc_midpoint of a1);
        label.bot("$\pi_Y$", straight_arc_midpoint of a2);
        label.bot("$s$", straight_arc_midpoint of a3);
        label.rt("$t$", straight_arc_midpoint of a4);
        label.llft("$\pi_X'$", straight_arc_midpoint of a5);
        label.urt("$\pi_Y'$", straight_arc_midpoint of a6);

        fill fullcircle scaled 0.25u shifted (center d1) withcolor white;
        label("$f$", straight_arc_midpoint of d1);
      endfig;
    \end{mplibcode}
  \end{equation*}

  This limit is called the \Def{pullback} or \Def{fibered product} of \( s \) and \( t \).
\end{definition}

\begin{definition}\label{def:categorical_cocone}\cite[definition 5.2.1]{Leinster2014}
  The dual notion of a cone\Tinyref{def:categorical_cone} is that of a cocone. Given a category \( \Bold A \), an index category \( \Bold I \) and a diagram \( D: \Bold I \to \Bold A \), we say that the family of morphisms
  \begin{equation*}
    \{ D(i) \overset {\iota_i} \to A \}_{i \in \Bold I}
  \end{equation*}
  is a \Def{cocone} for D if it is a cone for \( D^{-1}: \Cat{I}^{-1} \to \Bold{A}^{-1} \).

  Explicitly, a \Def{cocone} on \( D \) consists of
  \begin{itemize}
    \item an object \( A \in \Bold A \), called the \Def{vertex} of the cocone
    \item a family of \Def{coprojection} morphisms \( \{ \iota_i: D(i) \to A \}_{i \in \Bold I} \)
  \end{itemize}
  such that for all morphisms \( u: i \to j \) in \( \Bold I \), the following diagram commutes:
  \begin{AlignedEquation}\label{def:categorical_cocone/universal_property}
    \begin{mplibcode}
    	beginfig(1);
        input metapost/graphs;

        v1 := thelabel("$A$", origin);
        v2 := thelabel("$D(i)$", (-1, 1) scaled u);
        v3 := thelabel("$D(j)$", (1, 1) scaled u);

        a1 := straight_arc(v2, v1);
        a2 := straight_arc(v3, v1);
        a3 := straight_arc(v2, v3);

        draw_vertices(v);
        draw_arcs(a);

        label.llft("$\iota_i$", straight_arc_midpoint of a1);
        label.lrt("$\iota_j$", straight_arc_midpoint of a2);
        label.top("$D(u)$", straight_arc_midpoint of a3);
      endfig;
    \end{mplibcode}
  \end{AlignedEquation}
\end{definition}

\begin{definition}\label{def:categorical_colimit}\cite[definition 5.1.19(b)]{Leinster2014}
  Analogously to limits\Tinyref{def:categorical_limit}, we define the \Def{colimit} \( \varinjlim D \) of \( D \) to be a cocone 
  \begin{equation*}
    \{ D(i) \overset {\iota_i} \to L \}_{i \in \Bold I}
  \end{equation*}
  such that for every cocone
  \begin{equation*}
    \{ D(i) \overset {\iota_i'} \to L' \}_{i \in \Bold I}
  \end{equation*}
  there exists exactly one morphism \( f: L \to L' \) such that \( \iota_i' = f \circ \iota_i, i \in \Bold I \), i.e. the following diagram commutes:
  \begin{equation*}
    \begin{mplibcode}
    	beginfig(1);
        input metapost/graphs;

        v1 := thelabel("$D(i)$", origin);
        v2 := thelabel("$L'$", (-1, -1) scaled u);
        v3 := thelabel("$L$", (1, -1) scaled u);

        a1 := straight_arc(v1, v2);
        a2 := straight_arc(v1, v3);

        d1 := straight_arc(v3, v2);

        draw_vertices(v);
        draw_arcs(a);

        drawarrow d1 dotted;

        label.ulft("$\iota_i'$", straight_arc_midpoint of a1);
        label.urt("$\iota_i$", straight_arc_midpoint of a2);
        label.top("$f$", straight_arc_midpoint of d1);
      endfig;
    \end{mplibcode}
  \end{equation*}

  If all small colimits exist, we say that \( \Bold A \) is a \Def{cocomplete category}.
\end{definition}

\begin{definition}\label{def:cocomplete_category}
  If a category is both complete\Tinyref{def:categorical_limit} and cocomplete\Tinyref{def:categorical_colimit}, it is said to be a \Def{cocomplete category}.
\end{definition}

\begin{definition}\label{def:categorical_coproduct}\cite[definition 5.2.2]{Leinster2014}
  If the index category \( \Bold I \) is discrete, specifying a functor \( D: \Bold I \to \Bold A \) is analogous to specifying a \( \Bold I \)-indexed family \( \{ X \}_{i \in \Bold I} \) of objects in \( \Bold A \)\Tinyref{def:categorical_product}.

  The \Def{coproduct} or \Def{categorical sum}
  \begin{equation*}
    S \coloneqq \coprod_{i \in \Bold I} X_i = \sum_{i \in \Bold I} X_i.
  \end{equation*}
  satisfies the following universal property: for any object \( S' \) and any family of morphisms \( \{ \iota_i': {X_i} \to S' \}_{i \in \Bold I} \) there exists exactly one morphism \( f: S \to S' \) such that for every \( i \in \Bold I \) we have \( \iota_i' = f \circ \iota_i \), i.e. the following diagram commutes:
  \begin{equation*}
    \begin{mplibcode}
    	beginfig(1);
        input metapost/graphs;

        v1 := thelabel("$X_j$", origin);
        v2 := thelabel("$S'$", (-1, 1) scaled u);
        v3 := thelabel("$S$", (1, 1) scaled u);

        a1 := straight_arc(v1, v2);
        a2 := straight_arc(v1, v3);

        d1 := straight_arc(v3, v2);

        draw_vertices(v);
        draw_arcs(a);

        drawarrow d1 dotted;

        label.llft("$\iota_i'$", straight_arc_midpoint of a1);
        label.lrt("$\iota_i$", straight_arc_midpoint of a2);
        label.top("$f$", straight_arc_midpoint of d1);
      endfig;
    \end{mplibcode}
  \end{equation*}

  The function \( f \) is also denoted as \( \{ f_i \}_{i \in \Bold I} \).

  In particular, for two objects \( X, Y \in \Bold A \), the coproduct is an object \( X + Y \) with coprojections \( \pi_X: X \to X \times Y \) and \( \pi_Y: Y \to X \times Y \) such that for each object $S'$ and morphisms $\iota_X': X \to S'$ and $\iota_Y': X \to P'$ the following diagram commutes:
  \begin{equation*}
    \begin{mplibcode}
    	beginfig(1);
        input metapost/graphs;

        v1 := thelabel("$S'$", (-1, 1) scaled u);
        v2 := thelabel("$X + Y$", origin);
        v3 := thelabel("$X$", (0, -1) scaled u);
        v4 := thelabel("$Y$", (1, 0) scaled u);

        a1 := straight_arc(v3, v1);
        a2 := straight_arc(v4, v1);
        a3 := straight_arc(v3, v2);
        a4 := straight_arc(v4, v2);

        d1 := straight_arc(v2, v1);

        draw_vertices(v);
        draw_arcs(a);

        drawarrow d1 dotted;

        label.llft("$\iota_X'$", straight_arc_midpoint of a1);
        label.urt("$\iota_Y'$", straight_arc_midpoint of a2);
        label.rt("$\iota_X$", straight_arc_midpoint of a3);
        label.bot("$\iota_Y$", straight_arc_midpoint of a4);

        fill fullcircle scaled 0.25u shifted (center d1) withcolor white;
        label("$f$", straight_arc_midpoint of d1);
      endfig;
    \end{mplibcode}
  \end{equation*}
\end{definition}

\begin{remark}\label{remark:empty_categorical_coproduct}
  The coproduct \( \prod_{i \in \varnothing} X_i \) of an empty family of objects is the initial object of the category.
\end{remark}

\begin{definition}\label{def:categorical_coequalizer}\cite[definition 5.2.7]{Leinster2014}
  As for equalizers,\Tinyref{def:categorical_coequalizer}, assume that the index category \( \Bold I \cong \Cat{I}^{-1} \) consists of two objects and two unidirectional morphisms:
  \begin{equation*}
    \begin{mplibcode}
    	beginfig(1);
        input metapost/graphs;

        v1 := thelabel("$\bullet$", (1, 0) scaled u);
        v2 := thelabel("$\bullet$", (2, 0) scaled u);

        a1 := straight_arc_shifted(v1, v2, (0, safe_arc_spacing));
        a2 := straight_arc_shifted(v1, v2, (0, -safe_arc_spacing));

        draw_vertices(v);
        draw_arcs(a);
      endfig;
    \end{mplibcode}
  \end{equation*}

  Cocones with vertex \( A \) are then given by commutative diagrams of shape
  \begin{equation*}
    \begin{mplibcode}
    	beginfig(1);
        input metapost/graphs;

        v1 := thelabel("$A$", (3, 0) scaled u);
        v2 := thelabel("$X$", (1, 0) scaled u);
        v3 := thelabel("$Y$", (2, 0) scaled u);

        a1 := straight_arc(v3, v1);
        a2 := straight_arc_shifted(v2, v3, (0, safe_arc_spacing));
        a3 := straight_arc_shifted(v2, v3, (0, -safe_arc_spacing));

        draw_vertices(v);
        draw_arcs(a);

        label.top("$f$", straight_arc_midpoint of a1);
        label.top("$s$", straight_arc_midpoint of a2);
        label.bot("$t$", straight_arc_midpoint of a3);
      endfig;
    \end{mplibcode}
  \end{equation*}

  The \Def{coequalizer} \( (L, l) \) then satisfies the universal property: for any \( \Bold I \)-cocone \( (L', l') \), there exists a unique morphism \( f: L \to L' \) such that the following diagram commutes:
  \begin{equation*}
    \begin{mplibcode}
    	beginfig(1);
        input metapost/graphs;

        v1 := thelabel("$X$", origin);
        v2 := thelabel("$Y$", (1, 0) scaled u);
        v3 := thelabel("$L'$", (2, 1) scaled u);
        v4 := thelabel("$L$", (2, -1) scaled u);

        a1 := straight_arc_shifted(v1, v2, (0, safe_arc_spacing));
        a2 := straight_arc_shifted(v1, v2, (0, -safe_arc_spacing));
        a3 := straight_arc(v2, v3);
        a4 := straight_arc(v2, v4);

        d1 := straight_arc(v4, v3);

        draw_vertices(v);
        draw_arcs(a);

        drawarrow d1 dotted;

        label.top("$s$", straight_arc_midpoint of a1);
        label.bot("$t$", straight_arc_midpoint of a2);
        label.ulft("$l'$", straight_arc_midpoint of a3);
        label.llft("$l$", straight_arc_midpoint of a4);
        label.rt("$f$", straight_arc_midpoint of d1);
      endfig;
    \end{mplibcode}
  \end{equation*}
\end{definition}

\begin{definition}\label{def:categorical_pushout}\cite[definition 5.2.11]{Leinster2014}
  A \Def{pushout} in \( \Bold A \) is a \Def{pullback} in \( \Bold A^{-1} \).

  Explicitly, the index category \( \Bold I \) has the shape
  \begin{equation*}
    \bullet \longleftarrow \bullet \longrightarrow \bullet
  \end{equation*}

  Cocones of shape \( \Bold I \) with vertex \( A \) are then given by commutative diagrams of shape
  \begin{equation*}
    \begin{mplibcode}
    	beginfig(1);
        input metapost/graphs;

        v1 := thelabel("$A$", origin);
        v2 := thelabel("$X$", (-1, 0) scaled u);
        v3 := thelabel("$Y$", (0, 1) scaled u);
        v4 := thelabel("$Z$", (-1, 1) scaled u);

        a1 := straight_arc(v2, v1);
        a2 := straight_arc(v3, v1);
        a3 := straight_arc(v4, v2);
        a4 := straight_arc(v4, v3);

        draw_vertices(v);
        draw_arcs(a);

        label.bot("$\iota_X$", straight_arc_midpoint of a1);
        label.rt("$\iota_Y$", straight_arc_midpoint of a2);
        label.lft("$s$", straight_arc_midpoint of a3);
        label.top("$t$", straight_arc_midpoint of a4);
      endfig;
    \end{mplibcode}
  \end{equation*}

  The pushout \( (L, \iota_X, \iota_Y) \) of \( D \) then satisfies the universal property: for any \( \Bold I \)-cocone \( (L', \iota_X', \iota_Y') \), there exists a unique morphism \( f: L \to L' \) such that the following diagram commutes:
  \begin{equation*}
    \begin{mplibcode}
    	beginfig(1);
        input metapost/graphs;

        v1 := thelabel("$L$", origin);
        v2 := thelabel("$X$", (-1, 0) scaled u);
        v3 := thelabel("$Y$", (0, 1) scaled u);
        v4 := thelabel("$Z$", (-1, 1) scaled u);
        v5 := thelabel("$L'$", (1, -1) scaled u);

        a1 := straight_arc(v2, v1);
        a2 := straight_arc(v3, v1);
        a3 := straight_arc(v4, v2);
        a4 := straight_arc(v4, v3);
        a5 := straight_arc(v2, v5);
        a6 := straight_arc(v3, v5);

        draw_vertices(v);
        draw_arcs(a);

        d1 := straight_arc(v1, v5);

        draw_vertices(v);
        draw_arcs(a);

        drawarrow d1 dotted;

        label.top("$\iota_X$", straight_arc_midpoint of a1);
        label.lft("$\iota_Y$", straight_arc_midpoint of a2);
        label.lft("$s$", straight_arc_midpoint of a3);
        label.top("$t$", straight_arc_midpoint of a4);
        label.llft("$\iota_X'$", straight_arc_midpoint of a5);
        label.urt("$\iota_Y'$", straight_arc_midpoint of a6);

        fill fullcircle scaled 0.25u shifted (center d1) withcolor white;
        label("$f$", straight_arc_midpoint of d1);
      endfig;
    \end{mplibcode}
  \end{equation*}
\end{definition}

\begin{definition}\label{def:categorical_limit_preservation}\cite[definitions 5.3.1, 5.3.5]{Leinster2014}
  Let \( F: \Bold A \to \Bold B \) be a functor. We say that
  \begin{defenum}
    \DItem{def:categorical_limit_preservation/preserve} \( F \) \Def{preserves} limits of shape \( \Bold I \) for some index category \( \Bold I \) if, given a \( \Bold I \)-shaped limit cone
    \begin{equation*}
     \{ L \overset {\pi_i} \to D(i) \}_{i \in \Bold I},
    \end{equation*}
    its image
    \begin{equation*}
      \{ F(L) \overset {F(\pi_i)} \to F(D(i)) \}_{i \in \Bold I}
    \end{equation*}
    is also a limit cone. We say that \( F \) simply preserves limits if it preserves limits for every index category \( \Bold I \).

    \DItem{def:categorical_limit_preservation/reflect} \( F \) \Def{reflects} limits of shape \( \Bold I \) if, given any \( \Bold I \)-shaped cone, if its image is a limit cone, then is it itself a limit cone.

    \DItem{def:categorical_limit_preservation/create} \( F \) \Def{creates} limits of shape \( \Bold I \) if it both preserves and reflects limits.

    \DItem{def:categorical_limit_preservation/lift} \( F \) \Def{lifts} limits of shape \( \Bold I \) if, given a diagram \( D: \Bold I \to \Bold B \), any limit cone \( \varprojlim D \) is the image of some limit cone in \( A \).
  \end{defenum}
\end{definition}

\begin{remark}\label{remark:categorical_colimit_preservation}
  Analogous definitions can be given for colimits.
\end{remark}
