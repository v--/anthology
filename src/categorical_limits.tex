\section{Limits}\label{sec:categorical_limits}

\begin{note}\label{def:categorical_limit_examples}
  Examples of limits and colimits can be found in \cref{thm:set_categorical_limits}, \cref{thm:group_categorical_limits} and \cref{sec:initial_final_topologies}.
\end{note}

\begin{definition}\label{def:diagonal_functor}\cite[143]{Leinster2014}
  Let \( \Bold I \) be a small index category and let \( \Bold A \) be any category. For each object \( A \in \Bold A \), we define the functor \( \Delta A: \Bold I \to \Bold A \) as
  \begin{itemize}
    \item For every object \( i \in \Bold I \), define \( \Delta A(i) = A \)
    \item For every morphism \( u: i \to j \), define \( \Delta A(u) = \Id_A \)
  \end{itemize}

  We combine these functors for every object \( A \in \Bold A \) to obtain the functor \( \Delta: \Bold A \to [\Bold I, \Bold A] \).
\end{definition}

\begin{definition}\label{def:categorical_cone}\cite[definition 5.1.19(a)]{Leinster2014}
  Let \( \Bold A \) be a category and \( \Bold I \) be an \underLine{index} category\Tinyref{def:categorical_diagram}. Let \( D: \Bold I \to \Bold A \) be a diagram. A \underLine{cone} on \( D \) can be defined equivalently as:

  \begin{defenum}
    \item\label{def:categorical_cone/explicit} a family of \underLine{projection} morphisms \( \{ f_i: A \to D(i) \}_{i \in \Bold I} \) from the \underLine{vertex} \( A \) such that for all morphisms \( u: i \to j \) in \( \Bold I \), the following diagram commutes:
    \begin{equation}\label{def:categorical_cone/universal_property}
      \begin{tikzcd}[baseline=(current bounding box.center)]
        & A \arrow[ld, "f_i"'] \arrow[rd, "f_j"] & \\
        D(i) \arrow[rr, "D(u)"'] & & D(j)
      \end{tikzcd}
    \end{equation}

    \item\label{def:categorical_cone/natural} a natural transformation in \( [\Bold I, \Bold A](\Delta A, D) \).

    \item\label{def:categorical_cone/comma} an object of the comma category \( (\Delta \downarrow D) \) (see the equivalence proof for details).
  \end{defenum}
\end{definition}
\begin{proof}
  (\ref{def:categorical_cone/explicit} \( \iff \) \ref{def:categorical_cone/natural}) Let \( i, j \in \Bold I \) and \( u: i \to j \). Then a natural transformation \( f \) in  satisfies the following commutative diagram:
  \begin{Center}
    \begin{tikzcd}
      \Delta A(i) \arrow[r, "\Delta A(u)"] \arrow[d, "f_i"] & \Delta A(j) \arrow[d, "f_j"] \\
      D(i) \arrow[r, "D(u)"]                       & D(j)
    \end{tikzcd}
  \end{Center}

  Since \( \Delta A(i) = \Delta A(j) = A \), the above diagram is the same as \cref{def:categorical_cone/universal_property}.

  (\ref{def:categorical_cone/natural} \( \iff \) \ref{def:categorical_cone/comma}) We can regard \( D: \Bold I \to \Bold A \) as an object in the functor category \( [\Bold I, \Bold A] \). Since \( \Delta: \Bold A \to [\Bold I, \Bold A] \), an object \( (A, h) \) in \( (\Delta \downarrow D) \) consists of an object \( A \) of \( \Bold A \) and a natural transformation from \( \Delta A \) to \( D \). The converse also applies.
\end{proof}

\begin{definition}\label{def:categorical_limit}\cite[definitions 5.1.19(b), definition 6.3.6]{Leinster2014}
  Let \( \Bold A \) be a category and \( \Bold I \) be an index category. The (unique up to an isomorphism, if it exists) \underLine{limit} or \underLine{limit cone} \( \varprojlim D \) of \( D \) is a cone \( \{ L \overset {p_i} \to D(i) \}_{i \in \Bold I} \) such that for every cone \( \{ A \overset {f_i} \to D(i) \}_{i \in \Bold I} \) there exists exactly one morphism \( f: A \to L \) such that \( f \circ p_i = f_i, i \in \Bold I \), i.e. the following diagram commutes:
  \begin{Center}
    \begin{tikzcd}
                                                  & D(i) & \\
      A \arrow[ru, "f_i"] \arrow[rr, "f", dotted] &      & L \arrow[lu, "p_i"']
    \end{tikzcd}
  \end{Center}

  If the diagram \( \Bold I \) is small, its limit is called a \underLine{small limit}. If a category \( \Bold A \) has all small limits, it is called \underLine{complete}.
\end{definition}

\begin{definition}\label{def:categorical_product}\cite[definition 5.1.1, 5.1.7]{Leinster2014}
  If the index category \( \Bold I \) is discrete, then any diagram \( D: \Bold I \to \Bold A \) is simply an indexed family \( \{ X_i \}_{i \in \Bold I} \) of objects of \( \Bold A \). In this case, the limit \( L \) does not depend on the functor \( D \). We call it the \underLine{product in \( \Bold A \) indexed by \( \Bold I \)} and denote it by \( \prod_{i \in \Bold I} X_i \).

  Explicitly, the \underLine{product of \( \{ X \}_{i \in \Bold I} \)} is an object \( \prod_{i \in \Bold I} X_i \) with associated \underLine{projection morphisms} \( \{ p_j: \prod_{i \in \Bold I} X_i \to X_j \}_{j \in \Bold I} \), that satisfy the following universal property: for any object \( A \) and any family of morphisms \( \{ f_j: A \to {X_j} \}_{j \in \Bold I} \) there exists exactly one morphism \( f: A \to \prod_{i \in \Bold I} X_i \) such that for every \( j \in \Bold I \) we have \( p_j \circ f = f_j \), i.e. the following diagram commutes:
  \begin{Center}
    \begin{tikzcd}
      A \arrow[rd, "f_j"'] \arrow[rr, "f", dotted] & & \prod_{i \in \Bold I} X_i \arrow[ld, "p_j"] \\
      & X_j &
    \end{tikzcd}
  \end{Center}

  The function \( f \) is also denoted as \( \{ f_i \}_{i \in \Bold I} \).

  In particular, for two objects \( X, Y \in \Bold A \) (i.e. when \( \Bold I \) is a two-object discrete category), the product is an object \( X \times Y \) with projections \( p_X: X \times Y \to X \) and \( p_Y: X \times Y \to Y \) such that the following diagram commutes:
  \begin{Center}
    \begin{tikzcd}
      A \arrow[rd, "f" description, dotted] \arrow[rrd, "f_Y"] \arrow[rdd, "f_X"'] &                                               &   \\
                                                                                   & X \times Y \arrow[r, "p_Y"'] \arrow[d, "p_X"] & Y \\
                                                                                   & X                                             &
    \end{tikzcd}
  \end{Center}
\end{definition}

\begin{note}\label{note:small_categorical_product}
  If the discrete category \( \Bold I \) is small, denote the set of its objects by \( I \). This allows us to talk about products of families \( \{ X_i \}_{i \in I} \) indexed by the set \( I \) rather than the category \( \Bold I \).
\end{note}

\begin{note}\label{note:empty_categorical_product}
  The product \( \prod_{i \in \varnothing} X_i \) of an empty family of objects is the terminal object of the category.
\end{note}

\begin{definition}\label{def:categorical_fork}\cite[112]{Leinster2014}
  A \underLine{fork} in the category \( \Bold A \) is a commutative diagram of the form
  \begin{Center}
    \begin{tikzcd}
      A \arrow[r, "f"] & X \arrow[r, shift left=1, "s"] \arrow[r, shift right=1, "t"'] & Y
    \end{tikzcd}
  \end{Center}

  Commutativity simply means that \( s \circ f = t \circ f \).
\end{definition}

\begin{definition}\label{def:categorical_equalizer}\cite[definition 5.1.11]{Leinster2014}
  Assume that the index category \( \Bold I \) consists of two objects and two unidirectional morphisms:
  \begin{Center}
    \begin{tikzcd}
      \bullet \arrow[r, shift left=1] \arrow[r, shift right=1] & \bullet
    \end{tikzcd}
  \end{Center}

  Diagrams \( D \) of shape \( \Bold I \) are simply subcategories of \( \Bold A \) of the shape
  \begin{Center}
    \begin{tikzcd}
      X \arrow[r, shift left=1, "s"] \arrow[r, shift right=1, "t"'] & Y
    \end{tikzcd}
  \end{Center}

  Cones with vertex \( A \) are then given by commutative diagrams of shape
  \begin{Center}
    \begin{tikzcd}
      & A \arrow[ld, "f"'] \arrow[rd, "g"] & \\
      X \arrow[rr, shift left=1, "s"] \arrow[rr, shift right=1, "t"'] && Y
    \end{tikzcd}
  \end{Center}

  Since the morphism \( g: A \to Y \) is determined uniquely by \( f \) and \( s \), the cones are actually forks:
  \begin{Center}
    \begin{tikzcd}
      A \arrow[r, "f"] & X \arrow[r, shift left=1, "s"] \arrow[r, shift right=1, "t"'] & Y
    \end{tikzcd}
  \end{Center}

  The limit \( (L, l) \) of \( D \) then satisfies the universal property: for any fork \( (A, g) \), there exists a unique morphism \( f: A \to L \) such that the following diagram commutes:
  \begin{Center}
    \begin{tikzcd}
      A \arrow[rd, "g"] \arrow[dd, "f", dotted] & & \\
      & X \arrow[r, shift left=1, "s"] \arrow[r, shift right=1, "t"'] & Y \\
      L \arrow[ru, "l"'] & &
    \end{tikzcd}
  \end{Center}

  This limit is called the \underLine{equalizer of \( s \) and \( t \)}.
\end{definition}

\begin{definition}\label{def:categorical_pullback}\cite[definition 5.1.16]{Leinster2014}
  Assume that the index category \( \Bold I \) has the shape
  \begin{Center}
    \begin{tikzcd}
      \bullet \arrow[r] & \bullet & \bullet \arrow[l]
    \end{tikzcd}
  \end{Center}

  Cones of shape \( \Bold I \) with vertex \( A \) are then given by commutative diagrams of shape
  \begin{Center}
    \begin{tikzcd}
      A \arrow[r, "f_Y"] \arrow[d, "f_X"'] & Y \arrow[d, "t"] & \\
      X \arrow[r, "s"'] & Z
    \end{tikzcd}
  \end{Center}

  The limit \( (L, p_X, p_Y) \) then satisfies the universal property: for any \( \Bold I \)-cone with a vertex in \( A \in \Bold A \), there exists a unique morphism \( f: A \to L \) such that the following diagram commutes:
  \begin{Center}
    \begin{tikzcd}
      A \arrow[rd, "f" description, dotted] \arrow[rrd, "f_Y"] \arrow[rdd, "f_X"'] &                                               &   \\
                                                                                   & L \arrow[r, "p_Y"'] \arrow[d, "p_X"] & Y \arrow[d, "t"] \\
                                                                                   & X \arrow[r, "s"']                             & Z
    \end{tikzcd}
  \end{Center}

  This limit is called the \underLine{pullback or fibered product of \( s \) and \( t \)}.
\end{definition}

\begin{definition}\label{def:categorical_cocone}\cite[definition 5.2.1]{Leinster2014}
  The dual notion of a cone\Tinyref{def:categorical_cone} is that of a cocone. Given a category \( \Bold A \), an index category \( \Bold I \) and a diagram \( D: \Bold I \to \Bold A \), we say that the family of morphisms \( \{ D(i) \overset {f_i} \to A \}_{i \in \Bold I} \) is a \underLine{cocone for D} if it is a cone for \( D^{\Op}: \Bold{I}^{\Op} \to \Bold{A}^{\Op} \).

  Explicitly, a \underLine{cocone} on \( D \) consists of
  \begin{itemize}
    \item an object \( A \in \Bold A \), called the \underLine{vertex} of the cocone
    \item a family of \underLine{coprojection} morphisms \( \{ f_i: D(i) \to A \}_{i \in \Bold I} \)
  \end{itemize}
  such that for all morphisms \( u: i \to j \) in \( \Bold I \), the following diagram commutes:
  \begin{Center}
    \begin{tikzcd}
      & A & \\
      D(i) \arrow[ru, "f_i"] \arrow[rr, "D(u)"'] & & D(j) \arrow[lu, "f_j"']
    \end{tikzcd}
  \end{Center}
\end{definition}

\begin{definition}\label{def:categorical_colimit}\cite[definition 5.1.19(b)]{Leinster2014}
  Analogously to limits\Tinyref{def:categorical_limit}, we define the \underLine{colimit} \( \varinjlim D \) of \( D \) to be a cocone \( \{ D(i) \overset {p_i} \to L \}_{i \in \Bold I} \) such that for every cocone \( \{ D(i) \overset {f_i} \to A \}_{i \in \Bold I} \) there exists exactly one morphism \( f: L \to A \) such that \( f_i = f \circ p_i, i \in \Bold I \), i.e. the following diagram commutes:
  \begin{Center}
    \begin{tikzcd}
        & D(i) \arrow[rd, "p_i"] \arrow[ld, "f_i"'] & \\
      A &                                           & L \arrow[ll, "f", dotted]
    \end{tikzcd}
  \end{Center}

  If all small colimits exist, we say that \( \Bold A \) is a \underLine{cocomplete category}.
\end{definition}

\begin{definition}\label{def:cocomplete_category}
  If a category is both complete\Tinyref{def:categorical_limit} and cocomplete\Tinyref{def:categorical_colimit}, it is said to be a \underLine{cocomplete category}.
\end{definition}

\begin{definition}\label{def:categorical_coproduct}\cite[definition 5.2.2]{Leinster2014}
  If the index category \( \Bold I \) is discrete, specifying a functor \( D: \Bold I \to \Bold A \) is analogous to specifying a \( \Bold I \)-indexed family \( \{ X \}_{i \in \Bold I} \) of objects in \( \Bold A \)\Tinyref{def:categorical_product}.

  The \underLine{coproduct} \( \coprod_{i \in \Bold I} X_i \) or \underLine{categorical sum} \( \sum_{i \in \Bold I} X_i \) satisfies the following universal property: for any object \( A \) and any family of morphisms \( \{ f_j: {X_j} \to A \}_{j \in \Bold I} \) there exists exactly one morphism \( f: \coprod_{i \in \Bold I} X_i \to A \) such that for every \( j \in \Bold I \) we have \( f \circ p_j = f_j \), i.e. the following diagram commutes:
  \begin{Center}
    \begin{tikzcd}
      A &                                          & \prod_{i \in \Bold I} X_i \arrow[ll, "f"', dotted] \\
        & X_j \arrow[lu, "f_j"] \arrow[ru, "p_j"'] &
    \end{tikzcd}
  \end{Center}

  The function \( f \) is also denoted as \( \{ f_i \}_{i \in \Bold I} \).

  In particular, for two objects \( X, Y \in \Bold A \), the product is an object \( X + Y \) with coprojections \( p_X: X \to X \times Y \) and \( p_Y: Y \to X \times Y \) such that the following diagram commutes:
  \begin{Center}
    \begin{tikzcd}
      A & & \\
        & X + Y \arrow[lu, "f" description, dotted] & Y \arrow[llu, "f_Y"'] \arrow[l, "p_Y"] \\
        & X \arrow[luu, "f_X"] \arrow[u, "p_X"']    &
    \end{tikzcd}
  \end{Center}
\end{definition}

\begin{note}\label{note:empty_categorical_coproduct}
  The coproduct \( \prod_{i \in \varnothing} X_i \) of an empty family of objects is the initial object of the category.
\end{note}

\begin{definition}\label{def:categorical_coequalizer}\cite[definition 5.2.7]{Leinster2014}
  As for equalizers,\Tinyref{def:categorical_coequalizer}, assume that the index category \( \Bold I \cong \Bold{I}^{\Op} \) consists of two objects and two unidirectional morphisms:
  \begin{Center}
    \begin{tikzcd}
      \bullet \arrow[r, shift left=1] \arrow[r, shift right=1] & \bullet
    \end{tikzcd}
  \end{Center}

  Cocones with vertex \( A \) are then given by commutative diagrams of shape
  \begin{Center}
    \begin{tikzcd}
      X \arrow[r, shift left=1, "s"] \arrow[r, shift right=1, "t"'] & Y \arrow[r, "f"] & A
    \end{tikzcd}
  \end{Center}

  The \underLine{coequalizer} \( (L, l) \) then satisfies the universal property: for any \( \Bold I \)-cocone \( (A, g) \), there exists a unique morphism \( f: L \to A \) such that the following diagram commutes:
  \begin{Center}
    \begin{tikzcd}
      & & A \\
      X \arrow[r, shift left=1, "s"] \arrow[r, shift right=1, "t"'] & Y \arrow[rd, "l"'] \arrow[ru, "g"] & \\
      & & L \arrow[uu, "f"', dotted]
    \end{tikzcd}
  \end{Center}
\end{definition}

\begin{definition}\label{def:categorical_pushout}\cite[definition 5.2.11]{Leinster2014}
  A \underLine{pushout} in \( \Bold A \) is a \underLine{pullback} in \( \Bold A^{\Op} \).

  Explicitly, the index category \( \Bold I \) has the shape
  \begin{Center}
    \begin{tikzcd}
      \bullet & \bullet \arrow[l] \arrow[r] & \bullet
    \end{tikzcd}
  \end{Center}

  Cocones of shape \( \Bold I \) with vertex \( A \) are then given by commutative diagrams of shape
  \begin{Center}
    \begin{tikzcd}
      Z \arrow[r, "t"] \arrow[d, "s"'] & Y \arrow[d, "f_Y"] \\
      X \arrow[r, "f_X"']              & A
    \end{tikzcd}
  \end{Center}

  The pushout \( (L, p_X, p_Y) \) of \( D \) then satisfies the universal property: for any \( \Bold I \)-cocone with a vertex in \( A \in \Bold A \), there exists a unique morphism \( f: L \to A \) such that the following diagram commutes:
  \begin{Center}
    \begin{tikzcd}
      Z \arrow[r, "t"] \arrow[d, "s"']       & Y \arrow[d, "f_Y"'] \arrow[rdd, "p_Y"] & \\
      X \arrow[r, "f_X"] \arrow[rrd, "p_X"'] & L \arrow[rd, "f" description, dotted]  & \\
                                             &                                        & A
    \end{tikzcd}
  \end{Center}
\end{definition}

\begin{definition}\label{def:categorical_limit_preservation}\cite[definitions 5.3.1, 5.3.5]{Leinster2014}
  Let \( F: \Bold A \to \Bold B \) be a functor. We say that
  \begin{defenum}
    \item\label{def:categorical_limit_preservation/preserve} \( F \) \underLine{preserves limits of shape \( \Bold I \)} for some index category \( \Bold I \) if, given a \( \Bold I \)-shaped limit cone \mbox{\( \{ L \overset {p_i} \to D(i) \}_{i \in \Bold I} \)}, its image \mbox{\( \{ F(L) \overset {F(p_i)} \to F(D(i)) \}_{i \in \Bold I} \)} is also a limit cone. We say that \( F \) simply \underLine{preserves limits} if it preserves limits for every index category \( \Bold I \).

    \item\label{def:categorical_limit_preservation/reflect} \( F \) \underLine{reflects limits of shape \( \Bold I \)} if, given any \( \Bold I \)-shaped cone, if its image is a limit cone, then is it itself a limit cone.

    \item\label{def:categorical_limit_preservation/create} \( F \) \underLine{creates limits of shape \( \Bold I \)} if it both preserves and reflects limits.

    \item\label{def:categorical_limit_preservation/lift} \( F \) \underLine{lifts limits of shape \( \Bold I \)} if, given a diagram \( D: \Bold I \to \Bold B \), any limit cone \( \varprojlim D \) is the image of some limit cone in \( A \).
  \end{defenum}
\end{definition}

\begin{note}\label{note:categorical_colimit_preservation}
  Analogous definitions can be given for colimits.
\end{note}
