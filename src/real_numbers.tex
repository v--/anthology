\subsection{Real numbers}\label{subsec:real_numbers}

\begin{definition}\label{def:real_numbers}
  The \Def{real numbers} \( \R \) are the metric space completion\Tinyref{def:complete_metric_space} of \( \Q \) with respect to the absolute value.
\end{definition}

\begin{definition}\label{def:extended_real_numbers}
  We are sometimes interested in \Def{extended real numbers}. These can be any of the three sets
  \begin{itemize}
    \item \R \cup \{ +\infty \},
    \item \R \cup \{ -\infty \},
    \item \R \cup \{ -\infty, +\infty \},
  \end{itemize}
  where \( -\infty \) and \( +\infty \) are both sentinel values that act as the greatest\Tinyref{def:poset/largest_smallest_element} and/or least real number.

  We generally avoid performing arithmetic operations on \( \pm \infty \), however it is sometimes convenient to define
  \begin{align*}
    x + (+\infty) &\coloneqq +\infty, x \in \R
    x \cdot (+\infty) &\coloneqq +\infty, x \in \R
  \end{align*}

  We leave the operations
  \begin{align*}
    (-\infty) &+ (+\infty)
    (-\infty) &\cdot (+\infty)
  \end{align*}
  undefined.

  With these operations, the extended real numbers are no longer a field\Tinyref{def:field}.
\end{definition}

\begin{definition}\label{def:floor_ceiling_functions}
  Let \( x \in \R \) be a real number. In analogy with \cref{def:commutative_ring_division}, we define its floor function
  \begin{equation*}
    \Floor(x) \coloneqq \max \{ n \in \Z : n \leq x \}
  \end{equation*}
  and its ceiling function
  \begin{equation*}
    \Ceil(x) \coloneqq \min \{ n \in \Z : n \geq x \}.
  \end{equation*}
\end{definition}

\begin{definition}\label{def:eulers_constant}
  We define \Def{Euler's number} as
  \begin{equation*}
    e \coloneqq \sum_{i=0}^n \frac 1 {n!}.
  \end{equation*}
\end{definition}

\begin{definition}\label{def:pi}\cite[515]{Knapp2016BAlg}
  We define the number \( \pi \) as
  \begin{equation*}
    \pi \coloneqq \inf\{ e^{ix} = 1, x > 0 \}.
  \end{equation*}
\end{definition}
