\subsection{Convergence}\label{subsec:convergence}

Let \( (X, \Cal{T}) \) be a topological space.

\begin{definition}\label{def:topological_net}\cite[49]{Engelking1989}
  A \Def{net} or \Def{generalized sequence}\Tinyref{def:sequence} on the set \( X \) is any function from a directed set\Tinyref{def:order/directed} \( (I, \leq) \) to \( X \). For convenience, since nets are simply indexed families\Tinyref{def:indexed_family} over directed sets, we denote nets by
  \begin{equation*}
    \{ x_i \}_{i \in I},
  \end{equation*}
  because the preorder on the domain \( I \) is usually clear from the context.
\end{definition}

\begin{definition}\label{def:net_limit_point}\cite[49]{Engelking1989}
  We say that \( x \in X \) is a \Def{limit point} of the net \( \{ x_i \}_{i \in I} \) if for every neighborhood \( U \) of \( x \) there exists an index \( i_U \in I \) such that
  \begin{equation*}
    i \geq i_U \implies x_i \in U.
  \end{equation*}

  We also say that the net \( \{ x_i \}_{i \in I} \) \Def{converges} to \( x \) and that the net is \Def{convergent}. If the net does not converge to any value, we say that it is \Def{divergent}.

  We denote the set of all limit points of \( \{ x_i \}_{i \in I} \) by
  \begin{equation*}
    \lim_{i \in I} x_i.
  \end{equation*}

  Thus, we can view \( \lim \) as a multivalued\Tinyref{def:function/multivalued} operator from the set of all nets on \( X \) to \( X \).

  If \( x \) is only one limit point for a net, we use the convention in \cref{remark:singleton_sets} and write
  \begin{equation*}
    x = \lim_{i \in I} x_i.
  \end{equation*}

  If the net is a sequence\Tinyref{def:sequence}, we also use the following notations:
  \begin{itemize}
    \item \( x = \lim_{i \to \infty} x_i \)
    \item \( x = \lim x_i \)
    \item \( x_i \xrightarrow[i \to \infty]{} x \)
    \item \( x_i \to x \)
  \end{itemize}
\end{definition}

\begin{definition}\label{def:net_cluster_point}\cite[50]{Engelking1989}
  We say that \( x \in X \) is a \Def{cluster point} of the net \( \{ x_i \}_{i \in I} \) if for every neighborhood \( U \) of \( x \) and every index \( i_0 \in I \) there exists an index \( i_U \geq i_0 \) such that \( x_{i_U} \in U \).
\end{definition}

\begin{example}\label{ex:multiple_limit_points_of_net}
  Even limits of sequences need not be unique in arbitrary topological spaces. Let \( X = \{ y, z \} \) be a binary set with the indiscrete topology\Tinyref{def:standard_topologies/indiscrete} \( \{ \varnothing, X \} \). L

  Define the following sequence\Tinyref{def:sequence}
  \begin{align*}
    x_i \coloneqq \begin{cases}
      y, &i \text{ is even}, \\
      z, &i \text{ is odd}.
    \end{cases}
  \end{align*}

  Thus the only neighborhood of \( y \), the whole space \( X \), contains all members of the sequence. Since the same is true for \( z \), we have
  \begin{equation*}
    \lim_{i \in \Z_{>0}} x_i = \{ y, z \}.
  \end{equation*}
\end{example}

\begin{proposition}\label{thm:sequence_converges_iff_almost_entirely_in_neighborhood}
  A sequence \( \{ x_i \}_{i=1}^\infty \subseteq X \) converges to \( x \in X \) if and only if, given a neighborhood \( U \) of \( x \), only finitely many elements of the sequence are outside of \( U \).
\end{proposition}
\begin{proof}
  This is simply a restatement of \cref{def:net_limit_point} for the special case of sequences.
\end{proof}

\begin{proposition}\label{thm:limit_point_iff_in_closure}\cite[proposition 1.6.3]{Engelking1989}
  Fix a set \( A \subseteq X \). A point \( x \in X \) belongs to \( \Cl{A} \) if and only if there exists a net \( \{ x_i \}_{i \in I} \subseteq A \) such that \( x = \lim_{i \in I} x_i \).
\end{proposition}

\begin{definition}\label{def:convergence_of_function_at_point}
  Fix two topological spaces \( (X, \Cal{T}) \) and \( (Y, \Cal{O}) \), let \( D \subseteq X \) be a nonempty open set and let \( f: D \to Y \) be any function. We say that \( f \) \Def{converges} to \( y_0 \in Y \) at \( x_0 \in \Cl(D) \) and write
  \begin{equation*}
    \lim_{x \to x_0} f(x) = y_0
  \end{equation*}
  if for every neighborhood \( V \in \Cal{O} \) of \( y_0 \) there exists a neighborhood \( U \in \Cal{T} \) of \( x_0 \) such that
  \begin{equation*}
    f(U \cap D) \subseteq V.
  \end{equation*}

  Note that \( x_0 \) does not even need to belong to \( D \) and \( y_0 \) does not need to belong to \( f(D) \).
\end{definition}
