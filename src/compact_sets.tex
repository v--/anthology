\subsection{Compact sets}\label{sec:compact_sets}

Let \( (X, \Cal{T}) \) be a topological space.

\begin{definition}\label{def:compact_set}\cite[40]{Deimling1985}
  The set \( A \subseteq X \) is called \underLine{compact} if any of the following equivalent conditions hold:
  \begin{defenum}
    \item\label{def:compact_set/union} (\underLine{Finite union property}) Every open cover of \( X \) has a finite subcover.
    \item\label{def:compact_set/intersection} (\underLine{Finite intersection property}) The intersection is nonempty for every family of sets such that the intersection of any finite subfamily is nonempty.
  \end{defenum}

  If the closure if \( A \) is compact, we call \( A \)~\underLine{relatively compact} or \underLine{precompact} (although the term \enquote{precompact} is also used for totally bounded sets, see \ref{def:totally_bounded_set}).
\end{definition}
