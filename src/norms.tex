\subsection{Norms}\label{subsec:norms}

\begin{remark}\label{remark:normed_fields}
  Norms generalize distances of points in a plane, while absolute values generalize the absolute value over either \( \BR \) or \( \BC \). The axioms themselves differ minimally. Absolute values in a field are multiplicative norms over the field, however we cannot define absolute values in terms of norms since absolute values are needed for defining norms. Still, we will refer to fields with absolute values as \Def{normed fields}.
\end{remark}

\begin{definition}\label{def:absolute_value}\MarginCite{nLab:absolute_value}
  Let \( R \) be a \hyperref[def:semiring]{semiring}. We say that the function \( \Abs{\cdot}: V \to \BR_{>0} \) is an \Def{absolute value} if
  \begin{DefEnum}
    \ILabel{def:absolute_value/identity}(identity) \( x = 0_R \) if and only if \( \Abs{x} = 0 \)
    \ILabel{def:absolute_value/multiplicativity}(multiplicativity) For any \( x, y \in V \),
    \begin{equation*}
      \Abs{xy} = \Abs{x} \cdot \Abs{y}
    \end{equation*}

    \ILabel{def:absolute_value/subadditivity}(subadditivity) For any \( x, y \in V \),
    \begin{equation*}
      \Abs{x + y} \leq \Abs{x} + \Abs{y}
    \end{equation*}
  \end{DefEnum}
\end{definition}

\begin{definition}\label{def:norm}
  Let \( M \) be an \( R \)-module with absolute value \( \Abs{\cdot} \). We say that the function \( \Norm{\cdot}: M \to \BR_{\geq 0} \) is a \Def{norm} if
  \begin{DefEnum}
    \ILabel{def:norm/identity}(identity) \( x = 0_M \) if and only if \( \Norm x = 0_{\BR} \)

    \ILabel{def:norm/absolute_homogeneity}(absolute homogeneity)
    \begin{equation*}
      \Norm{t x} = \Abs{t} \Norm{x} \text{ for all } t \in R \text{ and } x \in M
    \end{equation*}

    \ILabel{def:norm/subadditivity}(subadditivity)
    \begin{equation*}
      \Norm{x + y} \leq \Norm{x} + \Norm{y} \text{ for all } x, y \in M
    \end{equation*}
  \end{DefEnum}

  If we remove \fullref{def:norm/identity}, then \( \Norm{\cdot} \) is called a \Def{seminorm}.

  If instead \( V \) is an \hyperref[def:algebra_over_ring]{associative} and \( \Norm{\cdot} \) satisfies the additional axiom
  \begin{DefEnum}
    \ILabel{def:norm/multiplicativity}(multiplicativity)
    \begin{equation*}
      \Norm{xy} = \Norm{x} \cdot \Norm{y} \text{ for all } x, y \in M,
    \end{equation*}
  \end{DefEnum}
  we say that it is a \Def{multiplicative norm}.
\end{definition}

\begin{definition}\label{def:norm_induced_metric}
  A norm \( \Norm \cdot \) on a real or complex vector space \( V \) induces the \hyperref[def:vector_space]{metric}
  \begin{align*}
     & \rho: V \times V \to \BR_{\geq 0}  \\
     & \rho(x, y) \coloneqq \Norm{x - y}.
  \end{align*}
\end{definition}
\begin{proof}
  The function is positive definite since \( \Norm \cdot \) is positive definite; we will show that the function is a metric.

  \SubProofOf{def:metric_space/M1} Follows from \fullref{def:norm/identity}.

  \SubProofOf{def:metric_space/M2} By \fullref{def:norm/absolute_homogeneity},
  \begin{equation*}
    \rho(x, y) = \Norm{ x - y } = \Norm{ (-1) (y - x) } = \Abs{-1} \Norm{y - x} = \rho(y, x).
  \end{equation*}

  \SubProofOf{def:metric_space/M3}
  \begin{equation*}
    \rho(x, y) + \rho(y, z) = \Norm{x - y} + \Norm{y - z} \geq \Norm{x - z} = \rho(x, z).
  \end{equation*}
\end{proof}

\begin{definition}\label{def:duality_mapping}\MarginCite[ex. 2.26]{Phelps1993}
  We define the \Def{duality mapping}
  \begin{align*}
     & D: E \MultTo X^*,                                                                                              \\
     & D(x) \coloneqq \{ x^* \in X^* \colon \Norm x = \Norm {x^*} \text{ and } \Prod{x^*} x = \Norm {x^*} \Norm x \}.
  \end{align*}

  We will usually use this mapping for unit vectors, so we may as well consider its restriction to the unit spheres, where
  \begin{align*}
     & D': S_X \MultTo S_{X^*},                                       \\
     & D'(x) \coloneqq \{ x^* \in S_{X^*} \colon \Prod{x^*} x = 1 \}.
  \end{align*}
\end{definition}

\begin{definition}\label{def:smooth_norm}\MarginCite[def. 2.36]{Phelps1993}
  The norm \( \Norm \cdot \) on \( X \) is called \Def{smooth} if any of  if for each \( x \in S_X \) the duality mapping is single-valued.
\end{definition}

\begin{definition}\label{def:rotund_norm}\MarginCite[def. 2.36]{Phelps1993}
  The norm \( \Norm \cdot \) on \( X \) is called \Def{rotund} or \Def{strictly convex} if any of the following equivalent conditions hold:
  \begin{DefEnum}
    \ILabel{def:rotund_norm/no_sphere_segments} There are no line segments in the unit sphere \( S_X \).
    \ILabel{def:rotund_norm/least_norm} Every convex subset of \( X \) has at most one point of least norm.
    \ILabel{def:rotund_norm/linearly_dependent}
    \begin{align}\label{def:rotund_norm/linearly_dependent/equation}
      \Norm{x + y} = \Norm x + \Norm y \implies x \text{ and } y \text{ are linearly dependent}.
    \end{align}
  \end{DefEnum}
\end{definition}
\begin{proof}
  \SubProofImplies{def:rotund_norm/no_sphere_segments}{def:rotund_norm/least_norm} Let the norm in \( E \) be rotund and let \( C \subseteq E \) be a (potentially empty) convex set. We will prove that \( C \) contains at most one point of least norm.

  If \( C \) is empty or otherwise contains no element of least norm, trivially contains at most one point of least norm.

  Now let \( C \) contain at least one element \( x \in C \) of least norm. Assume that \( y \in C \) is another element of least norm. Necessarily \( \Norm x = \Norm y \).

  Fix \( t \in (0, 1) \) and define \( z \coloneqq tx + (1-t)y \). Since \( C \) is convex, it contains \( z \). Since \( x \) and \( y \) are elements of least norm, we have \( \Norm z \geq \Norm x \). By the triangle inequality,
  \begin{align*}
    \Norm{z}
    =
    \Norm{tx + (1-t)y}
    \leq
    t \Norm x + (1-t) \Norm y
    =
    \Norm{x},
  \end{align*}
  thus \( \Norm z = \Norm x \).

  This implies that the entire segment \( [x, y] \) are elements of least norm in \( C \). Hence the segment \( [x, y] \) is contained in the sphere \( \Norm x S_E \), which contradicts the rotundity of the norm \( \Norm{\cdot} \).

  Hence \( C \) contains at most one element of least norm.

  \SubProofImplies{def:rotund_norm/least_norm}{def:rotund_norm/no_sphere_segments} Let every convex set \( C \subseteq E \) have at most one element of least norm.

  Assume\LEM that the norm \( \Norm{\cdot} \) is not rotund. Then the unit sphere \( S_E \) contains a line segment \( [x, y], x \neq y \). The set \( [x, y] \) is compact and, by the Weierstrass extreme value theorem, the norm attains its minimum on the segment in a point \( z \in [x, y] \). Since the segment is also convex and we assumed that convex sets have at most one element of least norm, it follows that this element \( z \) is unique.

  Then for any point \( s \in [x, y], s \neq z \), we have \( \Norm s > \Norm z = 1 \), thus \( s \) cannot be an element of the unit sphere. The obtained contradiction shows that the norm \( \Norm{\cdot} \) is rotund.

  \SubProofImplies{def:rotund_norm/no_sphere_segments}{def:rotund_norm/linearly_dependent} Let \( E \) be rotund let \( x, y \in E \) be distinct vectors such that
  \begin{align}\label{def:rotund_norm/linearly_dependent/assumption}
    \Norm{x + y} = \Norm x + \Norm y.
  \end{align}

  If either of them is the zero vector, then they are trivially linearly dependent.

  Assume that both \( x \) and \( y \) are nonzero and define
  \begin{align*}
    \xi \coloneqq \frac x {\Norm x}
     &  &
    \eta \coloneqq \frac y {\Norm y}
     &  &
    t \coloneqq \frac {\Norm x} {\Norm{x + y}}
  \end{align*}

  \Fullref{def:rotund_norm/linearly_dependent/assumption} implies that
  \begin{equation*}
    1 - t = 1 - \frac {\Norm x} {\Norm{x + y}} = \frac {\Norm{x + y} - \Norm x} {\Norm{x + y}} = \frac {\Norm y} {\Norm{x+y}}.
  \end{equation*}

  Since both \( \xi \) and \( \eta \) are in \( S_E \), by rotundity, their convex combination
  \begin{equation*}
    \nu \coloneqq t \xi + (1-t)\eta
  \end{equation*}
  should not be contained in \( S_E \) unless \( \xi = \eta \).

  Calculating the norm, we obtain
  \begin{align*}
    \Norm{\nu}
     & =
    \Norm{t \xi + (1-t)\eta}
    =    \\ &=
    \Norm{\frac {\Norm x \xi} {\Norm{x + y}} + \frac {\Norm y \eta} {\Norm{x + y}}}
    =    \\ &=
    \Norm{\frac {x + y} {\Norm{x + y}}}
    = 1,
  \end{align*}
  hence \( \nu \in S_E \). Thus \( \xi = \eta \) and \( x = \frac {\Norm x} {\Norm y} y \), so \( x \) and \( y \) are linearly dependent.

  \SubProofImplies{def:rotund_norm/linearly_dependent}{def:rotund_norm/no_sphere_segments} Let \fullref{def:rotund_norm/linearly_dependent/equation} hold and fix \( x, y \in S_E, t \in (0, 1) \). Define \( z \coloneqq tx + (1-t)y \).
  First, assume that the vectors \( tx \) and \( (1-t)y \) satisfy the left part of \fullref{def:rotund_norm/linearly_dependent/equation}, i.e.
  \begin{equation*}
    \Norm z = \Norm{tx + (1-t)y} = t \Norm x + (1-t) \Norm y = 1.
  \end{equation*}

  This does not refute rotundity since \( x \) and \( y \) are not necessarily distinct. It follows from \fullref{def:rotund_norm/linearly_dependent/equation} that \( tx \) and \( (1-t)y \) are linearly dependent, hence \( x \) and \( y \) are also linearly dependent. Since \( x \) and \( y \) both have unit norm, either \( y = x \) or \( y = -x \).

  If we assume that \( y = -x \), then
  \begin{align*}
    \Norm z
    =
    \Norm{tx + (1-t)y}
    =
    (2t - 1) \Norm x
    =
    2t - 1,
  \end{align*}
  which is only possible if \( t = 1 \) since \( \Norm z = 1 \). But \( t \) is strictly less than 1.

  Hence \( y \neq -x \) and the only remaining possibility is that \( y = x \).

  Now assume that the vectors \( tx \) and \( (1-t)y \) do not satisfy the left part of \fullref{def:rotund_norm/linearly_dependent/equation}. This implies \( \Norm z < 1 \). Thus \( x \) and \( y \) are necessarily distinct, but \( z \) is not contained in the unit sphere and the segment \( [x, y] \) is not contained in \( S_E \).

  We have shown that \( x, y \in S_E \) implies that either \( y = x \) or that the segment \( [x, y] \) is not contained in \( S_E \), thus the norm in \( E \) is rotund.
\end{proof}

\begin{theorem}\label{thm:smooth_rotund_norm_duality}\MarginCite[exer. 2.37(a)]{Phelps1993}
  If the norm in a Banach space \( X \) is such that its dual norm in \( X^* \) is rotund (resp. smooth), then it is itself smooth (resp. rotund).
\end{theorem}
\begin{proof}\mbox{}
  \begin{enumerate}
    \item First, let the dual norm \( \Norm{\cdot}^* \) be rotund and assume that \( \Norm{\cdot} \) is not smooth.

          Fix \( x \in S_X \). Since \( D(x) \) is nonempty (by \fullref{thm:hahn_banach_implies_duality_mapping_nonempty}) and since \( \Norm{\cdot} \) is not smooth, then there exist two different functionals \( x^*, y^* \in D(x) \), such that
          \begin{align*}
            \Prod {x^*} x
            =
            \Prod {y^*} x
            =
            1.
          \end{align*}

          We will show that the segment \( [x^*, y^*] \) is contained in \( S_{X^*} \), i.e. that the dual norm is not rotund.

          Fix any \( t \in (0, 1) \) and define \( z^* \coloneqq t x^* + (1-t) y^* \). We only need to show that \( \Norm{z^*} = 1 \).

          By the triangle inequality, we have
          \begin{align*}
            \Norm{z^*}
            =
            \Norm{t x^* + (1-t) y^*}
            \leq
            t \Norm{x^*} + (1-t) \Norm{y^*}
            =
            t + (1-t)
            =
            1.
          \end{align*}

          For the reverse inequality, note that
          \begin{align*}
            \Norm{z^*}
            \geq
            \Prod {z^*} x
            =
            t \Prod {x^*} x + (1-t) \Prod {y^*} x
            =
            t + (1-t)
            =
            1,
          \end{align*}
          thus \( \Norm{z^*} = 1 \). Hence \( [x^*, y^*] \) is contained in \( S_{X^*} \) and the dual space is not smooth. The obtained contradiction proves that the norm in \( X \) is rotund.

    \item Now let the dual norm \( \Norm{\cdot}^* \) be smooth and assume that \( \Norm{\cdot} \) is not rotund. Then there exist points \( x, y \in S_X \) such that the while segment \( [x, y] \) is contained in \( S_X \).

          Fix \( t \in (0, 1) \) and define \( z \coloneqq tx + (1-t)y \in S_X \). Denote by \( J: X \to X^{**} \) the canonical embedding into the double-dual. By \fullref{thm:hahn_banach_implies_duality_mapping_nonempty}, there exists a functional \( z^* \in X^* \), such that
          \begin{align*}
            \Prod {J(z)} {z^*}
            =
            \Prod{z^*} z
            =
            1.
          \end{align*}

          Because the dual norm \( \Norm{\cdot}^* \) is smooth, we cannot have \( \Prod{J(x)} {z^*} =  \Prod{z^*} x = 1 \) or \( \Prod{J(y)} {z^*} = \Prod{z^*} y = 1 \) and since \( \Norm{z^*} = 1 \), necessarily
          \begin{equation*}
            \Prod{z^*} x < 1 \text{ and } \Prod{z^*} y < 1.
          \end{equation*}

          If follows that
          \begin{align*}
            1
            =
            \Prod{z^*} z
            =
            t \Prod{z^*} x + (1-t) \Prod{z^*} y
            <
            t + (1-t)
            =
            1,
          \end{align*}
          which is a contradiction. Hence \( \Norm{\cdot} \) is rotund.
  \end{enumerate}
\end{proof}

\begin{proposition}\label{thm:hilbert_space_smooth_rotund}\MarginCite[exer. 2.37(c)]{Phelps1993}
  Norms in Hilbert spaces are both smooth and rotund.
\end{proposition}
\begin{proof}
  Let \( X \) be a Hilbert space, i.e. the norm is generated by an inner product and, due to Riesz's theorem, we identify the space \( X \) with its continuous dual \( X^* \).

  To prove that \( X \) is rotund, choose \( x, y \in S_X, x \neq y \). We will show that the segment \( [x, y] \) is not contained in \( S_X \).

  If \( x \) and \( y \) are linearly dependent, necessarily \( y = -x \) and all non-trivial convex combinations of \( x \) and \( y \) are contained in the open unit ball, hence \( [x, y] \not\subseteq S_X \).

  Not let \( x \) and \( y \) be linearly independent. By the Cauchy-Bunyakovsky-Schwarz inequality, we have
  \begin{align}\label{eq:hilbert_cauchy_inequality}
    \Prod x y \leq \Abs{\Prod x y} < \Norm x \Norm y = 1.
  \end{align}

  Fix \( t \in (0, 1) \) and let \( z \coloneqq tx + (1-t)y \). We will show that \( z \not\in S_X \). Indeed,
  \begin{align*}
    \Norm{z}^2
    =
    \Prod z z
     & =
    t^2 \Norm x^2 + t(1-t) \Prod x y + (1-t) t \Prod y x + (1-t)^2 \Norm y^2
    =    \\ &=
    t^2 + (1-t)^2 + 2 t(1-t) \Prod x y
    <    \\ &\overset {(\ref{eq:hilbert_cauchy_inequality})} <
    t^2 + (1-t)^2 + 2 t(1-t)
    =    \\ &=
    t^2 + 1 - 2t + t^2 + 2t - t^2
    =
    1.
  \end{align*}

  Thus \( \Norm{z}^2 < 1 \) and \( \Norm z < 1 \) and \( z \not\in S_X \).

  In both cases, no interior point of the segment \( [x, y] \) is contained in \( S_X \), hence the norm in \( X \) is rotund.

  Since we identify \( X \) with its dual, the norm in \( X^* \) is also rotund and by \fullref{thm:smooth_rotund_norm_duality}, the norm in \( X \) is also smooth.
\end{proof}

\begin{example}\label{thm:c0_l1_not_smooth_rotund}\MarginCite[exer. 2.37(c)]{Phelps1993}
  The norms in \( c_0 \) and \( l^1 \) are neither smooth nor rotund.
\end{example}
\begin{proof}
  Consider the space \( c_0 \) of all real sequences that converge to zero equipped with the uniform norm
  \begin{equation*}
    \Norm{x}_{c_0} \coloneqq \sup_i \Abs{x_i}.
  \end{equation*}

  Note that the dual space of \( c_0 \) is (isometrically isomorphic to) the space \( l^1 \) of absolutely summable sequences with norm
  \begin{equation*}
    \Norm{x}_{l^1} \coloneqq \sum_i \Abs{x_i}.
  \end{equation*}

  Let \( \{ e_n \}_{n=1}^\infty \) be the canonical basis of \( c_0 \), i.e. the coordinates \( e^{(i)}_n \) of \( e_n \) are given by the Dirac delta function, \( e^{(i)}_n \coloneqq \delta_{i,n} \).

  For every natural \( n \geq 1 \), define \( x_n \) to be the same as \( e_n \) except that the first coordinate of \( x_n \) is always \( 1 \).

  The corresponding norms of \( e_n \) are all equal to 1 and the norms of \( x_n \) are
  \begin{align*}
    \Norm{x_n}_{c_0} = 1
     &  &
    \Norm{x_n}_{l^1} = 2.
  \end{align*}

  For every \( n \) we have
  \begin{equation*}
    \Prod {e_1} {x_n} = \Prod {e_n} {x_n} = 1,
  \end{equation*}
  hence \( J_{c_0}(x_n) \) has at least two elements \( e_1 \) and \( e_n \) and the norm in \( c_0 \) is not smooth.

  Given that \( \{ x_1, x_2, \ldots \} \subseteq S_{c_0} \), consider the convex combinations of \( x_2 \) and \( x_3 \):
  \begin{align*}
    tx_2 + (1-t)x_3
    =
    (1, t, (1-t), 0, 0, \ldots).
  \end{align*}

  Evidently \( tx_2 + (1-t)x_3 \in S_{c_0} \) for every \( t \in (0, 1) \), hence the norm in \( c_0 \) is not rotund.

  The contrapositions\LEM to the statements in \fullref{thm:smooth_rotund_norm_duality} say that if \( X \) is not rotund (resp. smooth), then the dual space \( X^* \) is not smooth (resp. rotund). Thus \( l^1 \) is neither smooth or rotund as the dual of \( c_0 \).
\end{proof}
