\section{Order theory}\label{sec:order_theory}

\begin{remark}\label{remark:order_infix_notation}
  \Def{Orders} are special binary \hyperref[def:binary_relation]{relations}. We denote orders using symbols rather than letters because it is customary to write orders using infix notation \( a \sim b \) rather than \( (a, b) \in \sim \).
\end{remark}

\subsection{Preorders}\label{subsec:preorders}

\begin{definition}\label{def:preordered_set}\cite{nLab:preorder}
  A binary \hyperref[def:binary_relation]{relation} \( \leq \) on a set \( X \) is called a \Def{preorder} if it is \hyperref[def:binary_relation/reflexive]{reflexive} and \hyperref[def:binary_relation/transitive]{transitive}.

  A \Def{preordered set} is a set \( X \) equipped with a preorder \( \leq \). 

  We introduce the following terminology:
  \begin{defenum}
    \DItem{def:preordered_set/dual} The partially ordered set \( (X, \geq) \) where \( \geq \) is the converse \hyperref[def:binary_relation/converse]{relation}, is the called the \Def{dual preordered set}. See \fullref{def:thin_category} for a discussion of the duality.

    \DItem{def:preordered_set/subset} For every subset \( A \subseteq X \) the pair \( (A, \leq_A) \) is a partially ordered set and the restriction
    \begin{equation*}
      \leq_A \coloneqq \{ (a, b) \in \leq \colon a \in A \Tand b \in A \}.
    \end{equation*}
    is called the \Def{subset preorder}.

    \DItem{def:preordered_set/comparability} The elements \( x, y \in X \) are called \Def{comparable} if either \( x \leq y \) or \( y \leq x \).

    \DItem{def:preordered_set/upper_lower_bound}\cite[170]{Enderton1977} \( x \in X \) is an \Def{upper bound} for \( A \subseteq X \) (resp. \Def{lower bound} for \( A \) in the dual poset \( (X, \geq) \)) if \( y \leq x \) for any \( y \in A \).

    \DItem{def:preordered_set/bounded_set} The set \( A \subseteq X \) is called \Def{bounded from above} (resp. \Def{bounded from below}) if it has an upper bound (resp. lower bound). If a set is bounded from both directions, we simply say that it is \Def{bounded}.

    \DItem{def:preordered_set/largest_smallest_element}\cite[171]{Enderton1977} \( x \in X \) is a \Def{largest element} or \Def{maximum} \( \max X \) of \( X \) (resp. \Def{smallest element} or \Def{minimum} \( \min X \) of \( (X, \geq) \)) if \( y \leq x \) any \( y \in X \).

    \DItem{def:preordered_set/maximal_minimal_element}\cite[170]{Enderton1977} \( x \in X \) is a \Def{maximal element} of \( X \) (resp. \Def{minimal element} of \( (X, \geq) \)) if \( x \leq y \) implies \( x = y \) for any \( y \in X \).

    \DItem{def:preordered_set/supremum_infimum}\cite[170]{Enderton1977} \( x \in X \) is a \Def{supremum} (resp. \Def{infimum}) for \( A \subseteq X \) if \( x \) is the least upper bound (resp. greatest lower bound) of \( A \), i.e. the smallest element of the set \( U \subseteq X \) consisting of all upper bounds of \( X \).
  \end{defenum}
\end{definition}

\begin{definition}\label{def:thin_category}\cite{nLab:thin_category}
  A \hyperref[def:category]{category} \( \Cat{P} \) is called a \Def{thin category} if, for every two objects \( A, B \in \Bold{P} \), whenever \( f, g \in \Bold{P}(A, B) \), we have \( f = g \).

  If \( \Cat{P} \) is locally small, this is equivalent to saying that any set of morphisms \( \Bold{P}(A, B) \) is at most a singleton.
\end{definition}

\begin{proposition}\label{thm:preorder_category_correspondence}
  To every preordered \hyperref[def:preordered_set]{set} there corresponds exactly one \hyperref[def:category_cardinality]{small} \hyperref[def:thin_category]{thin} category.

  Furthermore, \hyperref[def:preordered_set/supremum_infimum]{infima} correspond to categorical \hyperref[def:categorical_product]{products}, suprema to \hyperref[def:categorical_coproduct]{coproducts} and dual preordered \hyperref[def:preordered_set/dual]{sets} correspond to dual \hyperref[def:opposite_category]{categories}.

  Compare this result to \cref{thm:partial_order_category_correspondence}.
\end{proposition}
\begin{proof}\mbox{}
  \begin{description}
    \Implies Let \( (P, \leq) \) be a preordered set. We define the category \( \Cat{P} \) as follows:
    \begin{description}
      \RItem{def:category/objects} The set of objects in \( \Cat{P} \) is the set \( P \).

      \RItem{def:category/morphisms} The set of morphisms between two elements \( x, y \in \Cat{P} \) is the singleton \( \{ (x, y) \} \) when \( x \leq y \) and the empty set otherwise.

      \RItem{def:category/composition} The composition of two morphisms \( (x, y) \) and \( (y, z) \) is simply \( (x, z) \) (such a morphism exists by transitivity of \( \leq \)).
    \end{description}

    The axiom \ref{def:category/identity} follows from reflexivity of \( \leq \) and the axiom \ref{def:category/associativity} is trivial. Therefore \( \Cat{P} \) is a category.

    We will only prove the equivalence of products and infima since the argument for suprema and coproducts is completely analogous.

    Let \( p \) be the categorical product of the set \( A \subseteq P \). Then \( p \leq x \) for all \( x \in A \), hence it is a lower bound. If \( q \) is another lower bound, then by definition of product, there exists a unique morphism \( q \leq p \). Therefore \( p \) is the infimum.

    \ImpliedBy Now assume that \( \Cat{P} \) is a thin small category. Define the relation \( \leq \) on the set \( \Cat{P} \) as \( x \leq y \iff \Cat{P}(x, y) \neq \varnothing \). This is a preorder because
    \begin{description}
      \RItem{def:binary_relation/reflexive} \( x \leq x \) because of the existence of identity morphisms.

      \RItem{def:binary_relation/transitive} If \( x \leq y \) and \( y \leq z \), then, by composition of morphisms, \( x \leq z \).
    \end{description}

    Note that the infimum of a set \( A \subseteq \Cat{P} \) (if it exists) has a unique morphism \( \inf A \leq x \) for any \( x \in A \). If \( b \leq x \) for all \( x \in A \) is another \hyperref[def:categorical_cone]{cone}, then necessarily \( b \leq \inf A \). Therefore the infimum is the categorical product.
  \end{description}

  We proved that for each partially ordered set there corresponds at least one thin small category and vice versa. The fact that to each poset corresponds at most one category \( \Cat{P} \) is obvious. Therefore we have a correspondence between the two.

  Duality is also obvious from our constructions.
\end{proof}

\begin{definition}\label{def:monotone_map}
  Let \( (P, \leq_P) \) and \( (Q, \leq_Q) \) be two preordered \hyperref[def:preordered_set]{sets}. We say that the function \( f: P \to Q \) is an \Def{order preserving map} or \Def{order homomorphism} or simply a \Def{monotone map} if
  \begin{equation*}
    x \leq_P y \Timplies f(x) \leq_Q f(y).
  \end{equation*}

  If
  \begin{equation*}
    x \neq y \Timplies f(x) \neq f(y),
  \end{equation*}
  we call the function \( f \) \Def{strictly monotone}.

  In particular, if \( P \) is the set of nonnegative \hyperref[def:integers]{integers}, then we speak of \Def{monotone sequences}
  \begin{equation*}
    \{ x_k \}_{k=1}^\infty,
  \end{equation*}
  where \( x_{k-1} \leq_Q x_k \) for all \( k = 1, 2, 3, \ldots \).
\end{definition}

\begin{definition}\label{def:directed_set}\cite[8]{Engelking1989}
  A preordered \hyperref[def:preordered_set]{set} \( (X, \leq) \) is called a \Def{directed set} (there is no established name for the relation itself) if every finite subset of \( X \) has an upper \hyperref[def:preordered_set/upper_lower_bound]{bound}, i.e. for all \( x, y \in X \) there exists \( z \in X \) such that \( x \leq z \) and \( y \leq z \).

  Since the set of all upper bounds of \( \{ a, b \} \) may not have a least \hyperref[def:preordered_set/largest_smallest_element]{smallest}, the upper bound condition is weaker than every two-element set having a supremum.

  Directed sets are used to define nets in topological spaces, see \fullref{def:topological_net}.
\end{definition}

\begin{definition}\label{def:category_of_preordered_sets}
  Partially ordered sets and monotone maps form a category as described in \fullref{def:first_order_model_category}.
\end{definition}
