\subsection{Separation axioms}\label{subsec:separation_axioms}

\begin{definition}\label{def:separation_axioms}
  We can classify topological spaces using the following separation axioms. We say that \( (X, \Cal{T}) \) is

  \begin{description}
    \DItem{def:separation_axioms/regular}[regular] every point \( x \in X \) and every closed set \( F \in \Cal{F}_{\Cal{T}} \) can be separated using neighborhoods, i.e. there exist disjoint open sets \( U \ni x \) and \( V \supseteq F \).
    \DItem{def:separation_axioms/completely_regular}[completely regular] (Tychonoff) every point \( x \in X \) and every closed set \( F \in \Cal{F}_{\Cal{T}} \) can be functionally separated, i.e. there exists a continuous function \( f: X \to [0, 1] \) such that \( f(x) = 0 \) and \( f(F) = 1 \).
    \DItem{def:separation_axioms/normal}[normal] (Urysohn) every two closed sets \( F, G \in \Cal{F}_{\Cal{T}} \) can be separated using neighborhoods, i.e. there exist disjoint open sets \( U \supseteq F \) and \( V \supseteq G \).
    \DItem{def:separation_axioms/T0}[T0] (Kolmogorov) for every two different points \( x, y \in X \), there exists an open set \( U \in \Cal{T} \) such that either \( x \in U \) or \( y \in U \).
    \DItem{def:separation_axioms/T0.5}[T0.5] every singleton set \( \{ x \} \) is either open or closed.
    \DItem{def:separation_axioms/T1}[T1] (Frechet) every singleton set \( \{ x \} \) is closed.
    \DItem{def:separation_axioms/T2}[T2] (Hausdorff) every two different points \( x, y \in X \) can be separated using neighborhoods, i.e. there exist disjoint open sets \( U \ni x \) and \( V \ni y \).
    \DItem{def:separation_axioms/T3}[T3] the space is \ref{def:separation_axioms/T0} and \ref{def:separation_axioms/regular}
    \DItem{def:separation_axioms/T3.5}[T3.5] the space is \ref{def:separation_axioms/T0} and \ref{def:separation_axioms/completely_regular}
    \DItem{def:separation_axioms/T4}[T4] the space is \ref{def:separation_axioms/T1} and \ref{def:separation_axioms/normal}
  \end{description}
\end{definition}

\begin{proposition}\label{thm:t2_iff_singleton_limits}
  The space \( (X, \Cal{T}) \) is Hausdorff (T2)\Tinyref{def:separation_axioms/T2} if and only if every net\Tinyref{def:topological_net} has at most one limit\Tinyref{def:net_limit_point}.
\end{proposition}
\begin{proof}
  \begin{description}
    \Implies Let \( X \) be Hausdorff and assume that there exists a net \( \{ x_i \}_{i \in I} \) such that \( y \) and \( z \) are not necessarily distinct limit points.

    Fix neighborhoods \( U \) of \( y \) and \( V \) of \( z \). Since both are limit points, there exist \( i_U \) and \( i_V \) such that \( i \geq i_U \) implies \( x_i \in U \) and \( i \geq i_V \) implies \( x_i \in V \).

    Since \( I \) is a directed set, there exists an upper bound \( i_0 \) of \( i_U \) and \( i_V \). Thus,
    \begin{equation*}
      i \geq i_0 \implies x_i \in U \cap V.
    \end{equation*}

    In particular, the intersection \( U \cap V \) is nonempty and is a neighborhood of both \( y \) and \( z \).

    If \( y \neq z \), then we have two distinct points such that no two neighborhoods of \( y \) and \( z \), respectively, are disjoint. This contradicts the assumption that \( X \) is Hausdorff. Thus\LEM \( y = z \).

    \ImpliedBy Conversely, if \( X \) is not Hausdorff\LEM, then for every two distinct points \( y \) and \( z \) and every two neighborhoods \( U \ni y \) and \( V \ni z \), their intersection \( U \cap V \) is nonempty.

    Let \( \Cal{U} \) and \( \Cal{V} \) be the sets of all neighborhoods of \( y \) and \( z \), respectively. Since they are both partially ordered by set inclusion \( \subseteq \), define the directed set \( (\Cal{U} \times \Cal{V}, \leq) \) with order
    \begin{equation*}
      (U, V) \leq (U', V') \iff U \supset V \land U' \supset V'.
    \end{equation*}

    For each \( (U, V) \in \Cal{U} \times \Cal{V} \), choose\AOC a point \( x_{(U, V)} \) from \( U \cap V \).

    Thus the net \( \{ x_{(U, V)} \}_{(U, V) \in \Cal{U} \cap \Cal{V}} \) has both \( y \) and \( z \) as its limit points, which contradicts our initial assumption.
  \end{description}
\end{proof}
