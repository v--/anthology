\subsection{Partial orders}\label{subsec:partial_orders}

\begin{Definition}\label{def:poset}
  Fix a set \( X \). A \Def{partially ordered set} structure or \Def{poset} structure can be defined equivalently as
  \begin{DefEnum}
    \ILabel{def:preordered_set/nonstrict} A \hyperref[def:preordered_set]{preorder} \( \leq \) on \( X \) such that \( \leq \) is \hyperref[def:binary_relation/antisymmetric]{antisymmetric} in addition to being \hyperref[def:binary_relation/reflexive]{reflexive} and \hyperref[def:binary_relation/transitive]{transitive}. This relation is called a \Def{nonstrict partial order} or simply a \Def{partial order}.

    \ILabel{def:preordered_set/strict} An \hyperref[def:binary_relation/irreflexive]{irreflexive} and \hyperref[def:binary_relation/transitive]{transitive} binary relation \( < \) on \( X \). This relation is called a \Def{strict partial order}.
  \end{DefEnum}

  In order for the equivalence to hold, the following condition needs to be satisfied:
  \begin{equation*}
    x \leq y \iff x < y \Tor x = y.
  \end{equation*}
\end{Definition}
\begin{proof}
  \begin{RefList}
    \IImplies{def:preordered_set/nonstrict}{def:preordered_set/strict} Let \( \leq \) be a nonstrict partial order. We will show that \( < \) is a strict partial order.
    \begin{RefList}
      \IRef{def:binary_relation/transitive} Let \( x < y \) and \( y < z \). Thus \( x \leq y \) and \( y \leq z \). By transitivity, \( x \leq z \).

      Additionally, \( x \neq y \) and \( y \neq z \). Assume\LEM that \( x = z \). By reflexivity of \( \leq \), we have \( z \leq x \) and, since \( y \leq z \), by transitivity we obtain \( y \leq x \). But since \( x \leq y \), by the antisymmetry of \( \leq \), we have \( x = y \), which contradicts the assumption that \( x < y \).

      Therefore \( x < z \).

      \IRef{def:binary_relation/irreflexive} Fix \( x \in X \). By reflexivity of \( \leq \), \( x \leq x \) and by definition,
      \begin{equation*}
        x < x \iff x \leq x \Tand x \neq x,
      \end{equation*}
      thus
      \begin{equation*}
        x < x \iff x \neq x.
      \end{equation*}

      Since the right side is false, the left side \( x < x \) is also false.
    \end{RefList}

    \IImplies{def:preordered_set/strict}{def:preordered_set/nonstrict} Let \( < \) be a strict partial order. We will show that \( \leq \) is a nonstrict partial order.
    \begin{RefList}
      \IRef{def:binary_relation/reflexive} Fix \( x \in X \) and assume\LEM that \( x \not\leq x \). Then \( x \neq x \) which contradicts the reflexivity of equality. Hence \( x \leq x \).

      \IRef{def:binary_relation/antisymmetric} Let \( x \leq y \) and \( y \leq x \), that is, either \( x = y \) or both \( x < y \) and \( y < x \) hold. Assume\LEM the latter. By the transitivity of \( \leq \), we have \( x < x \), which contradicts the irreflexivity of \( < \). Hence \( x = y \).

      \IRef{def:binary_relation/transitive} Let \( x \leq y \) and \( y \leq z \). Then we have four cases depending on which of \( x \), \( y \) and \( z \) are equal. Since both relations \( < \) and \( = \) are transitive, it follows that in all four cases \( x \leq z \).
    \end{RefList}
\end{RefListProof}

\begin{Proposition}\label{thm:preorder_to_partial_order}
  Let \( (X, \sim) \) be a preordered set. Use the \hyperref[def:derived_relations/symmetric]{symmetric closure} to define the equivalence relation
  \begin{equation*}
    \cong \coloneqq \Cl_S(\sim).
  \end{equation*}

  The quotient set \( X / \sim \) along with the induced relation
  \begin{equation*}
    [x] \leq [y] \iff x \sim y
  \end{equation*}
  is then a \hyperref[def:poset]{partially ordered set}.
\end{Proposition}

\begin{Proposition}\label{thm:partial_order_category_correspondence}
  To every \hyperref[def:poset]{poset} there corresponds exactly one \hyperref[def:category_cardinality]{small} \hyperref[def:thin_category]{thin} \hyperref[def:skeletal_category]{skeletal} category.

  Compare this result to \cref{thm:preorder_category_correspondence}.
\end{Proposition}
\begin{proof}
  The statement follows from \cref{thm:preorder_category_correspondence} by noting that the procedure \cref{thm:preorder_to_partial_order} on the poset makes the corresponding category skeletal.
\end{proof}

\begin{Definition}\label{def:category_of_posets}
  Partially ordered sets and monotone maps form a subcategory of \cref{def:category_of_preordered_sets}. We denote the category of posets by \( \Cat{Pos} \).
\end{Definition}
