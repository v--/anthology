\section{Dentable sets}\label{sec:dentable_sets}

\begin{definition}\cite[Example 3.2(a]{Phelps1993})
  \label{def:banach_support_function}
  Let \( X \) be a Banach space.

  We define the \underLine{support function \( \sigma_{A^*} \) for the set of functionals \( A^* \subseteq X^* \)} by
  \begin{align*}
    &\sigma_{A^*}: X \to \BB{R} \cup \{ \infty \} \\
    &\sigma_{A^*}(x) \coloneqq \sup \{ \Prod {x^*} x \colon x^* \in A^* \}
  \end{align*}

  and the \underLine{weak* support function \( \sigma^*_A \) for the set of points \( A \subseteq X \)} by
  \begin{align*}
    &\sigma^*_A: X^* \to \BB{R} \cup \{ \infty \} \\
    &\sigma^*_A(x^*) \coloneqq \sup \{ \Prod {x^*} x \colon x \in A \}.
  \end{align*}
\end{definition}

\begin{definition}\cite[definition 2.17]{Phelps1993}
  \label{def:banach_slice}
  Given a linear functional \( x^* \), a nonempty subset \( A \) of \( X \) and a \underLine{diameter} \( \alpha > 0 \), the value \( S(x^*, A, \alpha) \) is called a \underLine{slice} of \( A \), where
  \begin{align*}
    &S: X^* \times \Power(X) \times \BB{R}^{>0} \mapsto \Power(A) \\
    &S(x^*, A, \alpha) \coloneqq \{ x \in A \colon \Prod {x^*} x > \sigma_A^*(x^*) - \alpha \}.
  \end{align*}

  We define a weak* slice of \( A^* \subseteq X^* \) as \( S^*(x, A^*, \alpha) \), where
  \begin{align*}
    &S^*: X \times \Power(X) \times \BB{R}^{>0} \mapsto \Power(A) \\
    &S^*(x, A^*, \alpha) \coloneqq \{ x^* \in A^* \colon \Prod {x^*} x > \sigma_{A^*}(x) - \alpha \}.
  \end{align*}

  If we need to make the underlying space explicit, we will use \( S_X(x^*, A, \alpha) \) and \( S_X^*(x, A^*, \alpha) \).
\end{definition}

\begin{definition}\cite[definition 5.1]{Phelps1993}
  \label{def:dentability}
  A subset \( A \) of a Banach space \( X \) is called \underLine{dentable} if it admits slices of arbitrarily small diameter, i.e. for every \( \varepsilon > 0 \) there exist a functional \( x^* \in X^* \) and a diameter \( \alpha > 0 \), such that \( \Diam S(x^*, A, \alpha) < \varepsilon \).

  Weak* dentability is defined in an obvious way.
\end{definition}

\begin{definition}\cite[definition 5.2]{Phelps1993}
  \label{def:radon-nikodym-property}
  The space \( X \) is said to have the \underLine{Radon-Nikodym property (RNP)} if every nonempty bounded set \( A \) of \( X \) is dentable.
\end{definition}

\begin{proposition}
  \label{thm:weak_dentable_sets_are_dentable}
  Let \( X \) be a Banach space and \( A^* \subseteq X^* \) be a weak*-dentable set. Then \( A^* \) is dentable in \( X^* \).
\end{proposition}
\begin{proof}
  Let \( \varepsilon > 0 \) and let \( x \in X \) and \( \alpha > 0 \) be such that \( \Diam S^*(x, A^*, \alpha) < \varepsilon \).
  We denote by \( J(x) \) the embedding of \( x \in X \) into the double-dual \( X^{**} \) and by \( T(J(x), A^*, \alpha) \) the slice of \( A^* \) in \( X^* \). We have that
  \begin{align*}
    S^*(x, A^*, \alpha)
    &=
    \{ x^* \in A^* \colon \Prod {x^*} x > \sigma_{A^*}(x) - \alpha \}
    = \\ &=
    \{ x^* \in A^* \colon \Prod {x^*} x > \sup \{ \Prod {y^*} x \colon y^* \in A^* \} - \alpha \}
    = \\ &=
    \{ x^* \in A^* \colon \Prod {J(x)} {x^*} > \sup \{ \Prod {J(x)} {y^*} \colon y^* \in A^* \} - \alpha \}
    =
    T(J(x), A^*, \alpha),
  \end{align*}

  Since \( J \) is an isometry, this equality implies that
  \begin{align*}
    \Diam T(J(x), A^*, \alpha) = \Diam S(x, A^*, \alpha) < \varepsilon.
  \end{align*}

  Hence \( A^* \) admits arbitrarily small slices in \( X^* \), i.e. it is dentable in \( X^* \).
\end{proof}
