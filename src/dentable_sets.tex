\subsection{Dentable sets}\label{subsec:dentable_sets}

\begin{Definition}\label{def:dentability}\cite[definition 5.1]{Phelps1993}
  A subset \( A \) of a Banach space \( X \) is called \Def{dentable} if it admits slices of arbitrarily small diameter, i.e. for every \( \varepsilon > 0 \) there exist a functional \( x^* \in X^* \) and a diameter \( \alpha > 0 \), such that \( \Diam S(x^*, A, \alpha) < \varepsilon \).

  Weak* dentability is defined in an obvious way.
\end{Definition}

\begin{Definition}\label{def:radon-nikodym-property}\cite[definition 5.2]{Phelps1993}
  The space \( X \) is said to have the \Def{Radon-Nikodym property (RNP)} if every nonempty bounded set \( A \) of \( X \) is dentable.
\end{Definition}

\begin{Proposition}\label{thm:weak_dentable_sets_are_dentable}
  Let \( X \) be a Banach space and \( A^* \subseteq X^* \) be a weak*-dentable set. Then \( A^* \) is dentable in \( X^* \).
\end{Proposition}
\begin{proof}
  Let \( \varepsilon > 0 \) and let \( x \in X \) and \( \alpha > 0 \) be such that \( \Diam S^*(x, A^*, \alpha) < \varepsilon \).
  We denote by \( J(x) \) the embedding of \( x \in X \) into the double-dual \( X^{**} \) and by \( T(J(x), A^*, \alpha) \) the slice of \( A^* \) in \( X^* \). We have that
  \begin{align*}
    S^*(x, A^*, \alpha)
    &=
    \{ x^* \in A^* \colon \Prod {x^*} x > \sigma_{A^*}(x) - \alpha \}
    = \\ &=
    \{ x^* \in A^* \colon \Prod {x^*} x > \sup \{ \Prod {y^*} x \colon y^* \in A^* \} - \alpha \}
    = \\ &=
    \{ x^* \in A^* \colon \Prod {J(x)} {x^*} > \sup \{ \Prod {J(x)} {y^*} \colon y^* \in A^* \} - \alpha \}
    =
    T(J(x), A^*, \alpha),
  \end{align*}

  Since \( J \) is an isometry, this equality implies that
  \begin{equation*}
    \Diam T(J(x), A^*, \alpha) = \Diam S(x, A^*, \alpha) < \varepsilon.
  \end{equation*}

  Hence \( A^* \) admits arbitrarily small slices in \( X^* \), i.e. it is dentable in \( X^* \).
\end{proof}
