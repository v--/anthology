\section{Real analysis}\label{sec:real_analysis}
\subsection{Real vector space geometry}\label{subsec:real_vector_space_geometry}

\begin{remark}\label{remark:geometry_of_vector_spaces}
  Let \( X \) be a vector space\Tinyref{def:vector_space} over the normed\Tinyref{remark:normed_fields} extension field\Tinyref{def:field_extension} \( F \) of \( \R \). We will list here definitions referring to this general setting and later work with either \( F = \R \) or \( F = \C \).
\end{remark}

\begin{definition}\label{def:special_linear_combinations}
  Let \( t_1, \ldots, t_n \in F \) and \( x_1, \ldots, x_n \in X \). Consider the linear combination\Tinyref{def:linear_combination}
  \begin{equation*}
    x \coloneqq \sum_{k=1}^n t_k x_k.
  \end{equation*}

  We say that it is
  \begin{defenum}
    \DItem{def:special_linear_combinations/affine} an \Def{affine combination} if \( \sum_{k=1}^n t_k = 1 \).
    \DItem{def:special_linear_combinations/conic} a \Def{conic combination} if all of the coefficients are nonnegative real numbers.
    \DItem{def:special_linear_combinations/convex} a \Def{convex combination} if it is both affine and conic. A convex combination of two elements \( x, y \in X \) is usually written as \( tx + (1-t)y \) for some scalar \( t \in [0, 1] \).
  \end{defenum}
\end{definition}

\begin{definition}\label{def:linear_combination_hulls}
  Let \( A \) be a subset of \( X \). We define the \Def{convex hull} \( \Conv A \) (resp. \Def{affine hull} or \Def{conic hull}) of \( A \) by the set of all convex (resp. affine or conic) combinations of finite subsets of \( A \).

  The convex hull \( \Conv A \) of two elements \( x, y \in X \) is often called the \Def{segment between \( x \) and \( y \)} and is denoted as
  \begin{equation*}
    [x, y] \coloneqq \{ tx + (1-t)y \colon t \in [0, 1] \}.
  \end{equation*}

  If the set \( A \) is equal to its convex (resp. affine or conic) hull, we say that it is a \Def{convex set} (resp. \Def{affine set} or \Def{cone})
\end{definition}

\begin{definition}\label{def:simplex}
  Let \( v_1, \ldots, v_n \) be linearly independent vectors and \( v_0 \) be any other vector. The convex hull \( S \) of the vectors \( v_0 + v_1, \ldots, v_0 + v_n \) is called an \Def{n-simplex}.

  The convex hull of any nonempty subset of \( \{ v_0, \ldots, v_n \} \) of cardinality \( m + 1 \) is an \( m \)-simplex and is called an \Def{\( m \)-face of \( S \)}.
\end{definition}

\begin{remark}\label{remark:vector_space_set_operations}
  In analysis, it is customary to perform addition and scalar multiplication with sets in vector spaces. That is, if \( A \) and \( B \) are sets, it is customary to write
  \begin{align*}
    A + B \coloneqq \{ a + b \colon a \in A, b \in B \},
  \end{align*}
  and if \( \alpha \) is a scalar,
  \begin{equation*}
    \alpha A \coloneqq \{ \alpha a \colon a \in A \}.
  \end{equation*}

  If we use the convention in \cref{remark:singleton_sets}, this turns sets in vector spaces into an ill-defined algebraic structure, but it is useful when no metric is available, e.g. in topological vector spaces\Tinyref{def:topological_vector_space} or set-valued analysis.
\end{remark}

\begin{definition}\label{def:absorbing_set}\cite[25]{Rudin1991}
  We say that the set \( A \) is \Def{absorbing} if, for any point \( x \in X \), there exists a positive real number \( t > 0 \) such that \( tx \in A \).
\end{definition}

\begin{definition}\label{def:balanced_set}\cite[6]{Rudin1991}
  We say that the set \( A \) is \Def{balanced} if, for any scalar \( t \in F \) with \( \Abs{t} \leq 1 \) we have \( tA \subseteq A \).
\end{definition}

\begin{definition}\label{def:minkowski_functional}
  Let \( A \) is an absorbing\Tinyref{def:absorbing_set} convex\Tinyref{def:special_linear_combinations/convex} set.

  We define the corresponding \Def{Minkowski functional}
  \begin{align*}
    &\mu_A: X \to [0, \infty),
    &\mu_A(x) = \inf \{ t > 0 \colon x \in tA \}.
  \end{align*}
\end{definition}
\begin{proof}
  We will prove that \( \mu_A(x) \) is always a nonnegative real number. Obviously
  \begin{equation*}
    \mu_A(x) \geq 0
  \end{equation*}
  since the infimum over \( \R_{>0} \) is \( 0 \).

  Now fix \( x \in X \). Since \( A \) is an absorbing set, there exists \( t_0 > 0 \) such that \( t_0 x \in A \). We need to take the infimum of all such numbers. This infimum exists since \( \R \) is complete and the set over which we take the minimum is bounded.
\end{proof}

\begin{proposition}\label{thm:rn_bounded_iff_totally_bounded}
  A set in \( \R^n \) is totally bounded\Tinyref{def:totally_bounded_set} if and only if it is bounded\Tinyref{def:metric_space/bounded_set}.
\end{proposition}
\begin{proof}
  \begin{description}
    \Implies Follows from \cref{thm:totally_bounded_sets_are_bounded}.
    \ImpliedBy Let \( A \) be a bounded set in \( \R^n \) and let \( \B(x, r) \) be a ball containing \( A \). Fix \( \varepsilon > 0 \).

    Denote by \( e_1, \ldots, e_n \) the basis\Tinyref{def:left_module_hamel_basis} of \( \R^n \). Denote by \( m \) the smallest integer such that \( m \varepsilon \geq r \).

    We can create a grid around \( \B(x, r) \) as follows:

    Define the set
    \begin{equation*}
      \left\{ x + \sum_{i=1}^n [k_i \varepsilon] e_i \colon \forall i = 1, \ldots, n: k_i = 1, \ldots, m \right\}.
    \end{equation*}

    is finite. Furthermore, it is an \( \varepsilon \)-net of \( A \). Indeed, let \( y \in A \). Denote its coordinates along \( e_1, \ldots, e_n \) by \( y_1, \ldots, y_n \). Then \( y \) is contained in the ball
    \begin{equation*}
      \B\left(x + \sum_{i=1}^n [\Ceil(y_i) \varepsilon] e_i, \varepsilon \right).
    \end{equation*}
  \end{description}
\end{proof}

\begin{theorem}[Heine-Borel theorem]\label{thm:heine_borel}
  A set in \( \R^n \) is compact in the sense of \cref{def:compact_set} if and only if it is closed and bounded.
\end{theorem}
\begin{proof}
  Follows from \cref{thm:rn_bounded_iff_totally_bounded} and \cref{thm:complete_metric_space_compact_conditions/closed_totally_bounded}.
\end{proof}

\begin{proposition}\label{thm:real_supremum_of_closure}
  The supremum (resp. infimum) of a set \( A \subseteq \R \), if it exists, is equal to the supremum (resp. infimum) of \( \Cl A \).
\end{proposition}
\begin{proof}
  \begin{description}
    \Implies Denote by \( M \) the supremum of \( A \). Assume\LEM that it is not a supremum of \( \Cl A \), that is, there exists an upper bound \( M' \) of \( \Cl A \) such that \( M' < M \). But this is impossible because \( A \subseteq \Cl A \).

    Therefore \( M \) is a supremum of \( \Cl A \).

    \ImpliedBy Denote by \( M \) the supremum of \( \Cl A \). Assume\LEM that it is not a supremum of \( A \), that is, there exists an upper bound \( M' \) of \( A \) such that \( M' < M \).

    Let \( \{ x_i \}_{i=1}^\infty \subseteq A \) be a sequence that converges to \( M \). Then
    \begin{equation*}
      x_i < M' < M.
    \end{equation*}

    By \cref{thm:squeeze_lemma/sequences}, we have \( M' = M \), which contradicts our choice of \( M' \). Thus \( M \) is the supremum of \( A \).
  \end{description}
\end{proof}

\begin{proposition}\label{thm:real_bounded_set_has_supremum}
  Every nonempty bounded set\Tinyref{def:metric_space/bounded_set} in \( \R \) has a supremum and infimum.
\end{proposition}
\begin{proof}
  Let \( A \subseteq \R \) be a nonempty bounded set. By \cref{thm:heine_borel}, the set \( \Cl A \) is compact. By \cref{thm:weierstrass_extreme_value_theorem}, the identity function \( \Id: \R \to \R \) attains its minimum \( m \) and maximum \( M \) on \( \Cl A \). Note that both \( m \) and \( M \) do not have to belong to \( A \), but \( m \) is a lower bound and \( M \) is an upper bound of the set \( A \).

  If we take any other upper bound \( M' \) of \( A \), then by \cref{thm:real_supremum_of_closure},
  \begin{equation*}
    M' \geq \sup A = \sup \Cl A = M.
  \end{equation*}

  Hence \( M \) is the least upper bound of \( A \).

  We can analogously prove that \( m \) is the greatest lower bound of \( A \).
\end{proof}
