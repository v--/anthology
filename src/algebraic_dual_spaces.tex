\subsection{Algebraic dual spaces}\label{subsec:algebraic_dual_spaces}

\begin{definition}\label{def:dual_space}
  Let \( V \) be a vector space over \( F \). The vector space\Tinyref{thm:functions_over_ring_form_algebra} of all linear functions\Tinyref{def:linear_operator} from \( V \) to \( F \) is called the \Def{algebraic dual space} of \( V \) and is denoted by \( V^* \). The functions themselves are called \Def{linear functionals}.
\end{definition}

\begin{definition}\label{def:double_dual_canonical_embedding}
  Fix a vector space \( V \). We define the \Def{canonical embedding} into the double dual \( V^{**} \) of \( V \) by
  \begin{align*}
    &\Phi: V \to V^{**} \\
    &\Phi(x) \coloneqq (\varphi \mapsto \varphi(x)),
  \end{align*}
  where \( \varphi \in V^* \).
\end{definition}

\begin{proposition}\label{thm:finite_dimensional_dual_space_is_isomorphic}
  The dual vector space of a finite-dimensional vector space has the same dimension.
\end{proposition}
\begin{proof}
  Let \( V \) be an \( n \)-dimensional vector space over \( F \) and let \( B \) be a basis of \( V \). For each \( b \in B \), define its dual vector on \( V^* \) as the linear extension\Tinyref{thm:linear_map_iff_function_on_basis} of the functions
  \begin{align*}
    &\varphi: B \to F \\
    &\varphi(x) \coloneqq \begin{cases}
      1, &x = b \\
      0, &x \neq b
    \end{cases}
  \end{align*}
  from the basis to the whole space. Denote the dual basis vector of \( b \) by \( b^* \).

  We will now show that the set \( B^* \coloneqq \{ b^* \colon b \in B \} \) forms a basis of \( V^* \).

  Fix \( x^* \in V^* \). Define
  \begin{equation*}
    y^* \coloneqq \sum_{b \in B} x^*(b) b^*.
  \end{equation*}

  The linear functions \( x^* \) and \( y^* \) evidently agree on the basis \( B \). By \cref{thm:linear_maps_agree_on_free_module_if_they_agree_on_basis}, they agree on the whole space.

  Hence \( B^* \) is a basis of \( V^* \). Note that it has the same cardinality as the basis of \( B \).
\end{proof}

\begin{remark}\label{remark:finite_dimensional_dual_space_isomorphism}
  By \cref{thm:finite_dimensional_spaces_are_isomorphic}, if \( V \) is an \( n \)-dimensional vector space over \( F \), both \( V \) and its dual \( V^* \) are isomorphic to \( F^n \).

  In practice, it is useful to distinguish between vectors and functionals. This is why we can regard vectors \( x \in V \) as column vectors\Tinyref{def:array/column_vector} and functionals \( x^* \in V^* \) as row vectors\Tinyref{def:array/row_vector}. This is consistent with \cref{thm:finite_dimensional_operators_are_isomorphic_to_matrices}, where we regard linear operators as matrices that act on vectors by multiplication.

  For example, if we have the differentiable\Tinyref{def:derivatives/classical} function \( f(x, y) = xy \), we can regard its gradient at the point \( (\Ol x, \Ol y) \) as the row vector
  \begin{align*}
    f'(\Ol x, \Ol y) =
    \begin{pmatrix}
      \Ol y & \Ol x
    \end{pmatrix}.
  \end{align*}

  This is a linear functional that can acts on regular (column) vector by multiplying them from the left.
\end{remark}
