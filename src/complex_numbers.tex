\subsection{Complex numbers}\label{subsec:complex_numbers}

\begin{definition}\label{def:complex_numbers}
  We give a few equivalent definition of the field\Tinyref{def:field} \( \C \) \Def{complex numbers}. Informally, there are numbers of the form \( a + bi \), where \( a, b \in \R \) and \( i = \sqrt{-1} \). In order to find the multiplicative inverse of the nonzero polynomial \( a + bi \), we assume that division is well defined and proceed as follows:
  \begin{equation}\label{def:complex_numbers/inverse}
    \frac 1 {a + bi} = \frac {a - bi} {(a + bi)(a - bi)} = \frac{a - bi}{a^2 + b^2}.
  \end{equation}

  The closest to this informal definition is \cref{def:complex_numbers/polynomials}.

  \begin{defenum}
    \DItem{def:complex_numbers/polynomials} The most \enquote{algebraic} way to define complex numbers is as the polynomial\Tinyref{def:polynomial} quotient ring\Tinyref{thm:polynomial_quotient_rings_equinumerous_with_module_of_polynomials}
    \begin{equation*}
      \C \coloneqq \R[X] / \Gen{X^2 + 1}.
    \end{equation*}

    Elements of \( \C \) can be identified with real polynomials of the form \( bX + a \). See \cref{ex:polynomial_quotient_rings_gaussian_integers} for a broader discussion.
    Define \( i \coloneqq X \). We have
    \begin{equation*}
      i \cdot i = X^2 = -1 \pmod {X^2 + 1}.
    \end{equation*}

    Thus \( i \) is indeed the square root of \( -1 \). We will write
    \begin{equation*}
      a + bi = bX + a.
    \end{equation*}

    It is shown in \cref{ex:polynomial_quotient_rings_gaussian_integers} that multiplication modulo \( X^2 + 1 \) gives
    \begin{equation}\label{def:complex_numbers/polynomials/multiplication}
     (bX + a) (dX + c) = (ad + bc)X + (ac - bd) \pmod {X^2 + 1}.
    \end{equation}

    The multiplicative inverse of \( a + bi \) is then indeed \cref{def:complex_numbers/inverse}.

    The canonical embedding \( \iota: \R \to \C \) is then the standard polynomial embedding.

    \DItem{def:complex_numbers/matrices} The complex numbers can also be defined as the matrix ring\Tinyref{def:algebra_of_matrices}
    \begin{equation*}
      \C \coloneqq \left\{
        \begin{pmatrix}
          a & b \\
          -b & a
        \end{pmatrix}
      \colon a, b \in \R \right\}
    \end{equation*}
    with the usual matrix multiplication. The canonical embedding \( \iota: \R \to \C \) is then
    \begin{equation*}
      \iota(a) \coloneqq \begin{pmatrix}
        a & 0 \\
        0 & a
      \end{pmatrix}
    \end{equation*}

    \DItem{def:complex_numbers/tuples} Finally, we can define \( \C \) is the algebra\Tinyref{def:algebra_over_ring} obtained from the vector space \( \R^2 \) with the multiplication operation emulating \cref{def:complex_numbers/polynomials/multiplication} as
    \begin{align*}
      &\cdot: \C \times \C \to \C \\
      &(a, b) \cdot (c, d) \coloneqq (ac - bd, ad + bc).
    \end{align*}

    The canonical embedding \( \iota: \R \to \C \) is then
    \begin{equation*}
      \iota(a) \coloneqq (a, b).
    \end{equation*}
  \end{defenum}

  We define the unary \Def{complex conjugation} operation as \( \Ol{a + bi} \coloneqq a - bi \) and the \Def{absolute value}\Tinyref{def:absolute_value} as
  \begin{equation*}
    \Abs{a + bi} \coloneqq \sqrt{a^2 + b^2}.
  \end{equation*}

  For a complex number \( z = a + bi \) we denote
  \begin{align*}
    \Re z = a && \Im z = b
  \end{align*}
  and call them the \Def{real} and \Def{imaginary} parts of \( z \).
\end{definition}

\begin{theorem}[Fundamental theorem of algebra]\label{thm:fundamental_theorem_of_algebra}
  The field \( \C \) of complex numbers is algebraically closed\Tinyref{def:algebraically_closed_field}.
\end{theorem}

\begin{definition}\label{def:gaussian_integers}
  The \Def{Gaussian integers} are a subring of the complex numbers. They are defined as numbers with integer real and imaginary parts and are commonly denoted as
  \begin{equation*}
    \Z[i] \coloneqq \{ a + bi \colon a, b \in \Z \}.
  \end{equation*}
\end{definition}

\begin{theorem}\label{thm:linear_functionals_over_c}
  Let \( X \) be a vector space\Tinyref{def:vector_space} over \( \C \). There is a bijection between the real-valued and the complex-valued linear functionals on \( X \).
\end{theorem}
\begin{proof}
  Let \( c: X \to \C \) be a complex-valued linear functional. Denote \( a(x) \coloneqq \Re c(x) \) and \( b(x) \coloneqq \Im c(x) \). Then \( a: X \to \R \) and \( b: X \to \R \) are linear functionals. We will show that \( a(x) \) uniquely determines \( b(x) \) and hence \( c(x) \).

  Note that \( c(ix) = a(ix) + i b(ix) = i a(x) - b(x) \). Therefore \( b(x) = a(ix) - c(ix) \) and
  \begin{equation*}
    c(x) = a(x) + i (a(ix) - c(ix)) = a(x) - a(x) + c(x) = c(x).
  \end{equation*}
\end{proof}

\begin{remark}\label{remark:linear_functionals_over_c}
  \Cref{thm:linear_functionals_over_c} allows us to identify the dual space \( X* \) of a complex vector space \( X \) with \( \Hom(X, \R) \) in the case of an algebraic dual\Tinyref{def:dual_vector_space} or with the corresponding subspace in the case of a continuous dual space\Tinyref{def:continuous_dual_space}.

  This allows us to reuse some of the theory for real vector spaces, for example hyperplane separation\Tinyref{def:hyperplane_separation}.
\end{remark}
