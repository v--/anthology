\subsection{Real numbers}\label{subsec:real_numbers}

\begin{definition}\label{def:complex_numbers}
  The \Def{complex numbers} \( \Co \) are the algebra\Tinyref{def:algebra_over_ring} obtained from the vector space \( \R^2 \) with the multiplication operation
  \begin{align*}
    &\cdot: \Co \times \Co \to \Co \\
    &(a, b) \cdot (c, d) \coloneqq (ac - bd, ad + bc).
  \end{align*}

  We define the unary \Def{conjugation} operation as \( \Ol{(a, b)} \coloneqq (a, -b) \).

  We also define the canonical embedding
  \begin{align*}
    &\iota: \R \to \Co \\
    &\iota(x) \coloneqq (x, 0)
  \end{align*}
  and the shorthand \( i \coloneqq (0, 1) \) so that for any \( a, b \in \R \) we have
  \begin{equation*}
    (a, b) = \iota(a) + bi
  \end{equation*}
  usually written as
  \begin{equation*}
    (a, b) = a + bi.
  \end{equation*}

  For a complex number \( z = a + bi \) we denote
  \begin{align*}
    \Real z = a && \Imag z = b
  \end{align*}
  and call them the \Def{real} and \Def{imaginary} parts of \( z \).
\end{definition}

\begin{theorem}[Fundamental theorem of algebra]\label{thm:fundamental_theorem_of_algebra}
  Every nonconstant complex polynomial has at least one root.
\end{theorem}

\begin{corollary}\label{thm:complex_polynomials_have_n_roots}
  A complex polynomial of degree \( n \) has exactly \( n \) roots.
\end{corollary}
\begin{proof}
  Follows from \cref{thm:integral_domain_polynomial_root_limit} and \cref{thm:fundamental_theorem_of_algebra}.
\end{proof}

\begin{definition}\label{def:gaussian_integers}
  The \Def{Gaussian integers} are a subring of the complex numbers. They are defined as numbers with integer real and imaginary parts and are commonly denoted as
  \begin{equation*}
    \Z[i] \coloneqq \{ a + bi \colon a, b \in \Z \}.
  \end{equation*}
\end{definition}
