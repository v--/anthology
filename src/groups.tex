\section{Groups}\label{sec:groups}

\begin{definition}\label{def:group}(See \cref{ex:algebraic_theory_language})
  A \underLine{magma} is a nonempty set \( G \) with a function 
  \begin{align*}
    \cdot: G \times G \to G,
  \end{align*}
  called the \underLine{magma operation}. Denote \( \cdot(a, b) \) by juxtaposition, that is, \( ab \).

  \begin{description}
    \DItem{associativity}{def:group/associativity} If \( (ab)c = a(bc) \) for all \( a, b, c \in G \), we say that the operation is \underLine{associative} and we call the pair \( (G, \cdot) \) a \underLine{semigroup}.
    \DItem{identity}{def:group/identity} An \underLine{identity} is an element \( e \in G \) such that \( ea = ae = a \) for every \( a \in G \). If an identity exists, it is unique by~\cref{def:group_properties/unique_identity}. If \( (G, \cdot) \) is a semigroup with an identity, we call it a \underLine{monoid} (see also \cref{def:monoidal_category} and \cref{note:groupoids}). We can also consider the identity in a monoid to be a function with arity 0. If the monoid is not obvious from the context, we write \( e_G \).
    \DItem{inverse}{def:group/inverse} For each \( a \in G \), the \underLine{inverse} is an element \( a^{-1} \in G \) such that \( aa^{-1} = e \). It is unique for each element by~\cref{def:group_properties/unique_inverse}. If \( (G, \cdot) \) is a monoid and each element has an inverse, we call it a \underLine{group}. We can also consider the inverse in a group to be a function with arity 1.
    \DItem{commutativity}{def:group/commutativity} If \( ab = ba \) for all \( a, b \in G \), we say that the magma operation is \underLine{commutative}. If \( (G, \cdot) \) is a group with a commutative operation, we call it a \underLine{commutative group} or \underLine{abelian group}.
  \end{description}

  We define \( a^n \) for integer \( n \) as
  \begin{align*}
    a^n \coloneqq \begin{cases}
      e, &n = 0 \\
      a \cdot a^{n-1}, &n > 0 \\
      (a^{-n})^{-1}, &n < 0.
    \end{cases}
  \end{align*}

  We say that
  \begin{defenum}
    \item\label{def:group/trivial_group} the group \( \{ e \} \) is called the \underLine{trivial group}.
    \item\label{def:group/subgroup} the subset \( H \subseteq G \) is a \underLine{submagma (resp. subsemigroup, submonoid, subgroup, etc) of \( G \)} if \( H \) is closed under the magma operation, that is, \( ab \in H \) whenever \( a, b \in H \).
    \item\label{def:group/trivial_subgroup} the subgroup \( \{ e_G \} \) of \( G \) is called the \underLine{trivial subgroup}
    \item\label{def:group/proper_subgroup} all subgroups except for \( G \) itself are \underLine{proper subgroups}.
    \item\label{def:group/finite_order} \( a \) is of \underLine{finite order} \( \Ord(a) = n \) if \( n \) is the smallest positive integer such that \( a^n = e \).
    \item\label{def:group/infinite_order} \( a \) is of \underLine{infinite order} \( \Ord(a) = \infty \) if \( a^n \neq e \) for any positive integer \( n \).
    \item\label{def:group/absorbing_element} \( a \) is an \underLine{absorbing element} if \( ba = ab = a \) for any \( b \in G \).
  \end{defenum}
\end{definition}

\begin{note}\label{note:additive_group}
  Groups are often used to describe sets of invertible functions\Tinyref{def:function_invertibility} where the group operation is composition (see~\cref{note:groupoids} for a categorical viewpoint). As such, the group operation is usually denoted by juxtaposition as in~\cref{def:group}.

  Since composition of functions is not commutative in general, abelian groups are usually not sets of invertible functions. Since abelian groups are \( \BB{Z} \)-modules~\Tinyref{thm:abelian_group_iff_z_module}, we usually denote the group operation in abelian groups by \( a + b \) instead of \( ab \), the inverse by \( -a \) instead of \( a^{-1} \), and the unit by \( 0 \).

  To make a further distinction, if the operation is denoted by juxtaposition, we say that the group is a \underLine{multiplicative group}, and if the operation is denoted by \( + \), we say that the group is an \underLine{additive group}. This terminology usually, but not necessarily, coincides with the group being abelian.
\end{note}

\begin{definition}\label{def:symmetric_group}
  Let \( A \) be an arbitrary set. We associate with \( A \) the \underLine{symmetric group} \( S(A) \) whose elements are bijections\Tinyref{def:function_invertibility/bijection} from \( A \) to itself and whose group operation is function composition. We also call elements of \( S(A) \) \underLine{permutations} of \( A \), especially if \( A \) is a finite set.
\end{definition}

\begin{proposition}\label{def:group_properties}
  Any group \( G \) has the following basic properties:
  \begin{defenum}
    \item\label{def:group_properties/unique_identity} The identity \( e \) is unique (also holds for monoids).
    \item\label{def:group_properties/unique_inverse} The inverse \( a^{-1} \) of every element is unique.
    \item\label{def:group_properties/identity_inverse} The identity \( e \) is its own inverse.
    \item\label{def:group_properties/inverse_composition} \( (ab)^{-1} = b^{-1} a^{-1} \).
    \item\label{def:group_properties/double_inverse} For any \( a \in G \), \( a = (a^{-1})^{-1} \)
    \item\label{def:group_properties/negative_power} For any \( a \in G \) and positive integer \( n \), \( a^{-n} = (a^n)^{-1} = (a^{-1})^n \)
    \item\label{def:group_properties/cancellation} For any \( a, b, c \in G \), the cancellation laws \( a = b \iff ac = bc \iff ca = cb \) hold.
  \end{defenum}
\end{proposition}
\begin{proof}\mbox{}
  \begin{itemize}
    \item[\ref{def:group_properties/unique_identity}] If \( e' \) is another identity element of \( G \), then \( e' = e' e = e \).
    \item[\ref{def:group_properties/unique_inverse}] If \( b \) and \( c \) are both inverses of \( a \), then \( b = eb = cab = ce = c \).
    \item[\ref{def:group_properties/identity_inverse}] \( ee = e \).
    \item[\ref{def:group_properties/inverse_composition}]
    \begin{align*}
      (ab) (b^{-1} a^{-1})
      =
      a (b b^{-1}) a^{-1}
      =
      e
      =
      b^{-1} (a^{-1} a) b
      =
      (b^{-1} a^{-1}) (ab).
    \end{align*}

    \item[\ref{def:group_properties/double_inverse}]
    \begin{align*}
      (a^{-1})^{-1}
      =
      a a^{-1} (a^{-1})^{-1}
      =
      a.
    \end{align*}

    \item[\ref{def:group_properties/negative_power}] Using~\ref{def:group_properties/double_inverse},
    \begin{align*}
      a^{-n}
      =
      (a^n)^{-1}
      =
      a^{-1} \cdots a^{-1}
      =
      (a^{-1})^n.
    \end{align*}

    \item[\ref{def:group_properties/cancellation}] If \( a = b \), obviously \( ac = bc \) and \( ca = cb \). Now if \( ac = bc \), we have
    \begin{align*}
      a = acc^{-1} = bcc^{-1} = b.
    \end{align*}

    The case \( ca = bc \) is analogous.
  \end{itemize}
\end{proof}

\begin{definition}\label{def:groupoid}
  A \underLine{groupoid} is a category\Tinyref{def:category} in which all morphisms are isomorphisms\Tinyref{def:morphism_invertibility}.
\end{definition}

\begin{definition}\label{note:groupoids}
  Let \( \Bold G \) be a locally small with a single object \( g \). Then the endomorphisms of \( g \) form a monoid\Tinyref{def:group} under composition and the subcategory of \( \Bold G \) in which all morphisms are isomorphisms forms a group\Tinyref{def:group}. Thus, if \( \Bold G \) is a locally small groupoid with a single object \( g \), then the endomorphisms of \( g \) are automorphisms and, thus, they form a group under composition.
\end{definition}

\begin{definition}\label{def:group_cosets}
  Let \( H \subseteq G \) be a subgroup of \( G \) and let \( a \in G \). Consider the sets
  \begin{align*}
    aH \coloneqq \{ ah \colon h \in H \}
    &&
    Ha \coloneqq \{ ha \colon h \in H \}
  \end{align*}
  called the \underLine{left and right cosets of \( H \) with respect to \( a \)}.
\end{definition}

\begin{proposition}\label{thm:coset_partition}
  The left (resp. right) cosets of a subgroup \( H \) of \( G \) partition\Tinyref{def:set_partition} \( G \).
\end{proposition}
\begin{proof}
  To each element \( a \in G \) there corresponds a coset \( a \in aH \) (since \( H \) contains the identity as a subgroup).

  Two cosets \( aH \) and \( bH \) are either disjoint or equal. Indeed, if they are not disjoint, then there exists \( g \in aH \cap bH \) and thus \( g = ax = by \) for some \( x, y \in H \). Thus \( a = a x x^{-1} = b y x^{-1} \) and since \( y x^{-1} \in H \), we have that \( a \in bH \). Furthermore, for any \( z \in H \), we have \( az = b(y x^{-1} z) \in bH \), hence \( aH \subseteq bH \). After obtaining the converse inclusion, we conclude \( aH = bH \).
\end{proof}

\begin{definition}\label{def:group_homomorphism}
  Let \( M \) and \( N \) be magmas. We say that the function \( f: M \to N \) is a \underLine{magma homomorphism (resp. semigroup, monoid or group homomorphism)} if it is \underLine{compatible with the magma structure on \( M \) and \( N \)}, that is,
  \begin{description}
    \DItem{compatibility}{def:group_homomorphism/compatibility} \( f(ab) = f(a) f(b) \) for all \( a, b \in M \).
  \end{description}

  If \( M \) and \( N \) are monoids, the \underLine{kernel} of \( f \) is defined as the preimage\Tinyref{def:function_preimage} \( f^{-1}(e) \) of the identity \( e_N \).

  The terminology from~\cref{def:morphism_invertibility} applies to group homomorphisms because of the category \( \Bold{Grp} \) of groups\Tinyref{def:category_of_groups}.
\end{definition}

\begin{proposition}\label{thm:group_homomorphism_properties}
  Any group homomorphism \( f: G \to K \) has the following basic properties:
  \begin{defenum}
    \item\label{def:group_homomorphism_properties/preserves_identity} \( f \) preserves the identity, that is, \( f(e_G) = e_K \) (also holds for monoids).
    \item\label{def:group_homomorphism_properties/preserves_inverses} \( f \) preserves inverses, that is, \( f(a^{-1}) = f(a)^{-1} \) for any \( a \in G \).
    \item\label{def:group_homomorphism_properties/preserves_subgroup} If \( H \) is a subgroup of \( G \), its image \( f(H) \) is a subgroup of \( K \) (holds for any submagma).
    \item\label{def:group_homomorphism_properties/kernel_is_subgroup} The kernel \( \Ker(f) \) is a subgroup of \( G \).
  \end{defenum}
\end{proposition}
\begin{proof}\mbox{}
  \begin{itemize}
    \item[\ref{def:group_homomorphism_properties/preserves_identity}] We have
    \begin{align*}
      e_K f(e_G) = f(e_G e_G) = f(e_G) f(e_G).
    \end{align*}

    Cancelling on the right, we obtain
    \begin{align*}
      e_K = f(e_G).
    \end{align*}

    \item[\ref{def:group_homomorphism_properties/preserves_inverses}]
    \begin{align*}
      f(a^{-1})
      =
      f(a^{-1}) e_K
      =
      f(a^{-1}) f(a) f(a)^{-1}
      =
      f(a^{-1} a) f(a)^{-1}
      =
      f(a)^{-1}.
    \end{align*}

    \item[\ref{def:group_homomorphism_properties/preserves_subgroup}] If \( a, b \in G \), then
    \begin{align*}
      f(a) f(b) = f(ab) \in f(G),
    \end{align*}
    thus \( f(G) \) is closed under the group operation in \( K \).

    \item[\ref{def:group_homomorphism_properties/kernel_is_subgroup}] If \( a, b \in \Ker(f) \), then
    \begin{align*}
      f(ab) = f(a) f(b) = e_K e_k = e_k.
    \end{align*}

    Thus \( ab \in \Ker(f) \) and \( \Ker(f) \) is closed under the group operation.
  \end{itemize}
\end{proof}

\begin{definition}\label{def:normal_subgroup}
  Let \( N \) be a subgroup of \( G \). We say that \( N \) is a normal subgroup and write \( N \unlhd G \) if any of the following equivalent conditions hold:
  \begin{defenum}
    \item\label{def:normal_subgroup/direct} For any \( a \in G \), we have the set equality \( a N a^{-1} = N \).
    \item\label{def:normal_subgroup/cosets} The partitions induced by the left and rights cosets of \( N \) coincide (\( aN = Na \)) and form the \underLine{quotient group \( G / N \)}.
    \item\label{def:normal_subgroup/kernel} \( N \) is the kernel\Tinyref{def:group_homomorphism} of some group homomorphism (in particular, kernels are always normal subgroups).
  \end{defenum}
\end{definition}
\begin{proof}
  This is the group-theoretic analog to~\cref{thm:equivalence_partition}.

  (\ref{def:normal_subgroup/direct} \( \implies \) \ref{def:normal_subgroup/cosets}) For any \( a \in G \)
  \begin{align*}
    Na = (aNa^{-1})a = aN(a^{-1}a) = aN,
  \end{align*}
  thus every left coset is a right coset and vice versa.

  (\ref{def:normal_subgroup/cosets} \( \implies \) \ref{def:normal_subgroup/kernel}) Denote by \( G / N \) the family of all cosets of \( N \). Then \( G / N \) is itself a group with the inherited from \( G \) group structure
  \begin{itemize}
    \item \( aN \cdot bN \coloneqq (ab)N \)
    \item \( N = eN \) is an identity element of \( P \)
    \item \( a^{-1} N \) is the inverse of \( aN \)
  \end{itemize}

  Since it is possible for two elements \( a, x \in G \) to have the same coset \( aN = xN \), the group operation in \( G / N \) depends on the choice of representatives for each coset. In order for the operation to be well-defined, we need to make sure that the result does not depend on the choice of representatives. This happens to be true if and only if the subgroup \( N \) is normal.

  Indeed, let \( aN = xN \) and \( bN = yN \). If \( N \) is normal (in the sense of~\ref{def:normal_subgroup/cosets}), we have
  \begin{align*}
    (ab)N = a(yN) = a(Ny) = (aN)y = x(Ny) = (xy)N.
  \end{align*}

  Conversely, if the operation is well defined, then for any \( g \in N \)
  \begin{align*}
    N = a^{-1} a N = (a^{-1} N) (a N) = (a^{-1} N) N (a N) = (a^{-1} N) (g N) (a N) = (a^{-1} g a) N.
  \end{align*}

  Hence \( a^{-1} g a \in N \) and \( a^{-1} N a \subseteq N \). Thus
  \begin{align*}
    &Na = (a a^{-1} N)a = (aN) (a^{-1} N a) \subseteq (aN) N = aN,
    \\
    &aN = a(N a^{-1} a) = (a N a^{-1}) a \subseteq Na.
  \end{align*}

  Now define the homomorphism
  \begin{align*}
    &\varphi: G \to G / N \\
    &\varphi(a) = aN
  \end{align*}

  The preimage of the coset \( N \) consists of \( N \) itself. Since \( N \) (as an element of \( G / N \)) is the identity of \( G / N \), we conclude that \( N \) (as a subset of \( G \)) is the kernel of \( \varphi \).

  (\ref{def:normal_subgroup/kernel} \( \implies \) \ref{def:normal_subgroup/direct}) Let \( f: G \to K \) be a group homomorphism and fix any \( a \in G \). Denote \( N \coloneqq \Ker(f) \). Then \( aN = Na \) since
  \begin{align*}
    f(aN)
    =
    f(a) f(N)
    =
    f(a) f(e_G)
    =
    f(a)
    =
    f(N) f(a)
    =
    f(Na).
  \end{align*}

  Thus
  \begin{align*}
    N = aa^{-1}N = aNa^{-1}.
  \end{align*}
\end{proof}

\begin{proposition}\label{thm:abelian_normal_subgroups}
  All subgroups of an abelian group are normal.
\end{proposition}
\begin{proof}
  Let \( G \) be abelian and \( H \) be a subgroup of \( G \). Then \( aGa^{-1} = aa^{-1}H = H \) for any \( a \in G \) and thus \( H \) is normal.
\end{proof}

\begin{definition}\label{def:free_monoid}\cite[306]{Knapp2016BAlg}
  Let \( A \) be an arbitrary set. We associate with \( A \) its \underLine{free monoid} \( FM(A) \coloneqq (\Cal{A}^{*}, \cdot) \)\Tinyref{def:language}. It is a monoid due to~\cref{thm:set_of_all_words_is_monoid}. In category-theoretic terms, we define a functor\Tinyref{def:functor} \( FM: \Bold{Set} \to \Bold{Mon} \) that is left-adjoint\Tinyref{def:adjoint_functor} to the corresponding forgetful functor.
\end{definition}

\begin{definition}\label{def:free_group}\cite[306]{Knapp2016BAlg}
  Let \( A \) be an arbitrary set. We associate with \( A \) a \underLine{free group} \( F(A) \). As in\cref{def:free_monoid}, \( F(S) \) is left-adjoint to the corresponding forgetful functor.

  First, regard \( A \) as an alphabet\Tinyref{def:language} and define the set \( A^{-1} \) of words of the type \( a^{-1} \), where \( a \in A \) and \( \mbox{}^{-1} \) is a symbol not in \( A \). Consider the language \( W \coloneqq (A \cup A^{-1})^{*} \). If \( w = a_1 \ldots a_n \) is a word, its inverse word is
  \begin{align*}
    w^{-1} \coloneqq a_n^{-1} \ldots a_1^{-1}.
  \end{align*}

  By \cref{thm:set_of_all_words_is_monoid}, \( W \) is a monoid. It is not a group, however, since \( w w^{-1} \neq \varepsilon \). This is why we define a different group operation on a subset of \( W \).

  We define the \underLine{reduction function}
  \begin{align*}
    &r: W \to W \\
    &r(w) \coloneqq \begin{cases}
      r(ps), &w = pvs, \text{ where } v = aa^{-1} \text{ or } v = a^{-1}a \text{ for some } a \in A, \\
      w, &\text{otherwise}.
    \end{cases}
  \end{align*}

  The set of \underLine{reduced words} is the image \( F(A) \coloneqq \Img r(W) \). It forms a group under the operation
  \begin{align*}
    u \cdot_{F(A)} w \coloneqq r(u \cdot_{W} w).
  \end{align*}

  The group \( (F(A), \cdot_{F(A)}) \) is called the \underLine{free group generated by \( A \)}.
\end{definition}

\begin{definition}\label{def:group_direct_product}
  Let \( \{ X_i \}_{i \in I} \) be a nonempty family of groups.

  We define their \underLine{direct product} as the group \( \prod_{i \in I} X_i \), the group operation defined componentwise as
  \begin{align*}
    \{ x_i \}_{i \in I} \cdot \{ y_i \}_{i \in I}
    \coloneqq
    \{ x_i \cdot y_i \}_{i \in I}.
  \end{align*}

  We define their \underLine{direct sum} as the subgroup of \( \prod_{i \in I} X_i \)\Tinyref{def:group_direct_product} where only finitely many components of any group element are different from zero.
\end{definition}

\begin{note}\label{def:group_direct_sum_external_internal}\cite[126]{Knapp2016BAlg}
  If we are given a family of groups as in \cref{def:group_direct_product}, their sum \( \oplus_{i \in I} X_i \) is sometimes called an \underLine{external direct sum}.

  If instead we are given a group \( X \) and a family of subgroups \( \{ X_i \}_{i \in I} \), we say that \( X \) is their \underLine{internal direct sum} if the homomorphism
  \begin{align*}
    &\varphi: \prod_{i \in I} X_i \to X \\
    &\varphi(\{ x_i \}_{i \in I}) \coloneqq \cdot_{i \in I} x_i
  \end{align*}
  is an isomorphisms.

  The sum is well-defined since by definition there are only finitely many non-identity summands.

  This terminology also applies to finite direct products\Tinyref{def:group_direct_product} of groups, as well as similar constructions for other algebraic structures.
\end{note}

\begin{definition}\label{def:group_free_product}\cite[323]{Knapp2016BAlg}
  Let \( \{ X_i \}_{i \in I} \) be a nonempty family of groups.

  Similarly to \cref{def:free_group}, consider the language\Tinyref{def:language}
  \begin{align*}
    W \coloneqq \left( \bigcup_{i \in I} X_i \right)^{*}.
  \end{align*}

  We define the \underLine{reduction function}
  \begin{align*}
    &r: W \to W \\
    &r(w) \coloneqq \begin{cases}
      r(ps), &w = p e_i s \text{ for some } i \in I, \\
      r(pts), &w = puvs, \text{ where } u \neq e_i \text{ and } v \neq e_i \text{ and } t = u \cdot_i v \text{ for some } i \in I, \\
      w, &\text{otherwise}.
    \end{cases}
  \end{align*}

  The set of \underLine{reduced words} is the image \( \ast_{i \in I} X_i \coloneqq \Img r(W) \). It forms a group under the operation
  \begin{align*}
    u \cdot_\ast w \coloneqq r(u \cdot_{W} w).
  \end{align*}

  The group \( (\ast_{i \in I} X_i, \cdot_\ast) \) is called the \underLine{free product of the family \( \{ X_i \}_{i \in I} \)}.
\end{definition}

\begin{definition}\label{def:category_of_groups}
  The class\Tinyref{def:set_zfc} of all groups forms the category\Tinyref{def:category} \( \Bold{Grp} \), where for every two groups \( X, Y \in \Bold{Grp} \), the morphisms \( \Bold{Grp}(X, Y) \) are the homomorphisms\Tinyref{def:group_homomorphism} from \( X \) to \( Y \) and composition is the usual function composition\Tinyref{def:function_composition}.

  The category \( \Bold{Ab} \) of abelian groups is a full subcategory of \( \Bold{Grp} \).

  Both categories are concrete\Tinyref{def:concrete_category}, while \( \Bold{Ab} \) is abelian\Tinyref{def:abelian_category}.
\end{definition}

\begin{proposition}\label{thm:group_categorical_limits}
  We are interested in categorical limits\Tinyref{def:categorical_limit} and colimits\Tinyref{def:categorical_colimit} in \( \Bold{Grp} \). If \( \{ X_i \}_{i \in I} \) is an indexed family of groups, then
  \begin{defenum}
    \item\label{thm:group_categorical_limits/product} their categorical product\Tinyref{def:categorical_product} is their direct product\Tinyref{def:group_direct_product} \( \prod_{i \in I} X_i \), the projection morphisms being inherited from \cref{thm:set_categorical_limits/product}.

    \item\label{thm:group_categorical_limits/coproduct} their categorical coproduct\Tinyref{def:categorical_coproduct} is their free product\Tinyref{def:group_free_product} \( \ast_{i \in I} X_i \), the injection morphisms being
    \begin{align*}
      &\iota_j: X_j \to \ast_{i \in I} X_i \\
      &\iota_j(x_j) \coloneqq x_j.
    \end{align*}
  \end{defenum}
\end{proposition}

\begin{proposition}\label{thm:abelian_group_categorical_limits}
  We are interested in categorical limits\Tinyref{def:categorical_limit} and colimits\Tinyref{def:categorical_colimit} in \( \Bold{Ab} \). If \( \{ X_i \}_{i \in I} \) is an indexed family of abelian groups, then
  \begin{defenum}
    \item\label{thm:abelian_group_categorical_limits/product} their categorical product\Tinyref{def:categorical_product} is the direct product as inherited from \cref{thm:group_categorical_limits}.

    \item\label{thm:abelian_group_categorical_limits/coproduct} their categorical coproduct\Tinyref{def:categorical_coproduct} is the direct sum\Tinyref{def:group_direct_product} \( \oplus_{i \in I} X_i \), the injection morphisms being
    \begin{align*}
      &\iota_j: X_j \to \oplus_{i \in I} X_i \\
      &\iota_j(x_j) \coloneqq \begin{dcases}
        \begin{drcases}
          x_j, &i = j \\
          e_i, &i \neq j
        \end{drcases}
      \end{dcases}_{i \in \Bold I}.
    \end{align*}

    Since \( \Bold{Ab} \) is a subcategory of \( \Bold{Grp} \), by \cref{thm:group_categorical_limits} we have that for abelian groups the notions of free product\Tinyref{def:group_free_product} and direct sum coincide.
  \end{defenum}
\end{proposition}

\begin{note}\label{note:abelian_group_biproducts}
  By \cref{thm:additive_category_biproducts}, finite direct products and finite direct sums of abelian groups coincide as biproducts. This is also obvious by definition, even for nonabelian groups. What is not obvious, however, is that finite free products and finite direct products coincide for abelian groups.
\end{note}

\begin{definition}\label{def:free_abelian_group}
  As a special case of \cref{def:free_module}, since abelian groups are \( \BB{Z} \)-modules by \cref{def:abelian_group_z_module}, we define \underLine{free abelian groups} to be free \( \BB{Z} \)-modules.

  This definition is different from free groups\Tinyref{def:free_group}.
\end{definition}
