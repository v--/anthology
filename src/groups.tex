\begin{definition}\label{def:group}(See \cref{ex:algebraic_theory_language})
  A \uline{magma} is a nonempty set $G$ with a function $\cdot: G \times G \to G$, called the \uline{magma operation}. Denote $\cdot(a, b)$ by juxtaposition, that is, $ab$.

  \begin{description}
    \DItem{Associativity}{def:group/associativity} If $(ab)c = a(bc)$ for all $a, b, c \in G$, we say that the operation is \uline{associative} and we call the pair $(G, \cdot)$ a \uline{semigroup}.
    \DItem{Identity}{def:group/identity} An \uline{identity} is an element $e \in G$ such that $ea = ae = a$ for every $a \in G$. If $(G, \cdot)$ is a semigroup with identity, we call it a \uline{monoid} (see also \cref{def:monoidal_category} and \cref{note:groupoids}). We can also consider the identity in a monoid to be a function with arity 0.
    \DItem{Inverse}{def:group/inverse} For each $a \in G$, the \uline{inverse} is an element $a^{-1} \in G$ such that $aa^{-1} = e$. If $(G, \cdot)$ is a monoid and each element has an inverse, we call it a \uline{group}. We can also consider the inverse in a group to be a function with arity 1.
    \DItem{Commutativity}{def:group/commutativity} If $ab = ba$ for all $a, b \in G$, we say that the magma operation is \uline{commutative}. If $(G, \cdot)$ is a group with a commutative operation, we call it a \uline{commutative group} or \uline{abelian group}.
  \end{description}

  We define $a^n$ for nonnegative integers $n$ as
  \begin{align*}
    a^n \coloneqq \begin{cases}
      e, &n = 0 \\
      a \cdot a^{n-1}, &n > 0 \\
    \end{cases}
  \end{align*}

  If $H \subseteq G$ is \uline{closed under the magma operation}, that is, $ab \in H$ whenever $a, b \in H$, we say that $H$ is a \uline{submagma (resp. subsemigroup, submonoid, subgroup, etc) of $G$}.

  The group $\{ e \}$ is called the \uline{trivial group}.

  The subgroup $\{ e_G \}$ of $G$ is called the \uline{trivial subgroup}. All subgroups except for $G$ itself are called \uline{proper subgroups}.
\end{definition}

\begin{note}\label{note:additive_group}
  Groups are often used to describe sets of invertible functions\Tinyref{def:function_invertibility} where the group operation is composition (see~\cref{note:groupoids} for a categorical viewpoint). As such, the group operation is usually denoted by juxtaposition as in~\cref{def:group}.

  Since composition of functions is not commutative in general, abelian groups are usually not sets of invertible functions. Since abelian groups are $\BB{Z}$-modules~\Tinyref{thm:abelian_group_iff_z_module}, we usually denote the group operation in abelian groups by $a + b$ instead of $ab$, the inverse by $-a$ instead of $a^{-1}$, and the unit by $0$.

  To make a further distinction, if the operation is denoted by juxtaposition, we say that the group is a \uline{multiplicative group}, and if the operation is denoted by $+$, we say that the group is an \uline{additive group}. This terminology usually, but not necessarily, coincides with the group being abelian.
\end{note}

\begin{definition}\label{def:groupoid}
  A \uline{groupoid} is a category\Tinyref{def:category} in which every morphism is an isomorphism\Tinyref{def:morphism_invertibility}.
\end{definition}

\begin{definition}\label{note:groupoids}
  Let $\Bold G$ be a locally small with a single object $g$. Then the endomorphisms of $g$ form a monoid\Tinyref{def:group} under composition and the subcategory of $\Bold G$ in which all morphisms are isomorphisms forms a group\Tinyref{def:group}. Thus, if $\Bold G$ is a locally small groupoid with a single object $g$, then the endomorphisms of $g$ are automorphisms and, thus, they form a group under composition.
\end{definition}

\begin{definition}\label{def:group_cosets}
  Let $H \subseteq G$ be a subgroup of $G$ and let $a \in G$. Consider the sets
  \begin{align*}
    aH \coloneqq \{ ah \colon h \in H \}
    &&
    Ha \coloneqq \{ ha \colon h \in H \}
  \end{align*}
  called the \uline{left and right cosets of $H$ with respect to $a$}.
\end{definition}

\begin{proposition}
  The left (resp. right) cosets of a subgroup $H$ of $G$ partition\Tinyref{def:set_partition} $G$.
\end{proposition}
\begin{proof}
  To each element $a \in G$ there corresponds a coset $a \in aH$ (since $H$ contains the identity as a subgroup).

  Two cosets $aH$ and $bH$ are either disjoint or equal. Indeed, if they are not disjoint, then there exists $g \in aH \cap bH$ and thus $g = ax = by$ for some $x, y \in H$. Thus $a = a x x^{-1} = b y x^{-1}$ and since $y x^{-1} \in H$, we have that $a \in bH$. Furthermore, for any $z \in H$, we have $az = b(y x^{-1} z) \in bH$, hence $aH \subseteq bH$. After obtaining the converse inclusion, we conclude $aH = bH$.
\end{proof}

\begin{definition}\label{def:magma_homomorphism}
  Let $M$ and $N$ be magmas. We say that the function $f: M \to N$ is a \uline{magma homomorphism (resp. semigroup, monoid or group homomorphism)} if it is \uline{compatible with the magma structure on $M$ and $N$}, that is,
  \begin{description}
    \DItem{Compatibility}{def:magma_homomorphism/compatibility} $f(ab) = f(a) f(b)$ for all $a, b \in M$.
  \end{description}

  If $M$ and $N$ are monoids, the \uline{kernel} of $f$ is defined as the preimage\cref{def:function_invertibility} $f^{-1}(e)$ of the identity $e_N$.
\end{definition}

\begin{definition}\label{def:normal_subgroup}
  Let $N$ be a subgroup of $G$. We say that $N$ is a normal subgroup and write $N \unlhd G$ if any of the following equivalent conditions hold:
  \begin{defenum}
    \item\label{def:normal_subgroup/direct} For any $a \in G$, we have the set equality $a N a^{-1} = N$.
    \item\label{def:normal_subgroup/cosets} The partitions induced by the left and rights cosets of $N$ coincide.
    \item\label{def:normal_subgroup/kernel} $N$ is the kernel\Tinyref{def:magma_homomorphism} of some group homomorphism.
  \end{defenum}
\end{definition}
\begin{proof}
  The proof is a restatement of~\cref{thm:equivalence_partition} with the additional regularity condition induced by~\cref{def:magma_homomorphism/compatibility}.
\end{proof}

\begin{proposition}\label{thm:abelian_normal_subgroups}
  All subgroups of an abelian group are normal.
\end{proposition}
