\section{Topological vector spaces}\label{sec:topological_vector_spaces}

\begin{definition}\label{def:topological_vector_space}
  Let $X$ be any vector space and let $\Cal{T}$ be a topology on $X$. The space $(X, +, \cdot, \Cal{T})$ is called a \underLine{topological vector space} if the linear and topological structure agree, that is, the operations $+: X \times X \to X$ and $\cdot: X \times \BB{R} \to X$ are continuous with respect to $\Cal{T}$.
\end{definition}

\begin{proposition}\label{thm:continuous_implies_locally_bounded}
  Let $X$ and $Y$ be real Hausdorff topological vector spaces, let $D \subseteq X$ and let $f: D \to Y$ be a continuous function.

  Then $f$ is locally bounded.
\end{proposition}
\begin{proof}
  Let $x_0 \in D$. The set $f(x_0) + B_Y \subseteq Y$ is obviously bounded and open. Since $f$ is continuous, $f^{-1}(f(x_0) + B_Y)$ is also open. Even though $f$ may not be surjective, $f^{-1}(f(x_0) + B_Y)$ is nonempty, since it contains $x_0$.

  This implies that $f^{-1}(f(x_0) + B_Y)$ is a neighborhood of $x_0$ with a bounded image. Hence $f$ is locally bounded.
\end{proof}
