\section{Functional analysis}\label{sec:functional_analysis}

In this section, \( \K \) will refer to either \( \R \) or \( \C \). See \cref{remark:real_field_extensions} for a justification.

\subsection{Topological vector spaces}\label{subsec:topological_vector_spaces}

\begin{definition}\label{def:topological_vector_space}
  Let \( X \) be any vector space and let \( \T \) be a topology on \( X \). The space \( (X, +, \cdot, \T) \) is called a \Def{topological vector space} if the linear and topological structure agree, that is, the operations \( +: X \times X \to X \) and \( \cdot: X \times \R \to X \) are continuous with respect to \( \T \).

  See \cref{def:continuous_dual_space} and \cref{def:category_of_topological_vector_spaces} for more nuances.
\end{definition}

Given that a topological vector space \( X \) has both a topological and an algebraic structure, we should adapt certain definitions.

\begin{definition}\label{def:continuous_dual_space}
  We define the \Def{continuous dual space} \( X^* \) of a topological space \( X \) as the vector space of all continuous\Tinyref{def:continuous_function} linear functionals. This differs drastically from \cref{def:dual_vector_space} because in the general case, the continuous dual space may be trivial, i.e. only contain the zero functional. See \cref{def:locally_convex_duality_pairing}.

  We use the same notation for both the algebraic dual spaces and the continuous dual space because the meaning is usually clear from the context. In particular, hyperplanes as defined in \cref{def:hyperplane} are only relevant to continuous linear functionals.
\end{definition}

\begin{definition}\label{def:category_of_topological_vector_spaces}
  The category \( \Cat{TopVect}_{\K} \) of topological vector spaces over \( \K \) is a subcategory of both \( \Cat{Top} \) and \( \Cat{Vect}_K \). Its morphisms are the continuous\Tinyref{def:continuous_function} linear maps\Tinyref{def:linear_operator}.
\end{definition}

\begin{definition}\label{def:locally_convex_space}\cite[1.8]{Rudin1991}
  We say that a topological vector space\Tinyref{def:topological_vector_space} is \Def{locally convex} if there exists a topological base\Tinyref{def:topological_base} of convex\Tinyref{def:convex_set} sets.
\end{definition}

\begin{remark}\label{def:locally_convex_duality_pairing}
  Given a Hausdorff locally convex space \( X \), \cref{thm:hahn_banach_implies_functionals_vanish_nowhere} shows that the canonical duality pairing as defined in \cref{def:locally_convex_duality_pairing} is nondegenerate. If the space is not locally convex, we cannot guarantee that the pairing will be nondegenerate and our restriction to continuous linear functionals could interfere with our habits of working with linear functionals.
\end{remark}
