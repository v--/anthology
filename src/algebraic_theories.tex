\section{Algebra}\label{sec:algebra}
\subsection{Algebraic theories}\label{subsec:algebraic_theories}

\begin{remark}\label{remark:algebra_vs_analysis}
  An distinction is made, out of all areas of mathematics, between analysis\Tinyref{sec:analysis} and algebra\Tinyref{sec:algebra} and between upholders of the two. The difference seems silly but it comes down to a personal philosophy and worldview. An interesting anecdote about this distinction can be found in \cite{Tilly2010}.

  As an overgeneralization, one can say that algebra studies structures with operations on them, along with subsets and functions that \enquote{behave well} with respect to these operations. On the other hand, analysis studies sets and functions that arise from certain problems that are usually not well-behaved with respect to an algebraic structure. Ironically, this is an algebraic view of the two fields, so it is natural to prefer algebra because it seems tidier.

  Another overgeneralization compares rather the techniques used: analysis makes less assumptions and provides results more readily while algebra is burdened by classifying objects. Analytic techniques like approximations, inequalities and inclusions use less formalities and are easier to grasp for people with a more practical mindset, while algebraic techniques provide results at the cost of encoding as much information as possible into the algebraic structures themselves.

  One of my teachers, prof. Ribarska, emphasized that analysis happens where there is an algebraic structure\Tinyref{def:algebraic_theory} that is also a topological space\Tinyref{def:topological_space}. This seems to imply that algebra is somehow more \enquote{fundamental} and in a purely formal sense it is so, however even the proof of the fundamental theorem of algebra requires analytic techniques.

  Other fields of mathematics can roughly be divided as \enquote{closer to} algebra, e.g. category theory\Tinyref{sec:category_theory} or combinatorics\Tinyref{sec:combinatorics}, or \enquote{closer to} analysis, e.g. differential equations\Tinyref{sec:diffeq} or metric spaces\Tinyref{subsec:metric_spaces}. In practice, most fields of mathematics have both type of people and use both types of techniques.
\end{remark}

\begin{definition}\label{def:algebraic_theory}\cite[remark 2.1.4]{Leinster2014}
  There are many algebraic structures and they all share similarities. The formal study of algebraic theories is complicated so the definitions will be very general and somewhat informal, yet allow defining fields unlike other definitions for algebraic theories. The definitions are my own so they may not be as elegant as more established (but more difficult) ones.

  An \Def{algebraic theory} is a first-order language\Tinyref{def:first_order_language} \( \Cal{L} \) with
  \begin{itemize}
    \item formal equality \( \doteq \)
    \item no predicate symbols
    \item a set of functional symbols, called \Def{operations}
    \item a set of axioms concerning the operations (see \cref{remark:propositional_axioms})
  \end{itemize}

  A model\Tinyref{def:first_order_model} of an algebraic theory is called an \Def{algebraic structure}.

  This definitions does not give much insight, however it is useful for defining common features of different structures. Fix an algebraic theory \( \Cal{L} \). Let \( \Cal{A} = (A, I) \) be a model. Most operations are either
  \begin{description}
    \item[nullary] (\( \odot: \varnothing \to A \)) - the \enquote{operation} is simply a formalism for constants, e.g. \ref{def:algebraic_theory/identity}.
    \item[unary] (\( \odot: A \to A \)) - the operation is usually a formalism for the existence of certain elements, e.g. \ref{def:algebraic_theory/invertibile_element}.
    \item[binary] (\( \odot: A \times A \to A \)) - the most common type of operations.
  \end{description}

  The following are special kinds of elements:
  \begin{defenum}
    \DItem{def:algebraic_theory/identity} we say that the element \( e \in A \) is a \Def{left identity} with respect to the binary operation \( \odot \) if, for all \( x \in A \),
    \begin{equation*}
      e \odot x = x,
    \end{equation*}
    a \Def{right identity} if
    \begin{equation*}
      x \odot e = x,
    \end{equation*}
    and simply an \Def{identity} if it is both a left and right identity. An operation with an identity is called \Def{unital}. Two-sided identities are unique since, if both \( e \) and \( e' \) are identities for \( \odot \), then \( e' = e' e = e \).

    The existence of an identity element can be added as a constant to the theory.

    \DItem{def:algebraic_theory/invertibile_element} if \( e \in A \) is a left identity with respect to the binary operation \( \odot \), we say that \( x \in A \) is \Def{left invertible} if there exists \( y \) such that
    \begin{equation*}
      y \odot x = e,
    \end{equation*}
    \Def{right invertible} if
    \begin{equation*}
      x \odot y = e,
    \end{equation*}
    and simply \Def{invertible} if it is both left and right invertible. 

    \DItem{def:algebraic_theory/absorbing_element} we say that the constant  \( o \in A \) is is \Def{left absorbing} with respect to the binary operation \( \odot \) if, for all \( x \in A \),
    \begin{equation*}
      o \odot x = o,
    \end{equation*}
    \Def{right absorbing} if
    \begin{equation*}
      x \odot o = o,
    \end{equation*}
    and simply \Def{absorbing} if it is both left and right absorbing.

    \DItem{def:algebraic_theory/idempotent_element} we say that the element \( x \in A \) is \Def{idempotent} with respect to the binary operation \( \odot \) if
    \begin{equation*}
      x \cdot x = x.
    \end{equation*}
  \end{defenum}

  Binary operations often have some of the following axioms imposed on them:
  \begin{defenum}
    \DItem{def:algebraic_theory/associativity} a binary operation \( \odot \) is said to be \Def{associative} if, for all \( x, y, z \in A \),
    \begin{equation*}
      (x \odot y) \odot z = x \odot (y \odot z).
    \end{equation*}

    \DItem{def:algebraic_theory/commutativity} a binary operation \( \odot \) is said to be \Def{commutative} if, for all \( x, y \in A \),
    \begin{equation*}
      x \odot y = y \odot x.
    \end{equation*}

    \DItem{def:algebraic_theory/distributivity} if \( \oplus, \odot: A \times A \to A \) are binary operations, then \( \odot \) is \Def{left distributive} \( \oplus \) if, for all \( x, y, z \in A \),
    \begin{equation*}
      z \odot (x \oplus y) = (z \odot x) \oplus (z \odot y),
    \end{equation*}
    \Def{right distributive} if
    \begin{equation*}
      (x \oplus y) \odot z = (x \odot z) \oplus (y \odot z)
    \end{equation*}
    and simply \Def{distributive} if it is both left and right distributive.

    \DItem{def:algebraic_theory/cancellative} a binary operation \( \odot \) is said to be \Def{left cancellative} if, for all \( x, y, z \in A \),
    \begin{equation*}
      x = y \iff z \odot x = z \odot y,
    \end{equation*}
    \Def{right cancellative} if
    \begin{equation*}
      x = y \iff x \odot z = y \odot z
    \end{equation*}
    and simply \Def{cancellative} if it is both left and right cancellative.
  \end{defenum}

  Minimal structures usually exist and are unique up to an isomorphism\Tinyref{def:first_order_structure/minimal}.

  Algebraic structures with homomorphisms\Tinyref{def:first_order_homomorphism} between them form categories as described in \cref{def:first_order_model_category}.
\end{definition}
