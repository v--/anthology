\subsection{Exponential function}\label{subsec:exponential_function}

\begin{definition}\label{def:exponential_function}
  We define the \Def{exponential function}
  \begin{equation}\label{def:exponential_function/series}
    \exp(z) \coloneqq \sum_{k=0}^\infty \frac {z^k} {k!}.
  \end{equation}

  We define \Def{Euler's number} as
  \begin{equation*}
    e \coloneqq \exp(1) = \sum_{i=0}^k \frac 1 {k!}.
  \end{equation*}

  \Cref{thm:exponential_function_properties/interpolates_power} justifies the notation \( e^z = \exp(z) \).
\end{definition}
\begin{proof}
  We will show that \( \exp(z) \) converges everywhere. By \cref{thm:power_series_radius_of_convergence}, the radius of convergence is
  \begin{equation*}
    \limsup_{k \to \infty} \frac {k!} {(k-1)!}
    =
    \limsup_{k \to \infty} k
    =
    +\infty
  \end{equation*}

  Hence the radius of convergence of \( \exp(x) \) is infinite.
\end{proof}

\begin{definition}\label{def:trigonometric_functions}
  We define the following \Def{trigonometric functions}:

  \begin{defenum}
    \DItem{def:trigonometric_functions/sine} We define the \Def{sine} function as
    \begin{equation*}
      \sin(z)
      \coloneqq
      -i \sum_{m \text{ is odd}}^\infty \frac {i^m z^m} {m!}
      =
      -i \sum_{k=0}^\infty \frac {i^{2k+1} z^{2k+1}} {(2k + 1)!}
      =
      \sum_{k=0}^\infty \frac {i^{2k} z^{2k+1}} {(2k + 1)!}
    \end{equation*}

    By \cref{thm:power_series_parity}, \( \sin(z) \) is an odd function.

    \DItem{def:trigonometric_functions/cosine} We define the \Def{cosine} function as
    \begin{equation*}
      \cos(z)
      \coloneqq
      \sum_{m \text{ is even}}^\infty \frac {i^m z^m} {m!}
      =
      \sum_{k=0}^\infty \frac {i^{2k} z^{2k}} {(2k)!}.
    \end{equation*}

    By \cref{thm:power_series_parity}, \( \cos(z) \) is an even function.
  \end{defenum}

  Comparing these definitions with \cref{def:exponential_function}, we obtain \Def{Euler's formula}
  \begin{equation}\label{def:trigonometric_functions/eulers_formula}
    \exp(iz) = \cos(z) + i \sin(z),
  \end{equation}
  thus both \( \sin(z) \) and \( \cos(z) \) converge everywhere in \( \BC \).

  From \cref{def:trigonometric_functions/eulers_formula} it also follows that
  \begin{align*}
    \sin(z) &= \Re(\exp(iz)) \\
    \cos(z) &= \Im(\exp(iz))
  \end{align*}
\end{definition}

\begin{proposition}[Pythagorean identity]\label{thm:pythagorean_identity}
  For \( t \in \BR \),
  \begin{equation*}
    \sin(t)^2 + \cos(t)^2 = 1.
  \end{equation*}
\end{proposition}
\begin{proof}
  We use Cauchy multiplication for
  \begin{align*}
    \sin(t)^2
    &=
    \sin(t) \cdot \sin(t)
    = \\ &=
    (-i \cdot -i) \left( \sum_{k=0}^\infty \frac {i^{2k+1} z^{2k+1}} {(2k+1)!} \right) \left( \sum_{k=0}^\infty \frac {i^{2k+1} z^{2k+1}} {(2k+1)!} \right)
    = \\ &=
    -\sum_{k=0}^\infty \sum_{m=0}^k \frac {i^{2m+1} z^{2m+1}} {(2m+1)!} \frac {i^{2(k-m)+1} z^{2(k-m)+1}} {(2(k-m)+1)!}
    = \\ &=
    -\sum_{k=0}^\infty \frac {i^{2k+2} z^{2k+2}} {(2k+2)!} \sum_{m=0}^k \binom {2k+2} {2m+1}
    = \\ &=
    -\sum_{k=1}^\infty \frac {i^{2k} z^{2k}} {(2k)!} \sum_{m=0}^{k-1} \binom {2k} {2m+1}
  \end{align*}
  and for
  \begin{align*}
    \cos(t)^2
    &=
    \cos(t) \cdot \cos(t)
    = \\ &=
    \left( \sum_{k=0}^\infty \frac {i^{2k} z^{2k}} {(2k)!} \right) \left( \sum_{k=0}^\infty \frac {i^{2k} z^{2k}} {(2k)!} \right)
    = \\ &=
    \sum_{k=0}^\infty \sum_{m=0}^k \frac {i^{2m} z^{2m}} {(2m)!} \frac {i^{2(k-m)} z^{2(k-m)}} {(2(k-m))!}
    = \\ &=
    \sum_{k=0}^\infty \frac {i^{2k} z^{2k}} {(2k)!} \sum_{m=0}^k \binom {2k} {2m}
  \end{align*}

  For the sum we obtain
  \begin{align*}
    \sin(t)^2 + \cos(t)^2
    &=
    1 + \sum_{k=1}^\infty \frac {i^{2k} z^{2k}} {(2k)!} \left[ \sum_{m=0}^{k-1} \binom {2k} {2m+1} - \sum_{m=0}^k \binom {2k} {2m} \right]
    = \\ &=
    1 + \sum_{k=1}^\infty \frac {i^{2k} z^{2k}} {(2k)!} \sum_{m=0}^k (-1)^m \binom {2k} m
    = \\ &=
    1 + \sum_{k=1}^\infty \frac {i^{2k} z^{2k}} {(2k)!} \underbrace{(1 - 1)^k}_0
    =
    1.
  \end{align*}
\end{proof}

\begin{proposition}\label{thm:exponential_function_properties}
  The exponential function \( \exp(z) \) has the following basic properties:

  \begin{propenum}
    \DItem{thm:exponential_function_properties/derivative} \( \exp(x) \) is its own derivative.

    \DItem{thm:exponential_function_properties/homomorphism} \( \exp(x + y) = \exp(x) \exp(y) \). Stated in another way, \( \exp \) is a homomorphism from the additive group of \( \BC \) to the multiplicative group.

    \DItem{thm:exponential_function_properties/interpolates_power} The notation \( \exp(x) \) is consistent with iterated multiplication as defined in \cref{def:semiring/dioid}, that is, \( \exp(n) = \underbrace{e \cdot \ldots \cdot e}_{n \text{times}} \) and for positive integers \( n \), \( \exp(n) =  \) and \( \exp(-n) =\tfrac 1 {\exp(n)} \).

    \DItem{thm:exponential_function_properties/negative_power}
    \begin{equation*}
      \exp(z) = \frac 1 {\exp(-z)}.
    \end{equation*}

    \DItem{thm:exponential_function_properties/real_positive} For real \( t \), \( e^t \) is a positive real number.

    \DItem{thm:exponential_function_properties/conjugate} \( \Ol{\exp(z)} = \exp(\Ol{z}) \).

    \DItem{thm:exponential_function_properties/unit_circle} The function \( t \mapsto \exp(it) \) is a bijection between the half-open interval \( [0, 2\pi) \) and the unit circle in \( \BC \).

    \DItem{thm:exponential_function_properties/real_bijective} \( t \mapsto \exp(t) \) is a bijection from \( \BR \) to \( [0, \infty) \).

    \DItem{thm:exponential_function_properties/bijection} \( \exp(z) \) is a bijection between the strip \( S \coloneqq \{ a + bi \colon 0 \leq b < 2\pi \} \) and the complex plane \( \BC \setminus \{ 0 \} \).

    \DItem{thm:exponential_function_properties/periodic} \( \exp(z) \) is \( 2\pi \)-periodic\Tinyref{def:periodic_function}.

    \DItem{thm:exponential_function_properties/compound_interest} For nonnegative real \( t \geq 0 \) we have
    \begin{equation*}
      \exp(t) = \lim_{n \to \infty} \left(1 + \frac t n \right)^n
    \end{equation*}
  \end{propenum}
\end{proposition}
\begin{proof}
  \RItem{thm:exponential_function_properties/homomorphism} The Cauchy product of \( \exp(x) \) and \( \exp(y) \) is
  \begin{align*}
    \exp(x) \exp(y)
    &=
    \left( \sum_{k=0}^\infty \frac {x^k} {k!} \right) \left( \sum_{k=0}^\infty \frac {y^k} {k!} \right)
    = \\ &=
    \sum_{k=0}^\infty \sum_{m=0}^k \frac {x^m} {m!} \frac {x^{k-m}} {(k-m)!}
    = \\ &=
    \sum_{k=0}^\infty \frac 1 {k!} \sum_{m=0}^k \binom{k}{m} x^m y^{k-m}
    \overset {\ref{thm:binomial_theorem}} = \\ &=
    \sum_{k=0}^\infty \frac {(x + y)^k} {k!}
    =
    \exp(x + y).
  \end{align*}

  \RItem{thm:exponential_function_properties/interpolates_power} We use induction on \( n \) to prove \( \exp(n) = e^n \). The case \( \exp(0) = 1 \) is obvious. If we assume that \( \exp(n) = e^n \), by \cref{thm:exponential_function_properties/homomorphism}, we have
  \begin{equation*}
    \exp(n + 1)
    =
    \exp(n) \exp(1)
    =
    e^n \cdot e
    =
    e^{n+1}.
  \end{equation*}

  Note that this works for negative \( n \) too.

  \RItem{thm:exponential_function_properties/negative_power} Note that
  \begin{equation*}
    1 = \exp(0) = \exp(z - z) = \exp(z) \exp(-z),
  \end{equation*}
  hence
  \begin{equation*}
    \exp(-z) = \frac 1 {\exp(z)}.
  \end{equation*}

  \RItem{thm:exponential_function_properties/real_positive} For \( t > 0 \), the following
  \begin{equation*}
    \exp(t) = \sum_{k=0}^\infty \frac {t^k} {k!}
  \end{equation*}
  is a series of positive real numbers. To see its convergence, we apply \fullref{thm:dalamberts_ratio_test}:
  \begin{equation*}
    \frac {t^k} {k!} \cdot \frac {(k-1)!} {t^{k-1}}
    =
    \frac t k
    \xrightarrow[k \to \infty]{} 0.
  \end{equation*}

  Thus \( \exp(t) \) is a nonnegative real number. Furthermore, since the sequence of partial sums is monotone, \( \exp(t) \) cannot be zero. Hence for \( t > 0 \), we have \( \exp(t) > 0 \).

  Notice that \( \exp(t) \exp(-t) > 0 \), hence if \( \exp(t) > 0 \), then \( \exp(-t) > 0 \).

  \RItem{thm:exponential_function_properties/conjugate} By \cref{def:trigonometric_functions/eulers_formula},
  \begin{align*}
    \Ol{\exp(a + bi)}
    &\overset {\ref{thm:exponential_function_properties/homomorphism}} =
    \Ol{\exp(a) \exp(bi)}
    \overset {\ref{thm:exponential_function_properties/real_positive}} = \\ &=
    \exp(a) \Ol{(\cos(b) + i\sin(b))}
    = \\ &=
    \exp(a) (\cos(b) - i\sin(b))
    \overset {\ref{thm:power_series_parity}} = \\ &=
    \exp(a) (\cos(-b) + i\sin(-b))
    = \\ &=
    \exp(a) \exp(-bi)
    = \\ &=
    \exp(a - bi)
    = \\ &=
    \exp(\Ol{a + bi}).
  \end{align*}

  \RItem{thm:exponential_function_properties/periodic} By \cref{def:pi},
  \begin{equation*}
    \exp(x + 2\pi i) = \exp(x) \exp(2 \pi i) = \exp(x).
  \end{equation*}

  Furthermore, this is also the minimal period because of the infimum in \cref{def:pi}.

  \RItem{thm:exponential_function_properties/unit_circle} For \( t \in \BR \) we have
  \begin{equation*}
    \Abs{\exp(it)}
    =
    \Abs{\cos(t) + i\sin(t)}
    =
    \sqrt{\cos(t)^2 + \sin(t)^2}
    \overset {\ref{thm:pythagorean_identity}} =
    1
  \end{equation*}
  by \fullref{thm:pythagorean_identity}.

  Furthermore, if \( r \) is another real number,
  \begin{equation}
    \exp(ir)
    =
    \exp(i(t + (r - t)))
    =
    \exp(it) \exp(i(r - t)).
  \end{equation}

  It follows that \( \exp(ir) \neq \exp(it) \) if and only if \( \exp(i(r - t)) \neq 0 \). If \( t, r \in [0, 2\pi) \) and \( t \neq r \), this is satisfied.

  Hence \( t \mapsto \exp(it) \) is indeed an injection of \( [0, 2\pi) \) into the unit circle of \( \BC \). It is also a surjection because of the intermediate value theorem.

  \RItem{thm:exponential_function_properties/real_bijective} First, assume\LEM that \( e^t \) is not injective on \( \BR \). Then there exist \( t, r \in \BR \), \( t \neq r \), such that \( e^t = e^r \). By \cref{thm:exponential_function_properties/real_positive}, both are positive real numbers. In particular, we can divide by \( e^t \) to obtain
  \begin{equation*}
    1
    =
    \frac {e^r} {e^t}
    \overset {\ref{thm:exponential_function_properties/negative_power}} =
    =
    e^r e^{-t}
    \overset {\ref{thm:exponential_function_properties/homomorphism}} =
    e^{r - t}.
  \end{equation*}

  We know that \( e^0 = 1 \) from \cref{thm:exponential_function_properties/interpolates_power}. Thus it is enough to show that \( e^t = 1 \) if and only if \( t = 0 \).

  Assume\LEM that \( e^t = 1 \) holds for some \( t > 0 \). The partial sums are monotonely increasing so in order for them to converge to \( 1 \), for any fixed index \( n \) we must have
  \begin{align*}
    0 &\leq \sum_{k=0}^n \frac {t^k} {k!} = 1 + \sum_{k=1}^n \frac {t^k} {k!} \leq 1,\\
    -1 &\leq \sum_{k=1}^n \frac {t^k} {k!} \leq 0.
  \end{align*}

  But \( \sum_{k=1}^n \frac {t^k} {k!} > 0 \) because \( t > 0 \). The obtained contradiction proves that \( e^t \neq 1 \) for positive \( t \).

  For negative \( t \), note that
  \begin{equation*}
    e^t e^{-t} = 1.
  \end{equation*}

  Since \( -t \) is positive, \( e^{-t} \neq 1 \) and hence \( e^t \neq 1 \).

  Therefore the function \( t \mapsto e^t \) is injective on \( \BR \). It is also surjective onto \( \BR^{>0} \) because of the intermediate value theorem.

  \RItem{thm:exponential_function_properties/bijection} Fix \( a + bi \in H \), that is, \( b \in [0, 2\pi) \). By \cref{thm:exponential_function_properties/homomorphism},
  \begin{equation*}
    e^{a + bi} = e^a e^{bi}.
  \end{equation*}

  By \cref{thm:exponential_function_properties/unit_circle}, \( b \mapsto e^{bi} \) is injective for \( b \in [0, 2\pi) \) and by \cref{thm:exponential_function_properties/real_bijective}, \( a \mapsto e^a \) is injective on \( \BR \). It follows that their product is also injective.

  \RItem{thm:exponential_function_properties/compound_interest}\cite[3.31]{Rudin1976} By \fullref{thm:binomial_theorem},
  \begin{align*}
    \left(1 + \frac t n \right)^n
    &=
    \sum_{k=0}^n \binom{n}{k} \left(\frac t n\right)^k 1^{n-k}
    = \\ &=
    \sum_{k=0}^n \frac {n!} {(n-k)! k!} \frac {t^k} {n^k}
    = \\ &=
    \sum_{k=0}^n \frac {n!} {(n-k)! n^k} \frac {t^k} {k!}
    = \\ &=
    \sum_{k=0}^n \left[ \prod_{j=1}^k \left(1 - \frac {k+j} n \right) \right] \frac {t^k} {k!}.
  \end{align*}

  Fix an index \( m \). Since the series is nonnegative, there exists an index \( N \) such that for \( n \geq N \)
  \begin{equation*}
    \sum_{k=0}^m \frac {t^k} {k!}
    \leq
    \sum_{k=0}^n \left[ \prod_{j=1}^k \left(1 - \frac {k+j} n \right) \right] \frac {t^k} {k!}.
  \end{equation*}

  Note that
  \begin{equation*}
    \left[ \prod_{j=1}^k \left(1 - \frac {k+j} n \right) \right] \frac {t^k} {k!}
    \leq
    \frac {t^k} {k!},
  \end{equation*}
  hence
  \begin{equation*}
    \sum_{k=0}^m \frac {t^k} {k!}
    \leq
    \sum_{k=0}^n \left[ \prod_{j=1}^k \left(1 - \frac {k+j} n \right) \right] \frac {t^k} {k!}
    \leq
    \sum_{k=0}^n \frac {t^k} {k!}.
  \end{equation*}

  By \fullref{thm:squeeze_lemma},
  \begin{equation*}
    \lim_{n \to \infty} \left(1 + \frac t n \right)^n
    =
    \lim_{n \to \infty} \sum_{k=0}^n \frac {t^k} {k!}
    =
    \exp(t).
  \end{equation*}
\end{proof}

\begin{proposition}\label{thm:trigonometric_function_properties}
  \mbox{}
  \begin{propenum}
    \DItem{thm:trigonometric_function_properties/period} Both \( \sin \) and \( \cos \) are periodic with base period \( 2\pi \).

    \DItem{thm:trigonometric_function_properties/bijective} Both \( \sin \) and \( \cos \) bijective from \( [0, 2\pi) \) to \( [-1, 1] \).

    \DItem{thm:trigonometric_function_properties/sine_characterization}
    \begin{equation*}
       \sin(x) = \Re(e^x) = \frac {e^{ix} - e^{-ix}} {2i}
    \end{equation*}

    \DItem{thm:trigonometric_function_properties/cosine_characterization}
    \begin{equation*}
       \cos(x) = \Im(e^x) = \frac {e^{ix} + e^{-ix}} 2
    \end{equation*}
  \end{propenum}
\end{proposition}

\begin{definition}\label{def:logarithm}
  We define the \Def{natural logarithm} \( \ln(x) \) as the inverse of \( e^x \) from \( \BC \setminus \{ 0 \} \) to the strip \( S \coloneqq \{ a + bi \colon 0 \leq b < 2\pi \} \).

  We also define the \Def{base \( b \) logarithm} \( \log_b(x) \) for \( b > 0 \) over the same domain as
  \begin{equation*}
    \log_b(x) \coloneqq \frac {\ln(x)} {\ln(b)}.
  \end{equation*}
\end{definition}
\begin{proof}
  The well-definedness follows from \cref{thm:exponential_function_properties/bijection}.
\end{proof}

\begin{proposition}\label{thm:logarithm_properties}
  \mbox{}
  \begin{propenum}
    \DItem{thm:logarithm_properties/homomorphism} \( \ln(xy) = \ln(x) \ln(y) \)
  \end{propenum}
\end{proposition}
\begin{proof}
  \begin{description}
    \RItem{thm:logarithm_properties/homomorphism} Follows from \cref{thm:exponential_function_properties/homomorphism}.
  \end{description}
\end{proof}

\begin{definition}\label{def:power_function}
  We define the power function
  \begin{align*}
    &(-)^{-}: \BR^{>0} \times \BK \to \BK \\
    &x^y \coloneqq e^{y \ln x}.
  \end{align*}
\end{definition}

\begin{proposition}\label{thm:power_function_properties}
  \mbox{}
  \begin{propenum}
    \DItem{thm:power_function_properties/composition} \( (x^y)^z = x^{yz} \).
  \end{propenum}
\end{proposition}
\begin{proof}
  \begin{description}
    \RItem{thm:power_function_properties/composition}
    \begin{equation*}
      (x^y)^z
      =
      e^{z \ln(e^{y \ln(x)})}
      =
      e^{z y \ln(x)}
      =
      x^{yz}.
    \end{equation*}
  \end{description}
\end{proof}

\begin{definition}\label{def:inverse_trigonometric_functions}
  \cref{thm:trigonometric_function_properties/bijective} allows us to define the \Def{inverse trigonometric functions} \( \arcsin \) and \( \arccos \) from \( [-1, 1] \) to \( [0, 2\pi) \).
\end{definition}

\begin{definition}\label{def:hyperbolic_trigonometric_functions}
  In analogy with \cref{thm:trigonometric_function_properties/sine_characterization} and \cref{thm:trigonometric_function_properties/cosine_characterization}, we define \Def{hyperbolic trigonometric functions}.

  \begin{defenum}
    \DItem{def:hyperbolic_trigonometric_functions/sine} We define the \Def{hyperbolic sine} function as
    \begin{equation*}
      \sinh(x) \coloneqq - \frac {e^x - e^{-x}} 2 \\
    \end{equation*}

    \DItem{def:hyperbolic_trigonometric_functions/cosine} We define the \Def{hyperbolic cosine} function as
    \begin{equation*}
      \cosh(x) \coloneqq \frac {e^x + e^{-x}} 2
    \end{equation*}
  \end{defenum}

  Compare \cref{def:quadratic_plane_curve/ellipse/parametric_equations} and \cref{def:quadratic_plane_curve/hyperbola/parametric_equations} for a justification of the naming.
\end{definition}
