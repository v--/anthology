\subsection{Power series}\label{subsec:power_series}

\begin{definition}\label{def:convergent_power_series}
  Let \( \BK[[X]] \) be the space of power series defined in \cref{def:formal_power_series}.

  To each formal power series
  \begin{equation*}
    \sum_{k=0}^\infty a_k X^k
  \end{equation*}
  there corresponds a function, called a \Def{power series}
  \begin{equation}\label{def:convergent_power_series/series}
    p(x) \coloneqq \sum_{k=0}^\infty a_k x^k.
  \end{equation}

  We sometimes slightly generalize this notion slightly by using a \enquote{shift} by \( \alpha \in \BK \): define the function
  \begin{equation}\label{def:convergent_power_series/shifted_series}
    p(x) \coloneqq \sum_{k=0}^\infty a_k (x - \alpha)^k.
  \end{equation}

  If the limit exists (as a numeric series\Tinyref{def:convergent_series}) for a certain \( x \in \BK \), we say that the series \Def{converges} at \( x \).

  The series is no longer \enquote{formal} because it is now a proper function instead of an abstract algebraic object, although a power series may only be defined in a subset of \( \BK \) (that is, a partial function\Tinyref{def:function/partial}).
\end{definition}

\begin{theorem}\label{thm:power_series_radius_of_convergence}
  For every power series \cref{def:convergent_power_series/series}, there exists a nonnegative extended real number \( r \in [0, +\infty] \), called its \Def{radius of convergence}, such that \cref{def:convergent_power_series/series} converges absolutely if \( \Abs{x} < r \) and diverges if \( \Abs{x} > r \).

  The behavior of the series is more complicated when \( \Abs{x} = r \) (unless \( r = 0 \), in which case the power series converges if and only if \( x = 0 \)).
\end{theorem}
\begin{proof}
  Define
  \begin{equation*}
    q \coloneqq \limsup_{n \to \infty} \sqrt[n]{\Abs{a_n}},
  \end{equation*}
  where we put \( q = +\infty \) if the limit does not exist. We have
  \begin{equation*}
    \limsup_{n \to \infty} \sqrt[n]{\Abs{x^n a_n}} = \Abs{x} q.
  \end{equation*}

  By \cref{thm:cauchys_root_test}, \cref{def:convergent_power_series/series} converges absolutely if \( \Abs{z} q < 1 \) and diverges if \( \Abs{z} q > 1 \).

  Thus \( r \coloneqq \tfrac 1 q \) is the desired radius of convergence.

  Note that we may also use \cref{thm:dalamberts_ratio_test} for finding the same radius of convergence by \cref{remark:nonnegative_series_convergence_test_equivalence}.
\end{proof}

\begin{proposition}\label{thm:power_series_parity}
  Power series of the form
  \begin{equation}\label{thm:power_series_parity/odd}
    f_o(z) \coloneqq \sum_{m \text{ is odd}} a_m z^m = \sum_{k=0}^\infty a_{2k+1} z^{2k+1}
  \end{equation}
  are odd functions\Tinyref{def:function_pairity} and power series of the form
  \begin{equation}\label{thm:power_series_parity/even}
    f_e(z) \coloneqq \sum_{m \text{ is even}} a_m z^m = \sum_{k=0}^\infty a_{2k} z^{2k}
  \end{equation}
  are even functions.
\end{proposition}
\begin{proof}
  If \cref{thm:power_series_parity/odd} converges for \( z \in \BC \),
  \begin{equation*}
    f_o(-z)
    =
    \sum_{k=0}^\infty a_{2k+1} (-z)^{2k+1}
    =
    \sum_{k=0}^\infty a_{2k+1} (-1)^{2k+1} z^{2k+1}
    =
    - \sum_{k=0}^\infty a_{2k+1} z^{2k+1}
    =
    - f_o(z).
  \end{equation*}

  Analogously, since \( (-1)^{2k} = 1 \), we have \( f_e(-z) = f_e(z) \).
\end{proof}

\begin{proposition}\label{thm:power_series_are_locally_uniform_convergent}
  A power series is locally uniformly convergent\Tinyref{def:function_net_convergence/locally_uniform} in the interior of its domain of convergence.
\end{proposition}
\begin{proof}
  Assume that the series \cref{def:convergent_power_series/series} converges inside the ball \( B(0, R) \). Fix \( x \in B(0, R) \) and \( R_x < R - \Abs{x} \). Then the geometric series
  \begin{equation*}
    \sum_{k=0}^\infty a_k R_x^k
  \end{equation*}
  converges and dominates \cref{def:convergent_power_series/series} in the ball \( B(x, R_x) \). Thus by \cref{thm:weierstrass_series_criterion}, \cref{def:convergent_power_series/series} converges uniformly in \( B(x, R_x) \).

  Since the choice of \( x \in B(0, R) \) was arbitrary, we conclude that \cref{def:convergent_power_series/series} is locally uniformly convergent.
\end{proof}

\begin{theorem}\label{thm:series_termwise_operations}
  Suppose that the power series \cref{def:convergent_power_series/series} has a (potentially infinite) radius of convergence \( R \).

  \begin{defenum}
    \DItem{thm:series_termwise_operations/differentiation} \( p(x) \) is differentiable in \( B(0, R) \) and can be differentiated termwise as
    \begin{equation}\label{thm:series_termwise_operations/derivative}
      p'(x) = \sum_{k=0}^\infty a_{k+1} (k+1) x^k.
    \end{equation}

    Furthermore, \( p'(x) \) has the same radius of convergence as \( p(x) \).

    \DItem{thm:series_termwise_operations/integration} If the series is real and \( \Abs{x} < R \), \( p(x) \) is integrable in \( [0, x] \) (or \( [x, 0] \)) and can be integrated termwise as
    \begin{equation}\label{thm:series_termwise_operations/primitive}
      \int_0^x p(t) dt = \sum_{k=0}^\infty a_k \frac {x^{k+1}} {k+1}.
    \end{equation}
  \end{defenum}
\end{theorem}
\begin{proof}
  \begin{description}
    \RItem{thm:series_termwise_operations/differentiation} Note that the right-hand side of \cref{thm:series_termwise_operations/derivative} is a power series. Furthermore, its radius of convergence is, by \cref{thm:power_series_radius_of_convergence},
    \begin{equation*}
      \lim_{k \to \infty} \Abs{\frac {a_{k+1} (k+1) x^k} {a_{k+2} (k+2) x^{k+1}}}
      =
      \Abs{x} \lim_{k \to \infty} \frac {k+1} {k+2} \Abs{\frac {a_{k+1}} {a_{k+2}}}
      =
      R.
    \end{equation*}

    Fix \( x \in B(0, R) \) and choose \( r \in (\Abs{x}, R) \). Both series are uniformly convergent in \( B(0, r) \). By \cref{thm:derivative_limit_exchange/sequence}, the equality \cref{thm:series_termwise_operations/derivative} holds in \( B(0, r) \), hence it also holds for \( x \).

    \RItem{thm:series_termwise_operations/integration} Analogously to \cref{thm:series_termwise_operations/differentiation}, we conclude that the right-hand side of \cref{thm:series_termwise_operations/primitive} is a power series with radius of convergence \( R \).

    The rest follows directly from \cref{thm:riemann_intergral_limit_exchange}.
  \end{description}
\end{proof}
