\subsection{Power series}\label{subsec:power_series}

\begin{definition}\label{def:convergent_power_series}
  Let \( \K[[X]] \) be the space of power series defined in \cref{def:formal_power_series}.

  To each formal power series
  \begin{equation*}
    \sum_{k=0}^\infty a_k X^k
  \end{equation*}
  there corresponds a function, called a \Def{power series}
  \begin{equation}\label{def:convergent_power_series/series}
    p(x) \coloneqq \sum_{k=0}^\infty a_k x^k.
  \end{equation}

  We sometimes slightly generalize this notion slightly by using a \enquote{shift} by \( \alpha \in \K \): define the function
  \begin{equation}\label{def:convergent_power_series/shifted_series}
    p(x) \coloneqq \sum_{k=0}^\infty a_k (x - \alpha)^k.
  \end{equation}

  If the limit exists (as a numeric series\Tinyref{def:convergent_series}) for a certain \( x \in \K \), we say that the series \Def{converges} at \( x \).

  The series is no longer \enquote{formal} because it is now a proper function instead of an abstract algebraic object, although a power series may only be defined in a subset of \( \K \) (that is, a partial function\Tinyref{def:function/partial}).
\end{definition}

\begin{theorem}\label{thm:power_series_radius_of_convergence}
  For every power series \cref{def:convergent_power_series/series}, there exists a nonnegative extended real number \( r \in [0, +\infty] \), called its \Def{radius of convergence}, such that \cref{def:convergent_power_series/series} converges absolutely if \( \Abs{x} < r \) and diverges if \( \Abs{x} > r \).

  The behavior of the series is more complicated when \( \Abs{x} = r \) (unless \( r = 0 \), in which case the power series converges if and only if \( x = 0 \)).
\end{theorem}
\begin{proof}
  Define
  \begin{equation*}
    q \coloneqq \limsup_{n \to \infty} \sqrt[n]{\Abs{a_n}},
  \end{equation*}
  where we put \( q = +\infty \) if the limit does not exist. We have
  \begin{equation*}
    \limsup_{n \to \infty} \sqrt[n]{\Abs{x^n a_n}} = \Abs{x} q.
  \end{equation*}

  By \cref{thm:cauchys_root_test}, \cref{def:convergent_power_series/series} converges absolutely if \( \Abs{z} q < 1 \) and diverges if \( \Abs{z} q > 1 \).

  Thus \( r \coloneqq \tfrac 1 q \) is the desired radius of convergence.

  Note that we may also use \cref{thm:dalamberts_ratio_test} for finding the same radius of convergence by \cref{remark:nonnegative_series_convergence_test_equivalence}.
\end{proof}

\begin{proposition}\label{thm:power_series_parity}
  Power series of the form
  \begin{equation}\label{thm:power_series_parity/odd}
    f_o(z) \coloneqq \sum_{m \text{ is odd}} a_m z^m = \sum_{k=0}^\infty a_{2k+1} z^{2k+1}
  \end{equation}
  are odd functions\Tinyref{def:function_pairity} and power series of the form
  \begin{equation}\label{thm:power_series_parity/even}
    f_e(z) \coloneqq \sum_{m \text{ is even}} a_m z^m = \sum_{k=0}^\infty a_{2k} z^{2k}
  \end{equation}
  are even function.
\end{proposition}
\begin{proof}
  If \cref{thm:power_series_parity/odd} converges for \( z \in \C \),
  \begin{equation*}
    f_o(-z)
    =
    \sum_{k=0}^\infty a_{2k+1} (-z)^{2k+1}
    =
    \sum_{k=0}^\infty a_{2k+1} (-1)^{2k+1} z^{2k+1}
    =
    - \sum_{k=0}^\infty a_{2k+1} z^{2k+1}
    =
    - f_o(z).
  \end{equation*}

  Analogously, since \( (-1)^{2k} = 1 \), we have \( f_e(-z) = f_e(z) \).
\end{proof}
