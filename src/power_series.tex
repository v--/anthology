\section{Complex analysis}\label{subsec:complex_analysis}
\subsection{Power series}\label{subsec:power_series}

Here \( \K \) will refer to either \( \R \) or \( \C \).

\begin{definition}\label{def:convergent_series}
  When extending addition to a countable amount of terms, we need to impose some regularity conditions to avoid contradictions. The topologies of \( \R \) and \( \C \) are complete and allow us to define convergent and divergent series.

  A \Def{series} is an infinite sequence \( a_0, a_1, \ldots \in \K \), usually written as
  \begin{equation}\label{def:convergent_series/series}
    \sum_{i=0}^\infty a_i.
  \end{equation}

  To each series, there corresponds its sequence of \Def{partial sums}
  \begin{equation*}
    S_n \coloneqq \sum_{i=0}^n a_i, n = 0, 1, 2, \ldots.
  \end{equation*}

  We say that the series \cref{def:convergent_series/series} \Def{converges} to a value \( a \) if \( \lim_{n \to \infty} S_n = a \) in the sense of \cref{thm:metric_convergence_iff_metric_topology_convergence}.

  If a series does not converge, we say that it is \Def{divergent}.

  If the related series
  \begin{equation*}
    \sum_{i=0}^\infty \Abs{a_i}
  \end{equation*}
  converges, we say that \cref{def:convergent_series/series} is \Def{absolutely convergent}.
\end{definition}

\begin{proposition}\label{thm:convergent_series_coefficients_vanish}
  The coefficients of the convergent series
  \begin{equation*}
    \sum_{i=0}^\infty a_i
  \end{equation*}
  vanish as \( i \to \infty \), that is,
  \begin{equation*}
    \lim_{i \to \infty} a_0 = 0.
  \end{equation*}
\end{proposition}
\begin{proof}
  Since the series is convergent, its sequence of partial sums converges, i.e. the partial sums get arbitrarily close to each other. Then
  \begin{equation*}
    \Abs{a_n} = \Abs{S_n - S_{n-1}} \to 0.
  \end{equation*}
\end{proof}

\begin{theorem}[Riemann's series theorem]\label{thm:riemanns_series_theorem}\cite[\textnumero 247]{Фихтенгольц1968}
  If the series is not absolutely convergent, then there exist both infinitely many positive and infinitely many negative coefficients.

  If the real series
  \begin{equation*}
    \sum_{i=0}^\infty a_i
  \end{equation*}
  is convergent\Tinyref{def:convergent_series} but not absolutely convergent, then for any extended real number \( x \in \R \cup \{ -\infty, +\infty \} \) there exists a permutation\Tinyref{def:symmetric_group} \( p \) of the coefficients \( a_0, a_1, a_2 \)
  such that
  \begin{equation*}
    \sum_{i=0}^\infty p(a_i) = x.
  \end{equation*}
\end{theorem}
\begin{proof}
  First, assume that \( x \) is finite.

  Define the permuted series
  \begin{equation*}
    \sum_{i=0}^\infty b_i
  \end{equation*}
  as follows:
  \begin{algenum}
    \DItem{thm:riemanns_series_theorem/positive} Assign to \( b_n \) only nonnegative elements of the sequence \( \{ a_i \}_{i=0}^\infty \) until \( \sum_{i=0}^n b_i \geq x \). Then go to \cref{thm:riemanns_series_theorem/negative}.
    \DItem{thm:riemanns_series_theorem/negative} Assign to \( b_n \) only negative elements of the sequence \( \{ a_i \}_{i=0}^\infty \) until \( \sum_{i=0}^n b_i \geq x \). Then go to \cref{thm:riemanns_series_theorem/positive}.
  \end{algenum}

  This mutual recursion builds a series that converges to \( x \) because the coefficients \( \{ a_i \}_{i=0}^\infty \) get arbitrarily close to each other.

  If \( x = +\infty \), we can add positive coefficients until \( \sum_{i=0}^n b_i \geq 1 \), then add a single negative coefficient, then continue adding positive coefficients until \( \sum_{i=0}^n b_i \geq 2 \) and so on.

  If \( x = -\infty \), we use the same process but with milestones of \( -1, -2, -3, \ldots \).
\end{proof}

\begin{definition}\label{def:convergent_power_series}
  Let \( \K[[X]] \) be the space of power series defined in \cref{def:formal_power_series}.

  To each formal power series
  \begin{equation*}
    \sum_{i=0}^\infty a_i X^i
  \end{equation*}
  there corresponds a function, called a \Def{power series}
  \begin{equation*}
    x \mapsto \sum_{i=0}^\infty a_i x^i.
  \end{equation*}

  It is no longer \enquote{formal} because it is now a proper function instead of an abstract algebraic object, although a power series may only be defined in a subset of \( \K \) (that is, a partial function\Tinyref{def:function/partial}).
\end{definition}
