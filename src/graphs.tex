\subsection{Graphs}\label{subsec:graphs}

\begin{remark}\label{remark:directed_and_undirected_graphs}
  Unfortunately, the word \enquote{graph} has at least three popular meanings withing mathematics:
  \begin{enumerate}
    \item graphs of functions (see \cref{def:function_graph})
    \item directed graphs (see \cref{def:directed_graph})
    \item undirected graphs (see \cref{def:undirected_graph})
  \end{enumerate}

  Graphs of functions are different enough from the other two notions to not cause any confusion, however it is often not clear from the context whether \enquote{graph} refers to directed or undirected graphs. Both are formalisms corresponding to dots in the plane connected with (directed or undirected) lines (see \cref{ex:directed_graph}).

  We define undirected graphs as a special case of directed graphs. This approach makes some definitions more awkward. In programming, however, implementing undirected graphs as a special case of directed graphs is often more versatile (see \cite[section 5.4]{Erickson2019} and \cite[chapter 1, section 2.4]{Gondran1984}).
\end{remark}

\begin{definition}\label{def:directed_graph}\cite[1-graphs from chapter 1, section 1.1]{Gondran1984}
  A \Def{directed graph} \( G = (V, E) \) is a pair where
  \begin{itemize}
    \item \( V \) is a set, whose elements are called \Def{vertices} or \Def{nodes}.
    \item \( E \subseteq V^2 \) is a relation\Tinyref{def:relation} over \( V \), whose elements are called \Def{arcs}. If \( u, v \in V \) are vertices and \( e = (u, v) \) is an arc, we say that \( u \) is the \Def{head} or \Def{initial endpoint} and \( v \) is the \Def{tail} or \Def{terminal endpoint} of the arc. We denote $u = h(e)$ and $v = t(e)$.
  \end{itemize}

  For readability, we use the infix notation \( u \to v \) rather than \( (u, v) \) for arcs.

  We say that
  \begin{defenum}
    \DItem{def:directed_graph/order} the number
    \begin{equation*}
      \Ord G \coloneqq \Card V
    \end{equation*}
    is the \Def{order} of the graph \( G = (V, E) \).
    \DItem{def:directed_graph/empty} a graph \( G = (V, E) \) is \Def{empty} if \( E = \varnothing \).
    \DItem{def:directed_graph/subgraph} the graph \( G' = (V', E') \) is a \Def{subgraph} of \( G = (V, E) \) if we have both \( V' \subseteq V \) and \( E' \subseteq E \).
    \DItem{def:directed_graph/loop} the arc \( u \to v \) is a \Def{loop} if \( u = v \).
    \DItem{def:directed_graph/simple}\cite[chapter 1, section 1.3]{Gondran1984} the graph \( G \) is \Def{simple} if it has no loops.
  \end{defenum}
\end{definition}

\begin{definition}\label{def:graph_matrices}
  Let \( G = (V, E) \) be a simple finite directed graph.
  \begin{defenum}
    \DItem{def:graph_matrices/incidence}\cite[chapter 1, section 2.1]{Gondran1984} The \Def{incidence matrix} \( I = \{ a_{ve} \}_{v \in V, e \in E} \) of \( G \) is defined as
    \begin{equation*}
      a_{ve} \coloneqq \begin{cases}
        1, &v \text{ is the head of } e \\
        -1, &v \text{ is the tail of } e \\
        0, &\text{otherwise}
      \end{cases}
    \end{equation*}

    \DItem{def:graph_matrices/adjacency}\cite[chapter 1, section 2.3]{Gondran1984} The \Def{adjacency matrix} \( A = \{ a_{ve} \}_{u, v \in V} \) of \( G \) is defined as
    \begin{equation*}
      a_{uv} \coloneqq \begin{cases}
        1, &(u, v) \in E \\
        0, &\text{otherwise}
      \end{cases}
    \end{equation*}
  \end{defenum}
\end{definition}

\begin{example}\label{ex:directed_graph}
  Consider the directed graph \( G = (V, E) \), corresponding to the following drawing
  \begin{AlignedEquation}\label{ex:directed_graph/embedding}
    \begin{mplibcode}
      u := 0.6cm;

      beginfig(1);
        input metapost/graphs;
        lax_bboxmargin := 2pt;

        v1 := thelabel("$a$", origin);
        v2 := thelabel("$b$", (2, 2) scaled u);
        v3 := thelabel("$c$", (5, 2) scaled u);
        v4 := thelabel("$d$", (2, -2) scaled u);
        v5 := thelabel("$e$", (5, -2) scaled u);
        v6 := thelabel("$f$", (7, 0) scaled u);

        a1 := straight_arc(v1, v2);
        a2 := straight_arc(v1, v4);
        a3 := straight_arc(v2, v3);
        a5 := straight_arc(v4, v3);
        a4 := straight_arc(v4, v5);
        a6 := straight_arc(v3, v6);
        a7 := straight_arc(v5, v6);

        draw_vertices(v);
        draw_arcs(a);

        label.ulft("$1$", straight_arc_midpoint of a1);
        label.llft("$2$", straight_arc_midpoint of a2);
        label.top("$3$", straight_arc_midpoint of a3);
        label.bot("$5$", straight_arc_midpoint of a4);
        label.ulft("$4$", straight_arc_midpoint of a5);
        label.urt("$6$", straight_arc_midpoint of a6);
        label.lrt("$7$", straight_arc_midpoint of a7);
      endfig;
    \end{mplibcode}
  \end{AlignedEquation}

  The vertices are labeled \( a \) through \( f \) and the arcs are labeled 1 through 7.

  The corresponding incidence matrix is
  \begin{align*}
    \bordermatrix{
        & 1  & 2  & 3  & 4  & 5  & 6  & 7  \cr
      a & 1  & 1  &    &    &    &    &    \cr
      b & -1 &    & 1  &    &    &    &    \cr
      c &    &    & -1 & -1 &    & 1  &    \cr
      d &    & -1 &    & 1  & 1  &    &    \cr
      e &    &    &    &    & -1 &    & 1  \cr
      f &    &    &    &    &    & -1 & -1
    }
  \end{align*}
  and the adjacency matrix is
  \begin{align*}
    \bordermatrix{
        & a  & b  & c  & d  & e  & f  \cr
      a &    & 1  &    & 1  &    &    \cr
      b &    &    & 1  &    &    &    \cr
      c &    &    &    &    &    & 1  \cr
      d &    &    & 1  &    & 1  &    \cr
      e &    &    &    &    &    & 1  \cr
      f &    &    &    &    &    &
    }
  \end{align*}
\end{example}

\begin{definition}[Paths in directed graphs]\label{def:graph_paths}
  Let \( G = (V, E) \) be a directed graph.

  \begin{defenum}
    \DItem{def:graph_paths/adjacent_vertices} Two vertices \( u \)  and \( v \)  are called \Def{adjacent} if there exists an arc from \( u \)  to \( v \) .

    \DItem{def:graph_paths/adjacent_arcs}\cite[chapter 1, section 1.4]{Gondran1984} Two arcs are called \Def{adjacent} if they have a common endpoint. Thus the first three pairs of arcs are adjacent and the forth is not (assuming all vertices are distinct):
    \begin{enumerate}
      \item \( u \to v \) and \( v \to w \)
      \item \( u \to v \) and \( u \to w \)
      \item \( u \to w \) and \( v \to w \)
      \item \( u \to u' \) and \( v \to v' \)
    \end{enumerate}

    \DItem{def:graph_paths/undirected_path}\cite[chapter 1, section 3.1]{Gondran1984} An \Def{undirected path} or \Def{chain} is a sequence of distinct arcs
    \begin{equation*}
      p \coloneqq ( e_1, \ldots, e_n ),
    \end{equation*}
    such that any two consecutive arcs are adjacent, that is, the arcs \( e_i \) and \( e_{i+1} \) are adjacent for \( i = 1, \ldots, n - 1 \). We say that \( u \) is the \Def{head} of \( p \) if it is an endpoint of \( e_1 \) but not \( e_2 \) and that \( v \) is the \Def{tail} of \( p \) if it is an endpoint of \( e_n \) but not \( e_{n-1} \). The number $n$ is called the \Def{length} of the path and is denoted by \( \Len p \).

    In the graph \cref{ex:directed_graph/embedding}, \( (1, 2, 3) \) is a path with head \( a \), tail \( f \) and length 3, while \( (3, 4) \) is a path with head \( b \), tail \( d \) and length 2.

    If all vertices in a path are distinct, that is, if there are exactly $n + 1$ distinct vertices, we say that the path is \Def{simple}.

    Some authors (e.g. \cite[section 5.2]{Erickson2019}) call undirected paths \Def{walks} and reserve the term \enquote{path} for simple undirected paths.

    If the head and the tail of a path coincide, we say that the path is a \Def{cycle}.

    A \Def{simple cycle} is a cycle where all non-endpoint vertices are distinct.

    \DItem{def:graph_paths/directed_path}\cite[chapter 1, section 3.2]{Gondran1984} If the tail of each non-endpoint arc in a path coincides with the head of the next arc, we say that the path is a \Def{directed path}.

    In the graph \cref{ex:directed_graph/embedding}, \( (1, 2, 3) \) is a directed path, while \( (3, 4) \) is not.

    A directed cycle is also called a \Def{circuit}.

    \DItem{def:graph_paths/dag}\cite[231]{Erickson2019} A \Def{directed acyclic graph} or \Def{dag} is a directed graph without directed cycles.

    \DItem{def:graph_paths/eulerian_path}\cite[chapter 8, section 1.1]{Gondran1984} A path (either directed or undirected) is called \Def{Eulerian} if it contains every arc of the graph exactly once, that is, the path induces an ordering of the arcs. A graph with an Eulerian cycle is called an \Def{Eulerian graph}.

    \DItem{def:graph_paths/hamiltonian_path}\cite[chapter 8, section 3.1]{Gondran1984} A simple path (either directed or undirected) is called \Def{Hamiltonian} if it contains every vertex of the graph, that is, it induces an ordering of the vertices. A graph with a Hamiltonian cycle is called a \Def{Hamiltonian graph}.
  \end{defenum}
\end{definition}

\begin{definition}[Incidence in directed graphs]\label{def:graph_incidence}
  Let \( G = (V, E) \) be a directed graph. We define the multivalued functions\Tinyref{def:function/multivalued} with signature \( \Power(V) \MultTo E \):
  \begin{align*}
    &w^+(A) \coloneqq \{ (u, v) \in E \colon u \in A \} \\
    &w^-(A) \coloneqq \{ (u, v) \in E \colon v \in A \} \\
    &w(A) \coloneqq w^+(A) \cup w^-(A).
  \end{align*}

  That is, for a set \( A \) of vertices, \( w^+(A) \) gives us the set of arcs whose head is in \( A \), \( w^-(A) \) gives us the set of arcs whose tail is in \( A \) and \( w(A) \) gives us all arcs with at least one endpoint in \( A \).

  \begin{defenum}
    \DItem{def:graph_incidence/incident_arcs} The arc \( e \) is said to be \Def{incident} with the vertex \( v \) if \( e \in w(v) \), that is, if \( v \) is an endpoint of \( e \).

    \DItem{def:graph_incidence/degree}\cite[chapter 1, section 1.4]{Gondran1984} Given a vertex \( v \), the \Def{degree} \( d(v) \) (resp. \Def{in-degree} \( d^+(v) \) and \Def{out-degree} \( d^-(v) \)) of the vertex is defined as
    \begin{equation*}
      d(v) \coloneqq \Card w(v).
    \end{equation*}

    The degree of the graph is then defined as
    \begin{equation*}
      d(G) \coloneqq \max_{v \in V} d(v).
    \end{equation*}
  \end{defenum}
\end{definition}

\begin{definition}[Connectivity in directed graphs]\label{def:graph_connectivity}
  Let \( G = (V, E) \) be a directed graph.

  We say that
  \begin{defenum}
    \DItem{def:graph_connectivity/reachable_vertices} the vertex \( v \) is \Def{reachable} from the vertex \( u \) if there exists a directed path\Tinyref{def:graph_paths/directed_path} from \( u \) to \( v \).

    \DItem{def:graph_connectivity/strongly_connected_graph}\cite[chapter 1, section 3.5]{Gondran1984} the graph \( G \) is \Def{strongly connected} if every pair of distinct vertices are reachable, that is, if there exists a directed path between every pair of distinct vertices.

    \DItem{def:graph_connectivity/weakly_connected_graph}\cite[chapter 1, section 3.3]{Gondran1984} the graph \( G \) is \Def{weakly connected} if there exists an undirected path between every pair of distinct vertices.

    \DItem{def:graph_connectivity/connected_component}\cite[chapter 1, sections 3.3 and 3.5]{Gondran1984} the subgraph \( G' \) of \( G \) is a \Def{connected component} (resp. \Def{strongly connected component}) if it is connected (resp. strongly connected) and there exists no connected (resp. strongly connected) subgraph of \( G \) that properly contains \( G' \).

    \DItem{def:graph_connectivity/connectivity_number}\cite[chapter 1, sections 3.3 and 3.5]{Gondran1984} \( G \) has \Def{connectivity number} (resp. \Def{strong connectivity number}) $n$ if it has $n$ connected (resp. strongly connected) components.

    \DItem{def:graph_connectivity/cut}\cite[chapter 1, section 3.4]{Gondran1984} the set \( U \subseteq V \) of vertices is a \Def{cut} (resp. \Def{directed cut}) if removing \( T \) from the graph would increase the connectivity number (resp. strong connectivity number) of the graph.

    \DItem{def:graph_connectivity/cocycle}\cite[chapter 1, section 4.4]{Gondran1984} the set \( F \subseteq E \) of arcs is a \Def{cocycle} (resp. \Def{cocircuit}) if there exists a set \( U \subseteq V \) of vertices such that \( F = w(T) \) (resp. \( F \in \{ w^+(T), w^-(T) \} \)).
  \end{defenum}
\end{definition}

\begin{definition}[Adjacency in directed graphs]\label{def:graph_adjacency}
  Let $G = (V, E)$ be an undirected graph.

  \begin{defenum}
    \DItem{def:graph_adjacency/clique}\cite[chapter 1, section 1.4]{Gondran1984} The set \( U \subseteq V \) is called a \Def{clique} if all two vertices in \( U \) are adjacent.

    \DItem{def:graph_adjacency/complete_graph}\cite[chapter 1, section 1.4]{Gondran1984} If \( V \) itself is a clique, we say that \( G \) is a \Def{complete graph}.

    \DItem{def:graph_adjacency/anticlique}\cite[120]{Erickson2019} Dually, \( U \subseteq V \) is an \Def{anticlique} or \Def{independent set} of vertices if no two vertices in \( U \) are adjacent.

    \DItem{def:graph_adjacency/matching}\cite[chapter 5, exercise 11]{Gondran1984} The set \( F \subseteq E \) of arcs is a \Def{matching} or \Def{independent set} of arcs if no two arcs in \( F \) are adjacent.

    \DItem{def:graph_adjacency/bipartite_graph}\cite[7]{Gondran1984} The graph is called \Def{bipartite} if there exists a partition \( \{ A, B \} \) of \( V \) such that both \( A \) and \( B \) are anticliques. We also write \( G = (A, B, E) \).

    If \( G \) is undirected and if for every pair of vertices \( a \in A, b \in B \) there is an arc \( a \to b \), we say that \( G \) is a complete bipartite graph.
  \end{defenum}
\end{definition}

\begin{definition}\label{def:undirected_graph}
  An \Def{undirected graph} is a directed graph \( G = (V, E) \) where \( E \) is a symmetric relation (see \cref{remark:directed_and_undirected_graphs}). When dealing with undirected graphs, instead of speaking about the arcs \( u \to v \) and \( v \to u \), we speak about the \Def{edges} \( \{ u, v \} \). Thus, we can also define an undirected graph to be the tuple \( G = (V, E) \), where
  \begin{itemize}
    \item \( V \) is a set of \Def{vertices}.
    \item \( E \subseteq \Power(V) \) is a family of unordered pairs of vertices, that is, singletons and two-element sets.
  \end{itemize}

  Defining undirected graphs as a special case of directed graphs allows us to somewhat unify their study and usage, however we need to keep in mind some remarks:
  \begin{itemize}
    \item All paths are directed and hence we only speak of \Def{paths} and \Def{cycles}. It is necessary, however, to not allow consecutive arcs to represent the same edge, that is, we must treat paths as sequences of edges rather than sequences of arcs.

    Otherwise, since every edge corresponds to two \enquote{inverse} arcs, for all adjacent vertices \( u \) and \( v \) the path \( (u \to v, v \to u) \) is a cycle and hence all undirected graphs would be cyclic.

    \item The incidence matrix is usually defined as \( I = \{ a_{ve} \}_{v \in V, e \in E} \), where
    \begin{equation*}
      a_{ve} \coloneqq \begin{cases}
        1, &v \text{ is an endpoint of } e \\
        0, &\text{otherwise}
      \end{cases}
    \end{equation*}

    \item The adjacency matrix is symmetric if and only if the graph is undirected.

    \item If \( G \) contains to cycles, we say that it is \Def{acyclic}.

    \item The notions of connectedness and strong connectedness coincide. Connected acyclic graphs are called \Def{trees}. Undirected acyclic graphs are often called \Def{forests} since the connected components are trees.

    \item The in-degrees and out-degrees of vertices coincide with the degree. Also, \( w(A) = w^+(A) = w^-(A) \).
  \end{itemize}
\end{definition}

\begin{example}\label{ex:petersen_graph}\cite[347]{Gondran1984}
  The Petersen graph
  \begin{AlignedEquation}\label{ex:petersen_graph/embedding}
    \begin{mplibcode}
      u := 1cm;

      beginfig(1);
        input metapost/graphs;
        lax_bboxmargin := 2pt;

        for i = 1 upto 5:
          v[i] := thelabel("$\bullet$", dir(18 + i * 72) scaled u);
          v[5 + i] := thelabel("$\bullet$", dir(18 + i * 72) scaled 2u);
        endfor;

        for i = 1 upto 4:
          a[i] := straight_arc(v[5 + i], v[5 + i + 1]);
          a[5 + i] := straight_arc(v[i], v[5 + i]);
        endfor;

        a5 := straight_arc(v10, v6);
        a10 := straight_arc(v5, v10);

        a11 := straight_arc(v1, v3);
        a12 := straight_arc(v1, v4);
        a13 := straight_arc(v2, v4);
        a14 := straight_arc(v2, v5);
        a15 := straight_arc(v3, v5);

        draw_vertices(v);
        draw_edges(a);
      endfig;
    \end{mplibcode}
  \end{AlignedEquation}
  is connected, but not acyclic nor Hamiltonian\Tinyref{def:graph_paths/hamiltonian_path}.
\end{example}

\begin{definition}\label{def:graph_homomorphism}
  Let \( G = (V, E) \) and \( G' = (V', E') \) be directed graphs. We say that the function \( f: V \to V' \) is a \Def{graph homomorphism} if for every two vertices \( u, v \in V \),
  \begin{equation*}
    (u, v) \in E \text{ implies } (f(u), f(v)) \in E'.
  \end{equation*}

  The terminology from \cref{def:morphism_invertibility} applies to graph homomorphisms because of the category \( \Cat{Graph} \) of graphs\Tinyref{def:category_of_graphs}.
\end{definition}

\begin{definition}\label{def:category_of_graphs}
  We denote by \( \Cat{DGraph} \) the category\Tinyref{def:category} where
  \begin{itemize}
    \item the class\Tinyref{def:set_zfc} of objects is the class of all directed graphs\Tinyref{def:directed_graph}.
    \item the morphisms between two graphs the homomorphisms between them, with morphism composition being the usual function composition\Tinyref{def:function_composition}.
  \end{itemize}

  We denote by \( \Cat{UGraph} \) the category of undirected graphs. Given the convention established in \cref{remark:directed_and_undirected_graphs} about undirected graphs being a special case of directed graphs, we can regard \( \Bold{UGraph} \) as a subcategory of \( \Bold{DGraph} \).

  Both categories is locally small\Tinyref{def:category_cardinality}.
\end{definition}
