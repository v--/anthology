\subsection{Prime ideals}\label{subsec:prime_ideals}

\begin{definition}\label{def:prime_ring_ideal}\cite[384]{Knapp2016BAlg}
  An ideal \( P \) is called \Def{prime} if it is proper and satisfies any of the equivalent conditions:
  \begin{defenum}
    \DItem{def:prime_ring_ideal/direct} If \( x, y \in R \) are such that \( xy \in P \), then either \( x \in P \) or \( y \in P \).
    \DItem{def:prime_ring_ideal/ideals} If \( I, J \subseteq R \) are ideals such that \( IJ \subseteq P \), then either \( I \subseteq P \) or \( J \subseteq P \).
    \DItem{def:prime_ring_ideal/quotient} The quotient \( R / P \) is an integral domain.
  \end{defenum}

  An element \( r \in R \) is called \Def{prime} if the ideal \( \Gen r \) is prime.
\end{definition}
\begin{proof}
  \Implies[def:prime_ring_ideal/direct][def:prime_ring_ideal/ideals] Fix ideals \( I, J \) of \( R \) such that \( IJ \subseteq P \).

  Assume\LEM that neither \( I \not\subseteq P \) nor \( J \not\subseteq P \). Take \( x \in I \setminus P \) and \( y \in J \setminus P \). It follows that \( xy \in P \) and either \( x \in P \) or \( y \in P \). This contradicts our assumption.

  The obtained contradiction proves that either \( I \subseteq P \) or \( J \subseteq P \).

  \Implies[def:prime_ring_ideal/ideals][def:prime_ring_ideal/quotient] Fix an ideal \( P \) such that if \( I, J \subseteq R \) are ideals and \( IJ \subseteq P \), then either \( I \subseteq P \) or \( J \subseteq P \).

  We will prove that \( R / P \) is an integral domain. If \( R \) is an integral domain, this is obvious. If not, we fix nonzero \( x, y \in R \) so that \( xy = 0 \). Thus \( [x][y] = (x + P)(y + P) = xy + P = P = [0] \). We will show that either \( x = 0 \) or \( y = 0 \).

  Consider the ideals
  \begin{align*}
    \Gen{x} &= xR, \\
    \Gen{y} &= yR.
  \end{align*}

  By \cref{thm:product_of_principal_ideals}, we have \( \Gen{x} \Gen{y} = \Gen{xy} = \Gen{0} = \{ 0 \} \).

  Since \( \Gen{x} \Gen{y} \subseteq P \), then either \( \Gen{x} \subseteq P \) or \( \Gen{y} \subseteq P \). That is, either \( [x] = 0 \) or \( [y] = 0 \).

  Thus \( R / P \) is an integral domain.

  \Implies[def:prime_ring_ideal/quotient][def:prime_ring_ideal/direct] Suppose that \( R / P \) is an integral domain. Fix \( x, y \in R \) so that \( xy \in P \). If \( x = 0 \), obviously \( x = 0 \in R \) and similarly for \( y \). Suppose that both \( x \) and \( y \) are nonzero. We will show that either \( x \in P \) or \( y \in P \).

  We have
  \begin{equation*}
    [x][y] = [xy] = xy + P = P = [0].
  \end{equation*}

  Since \( R / P \) is an integral domain, either \( [x] = [0] \) or \( [y] = [0] \). That is, either \( x \in P \) or \( y \in P \).
\end{proof}

\begin{proposition}\label{thm:prime_ideal_iff_prime_quotient_ideal}
  If \( J \subseteq I \) are ideals of \( R \), then \( I \) is a prime ideal\Tinyref{def:prime_ring_ideal} in \( R \) if and only if \( I / J \) is a prime ideal in  \( R / J \).
\end{proposition}

\begin{definition}\label{def:irreducible_ring_element}
  A nonzero element \( r \in R \) of an integral domain is called \Def{reducible} if there exist non-invertible elements \( r_1, r_2 \in R \) such that
  \begin{equation*}
    r = r_1 r_2.
  \end{equation*}

  If \( r \) is not reducible, we say that it is \Def{irreducible}.
\end{definition}

\begin{definition}\label{def:coprime_ring_ideals}
  Two ring ideals \( I \subseteq R \) and \( J \subseteq R \) are said to be \Def{coprime} if \( I + J = R \).
\end{definition}

\begin{proposition}\label{thm:prime_implies_irreducible}\cite[389]{Knapp2016BAlg}
  All prime\Tinyref{def:prime_ring_ideal} elements in an integral domain are irreducible\Tinyref{def:irreducible_ring_element}.
\end{proposition}
\begin{proof}
  Let \( p \) be prime. Assume\LEM that \( p \) is reducible, that is, there exist non-invertible elements \( r_1, r_2 \in R \) such that
  \begin{equation*}
    p = r_1 r_2.
  \end{equation*}

  Since \( p \) is prime, it must divide either \( r_1 \) or \( r_2 \). Without loss of generality, assume that \( p | r_1 \) and \( r_1 = pc \) for some \( c \in R \).

  Then \( p = r_1 r_2 = pc r_2 \). By \cref{thm:semiring_properties/cancellable_iff_not_zero_divisor}, \( 1 = c r_2 \), which implies that \( r_2 \) is invertible with inverse \( c \). This contradicts our assumption that both \( r_1 \) and \( r_2 \) are invertible.

  The obtained contradiction proves that \( p \) is irreducible.
\end{proof}

\begin{definition}\label{def:maximal_ring_ideal}
  A two-sided ideal \( M \) is called \Def{maximal} if it is proper and satisfies any of the equivalent conditions:
  \begin{defenum}
    \DItem{def:maximal_ring_ideal/maximality} \( M \) is maximal with respect to set inclusion among proper two-sided ideals.
    \DItem{def:maximal_ring_ideal/quotient} The quotient \( R / M \) is a field.
  \end{defenum}
\end{definition}
\begin{proof}
  \Implies[def:maximal_ring_ideal/maximality][def:maximal_ring_ideal/quotient] Suppose that \( M \) is maximal among proper ideals. We will prove that every nonzero element of \( R / M \) is invertible.

  Fix \( x \not\in M \) so that \( [x] = x + M \neq M = [0] \). Define the set
  \begin{equation*}
    I \coloneqq Rx + M.
  \end{equation*}

  It is a ideal since both \( Rx \) and \( M \) are ideals. Furthermore, it contains \( M \) strictly because \( M \subseteq I \) and \( x \in I \). Since \( M \) is maximal, we have that \( I = R \).

  Hence there exists \( y \in R \) such that \( 1 = yx + M \). Hence \( [y] = y + M \) is an inverse of \( [x] \) in \( R / M \).

  Since \( [x] \in R / M \) was an arbitrary nonzero element, we conclude that \( R / M \) is a field.

  \Implies[def:maximal_ring_ideal/quotient][def:maximal_ring_ideal/maximality] Suppose that \( R / M \) is a field. Assume that \( M \) is not maximal. Then there exists a proper ideal \( I \supsetneq M \).

  Assume that \( I \neq M \) and take \( x \in I \setminus M \). Then \( x \not\in M \) and hence \( [x] \neq [0] \) and is invertible in \( R / M \). Denote by \( y \) any representative of this inverse. Thus \( [xy] - [1] = [0] \), that is, \( xy - 1 \in M \).

  Note that \( xy \in I \) because \( x \in I \) and \( y \in R \). Since \( I \) is closed under addition, it follows that \( 1 \in I \) and hence \( I = R \). But this contradicts our assumption that \( I \) is proper.

  The obtained contradiction proves that \( M \) is maximal.
\end{proof}

\begin{theorem}[Krull's theorem]\label{thm:krulls_theorem}\cite{Hodges1979}
  Every nontrivial commutative unital ring has a maximal ideal.
\end{theorem}

\begin{proposition}\label{thm:maximal_ideals_are_prime}
  Maximal ring ideals\Tinyref{def:maximal_ring_ideal} are prime\Tinyref{def:prime_ring_ideal}.
\end{proposition}
\begin{proof}
  If \( M \) is a maximal ideal of \( R \), by \cref{def:maximal_ring_ideal/quotient} \( R / M \) is a field. Thus \( R / M \) is an integral domain, which by \cref{def:prime_ring_ideal/quotient} means that \( M \) is a prime ideal.
\end{proof}

\begin{proposition}\label{thm:field_maximal_ideal_representation}\cite[exercise 8.1]{Коцев2016}
  If \( r_1, \ldots, r_n \) are elements of the field \( \BK \), then \( \Gen{X_1 - r_1, \ldots, X_n - r_n} \) is a maximal ideal of \( \BK[X_1, \ldots, X_n] \).
\end{proposition}

\begin{proposition}\label{thm:ufd_prime_iff_irreducible}
  An element in a unique factorization domain is prime\Tinyref{def:prime_ring_ideal} if and only if it is irreducible\Tinyref{def:irreducible_ring_element}.
\end{proposition}
\begin{proof}
  \begin{description}
    \Implies Follows from \cref{thm:prime_implies_irreducible}.

    \ImpliedBy Let \( r \) be an irreducible element and let \( p_1 p_2 \in \Gen{r} \). We will show that either  \( p_1 \in \Gen{r} \) or \( p_2 \in \Gen{r} \).

    Since \( \Gen{r} \) is an ideal, there exists an element \( q \in R \) such that \( qr = p_1 p_2 \). Because of unique factorization, there exists a unit \( u \in R \) such that \( uqr = p_1 p_2  \).

    Therefor either
    \begin{itemize}
      \item \( p_1 = 1 \), in which case \( p_2 = uqr \in \Gen{r} \).
      \item \( p_1 = u \), in which case \( p_2 = qr \in \Gen{r} \).
      \item \( p_1 = uq \), in which case \( p_2 = r \in \Gen{r} \).
      \item \( p_1 = ur \in \Gen{r} \).
      \item \( p_1 = qr \in \Gen{r} \).
      \item \( p_1 = uqr \in \Gen{r} \).
    \end{itemize}
  \end{description}
\end{proof}

\begin{proposition}\label{thm:prime_ideals_are_maximal_in_pid}
  Prime ring ideals\Tinyref{def:prime_ring_ideal} in a principal ideal domain are maximal\Tinyref{def:maximal_ring_ideal}.
\end{proposition}
\begin{proof}
  Let \( P \) be a prime ideal of \( R \) and let \( I \supsetneq P \) be an ideal strictly containing \( P \). We will show that \( I = R \).

  Since \( R \) is a principal ideal domain, both \( P \) and \( I \) are principal. Let \( p \) and \( i \) be their respective generators. Since \( I \) contains \( P \), there exists \( r \in R \) such that
  \begin{equation*}
    p = ir.
  \end{equation*}

  But \( p \) is prime, and thus irreducible by \cref{thm:ufd_prime_iff_irreducible}, and hence either \( i \) or \( r \) must be a unit. If\LEM \( r \) is a unit, then \( \Gen i = \Gen {ir} = \Gen p \), which contradicts our choice of \( I \supsetneq P \). It remains for \( i \) to be a unit.

  Therefore \( I = \Gen i = \Gen 1 = R \). This proves that \( P \) is maximal with respect to inclusion of ideals.
\end{proof}

\begin{definition}\label{def:krull_dimension}\cite[67]{Коцев2016}
  Consider chains\Tinyref{def:tower_diagram}
  \begin{equation*}
    P_0 \subsetneq P_1 \subsetneq \cdots
  \end{equation*}
  of prime ideals\Tinyref{def:prime_ring_ideal} in \( R \) under strict inclusion. The length of this chain is defined as the zero-based index of its last element and is allowed to be infinite. Zero-based means that a chain with only one ideal has length zero.

  We call the supremum of the lengths of these chains the \Def{Krull dimension} of the ring \( R \) and denote it by \( \dim R \).
\end{definition}

\begin{proposition}\label{thm:krull_dimension_properties}
  The Krull dimension\Tinyref{def:krull_dimension} of a ring \( R \) has the following basic properties:
  \begin{propenum}
    \DItem{thm:krull_dimension_properties/monotone} If \( R = T / S \) is the quotient of some rings \( S \subseteq T \), \( \dim R \leq \dim T \).
    \DItem{thm:krull_dimension_properties/pid} If \( R \) is a principal ideal domain, \( \dim R \in \{ 0, 1 \} \).
    \DItem{thm:krull_dimension_properties/field} If \( R \) is a field\Tinyref{def:field}, \( \dim R = 0 \).
    \DItem{thm:krull_dimension_properties/polynomials_over_field}\cite[exercise 8.19]{Коцев2016} If \( R = \BK[X_1, \ldots, X_n] \) for some field\Tinyref{def:field} \( \BK \), \( \dim R = n \).
  \end{propenum}
\end{proposition}

\begin{corollary}\label{thm:multivariate_polynomial_rings_are_not_pid}
  Multivariate polynomial rings are not principal ideal domains.
\end{corollary}
\begin{proof}
  Follows from \cref{thm:krull_dimension_properties/pid} and \cref{thm:krull_dimension_properties/polynomials_over_field}.
\end{proof}

\begin{definition}\label{def:radical_ideal}\cite[15]{Коцев2016}
  We define the \Def{radical ideal} of the ideal \( I \) of \( R \) as
  \begin{equation*}
    \sqrt I \coloneqq \{ x \in R \colon \exists n: x^n \in I \}.
  \end{equation*}
\end{definition}
\begin{proof}
  We verify that the set \( \sqrt I \) is an ideal of \( R \).

  It is closed under addition because if \( x^n \in I \) and \( y^m \in I \), then by \cref{thm:binomial_theorem},
  \begin{equation*}
    (x + y)^{n+m}
    =
    \sum_{k=0}^{n+m} \binom n k x^k y^{n+m-k}
  \end{equation*}

  In this sum, either \( k \geq n \) and thus the product \( x^k y^{n+m-k} \in I \), or \( k < n \), in which case \( n + m - k > n + m - n = m \) and the same holds. Thus
  \begin{equation*}
    (x + y)^{n+m} \in I
  \end{equation*}
  and \( x + y \in \sqrt I \).

  The radical is also closed under multiplication with \( R \) since if \( x^n \in I \), then \( (rx)^n = r^n x^n \in I \).

  Therefore it is an ideal of \( R \).
\end{proof}

\begin{definition}\label{def:primary_ring_ideal}\cite[74]{Коцев2016}
  We call the proper ideal \( P \) of \( R \) \Def{primary} if \( xy \in P \) implies that \( x \in P \) or \( y \in \sqrt P \).
\end{definition}

\begin{theorem}[Chinese remainder theorem]\label{thm:chinese_remained_theorem}\cite[theorem 8.27]{Knapp2016BAlg}
  Let \( I_1, \ldots, I_n \) be pairwise coprime\Tinyref{def:coprime_ring_ideals} ideals. Then
  \begin{equation*}
    R / \bigcap_{i=1}^n I_n \cong R / I_1 \times \cdots \times R / I_n.
  \end{equation*}
\end{theorem}
