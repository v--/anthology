\subsection{Initial and final topologies}\label{subsec:initial_final_topologies}

\begin{definition}\label{def:category_of_topological_spaces}
  We denote by \( \Cat{Top} \) the subcategory\Tinyref{def:subcategory} of \( \Bold{Set} \)\Tinyref{def:category_of_sets} where
  \begin{itemize}
    \item the class\Tinyref{def:set_zfc} of objects is the class of all topological spaces\Tinyref{def:topological_space}.
    \item the morphisms between two topological spaces are the continuous functions\Tinyref{def:continuous_function} between them.
  \end{itemize}
\end{definition}

\begin{theorem}\label{thm:top_complete_cocomplete}
  The category \( \Cat{Top} \) of is both complete \Tinyref{def:categorical_limit} and cocomplete \Tinyref{def:categorical_colimit}.
\end{theorem}

\begin{definition}\label{def:initial_topology}\cite{nLab:top}
  Let \( \{ (X_i, \Cal{T}_i) \}_{i \in I} \) be a family\Tinyref{def:indexed_family} of topological spaces. Let \( X \) be a bare set and let
  \begin{equation*}
    \{ f_i: X \to X_i \}_{i \in I}
  \end{equation*}
  be a family of functions.

  The topology on \( X \) generated by the subbase
  \begin{equation*}
    \Cal{P} \coloneqq \{ f_i^{-1}(U) \colon i \in I, U \in \Cal{T}_i \}
  \end{equation*}
  is called the \Def{initial (or weak) topology on \( X \) generated by the family} \( \{ f_i \}_{i \in I} \).

  It is the weakest topology that makes all functions in the family \( \{ f_i \}_{i \in I} \) continuous.
\end{definition}

\begin{definition}\label{def:final_topology}\cite{nLab:top}
  Dually, if the family of functions is of the type
  \begin{equation*}
    \{ f_i: X_i \to X \}_{i \in I},
  \end{equation*}
  then we define the \Def{final (or strong) topology on \( X \) generated by the family} \( \{ f_i \}_{i \in I} \) as the topology
  \begin{equation*}
    \Cal{T} \coloneqq \{ U \subseteq X \colon \forall i \in I, f_i^{-1}(U) \in \Cal{T}_i \}.
  \end{equation*}

  It is the strongest topology that makes all functions in the family \( \{ f_i \}_{i \in I} \) continuous.
\end{definition}

\begin{proposition}\label{thm:initial_final_topology_limit}\cite{nLab:top}
  Let \( D: \Bold I \to \Cat{Top} \) be a small diagram\Tinyref{def:categorical_diagram}. For each space in the image \( D(\Bold I) \), denote the set corresponding by \( X_i \) and the corresponding topology by \( \Cal{T}_i \).

  The limit (resp. colimit) \( (X, \Cal{T}) \) of \( D \) can then be described as
  \begin{defenum}
    \item \( (X, \{ f_i \}_{i \in \Cat{I}}) = \varprojlim UD \) (resp. \( \varinjlim UD \)) is the limit (resp. colimit) in \( \Bold{Set} \) of \( U \circ D \), where \( U: \Bold{Top} \to \Bold{Set} \) is the forgetful functor.
    \item \( \Cal{T} \) is the initial\Tinyref{def:initial_topology} (resp. final\Tinyref{def:final_topology}) topology on \( X \) generated by the family of functions \( \{ f_i \}_{i \in \Cat{I}} \).
  \end{defenum}

  In particular, the functor \( U \) lifts limits and colimits\Tinyref{def:categorical_limit_preservation/lift}.
\end{proposition}

\begin{definition}\label{def:topological_subspace}
  Let \( (X, \Cal{T}) \) be a topological space and let \( M \subseteq X \) be a subset of \( X \). The \Def{topological subspace} \( (M, \Cal{T}_M) \) is obtained by endowing \( M \) with the topology
  \begin{equation*}
    \Cal{T}_M \coloneqq \{ U \cap M \colon U \in \Cal{T} \}.
  \end{equation*}

  It is the initial topology generated by the canonical embedding \( \iota: M \to X \).
\end{definition}

\begin{definition}\label{def:topological_product}
  The \Def{topological product} or \Def{Tychonoff product} \( (\prod_{i \in I} X_i, \prod_{i \in I} \Cal{T}_i) \) of the family \( { (X_i, \Cal{T}_i) }_{i \in I} \) is simply the categorical product in the category \( \Cat{Top} \)\Tinyref{def:categorical_product}. The underlying set \( \prod_{i \in I} X_i \) is the Cartesian product\Tinyref{def:cartesian_product}\Tinyref{thm:set_categorical_limits/product} and the topology \( \prod_{i \in I} \Cal{T}_i \) is called the \Def{product topology}.

  Let \( { (X_i, \Cal{T}_i) }_{i \in I} \) and \( { (Y_i, \Cal{O}_i) }_{i \in I} \) be two families of topological spaces and let \( \{ f_i: X_i \to Y_i \}_{i \in I} \) be a family of arbitrary functions between them.

  We define the \Def{product \( \prod_{i \in I} f_i \) of \( \{ f_i \}_{i \in I} \)} as the function
  \begin{align*}
    &\left(\prod_{i \in I} f_i \right): \prod_{i \in I} X_i \to \prod_{i \in I} Y_i \\
    &\left(\prod_{i \in I} f_i \right)(\{ x_i \}_{i \in I}) \coloneqq \{ f_i (x_i) \}_{i \in I}.
  \end{align*}

  If all of the spaces \( (X_i, \Cal{T}_i) \) are equal to some space \( (X, \Cal{T}) \), we call the product of \( \{ f_i \}_{i \in I} \) the \Def{diagonal product} and denote it by
  \begin{equation*}
    \Delta_{i \in I} f_i: X \to \prod_{i \in I} Y_i.
  \end{equation*}
\end{definition}

\begin{definition}\label{def:topological_quotient}\cite[90]{Engelking1989}
  Let \( (X, \Cal{T}) \) be a topological space and let \( \cong \) be an equivalence relation\Tinyref{def:order/equivalence} on \( X \). The \Def{quotient space \( (X, \Cal{T}) / \sim \)} is obtained by endowing the quotient set \( X / \cong \) with the final topology given by the canonical projection map \( x \mapsto [x] \).
\end{definition}

\begin{definition}\label{def:topological_sum}\cite[74]{Engelking1989}
  The \Def{topological direct sum} \( (\oplus_{i \in I} X_i, \oplus_{i \in I} \Cal{T}_i) \) of the family \( { (X_i, \Cal{T}_i) }_{i \in I} \) is simply the categorical coproduct in the category \( \Cat{Top} \)\Tinyref{def:categorical_coproduct}. The underlying set \( \oplus_{i \in I} X_i \) is the disjoint union\Tinyref{def:disjoint_union}\Tinyref{thm:set_categorical_limits/coproduct} and the topology \( \oplus_{i \in I} \Cal{T}_i \) is called the \Def{direct sum topology}.

  Let \( { (X_i, \Cal{T}_i) }_{i \in I} \) and \( { (Y_i, \Cal{O}_i) }_{i \in I} \) be two families of topological spaces and let \( \{ f_i: X_i \to Y_i \}_{i \in I} \) be a family of arbitrary functions between them. Let \( \iota_{X_i}: X_i \to \oplus_{i \in I} X_i \) and \( \iota_{Y_i}: Y_i \to \oplus_{i \in I} Y_i \) be the corresponding canonical injections.

  We define the \Def{direct sum \( \oplus_{i \in I} f_i \) of \( \{ f_i \}_{i \in I} \)} as the function
  \begin{align*}
    &(\oplus_{i \in I} f_i): \oplus_{i \in I} X_i \to \oplus_{i \in I} Y_i \\
    &(\oplus_{i \in I} f_i){\restriction}_{X_i} \coloneqq \iota_{Y_i} \circ f_i.
  \end{align*}

  Obviously \( \oplus_{i \in I} f_i \) is continuous whenever all \( f_i \) are continuous.

  If all of the spaces \( (Y_i, \Cal{O}_i) \) are equal to some space \( (Y, \Cal{O}) \), we call the direct sum of \( \{ f_i \}_{i \in I} \) simply a \Def{sum} and denote it by
  \begin{equation*}
    \sum_{i \in I} f_i: \oplus_{i \in I} X_i \to Y.
  \end{equation*}
\end{definition}
