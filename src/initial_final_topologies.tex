\subsection{Initial and final topologies}\label{subsec:initial_final_topologies}

\begin{definition}\label{def:category_of_topological_spaces}
  Topological spaces with continuous \hyperref[def:global_continuity]{functions}form a subcategory of \( \Cat{Set} \) as described in \fullref{def:first_order_model_category}. We denote this category by \( \Cat{Top} \).
\end{definition}

\begin{theorem}\label{thm:top_complete_cocomplete}
  The category \( \Cat{Top} \) of is both \hyperref[def:categorical_limit]{complete} and \hyperref[def:categorical_colimit]{cocomplete}.
\end{theorem}

\begin{definition}\label{def:initial_topology}\cite{nLab:top}
  Let \( \{ (X_\al, \CT_\al) \}_{\al \in \CA} \) be a \hyperref[def:indexed_family]{family} of topological spaces. Let \( X \) be a bare set and let
  \begin{equation*}
    \{ f_\al: X \to X_\al \}_{\al \in \CA}
  \end{equation*}
  be a family of functions.

  The topology on \( X \) generated by the subbase
  \begin{equation*}
    \Cal{P} \coloneqq \{ f_\al^{-1}(U) \colon \al \in \CA, U \in \CT_\al \}
  \end{equation*}
  is called the \Def{initial} (or \Def{weak}) topology on \( X \) generated by the family \( \{ f_\al \}_{\al \in \CA} \).

  It is the weakest topology that makes all functions in the family \( \{ f_\al \}_{\al \in \CA} \) continuous.
\end{definition}

\begin{definition}\label{def:final_topology}\cite{nLab:top}
  Dually, if the family of functions is of the type
  \begin{equation*}
    \{ f_\al: X_\al \to X \}_{\al \in \CA},
  \end{equation*}
  then we define the \Def{final} (or \Def{strong}) topology on \( X \) generated by the family \( \{ f_\al \}_{\al \in \CA} \) as the topology
  \begin{equation*}
    \CT \coloneqq \{ U \subseteq X \colon \forall \al \in \CA, f_\al^{-1}(U) \in \CT_\al \}.
  \end{equation*}

  It is the strongest topology that makes all functions in the family \( \{ f_\al \}_{\al \in \CA} \) continuous.
\end{definition}

\begin{proposition}\label{thm:initial_final_topology_limit}\cite{nLab:top}
  Let \( D: \Bold I \to \Cat{Top} \) be a small \hyperref[def:categorical_diagram]{diagram}. For each space in the image \( D(\Bold I) \), denote the set corresponding by \( X_\al \) and the corresponding topology by \( \CT_\al \).

  The limit (resp. colimit) \( (X, \CT) \) of \( D \) can then be described as
  \begin{defenum}
    \item \( (X, \{ f_\al \}_{\al \in \Cat{I}}) = \varprojlim UD \) (resp. \( \varinjlim UD \)) is the limit (resp. colimit) in \( \Cat{Set} \) of \( U \circ D \), where \( U: \Bold{Top} \to \Cat{Set} \) is the forgetful functor.
    \item \( \CT \) is the \hyperref[def:initial_topology]{initial} (resp. \hyperref[def:final_topology]{final}) topology on \( X \) generated by the family of functions \( \{ f_\al \}_{\al \in \Cat{I}} \).
  \end{defenum}

  In particular, the functor \( U \) lifts limits and \hyperref[def:categorical_limit_preservation/lift]{colimits}.
\end{proposition}

\begin{definition}\label{def:topological_subspace}
  Let \( (X, \CT) \) be a topological space and let \( M \subseteq X \) be a subset of \( X \). The \Def{topological subspace} \( (M, \CT_M) \) is obtained by endowing \( M \) with the topology
  \begin{equation*}
    \CT_M \coloneqq \{ U \cap M \colon U \in \CT \}.
  \end{equation*}

  The topology \( \CT_M \) is called the \Def{subspace topology} or \Def{induced topology}.

  It is the initial topology generated by the canonical embedding \( \iota: M \to X \).
\end{definition}

\begin{definition}\label{def:topological_product}
  The \Def{topological product} or \Def{Tychonoff product} 
  \begin{equation*}
    \left( \prod_{\al \in \CA} X_\al, \prod_{\al \in \CA} \CT_\al \right)
  \end{equation*}
  of the family \( { (X_\al, \CT_\al) }_{\al \in \CA} \) is simply the categorical product in the category \( \Cat{Top} \)\Tinyref{def:categorical_product}. The underlying set \( \prod_{\al \in \CA} X_\al \) is the \hyperref[def:cartesian_product]{Cartesian product}\Tinyref{thm:set_categorical_limits/product} and the topology \( \prod_{\al \in \CA} \CT_\al \) is called the \Def{product topology}.

  Let \( { (X_\al, \CT_\al) }_{\al \in \CA} \) and \( { (Y_\al, \Cal{O}_\al) }_{\al \in \CA} \) be two families of topological spaces and let
  \begin{equation*}
    \{ f_\al: X_\al \to Y_\al \}_{\al \in \CA}
  \end{equation*}
  be a family of arbitrary functions between them.

  We define the \Def{product \( \prod_{\al \in \CA} f_\al \) of \( \{ f_\al \}_{\al \in \CA} \)} as the function
  \begin{align*}
    &\left(\prod_{\al \in \CA} f_\al \right): \prod_{\al \in \CA} X_\al \to \prod_{\al \in \CA} Y_\al \\
    &\left(\prod_{\al \in \CA} f_\al \right)(\{ x_\al \}_{\al \in \CA}) \coloneqq \{ f_\al (x_\al) \}_{\al \in \CA}.
  \end{align*}

  If all of the spaces \( (X_\al, \CT_\al) \) are equal to some space \( (X, \CT) \), we call the product of \( \{ f_\al \}_{\al \in \CA} \) the \Def{diagonal product} and denote it by
  \begin{equation*}
    \Delta_{\al \in \CA} f_\al: X \to \prod_{\al \in \CA} Y_\al.
  \end{equation*}
\end{definition}

\begin{definition}\label{def:topological_quotient}\cite[90]{Engelking1989}
  Let \( X \) be a topological space and let \( \cong \) be an equivalence \hyperref[def:equivalence_relation]{relation} on \( X \). The \Def{quotient space} \( (X, \CT) / \sim \) is obtained by endowing the quotient set \( X / \cong \) with the final \hyperref[def:final_topology]{topology} given by the canonical projection map \( x \mapsto [x] \).
\end{definition}

\begin{definition}\label{def:topological_sum}\cite[74]{Engelking1989}
  The \Def{topological direct sum}
  \begin{equation*}
    (\oplus_{\al \in \CA} X_\al, \oplus_{\al \in \CA} \CT_\al)
  \end{equation*}
  of the family \( { (X_\al, \CT_\al) }_{\al \in \CA} \) is simply the categorical coproduct in the category \( \Cat{Top} \)\Tinyref{def:categorical_coproduct}. The underlying set \( \oplus_{\al \in \CA} X_\al \) is the disjoint \hyperref[def:disjoint_union]{union}\Tinyref{thm:set_categorical_limits/coproduct} and the topology \( \oplus_{\al \in \CA} \CT_\al \) is called the \Def{direct sum topology}.

  Let \( { (X_\al, \CT_\al) }_{\al \in \CA} \) and \( { (Y_\al, \Cal{O}_\al) }_{\al \in \CA} \) be two families of topological spaces and let 
  \begin{equation*}
    \{ f_\al: X_\al \to Y_\al \}_{\al \in \CA}
  \end{equation*}
  be a family of arbitrary functions between them. Let \( \iota_{X_\al}: X_\al \to \oplus_{\al \in \CA} X_\al \) and \( \iota_{Y_\al}: Y_\al \to \oplus_{\al \in \CA} Y_\al \) be the corresponding canonical embeddings.

  We define the \Def{direct sum \( \oplus_{\al \in \CA} f_\al \) of \( \{ f_\al \}_{\al \in \CA} \)} as the function
  \begin{align*}
    &(\oplus_{\al \in \CA} f_\al): \oplus_{\al \in \CA} X_\al \to \oplus_{\al \in \CA} Y_\al \\
    &(\oplus_{\al \in \CA} f_\al){\restriction}_{X_\al} \coloneqq \iota_{Y_\al} \circ f_\al.
  \end{align*}

  Obviously \( \oplus_{\al \in \CA} f_\al \) is continuous whenever all \( f_\al \) are continuous.

  If all of the spaces \( (Y_\al, \Cal{O}_\al) \) are equal to some space \( (Y, \Cal{O}) \), we call the direct sum of \( \{ f_\al \}_{\al \in \CA} \) simply a \Def{sum} and denote it by
  \begin{equation*}
    \sum_{\al \in \CA} f_\al: \oplus_{\al \in \CA} X_\al \to Y.
  \end{equation*}
\end{definition}
