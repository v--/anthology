\begin{theorem}
  The category $\Bold{Top}$ of is both complete \Tinyref{def:categorical_limit} and cocomplete \Tinyref{def:categorical_colimit}.
\end{theorem}

\begin{definition}\label{def:initial_topology}\cite{nLab:top}
  Let $\{ (X_i, \Cal{T}_i) \}_{i \in I}$ be a family\Tinyref{def:indexed_family} of topological spaces. Let $X$ be a bare set and let
  \begin{align*}
    \{ f_i: X \to X_i \}_{i \in I}
  \end{align*}
  be a family of functions.

  The topology on $X$ generated by the subbase
  \begin{align*}
    \Cal{P} \coloneqq \{ f_i^{-1}(U) \colon i \in I, U \in \Cal{T}_i \}
  \end{align*}
  is called the \uline{initial (or weak) topology on $X$ generated by the family} $\{ f_i \}_{i \in I}$.

  It is the weakest topology that makes all functions in the family $\{ f_i \}_{i \in I}$ continuous.
\end{definition}

\begin{definition}\label{def:final_topology}\cite{nLab:top}
  Dually, if the family of functions is of the type
  \begin{align*}
    \{ f_i: X_i \to X \}_{i \in I},
  \end{align*}
  then we define the \uline{final (or strong) topology on $X$ generated by the family} $\{ f_i \}_{i \in I}$ as the topology
  \begin{align*}
    \Cal{T} \coloneqq \{ U \subseteq X \colon \forall i \in I, f_i^{-1}(U) \in \Cal{T}_i \}.
  \end{align*}

  It is the strongest topology that makes all functions in the family $\{ f_i \}_{i \in I}$ continuous.
\end{definition}

\begin{proposition}\label{thm:initial_final_topology_limit}\cite{nLab:top}
  Let $D: \Bold I \to \Bold{Top}$ be a small diagram\Tinyref{def:categorical_diagram}. For each space in the image $D(\Bold I)$, denote the set corresponding by $X_i$ and the corresponding topology by $\Cal{T}_i$.

  The limit (resp. colimit) $(X, \Cal{T})$ of $D$ can then be described as
  \begin{defenum}
    \item $(X, \{ f_i \}_{i \in \Bold{I}}) = \varprojlim_{\Bold{I}} UD$ (resp. $\varinjlim_{\Bold{I}} UD$) is the limit (resp. colimit) in $\Bold{Set}$ of $U \circ D$, where $U: \Bold{Top} \to \Bold{Set}$ is the forgetful functor.
    \item $\Cal{T}$ is the initial\Tinyref{def:initial_topology} (resp. final\Tinyref{def:final_topology}) topology on $X$ generated by the family of functions $\{ f_i \}_{i \in \Bold{I}}$.
  \end{defenum}

  In particular, the functor $U$ lifts limits and colimits\Tinyref{def:categorical_limit_preservation/lift}.
\end{proposition}

\begin{example}\label{ex:initial_final_topology}\cite{nLab:top}
  \begin{defenum}
    \item Products\Tinyref{def:categorical_product}\Tinyref{ex:categorical_product/set} in $\Bold{Top}$ are endowed with the the \uline{product topology}.
    \item Equalizers\Tinyref{def:categorical_equalizer}\Tinyref{ex:categorical_product/set} in $\Bold{Top}$ are \uline{subspaces} endowed with the the \uline{subspace topology}.
    \item Coequalizers\Tinyref{def:categorical_coequalizer}\Tinyref{ex:categorical_coequalizer/set} by projection maps of equivalence relations are called \uline{quotient spaces} and are endowed with the the \uline{quotient topology}.
  \end{defenum}
\end{example}
