\section{Nonsmooth analysis}\label{sec:nonsmooth_analysis}

\begin{remark}\label{remark:nonsmooth_analysis}
  Nonsmooth analysis studies generalized differentiability for functions which are not necessarily differentible. The generalized derivatives (see \fullref{subsec:nonsmooth_derivatives}) are not linear, which motivates the study of subdifferentials (see \fullref{subsec:subdifferentials}).

  Both optimization in Euclidean spaces and infinite-dimensional optimization studies (not necessarily linear) real-valued functionals. Hence we are only concerned with studying real-valued topological vector spaces.
\end{remark}

\subsection{Nonsmooth derivatives}\label{subsec:nonsmooth_derivatives}

\begin{remark}\label{remark:nonsmooth_differentiability}
  Unlike in \fullref{def:differentiability}, we do not introduce terminology for differentiability because actual differentiability refers to linear approximations of \( f: U \to Y \) with some consistency properties. We will say that \enquote{\( f \) has a Clarke derivative at \( x_0 \) in the direction \( h \)} rather than \enquote{\( f \) is Clarke differentiable at \( x_0 \) in the direction \( h \)}.
\end{remark}

\begin{definition}\label{def:nonsmooth_derivatives}
  Let \( X \) be a real Hausdorff topological vector spaces\Tinyref{def:topological_vector_space} and let \( U \subseteq X \) be an open set.

  We fix a point \( x_0 \in U \) and a direction \( h \in X \).

  \begin{defenum}
    \DItem{def:nonsmooth_derivatives/directional}\Fullref{def:differentiability} already introduced directional derivatives. Here we introduce a special notation for them:
    \begin{equation*}
      D_h^+ f(x_0) = f_+'(x_0)(h) \coloneqq \lim_{t \downarrow 0} \frac {f(x_0 + th) - f(x_0)} t.
    \end{equation*}

    \DItem{def:nonsmooth_derivatives/dini}\cite[definition 11.18]{Clarke2013} The upper (resp. lower) \Def{Dini derivative} is defined as
    \begin{align*}
      \Ol{D}_h f(x_0) = \Ol{f'}(x_0)(h) &\coloneqq \limsup_{t \downarrow 0} \frac {f(x_0 + th) - f(x_0)} t
      \\
      \Ul{D}_h f(x_0) = \Ul{f'}(x_0)(h) &\coloneqq \liminf_{t \downarrow 0} \frac {f(x_0 + th) - f(x_0)} t
    \end{align*}

    Dini derivatives are useful when the difference quotients are bounded but do not have a limit.

    \DItem{def:nonsmooth_derivatives/clarke}\cite[section 10.1]{Clarke2013} The \Def{generalized Clarke derivative} is defined as
    \begin{equation*}
      D_h^\circ f(x_0)
      =
      f^\circ(x_0)(h)
      \coloneqq
      \limsup_{\substack{y \to x_0 \\ t \downarrow 0}} \frac {f(y + th) - f(y)} t.
    \end{equation*}

    Refer to \fullref{subsec:clarke_gradients} for their usefulness.
  \end{defenum}
\end{definition}
