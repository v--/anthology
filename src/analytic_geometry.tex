\section{Geometry}\label{sec:geometry}

\begin{remark}\label{def:coordinates_in_geometry}
  Geometry is the multi-millennium evolution of attempts to measure parts of the earth. Ironically, it may be the main historical justification for the gradual axiomatization of mathematics. Completely abstract results about shapes date at least as early as in Ancient Greece. The important distinction between ancient geometry and modern geometry is the introduction of coordinates in the 17th century.

  An axiomatic approach for a theory of plane and solid figures was developed by Euclid in the third century BC. Later, Hilbert, Tarski and others independently proposed axioms systems that fit the requirements of modern logic systems. This is known today as \Def{synthetic Euclidean geometry} and is mostly of theoretical interest.

  Descartes' idea of coordinates connects problems of algebra and geometry in such a way that most of today's mathematics seamlessly switches between algebraic and geometric interpretations of the same problem. The study of classical Greek geometry in terms of coordinates is known as \Def{analytic geometry}\Tinyref{subsec:analytic_geometry}.
\end{remark}

\subsection{Analytic geometry}\label{subsec:analytic_geometry}

We need the concept of an affine plane\Tinyref{def:affine_plane} in order to define affine coordinate systems\Tinyref{def:affine_plane_coordinate_system} and justify the connection\Tinyref{def:euclidean_line_plane_and_space} between the Euclidean plane and \( \R^2 \).

\begin{definition}\label{def:affine_plane}\cite[1]{Hartshorne1967}
  An \Def{affine plane} is
  \begin{itemize}
    \item a set \( A^2 \), whose elements are called \Def{points},
    \item a family \( \Cal{L} \) of subsets of \( A^2 \), whose members are called \Def{lines},
    \item a \Def{parallel} relation \( l \parallel l' \) for lines that holds if either \( l = l' \) or if they have no points in common,
    \item a \Def{betweenness} relation for points that says if the point \( R \) is \Def{between} \( P \) and \( Q \).
    \item a \Def{collinearity} relation for a set \( B \) of points that holds if \( B \) is a subset of some line,
  \end{itemize}
  such that
  \begin{defenum}
    \DItem{def:affine_plane/A1}[A1] Given two distinct points, there exists only one line that contains both.
    \DItem{def:affine_plane/A2}[A2] Given a line \( l \) and a point \( P \not\in l \), there exists exactly one line \( l' \parallel l \) that contains \( P \).
    \DItem{def:affine_plane/A3}[A3] There exist three non-collinear points.
  \end{defenum}

  Some auxiliary definitions are
  \begin{defenum}
    \DItem{def:affine_plane/locus} A subset of \( A^2 \) is called a \Def{locus} in \( A^2 \).

    \DItem{def:affine_plane/half_plane} Every line \( l \) gives rise to two \Def{half-planes} \( B \) and \( C \) as follows:
    \begin{itemize}
      \item \( B \cup C = A^2 \setminus l \)
      \item \( B \) and \( C \) are disjoint
      \item If \( P \in B \) and \( Q \in C \), then there is a point \( R \in l \) between \( P \) and \( Q \)
    \end{itemize}

    \DItem{def:affine_plane/bound_vector} An ordered pair \( \Vec{(P, Q)} \) of points is called a \Def{bound vector} in \( A^2 \). It is called a \Def{zero vector} if \( P = Q \).

    \DItem{def:affine_plane/bound_vector_length} For the purpose of not being stuck in unnecessary formalisms, we assign a real number \( \Len(\Vec{(P, Q)}) \) to each vector and call it its \Def{length}.
  \end{defenum}
\end{definition}

\begin{definition}\label{def:affine_plane_coordinate_system}
\end{definition}

\begin{definition}\label{def:euclidean_line_plane_and_space}
  We list some informal definitions in order to justify the rest of the subsection, which is completely formal.

  The \Def{Euclidean plane} is a formalization of a straight infinite surface. \Cref{def:euclidean_plane_and_space/plane_figure} contains some highlighted lines that our mind maps to abstract geometric figures, without considering the size limitations of the page or the thickness of the lines.

  We denote the Euclidean plane by \( A^2 \), where \( A \) comes from \Def{affine}. This this context, \enquote{geometric figure} is synonymous with \enquote{locus} and the latter is defined as a subset of a set\Tinyref{def:set_zfc} denoted as \( A^2 \).

  \begin{figure}\label{def:euclidean_plane_and_space/plane_figure}
    \centering
    \begin{mplibcode}
      u := 1cm;

      beginfig(1);
        draw (0, -1) * u -- (3, 0) * u;
        draw (-1, 2) * u -- (3, 1) * u -- (1, 3) * u -- cycle;
        draw fullcircle scaled 1.5u shifted ((0, 0.5) * u);
      endfig;
    \end{mplibcode}
    \caption{A triangle, a circle and a line in the Euclidean plane.}
  \end{figure}

  Similarly, the \Def{Euclidean space} \( A^3 \) is an abstraction of the surrounding three-dimensional world.

  It remains to define \( A \) as the \Def{Euclidean line} and conceptually map it to an infinitely long and infinitely dense straight line.

  In practice, we put \( A \coloneqq \R \), the set of real numbers\Tinyref{def:real_numbers}. This justifies the names \Def{real line} for \( \R \), \Def{plane} for \( \R^2 \). This allows us to interpret \( \R \) and \( \R^2 \) visually as an infinitely long line and an infinite plane, respectively. Note that the identification of \( A^2 \) with \( \R^2 \) is actually more subtle - see \cref{def:euclidean_plane_coordinate_system}

  A more specific case is the \enquote{Euclidean space}. Instead of identifying it with \( \R^3 \), it is conventional to refer to finite-dimensional inner product spaces\Tinyref{def:inner_product_space} as \enquote{Euclidean spaces} \( (\R^n, \Prod \cdot \cdot) \), regardless of dimension.

  Note that \( \R^2 \) is a very concrete plane - see \cref{def:left_module_of_tuples} - that is given by pairs of coordinates.
\end{definition}
