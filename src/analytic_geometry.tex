\subsection{Analytic geometry}\label{subsec:analytic_geometry}

We will work in the plane \( \R^2 \), regarding it as the Euclidean plane\Tinyref{def:euclidean_plane} with a fixed coordinate system\Tinyref{def:euclidean_plane_coordinate_system} \( Oxy \).

\begin{remark}\label{remark:hyperbolic_trigonometric_functions}
  The similarity in the definition of trigonometric functions\Tinyref{def:trigonometric_functions} and hyperbolic trigonometric functions\Tinyref{def:hyperbolic_trigonometric_functions} also has geometric significance. Just like a circle can be described as the curve \( t \mapsto (\cos t, \sin t) \), a hyperbola can be described as \( t \mapsto (\cosh t, \sinh t) \).
\end{remark}
