\subsection{Fields}\label{subsec:fields}

\begin{definition}\label{def:field}
  As mentioned in \cref{def:semiring/field}, fields are commutative division rings.
\end{definition}

\begin{definition}\label{def:field_extension}
  If \( F \) and \( G \) are fields and \( F \) is a unital subring\Tinyref{def:first_order_structure/substructure} of \( G \), we say that \( F \) is a \Def{subfield} of \( G \) and that \( G \) is a \Def{field extension} of \( F \).

  Field extension are also denoted as \( G / F \) to highlight the roles of \( G \) and \( F \). This is not a quotient ring but simply a notation. See \cref{def:galois_group}.
\end{definition}

\begin{definition}\label{def:galois_group}\cite[124]{Knapp2016BAlg}
  Let \( F \) be a field extension\Tinyref{def:field_extension} of \( G \). The group \( \Gal(F / G) \) of automorphisms of \( F \) that leave \( G \) fixed is called the \Def{Galois group} of the field extension \( F / G \).
\end{definition}

\begin{definition}\label{def:field_characteristic}
  We define the \Def{characteristic} of a field \( F \) as
  \begin{itemize}
    \item the smallest number of times \( 1 \) must be added to itself in order to obtain \( 0 \).
    \item zero if \( 0 \) cannot be obtained in this way.
  \end{itemize}
\end{definition}

\begin{theorem}\label{thm:ring_of_integers_module_prime_is_field}
  The ring \( \Z_n \)\Tinyref{def:ring_of_integers_modulo} of integers modulo \( n \) is a field if \( n \) is a prime number\Tinyref{def:prime_number}.
\end{theorem}
\begin{proof}
  We only need to show that \( \Z_n \) has a multiplicative inverse for any nonzero element.

  Fix \( x \in \Z_n \). If \( y \) is a multiplicative inverse of \( x \), we should have
  \begin{equation*}
    xy \equiv 1 \pmod n,
  \end{equation*}
  which is the same as
  \begin{equation*}
    n \mid (xy - 1).
  \end{equation*}

  \Cref{thm:bezout_identity} gives us integers \( a, b \in \Z \) such that
  \begin{equation*}
    ax + bn = \gcd(x, n) = 1,
  \end{equation*}
  which is the same as
  \begin{equation*}
    -bn = xa - 1.
  \end{equation*}

  Define \( y \coloneqq \Rem(a, n) \). This is the multiplicative inverse of \( x \).
\end{proof}
