\section{Functional analysis}\label{sec:functional_analysis}

In this section, \( \BK \) will refer to either \( \BR \) or \( \BC \). See \fullref{remark:real_field_extensions} for a justification.

\subsection{Topological groups}\label{subsec:topological_groups}

\begin{definition}[Topological group]\label{def:topological_group}
  Let \( G \) be any group\Tinyref{def:magma/group} and let \( \CT \) be a topology on \( G \). The tuple \( (G, \cdot, \CT) \) is called a \Def{topological group} if the group structure and topological structure agree, that is, the operations \( \cdot: X \times X \to X \) and \( (-)^{-1}: X \to X \) are continuous with respect to \( \CT \).

  See \fullref{remark:hausdorff_topological_groups} and \fullref{def:category_of_topological_groups} for more nuances.
\end{definition}

\begin{remark}\label{remark:hausdorff_topological_groups}
  It is conventional to require the topology in a topological group to be \( T_1 \)\Tinyref{def:separation_axioms}. We will not do this due to our goal of not assuming more than is necessary.

  Due to \fullref{thm:topological_group_t0_iff_t3.5}, it is immaterial whether we require the topology to be \( T_0 \) or \( T_{3.5} \) or anywhere in between. It is customary to call the space \enquote{Hausdorff} (although stronger separation axioms actually hold) and require \( T_1 \) to hold (since it is simple to state).

  We will explicitly mention when we want a topological group to be Hausdorff. This is usually so when we speak of convergence.
\end{remark}

\begin{definition}\label{def:category_of_topological_groups}
  The category \( \Cat{TopGrp} \) of topological groups is a subcategory of both \( \Cat{Top} \) and \( \Cat{Grp} \). Its morphisms are the continuous\Tinyref{def:global_continuity} group homomorphisms\Tinyref{thm:group_homomorphism_single_condition}.
\end{definition}

\begin{proposition}\label{thm:neighborhood_translations_in_topological_groups}
  Fix \( x, y \in G \) in a topological group \( G \). If \( U \) is a neighborhood of \( x \), then both \( V = yx^{-1} U \) and \( W = U x^{-1}y \) are neighborhoods of \( y \).
\end{proposition}
\begin{proof}
  Since the group operations are continuous, for fixed \( x \) and \( y \), the function \( f(z) \coloneqq xy^{-1}z \) is continuous.

  Note that \( U = f(V) \), hence \( V \) is the preimage of \( U \) under \( f \) and it follows from the continuity of \( f \) that \( V \) is open.

  Since \( x \in U \), \( yx^{-1}x = ye = y \in V \). Therefore \( V \) is a neighborhood of \( y \).

  The proof that \( W \) is a neighborhood of \( y \) is analogous.
\end{proof}

\begin{corollary}\label{thm:origin_neighborhoods_in_topological_groups}
  In a topological group \( G \), every neighborhood is a translation of e neighborhood of the origin \( e \).
\end{corollary}

\begin{remark}\label{remark:origin_neighborhoods_in_topological_groups}
  \Fullref{thm:origin_neighborhoods_in_topological_groups} provides a lot of uniformity by allowing us to only consider neighborhoods of zero when working with topological groups.
\end{remark}

\begin{proposition}\label{thm:topological_group_t0_iff_t3.5}
  If a topological group is \( T_0 \), it is automatically \( T_{3.5} \).
\end{proposition}

\begin{proposition}\label{thm:topological_group_uniform_space}
  A Hausdorff topological group \( G \) can be made into a uniform space by the families of entourages
  \begin{align*}
    &V^l_A \coloneqq \{ (x, y) \in G \times G \colon x^{-1} y \in A \}, \\
    &V^r_A \coloneqq \{ (x, y) \in G \times G \colon x y^{-1} \in A \},
  \end{align*}
  where \( A \) is a symmetric\Tinyref{def:neighborhood_set_types/symmetric} neighborhood of the origin \( e \).

  If \( G \) is abelian, the two families of entourages coincide.
\end{proposition}

\begin{proposition}\label{thm:limits_are_topological_group_homomorphisms}
  If \( \{ a_\alpha \}_{\alpha \in \CA} \) and \( \{ b_\alpha \}_{\alpha \in \CA} \) are nets\Tinyref{def:topological_net} in a Hausdorff topological group \( X \) that converge to \( a \) and \( b \), correspondingly, then \( a_\alpha b_\alpha \to a b \).
\end{proposition}
\begin{proof}
  Special case of \fullref{thm:linearity_of_sequence_limits}.
\end{proof}
