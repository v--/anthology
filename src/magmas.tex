\subsection{Magmas}\label{subsec:magmas}

\begin{definition}\label{def:magma}
  A \Def{magma} is a set \( \CM \) equipped with a \hyperref[def:function/arity]{binary function} \( \cdot: \CM \times \CM \to \CM \), called the \Def{magma operation}. Unless specified otherwise, we denote this operation by juxtaposition as \( xy \) instead of \( x \cdot y \).

  \begin{DefEnum}
    \ILabel{def:magma/theory} In analogy to the \hyperref[def:pointed_set/theory]{theory of pointed sets}, we can define the \Def{theory of magmas} as an empty theory over a language with a single binary function.

    \ILabel{def:magma/homomorphism} A \hyperref[def:first_order_homomorphism]{homomorphism} between the magmas \( (\CM, \cdot_{\CM}) \) and \( (\CN, \cdot_{\CN}) \) is, explicitly, a function \( \varphi: \CM \to \CN \) such that
    \begin{equation}\label{eq:def:magma/homomorphism}
      \varphi(x \cdot_{\CM} y) = \varphi(x) \cdot_{\CN} \varphi(y) \Tforall x, y \in \CM.
    \end{equation}

    \ILabel{def:magma/category} We denote the \hyperref[def:first_order_model_category]{model category} for the theory of magmas by \( \Cat{Mag} \).

    \ILabel{def:magma/trivial} The \Def{trivial magma} is the empty set with an empty operation. It is the unique \hyperref[def:zero_objects/initial]{initial object} in \( \Cat{Mag} \).

    \ILabel{def:magma/opposite} The \Def{opposite magma} of \( (\CM, \cdot) \), also called the \Def{dual magma}, is the magma \( (\CM, \odot) \) with multiplication reversed:
    \begin{equation*}
      x \odot y \coloneqq y \cdot x.
    \end{equation*}

    We denote the opposite magma by \( \CM^{-1} \).

    \ILabel{def:magma/associative} We can add the following axiom to the theory:
    \begin{equation}\label{eq:def:magma/associative}
      (x \cdot y) \cdot z = x \cdot (y \cdot z) \Tforall x, y, z \in \CM.
    \end{equation}

    If \eqref{eq:def:magma/associative} is satisfied, we say that the operation \( \cdot \) and, by extension, the magma itself, are \Def{associative}. Associative magmas are usually called \Def{semigroups}. Associativity imposes no additional restrictions on the homomorphisms, hence semigroups are a \hyperref[def:subcategory]{full subcategory} of \( \Cat{Mag} \).

    \ILabel{def:magma/commutative} Another common axiom is \Def{commutativity}:
    \begin{equation}\label{eq:def:magma/commutative}
      x \cdot y = y \cdot x \Tforall x, y \in \CM.
    \end{equation}

    Commutative magmas also form a full subcategory. Obviously \( \CM = \CM^{-1} \) in a commutative magma.

    \ILabel{def:magma/cancellative} We say that the operation \( \cdot \) is \Def{left-cancellative} (resp. \Def{right-cancellative}) if, for all \( x, y \in \CM \), we have
    \begin{align}\label{eq:def:magma/cancellative}
      x = y \Twhenever z \cdot x = z \cdot y \Tforall z \in \CM
      &&
      (\text{resp. } x \cdot z = y \cdot z \Tforall z \in \CM).
    \end{align}

    The operation is \Def{cancellative} if it is both left and right cancellative. Cancellative magmas also form a full subcategory.
  \end{DefEnum}
\end{definition}

\begin{remark}\label{remark:magma_set_operations}
  It is customary to perform magma operations with sets. That is, if \( A \) and \( B \) are sets in the magma \( \CM \), it is customary to write
  \begin{BreakableAlign*}
    A \cdot B \coloneqq \{ a \cdot b \colon a \in A, b \in B \}.
  \end{BreakableAlign*}

  If we use the convention in \fullref{remark:singleton_sets}, this turns sets over a magma into a magma.
\end{remark}
