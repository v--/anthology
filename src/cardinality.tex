\section{Cardinality}\label{sec:cardinality}

\begin{definition}\label{def:set_domination}\cite[145]{Enderton1977}
  We say that the set $X$ is \ul{dominated by $Y$} and write $\Abs{X} \leq \Abs{Y}$ if there exists an injection\Tinyref{def:function_invertibility/injection} from $X$ to $Y$.
\end{definition}

\begin{definition}\label{def:equinumerous_sets}\cite[129]{Enderton1977}
  We say that the sets $X$ and $Y$ are \ul{equinumerous} and write $X \cong Y$ if there exists a bijection\Tinyref{def:function_invertibility/bijection} between $X$ and $Y$.
\end{definition}

\begin{theorem}[Cantor-Schröder-Bernstein]\label{thm:cantor_schroder_bernstein}\cite[147]{Enderton1977}
  If $\Abs{X} \leq \Abs{Y}$ and $\Abs{Y} \leq \Abs{X}$, then $X \cong Y$.
\end{theorem}

\begin{proposition}\label{thm:equinumerousity_equivalence}\cite[theorem 6A]{Enderton1977}
  Equinumerosity\Tinyref{def:equinumerous_sets} satisfies the equivalence relation axioms\Tinyref{def:order/equivalence} (however it is formally not an equivalence relation since we cannot formally define relations on the class of all sets; see \cref{def:set_zfc}).
\end{proposition}

\begin{theorem}[Cantor]\label{thm:cantor_power_set_theorem}\cite[theorem 6B]{Enderton1977}
  No set $X$ is equinumerous with its power set $\Power(X)$.
\end{theorem}
\begin{proof}
  Fix some function $f: X \to \Power(X)$. Define the set
  \begin{align*}
    Y \coloneqq \{ x \in X \colon x \not\in f(x) \}.
  \end{align*}

  Note that $Y \subseteq X$ and thus $Y \in \Power(X)$, however $Y$ is not in the image\Tinyref{def:function} $\Img f$ and thus $f$ is not a surjection\Tinyref{def:function_invertibility/surjection}.

  Since $f$ was arbitrary, we conclude that no function $f: X \to \Power(X)$ is a surjection and, hence, $X \not\cong \Power(X)$.
\end{proof}

\begin{definition}\label{def:finite_set}\cite[133]{Enderton1977}
  A set $A$ is \ul{finite} if it is isomorphic to a natural number as defined in \cref{def:natural_numbers_zfc} (using the convention that $\varnothing$ corresponds to zero).

  If $A$ is not finite, we say that it is \ul{infinite}.
\end{definition}

\begin{theorem}\label{def:pigeonhole_principle}[Pigeonhole Principle]\cite[Corollary 6C]{Enderton1977}
  No finite\Tinyref{def:finite_set} set is equinumerous to a proper subset\Tinyref{def:subset} of itself.
\end{theorem}

\begin{theorem}\label{thm:equinumerous_ordinal_existence}\cite[197]{Enderton1977}
  For every set, there exists at least one ordinal\Tinyref{def:ordinal} equinumerous to it.
\end{theorem}

\begin{definition}\label{def:cardinal}\cite[197]{Enderton1977}
  For each set $A$, define its \ul{cardinal} or \ul{cardinal number} $\Card A$ as the intersection of all ordinals that are equinumerous to $A$.

  If $\xi$ and $\eta$ are cardinal numbers, we define $\xi \leq \eta$ to mean that $\eta$ dominates\Tinyref{def:set_domination} $\xi$, i.e.
  \begin{align*}
    \xi \leq \eta \iff \Abs{\xi} \leq \Abs{\eta}.
  \end{align*}
\end{definition}

\begin{note}\label{note:cardinals}
  We can think of cardinal numbers as \enquote{choosing}\AOC a special set out of the equivalence classes obtained from \cref{thm:equinumerousity_equivalence}.

  Since the natural numbers as defined in \cref{def:natural_numbers_zfc} are ordinals and no two different natural numbers are equinumerous, we identify the cardinal numbers for finite sets\Tinyref{def:finite_set} with natural numbers.

  We give special names to
  \begin{itemize}
    \item $\aleph_0 \coloneqq \Card(\omega)$, the \ul{cardinality of the natural numbers}.
    \item $c \coloneqq \Card(\BB{R}) = \Card(\Power(\omega))$, the \ul{cardinality of the continuum}.
  \end{itemize}
\end{note}

\begin{proposition}\label{thm:cardinals_well_ordered}
  The class of all cardinals\Tinyref{def:cardinal} is well-ordered\Tinyref{def:poset/well_order} by the inclusion relation $\subseteq$, that is, every set of cardinals has a least element.
\end{proposition}
\begin{proof}
  Direct consequence of \cref{thm:ordinals_are_well_ordered} and \cref{def:cardinal}.
\end{proof}

\begin{hypothesis}\label{hyp:continuum_hypothesis}
  There exists no cardinal $\xi$ such that $\aleph_0 < \xi < c$.
\end{hypothesis}

\begin{definition}\label{def:cardinal_arithmetic}
  Let $\xi$ and $\eta$ be cardinal numbers. We define
  \begin{description}
    \DItem{addition}{def:cardinal_arithmetic/addition} $\xi + \eta \coloneqq \Card(\xi \coprod \eta)$, where $\coprod$ denotes disjoint unions\Tinyref{def:disjoint_union}.
    \DItem{multiplication}{def:cardinal_arithmetic/multiplication} $\xi \cdot \eta \coloneqq \Card(\xi \times \eta)$
    \DItem{exponentiation}{def:cardinal_arithmetic/exponentiation} $\xi^\eta \coloneqq \Card(\Bold{Set}(\eta, \xi))$\Tinyref{def:category_of_sets}
  \end{description}
\end{definition}
