\documentclass[numbers=endperiod, bibliography=totocnumbered]{scrartcl}

% Base packages
\usepackage[T2A]{fontenc}
\usepackage[utf8]{inputenc}
\usepackage[russian, english]{babel}
\usepackage[pdfencoding=unicode]{hyperref}
\usepackage[style=alphabetic, citestyle=alphabetic]{biblatex}
\usepackage{csquotes}

% Base math packages
\usepackage{amsmath}
\usepackage{amssymb}
\usepackage{amsthm}
\usepackage{mathtools}
\usepackage{cleveref}

% Misc packages
\usepackage[normalem]{ulem}
\usepackage{import}
\usepackage{tikz}
\usepackage{enumitem} % defenum
\usepackage{crossreftools} % Custom labels in references to lists
\usepackage{xspace} % TOC
\usepackage{tocbasic} % TOC

% Custom packages
\usepackage{packages/macros}
\usepackage{packages/theorem_styles}
\usepackage{packages/list_structures}
\usepackage{packages/toc}

% Tikz libraries
\usetikzlibrary{babel} % https://github.com/astoff/tikz-cd/issues/7
\usetikzlibrary{cd}

% Bibliography
\addbibresource{references.bib}

% Document
\title{Notebook}
\subtitle{\href{https://github.com/v--/notebook}{https://github.com/v--/notebook}}
\author{Ianis Vasilev, \Email{ianis@ivasilev.net}}
\date{}

\begin{document}
\sloppy

\maketitle

\begin{abstract}
  This document contains some of the mathematics notes and proofs that I write while studying. Having all these notes in one place is quite helpful for later reference.
\end{abstract}

\tableofcontents

\section{Analysis}\label{sec:analysis}
\subsection{Linear combinations}\label{sec:linear_combinations}
\import{src/}{linear_combinations.tex}
\subsection{Topological vector spaces}\label{sec:topological_vector_spaces}
\import{src/}{topological_vector_spaces.tex}
\subsection{Differentiability}\label{sec:differentiability}
\import{src/}{differentiability.tex}
\subsection{Subdifferentials}\label{sec:subdifferentials}
\import{src/}{subdifferentials.tex}
\subsection{Norms}\label{sec:norms}
\import{src/}{norms.tex}
\subsection{Dentable sets}\label{sec:dentable_sets}
\import{src/}{dentable_sets.tex}
\subsection{Asplund spaces}\label{sec:asplund_spaces}
\import{src/}{asplund_spaces.tex}
\subsection{Convex functions}\label{sec:convex_functions}
\import{src/}{convex_functions.tex}
\subsection{Clarke generalized gradients}\label{sec:clarke_gradients}
\import{src/}{clarke_gradients.tex}

\section{General topology}\label{sec:general_topology}
\subsection{Topological spaces}\label{sec:topological_spaces}
\import{src/}{topological_spaces.tex}
\subsection{Initial and final topologies}\label{sec:initial_final_topologies}
\import{src/}{initial_final_topologies.tex}
\subsection{Compact sets}\label{sec:compact_sets}
\import{src/}{compact_sets.tex}
\section{Baire spaces}\label{sec:baire_spaces}
\import{src/}{baire_spaces.tex}

\section{Metric spaces}\label{sec:metric_spaces}
\subsection{Hausdorff distance}\label{sec:hausdorff_distance}
\import{src/}{hausdorff_distance.tex}
\subsection{Totally bounded sets}\label{sec:totally_bounded_sets}
\import{src/}{totally_bounded_sets.tex}
\subsection{Noncompactness measures}\label{sec:noncompactness_measures}
\import{src/}{noncompactness_measures.tex}

\section{Logic}\label{sec:logic}
\subsection{First order logic}\label{sec:first_order_logic}
\import{src/}{first_order_logic.tex}

\section{Set theory}\label{sec:set_theory}
\subsection{Sets}\label{sec:sets}
\import{src/}{sets.tex}
\subsection{Relations}\label{sec:relations}
\import{src/}{relations.tex}
\subsection{Functions}\label{sec:functions}
\import{src/}{functions.tex}

\section{Category theory}\label{sec:category_theory}
\subsection{Categories}\label{sec:categories}
\import{src/}{categories.tex}
\subsection{Functors}\label{sec:functors}
\import{src/}{functors.tex}
\subsection{Limits}\label{sec:categorical_limits}
\import{src/}{categorical_limits.tex}
\subsection{Enriched categories}\label{sec:enriched_categories}
\import{src/}{enriched_categories.tex}

\section{Lists}

\listofaoc\label{list:aoc}
\listoflem\label{list:lem}
\listofusc\label{list:usc}

\printbibliography

\end{document}
