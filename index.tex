\documentclass[numbers=endperiod, bibliography=totocnumbered]{scrartcl}

% Base packages
\usepackage[T2A]{fontenc}
\usepackage[utf8]{inputenc}
\usepackage[english]{babel}
\usepackage[pdfencoding=unicode]{hyperref}
\usepackage[style=alphabetic, citestyle=alphabetic]{biblatex}
\usepackage{csquotes}

% Base math packages
\usepackage{amsmath}
\usepackage{amssymb}
\usepackage{amsthm}
\usepackage{mathtools}
\usepackage{cleveref}

% Misc packages
\usepackage{ulem}
\usepackage{import}
\usepackage{enumitem}

% Custom packages
\usepackage{packages/macros}
\usepackage{packages/theorem_styles}

% Bibliography
\addbibresource{references.bib}

% Document
\title{Anthology}
\subtitle{\href{https://github.com/v--/anthology}{https://github.com/v--/anthology}}
\author{Ianis Vasilev, \Email{ianis@ivasilev.net}}
\date{}

% https://tex.stackexchange.com/questions/171999/overfull-hbox-in-biblatex
\emergencystretch=1em
% https://tex.stackexchange.com/questions/13715/how-to-suppress-overfull-hbox-warnings-up-to-some-maximum
\hfuzz=20pt
% https://tex.stackexchange.com/questions/138/what-are-underfull-hboxes-and-vboxes-and-how-can-i-get-rid-of-them
\hbadness=3000

% https://tex.stackexchange.com/questions/156049/how-can-i-get-cleveref-enumitem-to-print-not-item-e-but-proposition-1
\newlist{defenum}{enumerate}{1}
\setlist[defenum]{label=\alph*), ref=\theproposition~(\alph*)}
\crefalias{defenumi}{definition}

\begin{document}

\maketitle

\begin{abstract}
  This document contains some of the mathematics notes and proofs that I write while studying. Having all these notes in one place is quite helpful for later reference.
\end{abstract}

\tableofcontents

\section{Analysis}\label{sec:analysis}
\subsection{Fundamental notions}\label{sec:analysis/fundamental_notions}
\subsubsection{Linear combinations}\label{sec:linear_combinations}
\import{src/}{linear_combinations.tex}
\subsubsection{Topological vector spaces}\label{sec:topological_vector_spaces}
\import{src/}{topological_vector_spaces.tex}
\subsubsection{Differentiability}\label{sec:differentiability}
\import{src/}{differentiability.tex}
\subsubsection{Subdifferentials}\label{sec:subdifferentials}
\import{src/}{subdifferentials.tex}

\subsection{Banach spaces}\label{sec:banach_spaces}
\subsubsection{Dentable sets}\label{sec:dentable_sets}
\import{src/}{dentable_sets.tex}
\subsubsection{Asplund spaces}\label{sec:asplund_spaces}
\import{src/}{asplund_spaces.tex}

\subsection{Convex analysis}\label{sec:convex_analysis}
\subsubsection{Convex functions}\label{sec:convex_functions}
\import{src/}{convex_functions.tex}

\subsection{Nonsmooth analysis}\label{sec:nonsmooth_analysis}
\subsubsection{Clarke generalized gradients}\label{sec:clarke_gradients}
\import{src/}{clarke_gradients.tex}

\section{Topology}\label{sec:topology}
\subsection{Fundamental notions}\label{sec:topology/fundamental_notions}
\subsubsection{Compact sets}\label{sec:compact_sets}
\import{src/}{compact_sets.tex}
\subsection{Baire spaces}\label{sec:baire_spaces}
\import{src/}{baire_spaces.tex}
\subsection{Metric spaces}\label{sec:metric_spaces}
\subsubsection{Hausdorff distance}\label{sec:hausdorff_distance}
\import{src/}{hausdorff_distance.tex}
\subsubsection{Totally bounded sets}\label{sec:totally_bounded_sets}
\import{src/}{totally_bounded_sets.tex}

\section{Logic}\label{sec:logic}
\subsection{Set theory}\label{sec:sets}
\subsubsection{Relations}\label{sec:relations}
\import{src/}{relations.tex}
\subsubsection{Functions}\label{sec:functions}
\import{src/}{functions.tex}

\printbibliography

\end{document}
